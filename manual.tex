% $Id: manual.tex,v 1.27 1997/02/21 15:19:37 roberto Exp roberto $

\documentstyle[fullpage,11pt,bnf]{article}

\newcommand{\rw}[1]{{\bf #1}}
\newcommand{\see}[1]{(see Section~\ref{#1})}
\newcommand{\nil}{{\bf nil}}
\newcommand{\Line}{\rule{\linewidth}{.5mm}}
\def\tecgraf{{\sf TeC\kern-.21em\lower.7ex\hbox{Graf}}}

\newcommand{\Index}[1]{#1\index{#1}}
\newcommand{\IndexVerb}[1]{{\tt #1}\index{#1}}
\newcommand{\Def}[1]{{\em #1}\index{#1}}
\newcommand{\Deffunc}[1]{\index{#1}}

\newcommand{\ff}{$\bullet$\ }

\newcommand{\Version}{2.5}

\makeindex

\begin{document}

\title{Reference Manual of the Programming Language Lua \Version}

\author{%
Roberto Ierusalimschy\quad
Luiz Henrique de Figueiredo\quad
Waldemar Celes
\vspace{1.0ex}\\
\smallskip
\small\tt lua@icad.puc-rio.br
\vspace{2.0ex}\\
%MCC 08/95 ---
\tecgraf\ --- Departamento de Inform\'atica --- PUC-Rio
}

\date{\small \verb$Date: 1997/02/21 15:19:37 $}

\maketitle

\thispagestyle{empty}
\pagestyle{empty}

\begin{abstract}
\noindent
Lua is an extension programming language designed to be used
as a configuration language for any program that needs one.
This document describes version \Version\ of the Lua programming language and
the API that allows interaction between Lua programs and their host C programs.
The document also presents some examples of using the main
features of the system.
\end{abstract}

\vspace{4ex}
\begin{quotation}
\small
\begin{center}{\bf Sum\'ario}\end{center}
\vspace{1ex}
\noindent
Lua \'e uma linguagem de extens\~ao projetada para ser usada como
linguagem de configura\c{c}\~ao em qualquer programa que precise de
uma.
Este documento descreve a vers\~ao \Version\ da linguagem de
programa\c{c}\~ao Lua e a Interface de Programa\c{c}\~ao (API) que permite
a intera\c{c}\~ao entre programas Lua e programas C hospedeiros.
O documento tamb\'em apresenta alguns exemplos de uso das principais
ca\-racte\-r\'{\i}sticas do sistema.
\end{quotation}


\vfill
\begin{quotation}
\noindent
\footnotesize
Copyright (c) 1994--1996 TeCGraf, PUC-Rio.  Written by Waldemar Celes Filho,
Roberto Ierusalimschy, Luiz Henrique de Figueiredo.  All rights reserved.
%
Permission is hereby granted, without written agreement and without license or
royalty fees, to use, copy, modify, and distribute this software and its
documentation for any purpose, subject to the following conditions:
%
The above copyright notice and this permission notice shall appear in all
copies or substantial portions of this software.
%
The name "Lua" cannot be used for any modified form of this software that does
not originate from the authors.  Nevertheless, the name "Lua" may and should be
used to designate the language implemented and described in this package,
even if embedded in any other system, as long as its syntax and semantics
remain unchanged.
%
The authors specifically disclaim any warranties, including, but not limited
to, the implied warranties of merchantability and fitness for a particular
purpose.  The software provided hereunder is on an "as is" basis, and the
authors have no obligation to provide maintenance, support, updates,
enhancements, or modifications.  In no event shall TeCGraf, PUC-Rio, or the
authors be liable to any party for direct, indirect, special, incidental, or
consequential damages arising out of the use of this software and its
documentation.
\end{quotation}
\vfill

\newpage

\tableofcontents

\newpage
\setcounter{page}{1}
\pagestyle{plain}


\section{Introduction}

Lua is an extension programming language designed to support
general procedural programming features with data description
facilities.
It is intended to be used as a
light-weight, but powerful, configuration language for any
program that needs one.
Lua has been designed and implemented by
W.~Celes,
R.~Ierusalimschy and
L.~H.~de Figueiredo.

Lua is implemented as a library, written in C.
Being an extension language, Lua has no notion of a ``main'' program:
it only works {\em embedded\/} in a host client,
called the {\em embedding\/} program.
This host program can invoke functions to execute a piece of
code in Lua, can write and read Lua variables,
and can register C functions to be called by Lua code.
Through the use of C functions, Lua can be augmented to cope with
many, completely different domains,
thus creating customized programming languages sharing a syntactical framework.

Lua is free-distribution software,
and provided as usual with no guarantees.
The implementation described in this manual is available
at the following URL's:
\begin{verbatim}
   http://www.inf.puc-rio.br/~roberto/lua.html
   ftp://ftp.icad.puc-rio.br/pub/lua/lua.tar.gz
\end{verbatim}


\section{Environment and Chunks}

All statements in Lua are executed in a \Def{global environment}.
This environment, which keeps all global variables and functions,
is initialized at the beginning of the embedding program and
persists until its end.

The global environment can be manipulated by Lua code or
by the embedding program,
which can read and write global variables
using functions in the library that implements Lua.

\Index{Global variables} do not need declaration.
Any variable is assumed to be global unless explicitly declared local
\see{localvar}.
Before the first assignment, the value of a global variable is \nil.

The unit of execution of Lua is called a \Def{chunk}.
The syntax%
\footnote{As usual, \rep{{\em a}} means 0 or more {\em a\/}'s,
\opt{{\em a}} means an optional {\em a} and \oneormore{{\em a}} means
one or more {\em a\/}'s.}
for chunks is:
\begin{Produc}
\produc{chunk}{\rep{statement \Or function} \opt{ret}}
\end{Produc}%
A chunk may contain statements and function definitions,
and may be in a file or in a string inside the host program.
A chunk may optionally end with a \verb|return| statement \see{return}.
When a chunk is executed, first all its functions and statements are compiled,
then the statements are executed in sequential order.
All modifications a chunk effects on the global environment persist
after its end.
Those include modifications to global variables and definitions
of new functions%
\footnote{Actually, a function definition is an
assignment to a global variable \see{TypesSec}.}.

Chunks may be pre-compiled; see program \IndexVerb{luac} for details.
Text files with chunks and their binary pre-compiled forms
are interchangeable.
Lua automatically detects the file type and acts accordingly.
\index{pre-compilation}

\section{\Index{Types}} \label{TypesSec}

Lua is a dynamically typed language.
Variables do not have types; only values do.
All values carry their own type.
Therefore, there are no type definitions in the language.

There are seven \Index{basic types} in Lua: \Def{nil}, \Def{number},
\Def{string}, \Def{function}, \Def{CFunction}, \Def{userdata},
and \Def{table}.
{\em Nil\/} is the type of the value \nil,
whose main property is to be different from any other value.
{\em Number\/} represents real (floating point) numbers,
while {\em string\/} has the usual meaning.

Functions are considered first-class values in Lua.
This means that functions can be stored in variables,
passed as arguments to other functions and returned as results.
When a function is defined in Lua, its body is compiled and stored
in a given variable.
Lua can call (and manipulate) functions written in Lua and
functions written in C; the latter have type {\em CFunction}.

The type {\em userdata\/} is provided to allow
arbitrary \Index{C pointers} to be stored in Lua variables.
It corresponds to \verb'void*' and has no pre-defined operations in Lua,
besides assignment and equality test.
However, by using fallbacks, the programmer may define operations
for {\em userdata\/} values; \see{fallback}.

The type {\em table\/} implements \Index{associative arrays},
that is, \Index{arrays} that can be indexed not only with numbers,
but with any value (except \nil).
Therefore, this type may be used not only to represent ordinary arrays,
but also symbol tables, sets, records, etc.
To represent \Index{records}, Lua uses the field name as an index.
The language supports this representation by
providing \verb'a.name' as syntactic sugar for \verb'a["name"]'.
Tables may also carry methods.
Because functions are first class values,
table fields may contain functions.
The form \verb't:f(x)' is syntactic sugar for \verb't.f(t,x)',
which calls the method \verb'f' from the table \verb't' passing
itself as the first parameter.

It is important to notice that tables are {\em objects}, and not values.
Variables cannot contain tables, only {\em references\/} to them.
Assignment, parameter passing and returns always manipulate references
to tables, and do not imply any kind of copy.
Moreover, tables must be explicitly created before used
\see{tableconstructor}.



\section{The Language}

This section describes the lexis, the syntax and the semantics of Lua.


\subsection{Lexical Conventions} \label{lexical}

Lua is a case sensitive language.
\Index{Identifiers} can be any string of letters, digits, and underscores,
not beginning with a digit.
The following words are reserved, and cannot be used as identifiers:
\index{reserved words}
\begin{verbatim}
      and       do        else      elseif
      end       function  if        local
      nil       not       or        repeat
      return    then      until     while
\end{verbatim}

The following strings denote other \Index{tokens}:
\begin{verbatim}
         ~=  <=  >=  <   >   ==  =   ..  +   -   *   /
         %   (   )   {   }   [   ]   ;   ,   .
\end{verbatim}

\Index{Literal strings} can be delimited by matching single or double quotes,
and can contain the C-like escape sequences
\verb-'\n'-, \verb-'\t'- and \verb-'\r'-.
Literal strings can also be delimited by matching \verb'[[ ... ]]'.
Literals in this bracketed form may run for several lines,
may contain nested \verb'[[ ... ]]' pairs,
and do not interpret escape sequences.
This form is specially convenient for
handling text that has quoted strings in it.

\Index{Comments} start anywhere outside a string with a
double hyphen (\verb'--') and run until the end of the line.
Moreover, if the first line of a chunk file starts with \verb'#',
this line is skipped%
\footnote{This facility allows the use of Lua as a script interpreter
in Unix systems \see{lua-sa}.}.

\Index{Numerical constants} may be written with an optional decimal part,
and an optional decimal exponent.
Examples of valid numerical constants are:
\begin{verbatim}
       4     4.0     0.4     4.57e-3     0.3e12
\end{verbatim}


\subsection{\Index{Coercion}} \label{coercion}

Lua provides some automatic conversions between values.
Any arithmetic operation applied to a string tries to convert
that string to a number, following the usual rules.
Conversely, whenever a number is used when a string is expected,
that number is converted to a string, according to the following rule:
if the number is an integer, it is written without exponent or decimal point;
otherwise, it is formatted following the \verb'%g'
conversion specification of the \verb'printf' function in the
standard C library.
For complete control on how numbers are converted to strings,
use the \verb|format| function \see{format}.


\subsection{\Index{Adjustment}} \label{adjust}

Functions in Lua can return many values.
Because there are no type declarations,
the system does not know how many values a function will return,
or how many parameters it needs.
Therefore, sometimes, a list of values must be {\em adjusted\/}, at run time,
to a given length.
If there are more values than are needed, then the last values are thrown away.
If there are more needs than values, then the list is extended with as
many  \nil's as needed.
Adjustment occurs in multiple assignment and function calls.


\subsection{Statements}

Lua supports an almost conventional set of \Index{statements},
similar to those in Pascal or C.
The conventional commands include
assignment, control structures and procedure calls.
Non-conventional commands include table constructors
\see{tableconstructor},
and local variable declarations \see{localvar}.

\subsubsection{Blocks}
A \Index{block} is a list of statements, which are executed sequentially.
Any statement can be optionally followed by a semicolon:
\begin{Produc}
\produc{block}{\rep{stat sc} \opt{ret}}
\produc{sc}{\opt{\ter{;}}}
\end{Produc}%
For syntactic reasons, a \IndexVerb{return} statement can only be written
as the last statement of a block.
This restriction also avoids some ``statement not reached'' errors.

\subsubsection{\Index{Assignment}} \label{assignment}
The language allows \Index{multiple assignment}.
Therefore, the syntax defines a list of variables on the left side,
and a list of expressions on the right side.
Both lists have their elements separated by commas:
\begin{Produc}
\produc{stat}{varlist1 \ter{=} explist1}
\produc{varlist1}{var \rep{\ter{,} var}}
\end{Produc}%
This statement first evaluates all values on the right side
and eventual indices on the left side,
and then makes the assignments.
Therefore, it can be used to exchange two values, as in
\begin{verbatim}
   x, y = y, x
\end{verbatim}
Before the assignment, the list of values is {\em adjusted\/} to
the length of the list of variables \see{adjust}.

A single name can denote a global or a local variable,
or a formal parameter:
\begin{Produc}
\produc{var}{name}
\end{Produc}%
Square brackets are used to index a table:
\begin{Produc}
\produc{var}{var \ter{[} exp1 \ter{]}}
\end{Produc}%
If \verb'var' results in a table value,
the field indexed by the expression value gets the assigned value.
Otherwise, the fallback {\em settable\/} is called,
with three parameters: the value of \verb'var',
the value of expression, and the value being assigned to it;
\see{fallback}.

The syntax \verb'var.NAME' is just syntactic sugar for
\verb'var["NAME"]':
\begin{Produc}
\produc{var}{var \ter{.} name}
\end{Produc}%

\subsubsection{Control Structures}
The \Index{condition expression} of a control structure may return any value.
All values different from \nil\ are considered true;
only \nil\ is considered false.
{\tt if}'s, {\tt while}'s and {\tt repeat}'s have the usual meaning.

\index{while-do}\index{repeat-until}\index{if-then-else}
\begin{Produc}
\produc{stat}{\rwd{while} exp1 \rwd{do} block \rwd{end} \OrNL
\rwd{repeat} block \rwd{until} exp1 \OrNL
\rwd{if} exp1 \rwd{then} block \rep{elseif}
   \opt{\rwd{else} block} \rwd{end}}
\produc{elseif}{\rwd{elseif} exp1 \rwd{then} block}
\end{Produc}

A {\tt return} is used to return values from a function or a chunk.
\label{return}
Because they may return more than one value,
the syntax for a \Index{return statement} is:
\begin{Produc}
\produc{ret}{\rwd{return} explist \opt{sc}}
\end{Produc}

\subsubsection{Function Calls as Statements} \label{funcstat}
Because of possible side-effects,
function calls can be executed as statements:
\begin{Produc}
\produc{stat}{functioncall}
\end{Produc}%
In this case, returned values are thrown away.
Function calls are explained in Section~\ref{functioncall}.

\subsubsection{Local Declarations} \label{localvar}
\Index{Local variables} may be declared anywhere inside a block.
Their scope begins after the declaration and lasts until the
end of the block.
The declaration may include an initial assignment:
\begin{Produc}
\produc{stat}{\rwd{local} declist \opt{init}}
\produc{declist}{name \rep{\ter{,} name}}
\produc{init}{\ter{=} explist1}
\end{Produc}%
If present, an initial assignment has the same semantics
of a multiple assignment.
Otherwise, all variables are initialized with \nil.


\subsection{\Index{Expressions}}

\subsubsection{\Index{Simple Expressions}}
Simple expressions are:
\begin{Produc}
\produc{exp}{\ter{(} exp \ter{)}}
\produc{exp}{\rwd{nil}}
\produc{exp}{\ter{number}}
\produc{exp}{\ter{literal}}
\produc{exp}{var}
\end{Produc}%
Numbers (numerical constants) and
string literals are explained in Section~\ref{lexical}.
Variables are explained in Section~\ref{assignment}.

\subsubsection{Arithmetic Operators}
Lua supports the usual \Index{arithmetic operators}.
These operators are the binary
\verb'+' (addition),
\verb'-' (subtraction),
\verb'*' (multiplication),
\verb'/' (division) and \verb'^' (exponentiation),
and the unary \verb'-' (negation).
If the operands are numbers, or strings that can be converted to
numbers, according to the rules given in Section~\ref{coercion},
then all operations except exponentiation have the usual meaning.
Otherwise, the fallback ``arith'' is called \see{fallback}.
An exponentiation always calls this fallback.
The standard mathematical library redefines this fallback,
giving the expected meaning to \Index{exponentiation}
\see{mathlib}.

\subsubsection{Relational Operators}
Lua provides the following \Index{relational operators}:
\begin{verbatim}
       <   >   <=  >=  ~=  ==
\end{verbatim}
All these return \nil\ as false and a value different from \nil\
(actually the number 1) as true.

Equality first compares the types of its operands.
If they are different, then the result is \nil.
Otherwise, their values are compared.
Numbers and strings are compared in the usual way.
Tables, CFunctions, and functions are compared by reference,
that is, two tables are considered equal only if they are the same table.
The operator \verb'~=' is exactly the negation of equality (\verb'==').
Note that the conversion rules of Section~\ref{coercion}
do not apply to equality comparisons.
Thus, \verb|"0"==0| evaluates to false.

The other operators work as follows.
If both arguments are numbers, then they are compared as such.
Otherwise, if both arguments can be converted to strings,
their values are compared using lexicographical order.
Otherwise, the ``order'' fallback is called \see{fallback}.
%Note that the conversion rules of Section~\ref{coercion}
%do apply to order operators.
%Thus, \verb|"2">"12"| evaluates to true.

\subsubsection{Logical Operators}
Like control structures, all logical operators
consider \nil\ as false and anything else as true.
The \Index{logical operators} are:
\index{and}\index{or}\index{not}
\begin{verbatim}
             and   or   not
\end{verbatim}
The operator \verb'and' returns \nil\ if its first argument is \nil;
otherwise it returns its second argument.
The operator \verb'or' returns its first argument
if it is different from \nil;
otherwise it returns its second argument.
Both \verb'and' and \verb'or' use \Index{short-cut evaluation},
that is,
the second operand is evaluated only if necessary.

\subsubsection{Concatenation}
Lua offers a string \Index{concatenation} operator,
denoted by ``\IndexVerb{..}''.
If operands are strings or numbers, then they are converted to
strings according to the rules in Section~\ref{coercion}.
Otherwise, the fallback ``concat'' is called \see{fallback}.

\subsubsection{Precedence}
\Index{Operator precedence} follows the table below,
from the lower to the higher priority:
\begin{verbatim}
             and   or
             <   >   <=  >=  ~=  ==
             ..
             +   -
             *   /
             not  - (unary)
             ^
\end{verbatim}
All binary operators are left associative,
except for \verb'^' (exponentiation),
which is right associative.

\subsubsection{Table Constructors} \label{tableconstructor}
Table \Index{constructors} are expressions that create tables;
every time a constructor is evaluated, a new table is created.
Constructors can be used to create empty tables,
or to create a table and initialize some fields.

The general syntax for constructors is:
\begin{Produc}
\produc{tableconstructor}{\ter{\{} fieldlist \ter{\}}}
\produc{fieldlist}{lfieldlist \Or ffieldlist \Or lfieldlist \ter{;} ffieldlist}
\produc{lfieldlist}{\opt{lfieldlist1}}
\produc{ffieldlist}{\opt{ffieldlist1}}
\end{Produc}

The form {\em lfieldlist1\/} is used to initialize lists.
\begin{Produc}
\produc{lfieldlist1}{exp \rep{\ter{,} exp} \opt{\ter{,}}}
\end{Produc}%
The expressions in the list are assigned to consecutive numerical indices,
starting with 1.
For example:
\begin{verbatim}
   a = {"v1", "v2", 34}
\end{verbatim}
is essentialy equivalent to:
\begin{verbatim}
   temp = {}
   temp[1] = "v1"
   temp[2] = "v2"
   temp[3] = 34
   a = temp
\end{verbatim}

The next form initializes named fields in a table:
\begin{Produc}
\produc{ffieldlist1}{ffield \rep{\ter{,} ffield} \opt{\ter{,}}}
\produc{ffield}{name \ter{=} exp}
\end{Produc}%
For example:
\begin{verbatim}
   a = {x = 1, y = 3}
\end{verbatim}
is essentialy equivalent to:
\begin{verbatim}
   temp = {}
   temp.x = 1    -- or temp["x"] = 1
   temp.y = 3    -- or temp["y"] = 3
   a = temp
\end{verbatim}


\subsubsection{Function Calls}  \label{functioncall}
A \Index{function call} has the following syntax:
\begin{Produc}
\produc{functioncall}{var realParams}
\end{Produc}%
Here, \verb'var' can be any variable (global, local, indexed, etc).
If its value has type {\em function\/} or {\em CFunction},
then this function is called.
Otherwise, the ``function'' fallback is called,
having as first parameter the value of \verb'var',
and then the original call parameters.

The form:
\begin{Produc}
\produc{functioncall}{var \ter{:} name realParams}
\end{Produc}%
can be used to call ``methods''.
A call \verb'var:name(...)'
is syntactic sugar for
\begin{verbatim}
  var.name(var, ...)
\end{verbatim}
except that \verb'var' is evaluated only once.

\begin{Produc}
\produc{realParams}{\ter{(} \opt{explist1} \ter{)}}
\produc{realParams}{tableconstructor}
\produc{explist1}{exp1 \rep{\ter{,} exp1}}
\end{Produc}%
All argument expressions are evaluated before the call;
then the list of \Index{arguments} is adjusted to
the length of the list of parameters \see{adjust};
finally, this list is assigned to the formal parameters.
A call of the form \verb'f{...}' is syntactic sugar for
\verb'f({...})', that is,
the parameter list is a single new table.

Because a function can return any number of results
\see{return},
the number of results must be adjusted before used.
If the function is called as a statement \see{funcstat},
its return list is adjusted to 0,
thus discarding all returned values.
If the function is called in a place that needs a single value
(syntactically denoted by the non-terminal \verb'exp1'),
then its return list is adjusted to 1,
thus discarding all returned values,
except the first one.
If the function is called in a place that can hold many values
(syntactically denoted by the non-terminal \verb'exp'),
then no adjustment is made.


\subsection{\Index{Function Definitions}}

Functions in Lua can be defined anywhere in the global level of a chunk.
The syntax for function definition is:
\begin{Produc}
\produc{function}{\rwd{function} var \ter{(} \opt{parlist1} \ter{)}
  block \rwd{end}}
\end{Produc}

When Lua pre-compiles a chunk,
all its function bodies are pre-compiled, too.
Then, when Lua ``executes'' the function definition,
its body is stored, with type {\em function},
into the variable \verb'var'.
It is in this sense that
a function definition is an assignment to a global variable.

Parameters act as local variables,
initialized with the argument values.
\begin{Produc}
\produc{parlist1}{name \rep{\ter{,} name}}
\end{Produc}

Results are returned using the \verb'return' statement \see{return}.
If control reaches the end of a function without a return instruction,
then the function returns with no results.

There is a special syntax for defining \Index{methods},
that is, functions that have an extra parameter \Def{self}.
\begin{Produc}
\produc{function}{\rwd{function} var \ter{:} name \ter{(} \opt{parlist1}
  \ter{)} block \rwd{end}}
\end{Produc}%
Thus, a declaration like
\begin{verbatim}
function v:f (...)
  ...
end
\end{verbatim}
is equivalent to
\begin{verbatim}
function v.f (self, ...)
  ...
end
\end{verbatim}
that is, the function gets an extra formal parameter called \verb'self'.
Notice that
the variable \verb'v' must have been previously initialized with a table value.


\subsection{Fallbacks} \label{fallback}

Lua provides a powerful mechanism to extend its semantics,
called \Def{fallbacks}.
A fallback is a programmer defined function
that is called whenever Lua does not know how to proceed.

Lua supports the following fallbacks,
identified by the given strings:
\begin{description}
\item[``arith'':]\index{arithmetic fallback}
called when an arithmetic operation is applied to non numerical operands,
or when the binary \verb'^' operation (exponentiation) is called.
It receives three arguments:
the two operands (the second one is \nil\ when the operation is unary minus)
and one of the following strings describing the offended operator:
\begin{verbatim}
  add  sub  mul  div  pow  unm
\end{verbatim}
Its return value is the final result of the arithmetic operation.
The default handler issues an error.
\item[``order'':]\index{order fallback}
called when an order comparison is applied to non numerical or
non string operands.
It receives three arguments:
the two operands and
one of the following strings describing the offended operator:
\begin{verbatim}
  lt gt le ge
\end{verbatim}
Its return value is the final result of the comparison operation.
The default handler issues an error.
\item[``concat'':]\index{concatenation fallback}
called when a concatenation is applied to non string operands.
It receives the two operands as arguments.
Its return value is the final result of the concatenation operation.
The default handler issues an error.
\item[``index'':]\index{index fallback}
called when Lua tries to retrieve the value of an index
not present in a table.
It receives as arguments the table and the index.
Its return value is the final result of the indexing operation.
The default handler returns \nil.
\item[``getglobal'':]\index{index getglobal}
called when Lua tries to retrieve the value of a global variable
which has a \nil\ value (or which has not been initialized).
It receives as argument the name of the variable.
Its return value is the final result of the expression.
The default handler returns \nil.
\item[``gettable'':]\index{gettable fallback}
called when Lua tries to index a non table value.
It receives as arguments the non table value and the index.
Its return value is the final result of the indexing operation.
The default handler issues an error.
\item[``settable'':]\index{settable fallback}
called when Lua tries to assign to an index in a non table value.
It receives as arguments the non table value,
the index, and the assigned value.
The default handler issues an error.
\item[``function'':]\index{function fallback}
called when Lua tries to call a non function value.
It receives as arguments the non function value and the
arguments given in the original call.
Its return values are the final results of the call operation.
The default handler issues an error.
\item[``gc'':]
called during garbage collection.
It receives as argument the table being collected.
After each run of the collector this function is called with argument \nil,
to signal the completion of the garbage collection.
Because this function operates during garbage collection,
it must be used with great care,
and programmers should avoid the creation of new objects
(tables or strings) in this function.
The default handler does nothing.
\item[``error'':]\index{error fallback}
called when an error occurs.
It receives as argument a string describing the error.
The default handler prints the message on the standard error output
(\verb|stderr|).
\end{description}

The function \IndexVerb{setfallback} is used to change a fallback handler.
Its first argument is the name of a fallback condition,
and the second argument is the new function to be called.
It returns the old handler function for the given fallback.

Section~\ref{exfallback} shows an example of the use of fallbacks.


\subsection{Error Handling} \label{error}

Because Lua is an extension language,
all Lua actions start from C code calling a function from the Lua library.
Whenever an error occurs during Lua compilation or execution,
the ``error'' fallback function is called,
and then the corresponding function from the library
(\verb'lua_dofile', \verb'lua_dostring',
\verb'lua_call', or \verb'lua_callfunction')
is terminated returning an error condition.

The only argument to the ``error'' fallback function is a string
describing the error.
The standard I/O library redefines this fallback,
using the debug facilities \see{debugI},
in order to print some extra information,
like the call stack.
To provide more information about errors,
Lua programs can include the compilation pragma \verb'$debug'.
\index{debug pragma}\label{pragma}
This pragma must be written in a line by itself.
When an error occurs in a program compiled with this option,
the error routine is able to print the number of the lines where the calls
(and the error) were made.
If needed, it is possible to change the ``error'' fallback handler
\see{fallback}.

Lua code can explicitly generate an error by calling the built-in
function \verb'error' \see{pdf-error}.


\section{The Application Program Interface}

This section describes the API for Lua, that is,
the set of C functions available to the host program to communicate
with the library.
The API functions can be classified in the following categories:
\begin{enumerate}
\item executing Lua code;
\item converting values between C and Lua;
\item manipulating (reading and writing) Lua objects;
\item calling Lua functions;
\item C functions to be called by Lua;
\item manipulating references to Lua Objects.
\end{enumerate}
All API functions and related types and constants
are declared in the header file \verb'lua.h'.

\subsection{Executing Lua Code}
A host program can execute Lua chunks written in a file or in a string
using the following functions:
\Deffunc{lua_dofile}\Deffunc{lua_dostring}
\begin{verbatim}
int lua_dofile   (char *filename);
int lua_dostring (char *string);
\end{verbatim}
Both functions return an error code:
0, in case of success; non zero, in case of errors.
More specifically, \verb'lua_dofile' returns 2 if for any reason
it could not open the file.
The function \verb'lua_dofile', if called with argument \verb'NULL',
executes the \verb|stdin| stream.
Function \verb'lua_dofile' is also able to execute pre-compiled chunks.
It automatically detects whether the file is text or binary,
and loads it accordingly (see program \IndexVerb{luac}).

\subsection{Converting Values between C and Lua} \label{valuesCLua}
Because Lua has no static type system,
all values passed between Lua and C have type
\verb'lua_Object'\Deffunc{lua_Object},
which works like an abstract type in C that can hold any Lua value.
Values of type \verb'lua_Object' have no meaning outside Lua;
for instance,
the comparisson of two \verb"lua_Object's" is undefined.

Because Lua has automatic memory management and garbage collection,
a \verb'lua_Object' has a limited scope,
and is only valid inside the {\em block\/} where it was created.
A C function called from Lua is a block,
and its parameters are valid only until its end.
It is good programming practice to convert Lua objects to C values
as soon as they are available,
and never to store \verb'lua_Object's in C global variables.

When C code calls Lua repeatedly, as in a loop,
objects returned by these calls accumulate,
and may create a memory problem.
To avoid this,
nested blocks can be defined with the functions:
\begin{verbatim}
void           lua_beginblock           (void);
void           lua_endblock             (void);
\end{verbatim}
After the end of the block,
all \verb'lua_Object''s created inside it are released.
The use of explicit nested blocks is encouraged.

To check the type of a \verb'lua_Object',
the following function is available:
\Deffunc{lua_type}
\begin{verbatim}
int            lua_type                 (lua_Object object);
\end{verbatim}
plus the following macros and functions:
\Deffunc{lua_isnil}\Deffunc{lua_isnumber}\Deffunc{lua_isstring}
\Deffunc{lua_istable}\Deffunc{lua_iscfunction}\Deffunc{lua_isuserdata}
\Deffunc{lua_isfunction}
\begin{verbatim}
int            lua_isnil                (lua_Object object);
int            lua_isnumber             (lua_Object object);
int            lua_isstring             (lua_Object object);
int            lua_istable              (lua_Object object);
int            lua_isfunction           (lua_Object object);
int            lua_iscfunction          (lua_Object object);
int            lua_isuserdata           (lua_Object object);
\end{verbatim}
All macros return 1 if the object is compatible with the given type,
and 0 otherwise.
The function \verb'lua_isnumber' accepts numbers and numerical strings,
whereas
\verb'lua_isstring' accepts strings and numbers \see{coercion},
and \verb'lua_isfunction' accepts Lua and C functions.
The function \verb'lua_type' can be used to distinguish between
different kinds of user data.

To translate a value from type \verb'lua_Object' to a specific C type,
the programmer can use:
\Deffunc{lua_getnumber}\Deffunc{lua_getstring}
\Deffunc{lua_getcfunction}\Deffunc{lua_getuserdata}
\begin{verbatim}
double         lua_getnumber            (lua_Object object);
char          *lua_getstring            (lua_Object object);
lua_CFunction  lua_getcfunction         (lua_Object object);
void          *lua_getuserdata          (lua_Object object);
\end{verbatim}
\verb'lua_getnumber' converts a \verb'lua_Object' to a floating-point number.
This \verb'lua_Object' must be a number or a string convertible to number
\see{coercion}; otherwise, the function returns 0.

\verb'lua_getstring' converts a \verb'lua_Object' to a string (\verb'char *').
This \verb'lua_Object' must be a string or a number;
otherwise, the function returns 0 (the \verb|NULL| pointer).
This function does not create a new string, but returns a pointer to
a string inside the Lua environment.
Because Lua has garbage collection, there is no guarantee that such
pointer will be valid after the block ends.

\verb'lua_getcfunction' converts a \verb'lua_Object' to a C function.
This \verb'lua_Object' must have type {\em CFunction\/};
otherwise, the function returns 0 (the \verb|NULL| pointer).
The type \verb'lua_CFunction' is explained in Section~\ref{LuacallC}.

\verb'lua_getuserdata' converts a \verb'lua_Object' to \verb'void*'.
This \verb'lua_Object' must have type {\em userdata\/};
otherwise, the function returns 0 (the \verb|NULL| pointer).

The reverse process, that is, passing a specific C value to Lua,
is done by using the following functions:
\Deffunc{lua_pushnumber}\Deffunc{lua_pushstring}
\Deffunc{lua_pushcfunction}\Deffunc{lua_pushusertag}
\Deffunc{lua_pushuserdata}
\begin{verbatim}
void           lua_pushnumber           (double n);
void           lua_pushstring           (char *s);
void           lua_pushcfunction        (lua_CFunction f);
void           lua_pushusertag          (void *u, int tag);
\end{verbatim}
plus the macro:
\begin{verbatim}
void           lua_pushuserdata         (void *u);
\end{verbatim}
All of them receive a C value,
convert it to a corresponding \verb'lua_Object',
and leave the result on the top of the Lua stack,
where it can be assigned to a Lua variable,
passed as parameter to a Lua function, etc. \label{pushing}

User data can have different tags,
whose semantics are only known to the host program.
Any positive integer can be used to tag a user datum.
When a user datum is retrieved,
the function \verb'lua_type' can be used to get its tag.

To complete the set,
the value \nil\ or a \verb'lua_Object' can also be pushed onto the stack,
with:
\Deffunc{lua_pushnil}\Deffunc{lua_pushobject}
\begin{verbatim}
void           lua_pushnil              (void);
void           lua_pushobject           (lua_Object object);
\end{verbatim}


\subsection{Manipulating Lua Objects}
To read the value of any global Lua variable,
one uses the function:
\Deffunc{lua_getglobal}
\begin{verbatim}
lua_Object     lua_getglobal            (char *varname);
\end{verbatim}
As in Lua, if the value of the global is \nil,
then the ``getglobal'' fallback is called.

To store a value previously pushed onto the stack in a global variable,
there is the function:
\Deffunc{lua_storeglobal}
\begin{verbatim}
void           lua_storeglobal          (char *varname);
\end{verbatim}

Tables can also be manipulated via the API.
The function
\Deffunc{lua_getsubscript}
\begin{verbatim}
lua_Object     lua_getsubscript         (void);
\end{verbatim}
expects on the stack a table and an index,
and returns the contents of the table at that index.
As in Lua, if the first object is not a table,
or the index is not present in the table,
the corresponding fallback is called.

To store a value in an index,
the program must push the table, the index, and the value onto the stack,
and then call the function:
\Deffunc{lua_storesubscript}
\begin{verbatim}
void lua_storesubscript (void);
\end{verbatim}
Again, the ``settable'' fallback is called if a non-table value is used.

Finally, the function
\Deffunc{lua_createtable}
\begin{verbatim}
lua_Object     lua_createtable          (void);
\end{verbatim}
creates and returns a new, empty table.

\begin{quotation}
\noindent
{\em Please note\/}:
Most functions from the Lua library receive parameters through Lua's stack.
Because other functions also use this stack,
it is important that these
parameters be pushed just before the corresponding call,
without intermediate calls to the Lua library.
For instance, suppose the user wants the value of \verb'a[i]',
where \verb'a' and \verb'i' are global Lua variables.
A simplistic solution would be:
\begin{verbatim}
  /* Warning: WRONG CODE */
  lua_Object result;
  lua_pushobject(lua_getglobal("a"));  /* push table */
  lua_pushobject(lua_getglobal("i"));  /* push index */
  result = lua_getsubscript();
\end{verbatim}
This code is incorrect because
the call \verb'lua_getglobal("i")' modifies the stack,
and invalidates the previous pushed value.
A correct solution could be:
\begin{verbatim}
  lua_Object result;
  lua_Object index = lua_getglobal("i");
  lua_pushobject(lua_getglobal("a"));  /* push table */
  lua_pushobject(index);               /* push index */
  result = lua_getsubscript();
\end{verbatim}
The functions {\em lua\_getnumber}, {\em lua\_getstring},
{\em lua\_getuserdata}, and {\em lua\_getcfunction},
plus the family \verb|lua_is*|,
are safe to be called without modifying the stack.
\end{quotation}

\subsection{Calling Lua Functions}
Functions defined in Lua by a chunk executed with
\verb'dofile' or \verb'dostring' can be called from the host program.
This is done using the following protocol:
first, the arguments to the function are pushed onto the Lua stack
\see{pushing}, in direct order, i.e., the first argument is pushed first.
Again, it is important to emphasize that, during this phase,
no other Lua function can be called.

Then, the function is called using
\Deffunc{lua_call}\Deffunc{lua_callfunction}
\begin{verbatim}
int            lua_call                 (char *functionname);
\end{verbatim}
or
\begin{verbatim}
int            lua_callfunction         (lua_Object function);
\end{verbatim}
Both functions return an error code:
0, in case of success; non zero, in case of errors.
Finally, the returned values (a Lua function may return many values)
can be retrieved with the macro
\Deffunc{lua_getresult}
\begin{verbatim}
lua_Object     lua_getresult             (int number);
\end{verbatim}
where \verb'number' is the order of the result, starting with 1.
When called with a number larger than the actual number of results,
this function returns \verb'LUA_NOOBJECT'.

Two special Lua functions have exclusive interfaces:
\verb'error' and \verb'setfallback'.
A C function can generate a Lua error calling the function
\Deffunc{lua_error}
\begin{verbatim}
void lua_error (char *message);
\end{verbatim}
This function never returns.
If the C function has been called from Lua,
then the corresponding Lua execution terminates,
as if an error had occurred inside Lua code.
Otherwise, the whole program terminates with a call to \verb|exit(1)|.
%%LHF: proponho lua_error(char* m, int rc), gerando exit(rc)

Fallbacks can be changed with:
\Deffunc{lua_setfallback}
\begin{verbatim}
lua_Object lua_setfallback (char *name, lua_CFunction fallback);
\end{verbatim}
The first parameter is the fallback name \see{fallback},
and the second is a CFunction to be used as the new fallback.
This function returns a \verb'lua_Object',
which is the old fallback value,
or \nil\ on failure (invalid fallback name).
This old value can be used for chaining fallbacks.

An example of C code calling a Lua function is shown in
Section~\ref{exLuacall}.


\subsection{C Functions} \label{LuacallC}
To register a C function to Lua,
there is the following macro:
\Deffunc{lua_register}
\begin{verbatim}
#define lua_register(n,f)       (lua_pushcfunction(f), lua_storeglobal(n))
/* char *n;         */
/* lua_CFunction f; */
\end{verbatim}
which receives the name the function will have in Lua,
and a pointer to the function.
This pointer must have type \verb'lua_CFunction',
which is defined as
\Deffunc{lua_CFunction}
\begin{verbatim}
typedef void (*lua_CFunction) (void);
\end{verbatim}
that is, a pointer to a function with no parameters and no results.

In order to communicate properly with Lua,
a C function must follow a protocol,
which defines the way parameters and results are passed.

To access its arguments, a C function calls:
\Deffunc{lua_getparam}
\begin{verbatim}
lua_Object     lua_getparam             (int number);
\end{verbatim}
where \verb'number' starts with 1 to get the first argument.
When called with a number larger than the actual number of arguments,
this function returns
\verb'LUA_NOOBJECT'\Deffunc{LUA_NOOBJECT}.
In this way, it is possible to write functions that work with
a variable number of parameters.
The funcion \verb|lua_getparam| can be called in any order,
and many times for the same index.

To return values, a C function just pushes them onto the stack,
in direct order \see{valuesCLua}.
Like a Lua function, a C function called by Lua can also return
many results.

Section~\ref{exCFunction} presents an example of a CFunction.


\subsection{References to Lua Objects}

As noted in Section~\ref{LuacallC}, \verb'lua_Object's are volatile.
If the C code needs to keep a \verb'lua_Object'
outside block boundaries,
it must create a \Def{reference} to the object.
The routines to manipulate references are the following:
\Deffunc{lua_ref}\Deffunc{lua_getref}
\Deffunc{lua_pushref}\Deffunc{lua_unref}
\begin{verbatim}
int            lua_ref (int lock);
lua_Object     lua_getref  (int ref);
void           lua_pushref (int ref);
void           lua_unref (int ref);
\end{verbatim}
The function \verb'lua_ref' creates a reference
to the object that is on the top of the stack,
and returns this reference.
If \verb'lock' is true, the object is {\em locked\/}:
this means the object will not be garbage collected.
Notice that an unlocked reference may be garbage collected.
Whenever the referenced object is needed,
a call to \verb'lua_getref'
returns a handle to it,
whereas \verb'lua_pushref' pushes the object on the stack.
If the object has been collected,
then \verb'lua_getref' returns \verb'LUA_NOOBJECT',
and \verb'lua_pushobject' issues an error.

When a reference is no longer needed,
it can be freed with a call to \verb'lua_unref'.



\section{Predefined Functions and Libraries}

The set of \Index{predefined functions} in Lua is small but powerful.
Most of them provide features that allow some degree of
\Index{reflexivity} in the language.
Some of these features cannot be simulated with the rest of the
Language nor with the standard Lua API.
Others are just convenient interfaces to common API functions.

The libraries, on the other hand, provide useful routines
that are implemented directly through the standard API.
Therefore, they are not necessary to the language,
and are provided as separate C modules.
Currently there are three standard libraries:
\begin{itemize}
\item string manipulation;
\item mathematical functions (sin, log, etc);
\item input and output (plus some system facilities).
\end{itemize}
In order to have access to these libraries,
the host program must call the functions
\verb-strlib_open-, \verb-mathlib_open-, and \verb-iolib_open-,
declared in \verb-lualib.h-.


\subsection{Predefined Functions}

\subsubsection*{\ff{\tt dofile (filename)}}\Deffunc{dofile}
This function receives a file name,
opens it, and executes its contents as a Lua chunk,
or as pre-compiled chunks.
When called without arguments,
it executes the contents of the standard input (\verb'stdin').
If there is any error executing the file, it returns \nil.
Otherwise, it returns the values returned by the chunk,
or a non \nil\ value if the chunk returns no values.
It issues an error when called with a non string argument.
\verb|dofile| is simply an interface to \verb|lua_dofile|.

\subsubsection*{\ff{\tt dostring (string)}}\Deffunc{dostring}
This function executes a given string as a Lua chunk.
If there is any error executing the string, it returns \nil.
Otherwise, it returns the values returned by the chunk,
or a non \nil\ value if the chunk returns no values.
\verb|dostring| is simply an interface to \verb|lua_dostring|.

\subsubsection*{\ff{\tt next (table, index)}}\Deffunc{next}
This function allows a program to traverse all fields of a table.
Its first argument is a table and its second argument
is an index in this table.
It returns the next index of the table and the
value associated with the index.
When called with \nil\ as its second argument,
the function returns the first index
of the table (and its associated value).
When called with the last index, or with \nil\ in an empty table,
it returns \nil.

In Lua there is no declaration of fields;
semantically, there is no difference between a
field not present in a table or a field with value \nil.
Therefore, the function only considers fields with non \nil\ values.
The order in which the indices are enumerated is not specified,
{\em even for numeric indices}.
If the table is modified in any way during a traversal,
the semantics of \verb|next| is undefined.

See Section~\ref{exnext} for an example of the use of this function.
This function cannot be written with the standard API.

\subsubsection*{\ff{\tt nextvar (name)}}\Deffunc{nextvar}
This function is similar to the function \verb'next',
but iterates over the global variables.
Its single argument is the name of a global variable,
or \nil\ to get a first name.
Similarly to \verb'next', it returns the name of another variable
and its value,
or \nil\ if there are no more variables.
There can be no assignments to global variables during the traversal;
otherwise the semantics of \verb|nextvar| is undefined.

See Section~\ref{exnext} for an example of the use of this function.
This function cannot be written with the standard API.

\subsubsection*{\ff{\tt tostring (e)}}\Deffunc{tostring}
This function receives an argument of any type and
converts it to a string in a reasonable format.

\subsubsection*{\ff{\tt print (e1, e2, ...)}}\Deffunc{print}
This function receives any number of arguments,
and prints their values in a reasonable format.
Each value is printed in a new line.
This function is not intended for formatted output,
but as a quick way to show a value,
for instance for error messages or debugging.
See Section~\ref{libio} for functions for formatted output.

\subsubsection*{\ff{\tt tonumber (e)}}\Deffunc{tonumber}
This function receives one argument,
and tries to convert it to a number.
If the argument is already a number or a string convertible
to a number \see{coercion}, then it returns that number;
otherwise, it returns \nil.

\subsubsection*{\ff{\tt type (v)}}\Deffunc{type}
This function allows Lua to test the type of a value.
It receives one argument, and returns its type, coded as a string.
The possible results of this function are
\verb'"nil"' (a string, not the value \nil),
\verb'"number"',
\verb'"string"',
\verb'"table"',
\verb'"function"' (returned both for C functions and Lua functions),
and \verb'"userdata"'.

Besides this string, the function returns a second result,
which is the \Def{tag} of the value.
This tag can be used to distinguish between user
data with different tags,
and between C functions and Lua functions.

\verb|type| is simply an interface to \verb|lua_type|.

\subsubsection*{\ff{\tt assert (v)}}\Deffunc{assert}
This function issues an {\em ``assertion failed!''} error
when its argument is \nil.

\subsubsection*{\ff{\tt error (message)}}\Deffunc{error}\label{pdf-error}
This function issues an error message and terminates
the last called function from the library
(\verb'lua_dofile', \verb'lua_dostring', \ldots).
It never returns.
\verb|error| is simply an interface to \verb|lua_error|.

\subsubsection*{\ff{\tt setglobal (name, value)}}\Deffunc{setglobal}
This function assigns the given value to a global variable.
The string \verb'name' does not need to be a syntactically valid variable name.
Therefore, this function can set global variables with strange names like
\verb|`m v 1'| or \verb'34'.
It returns the value of its second argument.
\verb|setglobal| is simply an interface to \verb|lua_storeglobal|.

\subsubsection*{\ff{\tt getglobal (name)}}\Deffunc{getglobal}
This function retrieves the value of a global variable.
The string \verb'name' does not need to be a syntactically valid variable name.

\subsubsection*{\ff{\tt setfallback (fallbackname, newfallback)}}
\Deffunc{setfallback}
This function sets a new fallback function to the given fallback.
It returns the old fallback function.
\verb|setfallback| is simply an interface to \verb|lua_setfallback|.

\subsection{String Manipulation}
This library provides generic functions for string manipulation,
such as finding and extracting substrings and pattern matching.
When indexing a string, the first character is at position 1,
not 0, as in C.
See page~\pageref{pm} for an explanation about patterns,
and Section~\ref{exstring} for some examples on string manipulation
in Lua.

\subsubsection*{\ff{\tt strfind (str, pattern [, init [, plain]])}}
\Deffunc{strfind}
This function looks for the first {\em match\/} of
\verb-pattern- in \verb-str-.
If it finds one, then it returns the indices on \verb-str-
where this occurence starts and ends;
otherwise, it returns \nil.
If the pattern specifies captures,
the captured strings are returned as extra results.
A third optional numerical argument specifies where to start the search;
its default value is 1.
A value of 1 as a forth optional argument
turns off the pattern matching facilities,
so the function does a plain ``find substring'' operation.

\subsubsection*{\ff{\tt strlen (s)}}\Deffunc{strlen}
Receives a string and returns its length.

\subsubsection*{\ff{\tt strsub (s, i [, j])}}\Deffunc{strsub}
Returns another string, which is a substring of \verb's',
starting at \verb'i'  and runing until \verb'j'.
If \verb'j' is absent,
it is assumed to be equal to the length of \verb's'.
In particular, the call \verb'strsub(s,1,j)' returns a prefix of \verb's'
with length \verb'j',
whereas the call \verb'strsub(s,i)' returns a suffix of \verb's',
starting at \verb'i'.

\subsubsection*{\ff{\tt strlower (s)}}\Deffunc{strlower}
Receives a string and returns a copy of that string with all
upper case letters changed to lower case.
All other characters are left unchanged.

\subsubsection*{\ff{\tt strupper (s)}}\Deffunc{strupper}
Receives a string and returns a copy of that string with all
lower case letters changed to upper case.
All other characters are left unchanged.

\subsubsection*{\ff{\tt strrep (s, n)}}\Deffunc{strrep}
Returns a string which is the concatenation of \verb-n- copies of 
the string \verb-s-.

\subsubsection*{\ff{\tt ascii (s [, i])}}\Deffunc{ascii}
Returns the ASCII code of the character \verb's[i]'.
If \verb'i' is absent, then it is assumed to be 1.

\subsubsection*{\ff{\tt format (formatstring, e1, e2, \ldots)}}\Deffunc{format}
\label{format}
This function returns a formated version of its variable number of arguments
following the description given in its first argument (which must be a string). 
The format string follows the same rules as the \verb'printf' family of
standard C functions.
The only differences are that the options/modifiers
\verb'*', \verb'l', \verb'L', \verb'n', \verb'p',
and \verb'h' are not supported,
and there is an extra option, \verb'q'.
This option formats a string in a form suitable to be safely read
back by the Lua interpreter;
that is,
the string is written between double quotes,
and all double quotes, returns and backslashes in the string
are correctly escaped when written.
For instance, the call
\begin{verbatim}
format('%q', 'a string with "quotes" and \n new line')
\end{verbatim}
will produce the string:
\begin{verbatim}
"a string with \"quotes\" and \
 new line"
\end{verbatim}

The options \verb'c', \verb'd', \verb'E', \verb'e', \verb'f',
\verb'g' \verb'i', \verb'o', \verb'u', \verb'X', and \verb'x' all
expect a number as argument,
whereas \verb'q' and \verb's' expect a string.
Note that the \verb'*' modifier can be simulated by building
the appropriate format string.
For example, \verb|"%*g"| can be simulated with
\verb|"%"..width.."g"|.

\subsubsection*{\ff{\tt gsub (s, pat, repl [, n])}}\Deffunc{gsub}
Returns a copy of \verb-s-,
where all occurrences of the pattern \verb-pat- have been
replaced by a replacement string specified by \verb-repl-.
This function also returns, as a second value,
the total number of substitutions made.

If \verb-repl- is a string, then its value is used for replacement.
Any sequence in \verb-repl- of the form \verb-%n-
with \verb-n- between 1 and 9
stands for the value of the n-th captured substring.

If \verb-repl- is a function, then this function is called every time a
match occurs, with all captured substrings as parameters
(see below).
If the value returned by this function is a string,
then it is used as the replacement string;
otherwise, the replacement string is the empty string.

An optional parameter \verb-n- limits 
the maximum number of substitutions to occur.
For instance, when \verb-n- is 1 only the first occurrence of
\verb-pat- is replaced.

As an example, in the following expression each occurrence of the form
\verb-$name- calls the function \verb|getenv|,
passing \verb|name| as argument
(because only this part of the pattern is captured).
The value returned by \verb|getenv| will replace the pattern.
Therefore, the whole expression:
\begin{verbatim}
  gsub("home = $HOME, user = $USER", "$(%w%w*)", getenv)
\end{verbatim}
may return the string:
\begin{verbatim}
home = /home/roberto, user = roberto
\end{verbatim}

\subsubsection*{Patterns} \label{pm}

\paragraph{Character Class:}
a \Def{character class} is used to represent a set of characters.
The following combinations are allowed in describing a character class:
\begin{description}
\item[{\em x}] (where {\em x} is any character not in the list \verb'()%.[*?')
--- represents the character {\em x} itself.
\item[{\tt .}] --- represents all characters.
\item[{\tt \%a}] --- represents all letters.
\item[{\tt \%A}] --- represents all non letter characters.
\item[{\tt \%d}] --- represents all digits.
\item[{\tt \%D}] --- represents all non digits.
\item[{\tt \%l}] --- represents all lower case letters.
\item[{\tt \%L}] --- represents all non lower case letter characters.
\item[{\tt \%s}] --- represents all space characters.
\item[{\tt \%S}] --- represents all non space characters.
\item[{\tt \%u}] --- represents all upper case letters.
\item[{\tt \%U}] --- represents all non upper case letter characters.
\item[{\tt \%w}] --- represents all alphanumeric characters.
\item[{\tt \%W}] --- represents all non alphanumeric characters.
\item[{\tt \%\em x}] (where {\em x} is any non alphanumeric character)  ---
represents the character {\em x}.
This is the standard way to escape the magic characters \verb'()%.[*?'.
\item[{\tt [char-set]}] --- 
Represents the class which is the union of all
characters in char-set.
To include a \verb']' in char-set, it must be the first character.
A range of characters may be specified by
separating the end characters of the range with a \verb'-';
e.g., \verb'A-Z' specifies the upper case characters.
If \verb'-' appears as the first or last character of char-set,
then it represents itself.
All classes \verb'%'{\em x} described above can also be used as
components in a char-set.
All other characters in char-set represent themselves.
\item[{\tt [\^{ }char-set]}] ---
represents the complement of char-set,
where char-set is interpreted as above.
\end{description}

\paragraph{Pattern Item:}
a \Def{pattern item} may be:
\begin{itemize}
\item
a single character class,
which matches any single character in the class;
\item
a single character class followed by \verb'*',
which matches 0 or more repetitions of characters in the class.
These repetition itens will always match the longest possible sequence.
\item
a single character class followed by \verb'-',
which also matches 0 or more repetitions of characters in the class.
Unlike \verb'*',
these repetition itens will always match the shortest possible sequence.
\item
a single character class followed by \verb'?',
which matches 0 or 1 occurrence of a character in the class;
\item
{\tt \%$n$}, for $n$ between 1 and 9;
such item matches a sub-string equal to the n-th captured string
(see below);
\item
{\tt \%b$xy$}, where $x$ and $y$ are two distinct characters;
such item mathes strings that start with $x$, end with $y$, 
and where the $x$ and $y$ are {\em balanced}.
That means that, if one reads the string from left to write,
counting plus 1 for an $x$ and minus 1 for a $y$,
the ending $y$ is the first where the count reaches 0.
For instance, the item \verb|%()| matches expressions with
balanced parentheses.
\end{itemize}

\paragraph{Pattern:}
a \Def{pattern} is a sequence of pattern items.
A \verb'^' at the beginning of a pattern anchors the match at the
beginning of the subject string.
A \verb'$' at the end of a pattern anchors the match at the
end of the subject string.

\paragraph{Captures:}
a pattern may contain sub-patterns enclosed in parentheses,
that describe \Def{captures}.
When a match succeeds, the sub-strings of the subject string
that match captures are stored ({\em captured\/}) for future use.
Captures are numbered according to their left parentheses.
For instance, in the pattern \verb|"(a*(.)%w(%s*))"|,
the part of the string matching \verb|"a*(.)%w(%s*)"| is
stored as the first capture (and therefore has number 1);
the character matching \verb|.| is captured with number 2,
and the part matching \verb|%s*| has number 3.

\subsection{Mathematical Functions} \label{mathlib}

This library is an interface to some functions of the standard C math library.
In addition, it registers a fallback for the binary operator \verb'^' that,
returns $x^y$ when applied to numbers \verb'x^y'.

The library provides the following functions:
\Deffunc{abs}\Deffunc{acos}\Deffunc{asin}\Deffunc{atan}
\Deffunc{atan2}\Deffunc{ceil}\Deffunc{cos}\Deffunc{floor}
\Deffunc{log}\Deffunc{log10}\Deffunc{max}\Deffunc{min}
\Deffunc{mod}\Deffunc{sin}\Deffunc{sqrt}\Deffunc{tan}
\Deffunc{random}\Deffunc{randomseed}
\begin{verbatim}
abs acos asin atan atan2 ceil cos floor log log10
max min  mod  sin  sqrt  tan  random randomseed
\end{verbatim}
Most of them
are only interfaces to the homonymous functions in the C library,
except that, for the trigonometric functions,
all angles are expressed in {\em degrees}, not radians.

The function \verb'max' returns the maximum
value of its numeric arguments.
Similarly, \verb'min' computes the minimum.
Both can be used with an unlimited number of arguments.

The functions \verb'random' and \verb'randomseed' are interfaces to
the simple random generator functions \verb'rand' and \verb'srand',
provided by ANSI C.
The function \verb'random' returns pseudo-random numbers in the range
$[0,1)$.


\subsection{I/O Facilities} \label{libio}

All input and outpu operations in Lua are done over two {\em current\/} files:
one for reading and one for writing.
Initially, the current input file is \verb'stdin',
and the current output file is \verb'stdout'.

Unless otherwise stated,
all I/O functions return \nil\ on failure and
some value different from \nil\ on success.

\subsubsection*{\ff{\tt readfrom (filename)}}\Deffunc{readfrom}

This function may be called in three ways.
When called with a file name,
it opens the named file,
sets it as the {\em current\/} input file,
and returns a {\em handle\/} to the file
(this handle is a user data containing the file stream \verb|FILE*|).
It does not close the current input file.
When called with a file handle, returned by a previous call,
it restores the file as the current input.
When called without parameters,
it closes the current input file,
and restores \verb'stdin' as the current input file.

If this function fails, it returns \nil,
plus a string describing the error.

\begin{quotation}
\noindent
{\em System dependent\/}: if \verb'filename' starts with a \verb'|',
then a \Index{piped input} is open, via function \IndexVerb{popen}.
Not all systems implement pipes.
Moreover,
the number of files that can be open at the same time is usually limited and
depends on the system.
\end{quotation}

\subsubsection*{\ff{\tt writeto (filename)}}\Deffunc{writeto}

This function may be called in three ways.
When called with a file name,
it opens the named file,
sets it as the {\em current\/} output file,
and returns a {\em handle\/} to the file
(this handle is a user data containing the file stream \verb|FILE*|).
It does not close the current output file.
Notice that, if the file already exists,
it will be {\em completely erased\/} with this operation.
When called with a file handle, returned by a previous call,
it restores the file as the current output.
When called without parameters,
this function closes the current output file,
and restores \verb'stdout' as the current output file.
\index{closing a file}
%%LHF: nao tem como escrever em stderr, tem?

If this function fails, it returns \nil,
plus a string describing the error.

\begin{quotation}
\noindent
{\em System dependent\/}: if \verb'filename' starts with a \verb'|',
then a \Index{piped output} is open, via function \IndexVerb{popen}.
Not all systems implement pipes.
Moreover,
the number of files that can be open at the same time is usually limited and
depends on the system.
\end{quotation}

\subsubsection*{\ff{\tt appendto (filename)}}\Deffunc{appendto}

This function opens a file named \verb'filename' and sets it as the
{\em current\/} output file.
It returns the file handle,
or \nil\ in case of error.
Unlike the \verb'writeto' operation,
this function does not erase any previous content of the file.
If this function fails, it returns \nil,
plus a string describing the error.

Notice that function \verb|writeto| is available to close an output file.

\subsubsection*{\ff{\tt remove (filename)}}\Deffunc{remove}

This function deletes the file with the given name.
If this function fails, it returns \nil,
plus a string describing the error.

\subsubsection*{\ff{\tt rename (name1, name2)}}\Deffunc{rename}

This function renames file named \verb'name1' to \verb'name2'.
If this function fails, it returns \nil,
plus a string describing the error.

\subsubsection*{\ff{\tt tmpname ()}}\Deffunc{tmpname}

This function returns a string with a file name that can safely
be used for a temporary file.

\subsubsection*{\ff{\tt read ([readpattern])}}\Deffunc{read}

This function reads the current input
according to a read pattern, that specifies how much to read;
characters are read from the current input file until
the read pattern fails or ends.
The function \verb|read| returns a string with the characters read,
even if the pattern succeeds only partially,
or \nil\ if the read pattern fails {\em and\/}
the result string would be empty.
When called without parameters,
it uses a default pattern that reads the next line
(see below).

A \Def{read pattern} is a sequence of read pattern items.
An item may be a single character class
or a character class followed by \verb'?' or by \verb'*'.
A single character class reads the next character from the input
if it belongs to the class, otherwise it fails.
A character class followed by \verb'?' reads the next character
from the input if it belongs to the class;
it never fails.
A character class followed by \verb'*' reads until a character that
does not belong to the class, or end of file;
since it can match a sequence of zero characteres, it never fails.%
\footnote{
Notice that the behavior of read patterns is different from
the regular pattern matching behavior,
where a \verb'*' expands to the maximum length {\em such that\/}
the rest of the pattern does not fail.
With the read pattern behavior 
there is no need for backtracking the reading.
}

A pattern item may contain sub-patterns enclosed in curly brackets,
that describe \Def{skips}.
Characters matching a skip are read,
but are not included in the resulting string.

Following are some examples of read patterns and their meanings:
\begin{itemize}
\item \verb|"."| returns the next character, or \nil\ on end of file.
\item \verb|".*"| reads the whole file.
\item \verb|"[^\n]*{\n}"| returns the next line
(skipping the end of line), or \nil\ on end of file.
This is the default pattern.
\item \verb|"{%s*}%S%S*"| returns the next word
(maximal sequence of non white-space characters),
or \nil\ on end of file.
\item \verb|"{%s*}[+-]?%d%d*"| returns the next integer
or \nil\ if the next characters do not conform to an integer format.
\end{itemize}

\subsubsection*{\ff{\tt write (value1, ...)}}\Deffunc{write}

This function writes the value of each of its arguments to the
current output file.
The arguments must be strings or numbers.
To write other values,
use \verb|tostring| before \verb|write|.
If this function fails, it returns \nil,
plus a string describing the error.

\subsubsection*{\ff{\tt date ([format])}}\Deffunc{date}

This function returns a string containing date and time
formatted according to the given string \verb'format',
following the same rules of the ANSI C function \verb'strftime'.
When called without arguments,
it returns a reasonable date and time representation that depends on
the host system.

\subsubsection*{\ff{\tt exit ([code])}}\Deffunc{exit}

This function calls the C function \verb-exit-,
with an optional \verb-code-,
to terminate the program.
The default value for \verb-code- is 1.

\subsubsection*{\ff{\tt getenv (varname)}}\Deffunc{getenv}

Returns the value of the environment variable \verb|varname|,
or \nil\ if the variable is not defined.

\subsubsection*{\ff{\tt execute (command)}}\Deffunc{execute}

This function is equivalent to the C function \verb|system|.
It passes \verb|command| to be executed by an operating system shell.
It returns an error code, which is system-dependent.


\section{The Debugger Interface} \label{debugI}

Lua has no built-in debugging facilities.
Instead, it offers a special interface,
by means of functions and {\em hooks},
which allows the construction of different
kinds of debuggers, profilers, and other tools
that need ``inside information'' from the interpreter.
This interface is declared in the header file \verb'luadebug.h'.

\subsection{Stack and Function Information}

The main function to get information about the interpreter stack
is
\begin{verbatim}
lua_Function lua_stackedfunction (int level);
\end{verbatim}
It returns a handle (\verb'lua_Function') to the {\em activation record\/}
of the function executing at a given level.
Level 0 is the current running function,
while level $n+1$ is the function that has called level $n$.
When called with a level greater than the stack depth,
\verb'lua_stackedfunction' returns \verb'LUA_NOOBJECT'.

The type \verb'lua_Function' is just another name
to \verb'lua_Object'.
Although, in this library,
a \verb'lua_Function' can be used wherever a \verb'lua_Object' is required,
when a parameter has type \verb'lua_Function'
it accepts only a handle returned by
\verb'lua_stackedfunction'.

Three other functions produce extra information about a function:
\begin{verbatim}
void lua_funcinfo (lua_Object func, char **filename, int *linedefined);
int lua_currentline (lua_Function func);
char *lua_getobjname (lua_Object o, char **name);
\end{verbatim}
\verb'lua_funcinfo' gives the file name and the line where the
given function has been defined.
If the ``function'' is in fact the main code of a chunk,
then \verb'linedefined' is 0.
If the function is a C function,
then \verb'linedefined' is -1, and \verb'filename' is \verb'"(C)"'.

The function \verb'lua_currentline' gives the current line where
a given function is executing.
It only works if the function has been compiled with debug
information \see{pragma}.
When no line information is available, it returns -1.

Function \verb'lua_getobjname' tries to find a reasonable name for
a given function.
Because functions in Lua are first class values,
they do not have a fixed name:
Some functions may be the value of many global variables,
while others may be stored only in a table field.
Function \verb'lua_getobjname' first checks whether the given
function is a fallback.
If so, it returns the string \verb'"fallback"',
and \verb'name' is set to point to the fallback name.
Otherwise, if the given function is the value of a global variable,
then \verb'lua_getobjname' returns the string \verb'"global"',
and \verb'name' points to the variable name.
If the given function is neither a fallback nor a global variable,
then \verb'lua_getobjname' returns the empty string,
and \verb'name' is set to \verb'NULL'.

\subsection{Manipulating Local Variables}

The following functions allow the manipulation of the
local variables of a given activation record.
They only work if the function has been compiled with debug
information \see{pragma}.
\begin{verbatim}
lua_Object lua_getlocal (lua_Function func, int local_number, char **name);
int lua_setlocal (lua_Function func, int local_number);
\end{verbatim}
\verb|lua_getlocal| returns the value of a local variable,
and sets \verb'name' to point to the variable name.
\verb'local_number' is an index for local variables.
The first parameter has index 1, and so on, until the
last active local variable.
When called with a \verb'local_number' greater than the
number of active local variables,
or if the activation record has no debug information,
\verb'lua_getlocal' returns \verb'LUA_NOOBJECT'.
Formal parameters are the first local variables.

The function \verb'lua_setlocal' sets the local variable
%%LHF: please, lua_setglobal!
\verb'local_number' to the value previously pushed on the stack
\see{valuesCLua}.
If the function succeeds, then it returns 1.
If \verb'local_number' is greater than the number
of active local variables,
or if the activation record has no debug information,
then this function fails and returns 0.

\subsection{Hooks}

The Lua interpreter offers two hooks for debugging purposes:
\begin{verbatim}
typedef void (*lua_CHFunction) (lua_Function func, char *file, int line);
extern lua_CHFunction lua_callhook;

typedef void (*lua_LHFunction) (int line);
extern lua_LHFunction lua_linehook;
\end{verbatim}
The first one is called whenever the interpreter enters or leaves a
function.
When entering a function,
its parameters are a handle to the function activation record,
plus the file and the line where the function is defined (the same
information which is provided by \verb'lua_funcinfo');
when leaving a function, \verb'func' is \verb'LUA_NOOBJECT',
\verb'file' is \verb'"(return)"', and \verb'line' is 0.

The other hook is called every time the interpreter changes
the line of code it is executing.
Its only parameter is the line number
(the same information which is provided by the call
\verb'lua_currentline(lua_stackedfunction(0))').
This second hook is only called if the active function
has been compiled with debug information \see{pragma}.

A hook is disabled when its value is \verb|NULL|,
which is the initial value of both hooks.


\section{Some Examples}

This section gives examples showing some features of Lua.
It does not intend to cover the whole language,
but only to illustrate some interesting uses of the system.


\subsection{\Index{Data Structures}}
Tables are a strong unifying data constructor.
They directly implement a multitude of data types,
like ordinary arrays, records, sets, bags, and lists.

Arrays need no explanations.
In Lua, it is conventional to start indices from 1,
but this is only a convention.
Arrays can be indexed by 0, negative numbers, or any other value (except \nil).
Records are also trivially implemented by the syntactic sugar
\verb'a.x'.

The best way to implement a set is to store
its elements as indices of a table.
The statement \verb's = {}' creates an empty set \verb's'. 
The statement \verb's[x] = 1' inserts the value of \verb'x' into
the set \verb's'.
The expression \verb's[x]' is true if and only if
\verb'x' belongs to \verb's'.
Finally, the statement \verb's[x] = nil' removes \verb'x' from \verb's'.

Bags can be implemented similarly to sets,
but using the value associated to an element as its counter.
So, to insert an element, 
the following code is enough:
\begin{verbatim}
if s[x] then s[x] = s[x]+1 else s[x] = 1 end
\end{verbatim}
and to remove an element:
\begin{verbatim}
if s[x] then s[x] = s[x]-1 end
if s[x] == 0 then s[x] = nil end
\end{verbatim}

Lisp-like lists also have an easy implementation.
The ``cons'' of two elements \verb'x' and \verb'y' can be
created with the code \verb'l = {car=x, cdr=y}'.
The expression \verb'l.car' extracts the header, 
while \verb'l.cdr' extracts the tail.
An alternative way is to create the list directly with \verb'l={x,y}',
and then to extract the header with \verb'l[1]' and
the tail with \verb'l[2]'.

\subsection{The Functions {\tt next} and {\tt nextvar}} \label{exnext}
\Deffunc{next}\Deffunc{nextvar}
This example shows how to use the function \verb'next' to iterate
over the fields of a table.
Function \IndexVerb{clone} receives any table and returns a clone of it.
\begin{verbatim}
function clone (t)           -- t is a table
  local new_t = {}           -- create a new table
  local i, v = next(t, nil)  -- i is an index of t, v = t[i]
  while i do
    new_t[i] = v
    i, v = next(t, i)        -- get next index
  end
  return new_t
end
\end{verbatim}

The next example prints the names of all global variables
in the system with non nil values.
Notice that the traversal is made with local variables,
to avoid changing a global variable:
\begin{verbatim}
function printGlobalVariables ()
  local i, v = nextvar(nil)
  while i do
    print(i)
    i, v = nextvar(i)
  end
end
\end{verbatim}


\subsection{String Manipulation} \label{exstring}

The first example is a function to trim extra white-spaces at the beginning
and end of a string.
\begin{verbatim}
function trim(s)
  local _, i = strfind(s, '^ *')
  local f, __ = strfind(s, ' *$')
  return strsub(s, i+1, f-1)
end
\end{verbatim}

The second example shows a function that eliminates all spaces
of a string.
\begin{verbatim}
function remove_blanks (s)
  return gsub(s, "%s%s*", "")
end
\end{verbatim}


\subsection{\Index{Variable number of arguments}}
Lua does not provide any explicit mechanism to deal with
variable number of arguments in function calls.
However, one can use table constructors to simulate this mechanism.
As an example, suppose a function to concatenate all its arguments.
It could be written like 
\begin{verbatim}
function concat (o)
  local i = 1
  local s = ''
  while o[i] do
    s = s .. o[i]
    i = i+1
  end
  return s
end
\end{verbatim}
To call it, one uses a table constructor to join all arguments:
\begin{verbatim}
  x = concat{"hello ", "john", " and ", "mary"}
\end{verbatim}

\subsection{\Index{Persistence}}
Because of its reflexive facilities,
persistence in Lua can be achieved within the language.
This section shows some ways to store and retrieve values in Lua,
using a text file written in the language itself as the storage media.

To store a single value with a name,
the following code is enough:
\begin{verbatim}
function store (name, value)
  write(format('\n%s =', name))
  write_value(value)
end
\end{verbatim}
\begin{verbatim}
function write_value (value)
  local t = type(value)
      if t == 'nil'    then write('nil')
  elseif t == 'number' then write(value)
  elseif t == 'string' then write(value, 'q')
  end
end
\end{verbatim}
In order to restore this value, a \verb'lua_dofile' suffices.

Storing tables is a little more complex.
Assuming that the table is a tree,
and that all indices are identifiers
(that is, the tables are being used as records),
then its value can be written directly with table constructors.
First, the function \verb'write_value' is changed to
\begin{verbatim}
function write_value (value)
  local t = type(value)
      if t == 'nil'    then write('nil')
  elseif t == 'number' then write(value)
  elseif t == 'string' then write(value, 'q')
  elseif t == 'table'  then write_record(value)
  end
end
\end{verbatim}
The function \verb'write_record' is:
\begin{verbatim}
function write_record(t)
  local i, v = next(t, nil)
  write('{')  -- starts constructor
  while i do
    store(i, v)
    write(', ')
    i, v = next(t, i)
  end
  write('}')  -- closes constructor
end
\end{verbatim}


\subsection{Inheritance} \label{exfallback}
The fallback for absent indices can be used to implement many
kinds of \Index{inheritance} in Lua.
As an example,
the following code implements single inheritance:
\begin{verbatim}
function Index (t,f)
  if f == 'parent' then  -- to avoid loop
    return OldIndex(t,f)
  end
  local p = t.parent
  if type(p) == 'table' then
    return p[f]
  else
    return OldIndex(t,f)
  end
end

OldIndex = setfallback("index", Index)
\end{verbatim}
Whenever Lua attempts to access an absent field in a table,
it calls the fallback function \verb'Index'.
If the table has a field \verb'parent' with a table value,
then Lua attempts to access the desired field in this parent object.
This process is repeated ``upwards'' until a value
for the field is found or the object has no parent.
In the latter case, the previous fallback is called to supply a value
for the field.

When better performance is needed,
the same fallback may be implemented in C,
as illustrated in Figure~\ref{Cinher}.
\begin{figure}
\Line
\begin{verbatim}
#include "lua.h"

int lockedParentName;  /* lock index for the string "parent" */
int lockedOldIndex;    /* previous fallback function */

void callOldFallback (lua_Object table, lua_Object index)
{
  lua_Object oldIndex = lua_getref(lockedOldIndex);
  lua_pushobject(table);
  lua_pushobject(index);
  lua_callfunction(oldIndex);
  if (lua_getresult(1) != LUA_NOOBJECT)
    lua_pushobject(lua_getresult(1));  /* return result */
}

void Index (void)
{
  lua_Object table = lua_getparam(1);
  lua_Object index = lua_getparam(2);
  lua_Object parent;
  if (lua_isstring(index) && strcmp(lua_getstring(index), "parent") == 0)
  {
    callOldFallback(table, index);
    return;
  }
  lua_pushobject(table);
  lua_pushref(lockedParentName);
  parent = lua_getsubscript();
  if (lua_istable(parent))
  {
    lua_pushobject(parent);
    lua_pushobject(index);
    lua_pushobject(lua_getsubscript()); /* return result from getsubscript */
  }
  else
    callOldFallback(table, index);
}
\end{verbatim}
\caption{Inheritance in C.\label{Cinher}}
\Line
\end{figure}
This code must be registered with:
\begin{verbatim}
  lua_pushstring("parent");
  lockedParentName = lua_ref(1);
  lua_pushobject(lua_setfallback("index", Index));
  lockedOldIndex = lua_ref(1);
\end{verbatim}
Notice how the string \verb'"parent"' is kept
locked in Lua for optimal performance.

\subsection{\Index{Programming with Classes}}
There are many different ways to do object-oriented programming in Lua.
This section presents one possible way to
implement classes,
using the inheritance mechanism presented above.
{\em Please note: the following examples only work
with the index fallback redefined according to
Section~\ref{exfallback}}.

As one could expect, a good way to represent a class is
with a table.
This table will contain all instance methods of the class,
plus optional default values for instance variables.
An instance of a class has its \verb'parent' field pointing to
the class,
and so it ``inherits'' all methods.

For instance, a class \verb'Point' can be described as in
Figure~\ref{Point}.
Function \verb'create' helps the creation of new points,
adding the parent field.
Function \verb'move' is an example of an instance method.
\begin{figure}
\Line
\begin{verbatim}
Point = {x = 0, y = 0}

function Point:create (o)
  o.parent = self
  return o
end

function Point:move (p)
  self.x = self.x + p.x
  self.y = self.y + p.y
end

...

--
-- creating points
--
p1 = Point:create{x = 10, y = 20}
p2 = Point:create{x = 10}  -- y will be inherited until it is set

--
-- example of a method invocation
--
p1:move(p2)
\end{verbatim}
\caption{A Class {\tt Point}.\label{Point}}
\Line
\end{figure}
Finally, a subclass can be created as a new table,
with the \verb'parent' field pointing to its superclass.
It is interesting to notice how the use of \verb'self' in
method \verb'create' allows this method to work properly even
when inherited by a subclass.
As usual, a subclass may overwrite any inherited method with
its own version.

\subsection{\Index{Modules}}
Here we explain one possible way to simulate modules in Lua.
The main idea is to use a table to store the module functions.

A module should be written as a separate chunk, starting with:
\begin{verbatim}
if modulename then return end  -- avoid loading twice the same module
modulename = {}  -- create a table to represent the module
\end{verbatim}
After that, functions can be directly defined with the syntax
\begin{verbatim}
function modulename.foo (...)
  ...
end
\end{verbatim}

Any code that needs this module has only to execute
\verb'dofile("filename")', where \verb'filename' is the file
where the module is written.
After this, any function can be called with
\begin{verbatim}
modulename.foo(...)
\end{verbatim}

If a module function is going to be used many times,
the program can give a local name to it.
Because functions are values, it is enough to write
\begin{verbatim}
localname = modulename.foo
\end{verbatim}
Finally, a module may be {\em opened},
giving direct access to all its functions,
as shown in the code in Figure~\ref{openmod}.
\begin{figure}
\Line
\begin{verbatim}
function open (mod)
  local n, f = next(mod, nil)
  while n do
    setglobal(n, f)
    n, f = next(mod, n)
  end
end
\end{verbatim}
\caption{Opening a module.\label{openmod}}
\Line
\end{figure}

\subsection{A CFunction} \label{exCFunction}\index{functions in C}
A CFunction to compute the maximum of a variable number of arguments
is shown in Figure~\ref{Cmax}.
\begin{figure}
\Line
\begin{verbatim}
void math_max (void)
{
 int i=1;   /* number of arguments */
 double d, dmax;
 lua_Object o;
 /* the function must get at least one argument */
 if ((o = lua_getparam(i++)) == LUA_NOOBJECT)
   lua_error ("too few arguments to function `max'");
 /* and this argument must be a number */
 if (!lua_isnumber(o))
   lua_error ("incorrect argument to function `max'");
 dmax = lua_getnumber (o);
 /* loops until there is no more arguments */
 while ((o = lua_getparam(i++)) != LUA_NOOBJECT)
 {
  if (!lua_isnumber(o))
    lua_error ("incorrect argument to function `max'");
  d = lua_getnumber (o);
  if (d > dmax) dmax = d;
 }
 /* push the result to be returned */
 lua_pushnumber (dmax);
}
\end{verbatim}
\caption{C function {\tt math\_max}.\label{Cmax}}
\Line
\end{figure}
After registered with
\begin{verbatim}
lua_register ("max", math_max);
\end{verbatim}
this function is available in Lua, as follows:
\begin{verbatim}
i = max(4, 5, 10, -34)  -- i receives 10
\end{verbatim}


\subsection{Calling Lua Functions} \label{exLuacall}

This example illustrates how a C function can call the Lua function
\verb'remove_blanks' presented in Section~\ref{exstring}.
\begin{verbatim}
void remove_blanks (char *s)
{
  lua_pushstring(s);  /* prepare parameter */
  lua_call("remove_blanks");  /* call Lua function */
  strcpy(s, lua_getstring(lua_getresult(1)));  /* copy result back to 's' */
}
\end{verbatim}


\section{\Index{Lua Stand-alone}} \label{lua-sa}

Although Lua has been designed as an extension language,
the language can also be used as a stand-alone interpreter.
An implementation of such an interpreter,
called simply \verb|lua|,
is provided with the standard distribution.
This program can be called with any sequence of the following arguments:
\begin{description}
\item[{\tt -v}] prints version information.
\item[{\tt -}] runs interactively, accepting commands from standard input
until an \verb|EOF|.
\item[{\tt -e stat}] executes \verb|stat| as a Lua chunk.
\item[{\tt var=exp}] executes \verb|var=exp| as a Lua chunk.
\item[{\tt filename}] executes file \verb|filename| as a Lua chunk.
\end{description}
All arguments are handled in order.
For instance, an invocation like
\begin{verbatim}
$ lua - a=1 prog.lua
\end{verbatim}
will first interact with the user until an \verb|EOF|,
then will set \verb'a' to 1,
and finally will run file \verb'prog.lua'.

Please notice that the interaction with the shell may lead to
unintended results.
For instance, a call like
\begin{verbatim}
$ lua a="name" prog.lua
\end{verbatim}
will {\em not\/} set \verb|a| to the string \verb|"name"|.
Instead, the quotes will be handled by the shell,
lua will get only \verb'a=name' to run,
and \verb'a' will finish with \nil,
because the global variable \verb|name| has not been initialized.
Instead, one should write
\begin{verbatim}
$ lua 'a="name"' prog.lua
\end{verbatim}

\section*{Acknowledgments}

The authors would like to thank CENPES/PETROBR\'AS which,
jointly with \tecgraf, used extensively early versions of
this system and gave valuable comments.
The authors would also like to thank Carlos Henrique Levy,
who found the name of the game.
Lua means {\em moon\/} in Portuguese.



\appendix

\section*{Incompatibilities with Previous Versions}

Although great care has been taken to avoid incompatibilities with
the previous public versions of Lua,
some differences had to be introduced.
Here is a list of all these incompatibilities.

\subsection*{Incompatibilities with \Index{version 2.4}}
The whole I/O facilities have been rewritten.
We strongly encourage programmers to adapt their code
to this new version.
However, we are keeping the old version of the libraries
in the distribution,
to allow a smooth transition.
The incompatibilities between the new and the old libraries are:
\begin{itemize}
\item The format facility of function \verb'write' has been supersed by
function \verb'format';
therefore this facility has been dropped.
\item Function \verb'read' now uses {\em read patterns\/} to specify
what to read;
this is incompatible with the old format options.
\item Function \verb'strfind' now accepts patterns,
so it may have a different behavior when the pattern includes
special characters.
\end{itemize}

\subsection*{Incompatibilities with \Index{version 2.2}}
\begin{itemize}
\item
Functions \verb'date' and \verb'time' (from \verb'iolib')
have been superseded by the new, more powerful version of function \verb'date'.
\item
Function \verb'append' (from \verb'iolib') now returns 1 whenever it succeeds,
whether the file is new or not.
\item
Function \verb'int2str' (from \verb'strlib') has been superseded by new
function \verb'format', with parameter \verb'"%c"'.
\item
The API lock mechanism has been superseded by the reference mechanism.
However, \verb-lua.h- provides compatibility macros,
so there is no need to change programs.
\item
The API function \verb'lua_pushliteral' now is just a macro to
\verb'lua_pushstring'.
\end{itemize}

\subsection*{Incompatibilities with \Index{version 2.1}}
\begin{itemize}
\item
The function \verb'type' now returns the string \verb'"function"'
both for C and Lua functions.
Because Lua functions and C functions are compatible,
this behavior is usually more useful.
When needed, the second result of function {\tt type} may be used
to distinguish between Lua and C functions.
\item
A function definition only assigns the function value to the
given variable at execution time.
\end{itemize}

\subsection*{Incompatibilities with \Index{version 1.1}}
\begin{itemize}
\item
The equality test operator now is denoted by \verb'==',
instead of \verb'='.
\item
The syntax for table construction has been greatly simplified.
The old \verb'@(size)' has been substituted by \verb'{}'.
The list constructor (formerly \verb'@[...]') and the record
constructor (formerly \verb'@{...}') now are both coded like
\verb'{...}'.
When the construction involves a function call,
like in \verb'@func{...}',
the new syntax does not use the \verb'@'.
More important, {\em a construction function must now
explicitly return the constructed table}.
\item
The function \verb'lua_call' no longer has the parameter \verb'nparam'.
\item
The function \verb'lua_pop' is no longer available,
since it could lead to strange behavior.
In particular,
to access results returned from a Lua function,
the new macro \verb'lua_getresult' should be used.
\item
The old functions \verb'lua_storefield' and \verb'lua_storeindexed'
have been replaced by
\begin{verbatim}
int lua_storesubscript (void);
\end{verbatim}
with the parameters explicitly pushed on the stack.
\item
The functionality of the function \verb'lua_errorfunction' has been
replaced by the {\em fallback\/} mechanism \see{error}.
\item
When calling a function from the Lua library,
parameters passed through the stack
must be pushed just before the corresponding call,
with no intermediate calls to Lua.
Special care should be taken with macros like
\verb'lua_getindexed' and \verb'lua_getfield'.
\end{itemize}

\newcommand{\indexentry}[2]{\item {#1} #2}
%\catcode`\_=12
\begin{theindex}
% $Id: manual.tex,v 1.56 2002/06/06 12:49:28 roberto Exp roberto $

\documentclass[11pt,twoside,draft]{article}
\usepackage{fullpage}
\usepackage{bnf}
\usepackage{graphicx}

% no need for subscripts...
\catcode`\_=12

%\newcommand{\See}[1]{Section~\ref{#1}}
\newcommand{\See}[1]{\S\ref{#1}}
%\newcommand{\see}[1]{(see~\See{#1} on page \pageref{#1})}
\newcommand{\see}[1]{(see~\See{#1})}
\newcommand{\seepage}[1]{(see page~\pageref{#1})}
\newcommand{\M}[1]{{\rm\emph{#1}}}
\newcommand{\T}[1]{{\tt #1}}
\newcommand{\Math}[1]{$#1$}
\newcommand{\nil}{{\bf nil}}
\newcommand{\False}{{\bf false}}
\newcommand{\True}{{\bf true}}
%\def\tecgraf{{\sf TeC\kern-.21em\lower.7ex\hbox{Graf}}}
\def\tecgraf{{\sf Tecgraf}}

\newcommand{\Index}[1]{#1\index{#1@{\lowercase{#1}}}}
\newcommand{\IndexVerb}[1]{\T{#1}\index{#1@{\tt #1}}}
\newcommand{\IndexEmph}[1]{\emph{#1}\index{#1@{\lowercase{#1}}}}
\newcommand{\IndexTM}[1]{\index{#1 event@{``#1'' event}}\index{tag method!#1}}
\newcommand{\Def}[1]{\emph{#1}\index{#1}}
\newcommand{\IndexAPI}[1]{\T{#1}\DefAPI{#1}}
\newcommand{\IndexLIB}[1]{\T{#1}\DefLIB{#1}}
\newcommand{\DefLIB}[1]{\index{#1@{\tt #1}}}
\newcommand{\DefAPI}[1]{\index{C API!#1@{\tt #1}}}
\newcommand{\IndexKW}[1]{\index{keywords!#1@{\tt #1}}}

\newcommand{\ff}{$\bullet$\ }

\newcommand{\Version}{5.0 (alpha)}

% changes to bnf.sty by LHF
\renewcommand{\Or}{$|$ }
\renewcommand{\rep}[1]{{\rm\{}\,#1\,{\rm\}}}
\renewcommand{\opt}[1]{{\rm [}\,#1\,{\,\rm]}}
\renewcommand{\ter}[1]{{\rm`{\tt#1}'}}
\newcommand{\Nter}[1]{{\tt#1}}
\newcommand{\NOTE}{\par\medskip\noindent\emph{NOTE}: }

\makeindex

\begin{document}

%{===============================================================
\thispagestyle{empty}
\pagestyle{empty}

{
\parindent=0pt
\vglue1.5in
{\LARGE\bf
The Programming Language Lua}
\hfill
\vskip4pt \hrule height 4pt width \hsize \vskip4pt
\hfill
Reference Manual for Lua version \Version
\\
\null
\hfill
Last revised on \today
\\
\vfill
\centering
\includegraphics[width=0.7\textwidth]{nolabel.ps}
\vfill
\vskip4pt \hrule height 2pt width \hsize
}

\newpage
\begin{quotation}
\parskip=10pt
\parindent=0pt
\footnotesize
\null\vfill

\noindent
Copyright \copyright\ 2002 Tecgraf, PUC-Rio.  All rights reserved.

Permission is hereby granted, free of charge,
to any person obtaining a copy of this software
and associated documentation files (the "Software"),
to deal in the Software without restriction,
including without limitation the rights to use, copy, modify,
merge, publish, distribute, sublicense,
and/or sell copies of the Software,
and to permit persons to whom the Software is furnished to do so,
subject to the following conditions:

The above copyright notice and this permission notice shall be
included in all copies or substantial portions of the Software.

THE SOFTWARE IS PROVIDED "AS IS", WITHOUT WARRANTY OF ANY KIND,
EXPRESS OR IMPLIED,
INCLUDING BUT NOT LIMITED TO THE WARRANTIES OF MERCHANTABILITY,
FITNESS FOR A PARTICULAR PURPOSE AND NONINFRINGEMENT.
IN NO EVENT SHALL THE AUTHORS OR COPYRIGHT HOLDERS BE LIABLE
FOR ANY CLAIM, DAMAGES OR OTHER LIABILITY,
WHETHER IN AN ACTION OF CONTRACT, TORT OR OTHERWISE,
ARISING FROM, OUT OF OR IN CONNECTION WITH THE SOFTWARE
OR THE USE OR OTHER DEALINGS IN THE SOFTWARE.


Copies of this manual can be obtained at
Lua's official web site,
\verb|www.lua.org|.

\bigskip
The Lua logo was designed by A. Nakonechny.
Copyright \copyright\ 1998.  All rights reserved.
\end{quotation}
%}===============================================================
\newpage

\title{\Large\bf Reference Manual of the Programming Language Lua \Version}

\author{%
Roberto Ierusalimschy\qquad
Luiz Henrique de Figueiredo\qquad
Waldemar Celes
\vspace{1.0ex}\\
\smallskip
\small\tt lua@tecgraf.puc-rio.br
\vspace{2.0ex}\\
%MCC 08/95 ---
\tecgraf\ --- Computer Science Department --- PUC-Rio
}

%\date{{\small \tt\$Date: 2002/06/06 12:49:28 $ $}}

\maketitle

\pagestyle{plain}
\pagenumbering{roman}

\begin{abstract}
\noindent
Lua is a powerful, light-weight programming language
designed for extending applications.
Lua is also frequently used as a general-purpose, stand-alone language.
Lua combines simple procedural syntax
(similar to Pascal)
with
powerful data description constructs
based on associative arrays and extensible semantics.
Lua is
dynamically typed,
interpreted from opcodes,
and has automatic memory management with garbage collection,
making it ideal for
configuration,
scripting,
and
rapid prototyping.

This document describes version \Version\ of the Lua programming language
and the Application Program Interface (API)
that allows interaction between Lua programs and their host C~programs.
\end{abstract}

\def\abstractname{Resumo}
\begin{abstract}
\noindent
Lua \'e uma linguagem de programa\c{c}\~ao
poderosa e leve,
projetada para estender aplica\c{c}\~oes.
Lua tamb\'em \'e frequentemente usada como uma linguagem de prop\'osito geral.
Lua combina programa\c{c}\~ao procedural
(com sintaxe semelhante \`a de Pascal)
com
poderosas constru\c{c}\~oes para descri\c{c}\~ao de dados,
baseadas em tabelas associativas e sem\^antica extens\'\i vel.
Lua \'e
tipada dinamicamente,
interpretada a partir de \emph{opcodes},
e tem gerenciamento autom\'atico de mem\'oria com coleta de lixo.
Essas caracter\'{\i}sticas fazem de Lua uma linguagem ideal para
configura\c{c}\~ao,
automa\c{c}\~ao (\emph{scripting})
e prototipagem r\'apida.

Este documento descreve a vers\~ao \Version\ da linguagem de
programa\c{c}\~ao Lua e a Interface de Programa\c{c}\~ao (API) que permite
a intera\c{c}\~ao entre programas Lua e programas C~hospedeiros.
\end{abstract}

\newpage
\null
\newpage
\tableofcontents

\newpage
\setcounter{page}{1}
\pagestyle{plain}
\pagenumbering{arabic}

%------------------------------------------------------------------------------
\section{Introduction}

Lua is an extension programming language designed to support
general procedural programming with data description
facilities.
Lua is intended to be used as a powerful, light-weight
configuration language for any program that needs one.
Lua is implemented as a library, written in C.

Being an extension language, Lua has no notion of a ``main'' program:
it only works \emph{embedded} in a host client,
called the \emph{embedding program} or simply the \emph{host}.
This host program can invoke functions to execute a piece of Lua code,
can write and read Lua variables,
and can register C~functions to be called by Lua code.
Through the use of C~functions, Lua can be augmented to cope with
a wide range of different domains,
thus creating customized programming languages sharing a syntactical framework.

Lua is free software,
and is provided as usual with no guarantees,
as stated in its copyright notice.
The implementation described in this manual is available
at Lua's official web site, \verb|www.lua.org|.

Like any other reference manual,
this document is dry in places.
For a discussion of the decisions behind the design of Lua,
see the papers below,
which are available at Lua's web site.
\begin{itemize}
\item
R.~Ierusalimschy, L.~H.~de Figueiredo, and W.~Celes.
Lua---an extensible extension language.
\emph{Software: Practice \& Experience} {\bf 26} \#6 (1996) 635--652.
\item
L.~H.~de Figueiredo, R.~Ierusalimschy, and W.~Celes.
The design and implementation of a language for extending applications.
\emph{Proceedings of XXI Brazilian Seminar on Software and Hardware} (1994) 273--283.
\item
L.~H.~de Figueiredo, R.~Ierusalimschy, and W.~Celes.
Lua: an extensible embedded language.
\emph{Dr. Dobb's Journal} {\bf  21} \#12 (Dec 1996) 26--33.
\item
R.~Ierusalimschy, L.~H.~de Figueiredo, and W.~Celes.
The evolution of an extension language: a history of Lua,
\emph{Proceedings of V Brazilian Symposium on Programming Languages} (2001) B-14--B-28.
\end{itemize}

%------------------------------------------------------------------------------
\section{Lua Concepts}\label{concepts}

This section describes the main concepts of Lua as a language.
The syntax and semantics of Lua are described in \See{language}.
The discussion below is not purely conceptual;
it includes references to the C~API \see{API},
because Lua is designed to be embedded in host programs.
It also includes references to the standard libraries \see{libraries}.


\subsection{Environment and Chunks}

All statements in Lua are executed in a \Def{global environment}.
This environment is initialized with a call from the embedding program to
\verb|lua_open| and
persists until a call to \verb|lua_close|
or the end of the embedding program.
If necessary,
the host programmer can create multiple independent global
environments, and freely switch between them \see{mangstate}.

The unit of execution of Lua is called a \Def{chunk}.
A chunk is simply a sequence of statements.
Statements are described in \See{stats}.

A chunk may be stored in a file or in a string inside the host program.
When a chunk is executed, first it is pre-compiled into opcodes for
a virtual machine,
and then the compiled statements are executed
by an interpreter for the virtual machine.
All modifications a chunk effects on the global environment persist
after the chunk ends.

Chunks may also be pre-compiled into binary form and stored in files;
see program \IndexVerb{luac} for details.
Text files with chunks and their binary pre-compiled forms
are interchangeable;
Lua automatically detects the file type and acts accordingly.
\index{pre-compilation}


\subsection{\Index{Values and Types}} \label{TypesSec}

Lua is a \emph{dynamically typed language}.
That means that
variables do not have types; only values do.
There are no type definitions in the language.
All values carry their own type.

There are seven \Index{basic types} in Lua:
\Def{nil}, \Def{boolean}, \Def{number},
\Def{string}, \Def{function}, \Def{userdata}, and \Def{table}.
\emph{Nil} is the type of the value \nil,
whose main property is to be different from any other value;
usually it represents the absence of a useful value.
\emph{Boolean} is the type of the values \False{} and \True.
In Lua, both \nil{} and \False{} make a condition fails,
and any other value makes it succeeds.
\emph{Number} represents real (double-precision floating-point) numbers.
\emph{String} represents arrays of characters.
\index{eight-bit clean}
Lua is 8-bit clean,
and so strings may contain any 8-bit character,
including embedded zeros (\verb|'\0'|) \see{lexical}.

Functions are \emph{first-class values} in Lua.
That means that functions can be stored in variables,
passed as arguments to other functions, and returned as results.
Lua can call (and manipulate) functions written in Lua and
functions written in C
\see{functioncall}.

The type \emph{userdata} is provided to allow the store of
arbitrary C data in Lua variables.
This type corresponds to a block of raw memory
and has no pre-defined operations in Lua,
except assignment and identity test.
However, by using \emph{metatables},
the programmer can define operations for userdata values
\see{metatables}.
Userdata values cannot be created or modified in Lua,
only through the C~API.
This guarantees the integrity of data owned by the host program.

The type \emph{table} implements \Index{associative arrays},
that is, \Index{arrays} that can be indexed not only with numbers,
but with any value (except \nil).
Moreover,
tables can be \emph{heterogeneous},
that is, they can contain values of all types.
Tables are the sole data structuring mechanism in Lua;
they may be used not only to represent ordinary arrays,
but also symbol tables, sets, records, graphs, trees, etc.
To represent \Index{records}, Lua uses the field name as an index.
The language supports this representation by
providing \verb|a.name| as syntactic sugar for \verb|a["name"]|.
There are several convenient ways to create tables in Lua
\see{tableconstructor}.

Like indices, the value of a table field can be of any type.
In particular,
because functions are first class values,
table fields may contain functions.
So, tables may also carry \emph{methods} \see{func-def}.

Tables, functions, and userdata values are \emph{objects}:
variables do not actually \emph{contain} these values,
only \emph{references} to them.
Assignment, parameter passing, and returns from functions
always manipulate references to these values,
and do not imply any kind of copy.

The library function \verb|type| returns a string describing the type
of a given value \see{pdf-type}.


\subsubsection{Metatables}

Each table or userdata object in Lua may have a \Index{metatable}.

You can change several aspects of the behavior
of an object by setting specific fields in its metatable.
For instance, when an object is the operand of an addition,
Lua checks for a function in the field \verb|"__add"| in its metatable.
If it finds one,
Lua calls that function to perform the addition.

We call the keys in a metatable \Index{events},
and the values \Index{metamethods}.
In the previous example, \verb|"add"| is the event,
and the metamethod is the function that performs the addition.

A metatable controls how an object behaves in arithmetic operations,
order comparisons, concatenation, and indexing.
A metatable can also defines a function to be called when a userdata
is garbage collected.
\See{metatable} gives a detailed description of which events you
can control with metatables.

You can query and change the metatable of an object
through the \verb|setmetatable| and \verb|getmetatable|
functions \see{pdf-getmetatable}.



\subsection{Coercion} \label{coercion}

Lua provides automatic conversion between
string and number values at run time.
Any arithmetic operation applied to a string tries to convert
that string to a number, following the usual rules.
Conversely, whenever a number is used when a string is expected,
the number is converted to a string, in a reasonable format.
The format is chosen so that
a conversion from number to string then back to number
reproduces the original number \emph{exactly}.
For complete control of how numbers are converted to strings,
use the \verb|format| function \see{format}.


\subsection{Variables}

There are two kinds of variables in Lua:
global variables
and local variables.
Variables are assumed to be global unless explicitly declared local
\see{localvar}.
Before the first assignment, the value of a variable is \nil.

All global variables live as fields in ordinary Lua tables.
Usually, globals live in a table called \Index{table of globals}.
However, a function can individually change its global table,
so that all global variables in that function will refer to that table.
This mechanism allows the creation of \Index{namespaces} and other
modularization facilities.

\Index{Local variables} are lexically scoped.
Therefore, local variables can be freely accessed by functions
defined inside their scope \see{visibility}.


\subsection{Garbage Collection}\label{GC}

Lua does automatic memory management.
That means that
you do not have to worry about allocating memory for new objects
and freeing it when the objects are no longer needed.
Lua manages memory automatically by running
a \Index{garbage collector} from time to time
and
collecting all dead objects
(all objects that are no longer accessible from Lua).
All objects in Lua are subject to automatic management:
tables, userdata, functions, and strings.

Using the C~API,
you can set garbage-collector metamethods for userdata \see{metatable}.
When it is about to free a userdata,
Lua calls the metamethod associated with event \verb|gc| in the
userdata's metatable.
Using such facility, you can coordinate Lua's garbage collection
with external resource management
(such as closing files, network or database connections,
or freeing your own memory).

Lua uses two numbers to control its garbage-collection cycles.
One number counts how many bytes of dynamic memory Lua is using,
and the other is a threshold.
When the number of bytes crosses the threshold,
Lua runs the garbage collector,
which reclaims the memory of all dead objects.
The byte counter is corrected,
and then the threshold is reset to twice the value of the byte counter.

Through the C~API, you can query those numbers,
and change the threshold \see{GC-API}.
Setting the threshold to zero actually forces an immediate
garbage-collection cycle,
while setting it to a huge number effectively stops the garbage collector.
Using Lua code you have a more limited control over garbage-collection cycles,
through the functions \verb|gcinfo| and \verb|collectgarbage|
\see{predefined}.


\subsubsection{Weak Tables}\label{weak-table}

A \IndexEmph{weak table} is a table whose elements are
\IndexEmph{weak references}.
A weak reference is ignored by the garbage collector.
In other words,
if the only references to an object are weak references,
then the garbage collector will collect that object.

A weak table can have weak keys, weak values, or both.
A table with weak keys allows the collection of its keys,
but prevents the collection of its values.
A table with both weak keys and weak values allows the collection of
both keys and values.
In any case, if either the key or the value is collected,
the whole pair is removed from the table.
The weakness of a table is set with the \verb|setmode| function.


%------------------------------------------------------------------------------
\section{The Language}\label{language}

This section describes the lexis, the syntax, and the semantics of Lua.
In other words,
this section describes
which tokens are valid,
how they can be combined,
and what their combinations mean.

\subsection{Lexical Conventions} \label{lexical}

\IndexEmph{Identifiers} in Lua can be any string of letters,
digits, and underscores,
not beginning with a digit.
This coincides with the definition of identifiers in most languages.
(The definition of letter depends on the current locale:
any character considered alphabetic by the current locale
can be used in an identifier.)

The following \IndexEmph{keywords} are reserved,
and cannot be used as identifiers:
\index{reserved words}
\begin{verbatim}
       and       break     do        else      elseif
       end       false     for       function  global
       if        in        local     nil       not
       or        repeat    return    then      true
       until     while
\end{verbatim}

Lua is a case-sensitive language:
\T{and} is a reserved word, but \T{And} and \T{\'and}
(if the locale permits) are two different, valid identifiers.
As a convention, identifiers starting with an underscore followed by
uppercase letters (such as \verb|_VERSION|)
are reserved for internal variables.

The following strings denote other \Index{tokens}:
\begin{verbatim}
       +     -     *     /     ^     %
       ~=    <=    >=    <     >     ==    =
       (     )     {     }     [     ]
       ;     :     ,     .     ..    ...
\end{verbatim}

\IndexEmph{Literal strings}
can be delimited by matching single or double quotes,
and can contain the C-like escape sequences
`\verb|\a|' (bell),
`\verb|\b|' (backspace),
`\verb|\f|' (form feed),
`\verb|\n|' (newline),
`\verb|\r|' (carriage return),
`\verb|\t|' (horizontal tab),
`\verb|\v|' (vertical tab),
`\verb|\\|' (backslash),
`\verb|\"|' (double quote),
`\verb|\'|' (single quote),
and `\verb|\|\emph{newline}' (that is, a backslash followed by a real newline,
which  results in a newline in the string).
A character in a string may also be specified by its numerical value,
through the escape sequence `\verb|\|\emph{ddd}',
where \emph{ddd} is a sequence of up to three \emph{decimal} digits.
Strings in Lua may contain any 8-bit value, including embedded zeros,
which can be specified as `\verb|\0|'.

Literal strings can also be delimited by matching \verb|[[| $\ldots$ \verb|]]|.
Literals in this bracketed form may run for several lines,
may contain nested \verb|[[| $\ldots$ \verb|]]| pairs,
and do not interpret escape sequences.
For convenience,
when the opening \verb|[[| is immediately followed by a newline,
the newline is not included in the string.
That form is specially convenient for
writing strings that contain program pieces or
other quoted strings.
As an example, in a system using ASCII
(in which `\verb|a|' is coded as~97,
newline is coded as~10, and `\verb|1|' is coded as~49),
the four literals below denote the same string:
\begin{verbatim}
       1)   "alo\n123\""
       2)   '\97lo\10\04923"'
       3)   [[alo
            123"]]
       4)   [[
            alo
            123"]]
\end{verbatim}

\IndexEmph{Numerical constants} may be written with an optional decimal part
and an optional decimal exponent.
Examples of valid numerical constants are
\begin{verbatim}
       3     3.0     3.1416  314.16e-2   0.31416E1
\end{verbatim}

\IndexEmph{Comments} start anywhere outside a string with a
double hyphen (\verb|--|);
If the text after \verb|--| is different from \verb|[[|,
the comment is a short comment,
that runs until the end of the line.
Otherwise, it is a long comment,
that runs until the corresponding \verb|]]|.
Long comments may run for several lines,
and may contain nested \verb|[[| $\ldots$ \verb|]]| pairs.
For convenience,
the first line of a chunk is skipped if it starts with \verb|#|.
This facility allows the use of Lua as a script interpreter
in Unix systems \see{lua-sa}.


\subsection{Variables}\label{variables}

Variables are places that store values.
%In Lua, variables are given by simple identifiers or by table fields.

A single name can denote a global variable, a local variable,
or a formal parameter in a function
(formal parameters are just local variables):
\begin{Produc}
\produc{var}{\Nter{Name}}
\end{Produc}%
Square brackets are used to index a table:
\begin{Produc}
\produc{var}{prefixexp \ter{[} exp \ter{]}}
\end{Produc}%
The first expression should result in a table value,
and the second expression identifies a specific entry inside that table.

The syntax \verb|var.NAME| is just syntactic sugar for
\verb|var["NAME"]|:
\begin{Produc}
\produc{var}{prefixexp \ter{.} \Nter{Name}}
\end{Produc}%

The expression denoting the table to be indexed has a restricted syntax;
\See{expressions} for details.

The meaning of assignments and evaluations of global and
indexed variables can be changed via metatables.
An assignment to a global variable \verb|x = val|
is equivalent to the assignment
\verb|_glob.x = val|,
where \verb|_glob| is the table of globals of the running function
(\see{global-table} for a discussion about the table of globals).
An assignment to an indexed variable \verb|t[i] = val| is equivalent to
\verb|settable_event(t,i,val)|.
An access to a global variable \verb|x|
is equivalent to \verb|_glob.x|
(again, \see{global-table} for a discussion about \verb|_glob|).
An access to an indexed variable \verb|t[i]| is equivalent to
a call \verb|gettable_event(t,i)|.
See \See{metatable} for a complete description of the
\verb|settable_event| and \verb|gettable_event| functions.
(These functions are not defined in Lua.
We use them here only for explanatory purposes.)


\subsection{Statements}\label{stats}

Lua supports an almost conventional set of \Index{statements},
similar to those in Pascal or C.
The conventional commands include
assignment, control structures, and procedure calls.
Non-conventional commands include table constructors
and variable declarations.

\subsubsection{Chunks}\label{chunks}
The unit of execution of Lua is called a \Def{chunk}.
A chunk is simply a sequence of statements,
which are executed sequentially.
Each statement can be optionally followed by a semicolon:
\begin{Produc}
\produc{chunk}{\rep{stat \opt{\ter{;}}}}
\end{Produc}%

\subsubsection{Blocks}
A \Index{block} is a list of statements;
syntactically, a block is equal to a chunk:
\begin{Produc}
\produc{block}{chunk}
\end{Produc}%

A block may be explicitly delimited to produce a single statement:
\begin{Produc}
\produc{stat}{\rwd{do} block \rwd{end}}
\end{Produc}%
\IndexKW{do}
Explicit blocks are useful
to control the scope of variable declarations.
Explicit blocks are also sometimes used to
add a \rwd{return} or \rwd{break} statement in the middle
of another block \see{control}.

\subsubsection{\Index{Assignment}} \label{assignment}
Lua allows \Index{multiple assignment}.
Therefore, the syntax for assignment
defines a list of variables on the left side
and a list of expressions on the right side.
The elements in both lists are separated by commas:
\begin{Produc}
\produc{stat}{varlist1 \ter{=} explist1}
\produc{varlist1}{var \rep{\ter{,} var}}
\produc{explist1}{exp \rep{\ter{,} exp}}
\end{Produc}%
Expressions are discussed in \See{expressions}.

Before the assignment,
the list of values is \emph{adjusted} to the length of
the list of variables.\index{adjustment}
If there are more values than needed,
the excess values are thrown away.
If there are less values than needed,
the list is extended with as many  \nil's as needed.
If the list of expressions ends with a function call,
then all values returned by that function call enter in the list of values,
before the adjust
(except when the call is enclosed in parentheses; see \See{expressions}).

The assignment statement first evaluates all its expressions,
and only then makes the assignments.
So, the code
\begin{verbatim}
       i = 3
       i, a[i] = i+1, 20
\end{verbatim}
sets \verb|a[3]| to 20, without affecting \verb|a[4]|
because the \verb|i| in \verb|a[i]| is evaluated
before it is assigned 4.
Similarly, the line
\begin{verbatim}
       x, y = y, x
\end{verbatim}
exchanges the values of \verb|x| and \verb|y|.

\subsubsection{Control Structures}\label{control}
The control structures
\rwd{if}, \rwd{while}, and \rwd{repeat} have the usual meaning and
familiar syntax:
\index{while-do statement}\IndexKW{while}
\index{repeat-until statement}\IndexKW{repeat}\IndexKW{until}
\index{if-then-else statement}\IndexKW{if}\IndexKW{else}\IndexKW{elseif}
\begin{Produc}
\produc{stat}{\rwd{while} exp \rwd{do} block \rwd{end}}
\produc{stat}{\rwd{repeat} block \rwd{until} exp}
\produc{stat}{\rwd{if} exp \rwd{then} block
  \rep{\rwd{elseif} exp \rwd{then} block}
   \opt{\rwd{else} block} \rwd{end}}
\end{Produc}%
Lua also has a \rwd{for} statement, in two flavors \see{for}.

The \Index{condition expression} \M{exp} of a
control structure may return any value.
All values different from \nil{} and \False{} are considered true
(in particular, the number 0 and the empty string are also true);
both \False{} and \nil{} are considered false.

The \rwd{return} statement is used to return values
from a function or from a chunk.\IndexKW{return}
\label{return}%
\index{return statement}%
Functions and chunks may return more than one value,
and so the syntax for the \rwd{return} statement is
\begin{Produc}
\produc{stat}{\rwd{return} \opt{explist1}}
\end{Produc}%

The \rwd{break} statement can be used to terminate the execution of a
\rwd{while}, \rwd{repeat}, or \rwd{for} loop,
skipping to the next statement after the loop:\IndexKW{break}
\index{break statement}
\begin{Produc}
\produc{stat}{\rwd{break}}
\end{Produc}%
A \rwd{break} ends the innermost enclosing loop.

\NOTE
For syntactic reasons, \rwd{return} and \rwd{break}
statements can only be written as the \emph{last} statement of a block.
If it is really necessary to \rwd{return} or \rwd{break} in the
middle of a block,
then an explicit inner block can used,
as in the idioms
`\verb|do return end|' and
`\verb|do break end|',
because now \rwd{return} and \rwd{break} are the last statements in
their (inner) blocks.
In practice,
those idioms are only used during debugging.
(For instance, a line `\verb|do return end|' can be added at the
beginning of a chunk for syntax checking only.)

\subsubsection{For Statement} \label{for}\index{for statement}

The \rwd{for} statement has two forms,
one for numbers and one generic.
\IndexKW{for}\IndexKW{in}

The numerical \rwd{for} loop repeats a block of code while a
control variable runs through an arithmetic progression.
It has the following syntax:
\begin{Produc}
\produc{stat}{\rwd{for} \Nter{Name} \ter{=} exp \ter{,} exp \opt{\ter{,} exp}
                    \rwd{do} block \rwd{end}}
\end{Produc}%
The \emph{block} is repeated for \emph{name} starting at the value of
the first \emph{exp}, until it reaches the second \emph{exp} by steps of the
third \emph{exp}.
More precisely, a \rwd{for} statement like
\begin{verbatim}
       for var = e1, e2, e3 do block end
\end{verbatim}
is equivalent to the code:
\begin{verbatim}
       do
         local var, _limit, _step = tonumber(e1), tonumber(e2), tonumber(e3)
         if not (var and _limit and _step) then error() end
         while (_step>0 and var<=_limit) or (_step<=0 and var>=_limit) do
           block
           var = var+_step
         end
       end
\end{verbatim}
Note the following:
\begin{itemize}\itemsep=0pt
\item Both the limit and the step are evaluated only once,
before the loop starts.
\item \verb|_limit| and \verb|_step| are invisible variables.
The names are here for explanatory purposes only.
\item The behavior is \emph{undefined} if you assign to \verb|var| inside
the block.
\item If the third expression (the step) is absent, then a step of~1 is used.
\item You can use \rwd{break} to exit a \rwd{for} loop.
\item The loop variable \verb|var| is local to the statement;
you cannot use its value after the \rwd{for} ends or is broken.
If you need the value of the loop variable \verb|var|,
then assign it to another variable before breaking or exiting the loop.
\end{itemize}

The generic \rwd{for} statement works over functions,
called \Index{generators}.
It calls its generator to produce a new value for each iteration,
stopping when the new value is \nil.
It has the following syntax:
\begin{Produc}
\produc{stat}{\rwd{for} \Nter{Name} \rep{\ter{,} \Nter{Name}} \rwd{in} explist1
                    \rwd{do} block \rwd{end}}
\end{Produc}%
A \rwd{for} statement like
\begin{verbatim}
       for var_1, ..., var_n in explist do block end
\end{verbatim}
is equivalent to the code:
\begin{verbatim}
       do
         local _f, _s, var_1 = explist
         while 1 do
           local var_2, ..., var_n
           var_1, ..., var_n = _f(_s, var_1)
           if var_1 == nil then break end
           block
         end
       end
\end{verbatim}
Note the following:
\begin{itemize}\itemsep=0pt
\item \verb|explist| is evaluated only once.
Its results are a ``generator'' function,
a ``state'', and an initial value for the ``iterator variable''.
\item \verb|_f| and \verb|_s| are invisible variables.
The names are here for explanatory purposes only.
\item The behavior is \emph{undefined} if you assign to any
\verb|var_i| inside the block.
\item You can use \rwd{break} to exit a \rwd{for} loop.
\item The loop variables \verb|var_i| are local to the statement;
you cannot use their values after the \rwd{for} ends.
If you need these values,
then assign them to other variables before breaking or exiting the loop.
\end{itemize}


\subsubsection{Function Calls as Statements} \label{funcstat}
Because of possible side-effects,
function calls can be executed as statements:
\begin{Produc}
\produc{stat}{functioncall}
\end{Produc}%
In this case, all returned values are thrown away.
Function calls are explained in \See{functioncall}.

\subsubsection{Local Declarations} \label{localvar}
\Index{Local variables} may be declared anywhere inside a block.
The declaration may include an initial assignment:\IndexKW{local}
\begin{Produc}
\produc{stat}{\rwd{local} namelist \opt{\ter{=} explist1}}
\produc{namelist}{\Nter{Name} \rep{\ter{,} \Nter{Name}}}
\end{Produc}%
If present, an initial assignment has the same semantics
of a multiple assignment \see{assignment}.
Otherwise, all variables are initialized with \nil.

A chunk is also a block \see{chunks},
and so local variables can be declared outside any explicit block.
Such local variables die when the chunk ends.

Visibility rules for local variables are explained in \See{visibility}.


\subsection{\Index{Expressions}}\label{expressions}

%\subsubsection{\Index{Basic Expressions}}
The basic expressions in Lua are the following:
\begin{Produc}
\produc{exp}{prefixexp}
\produc{exp}{\rwd{nil} \Or \rwd{false} \Or \rwd{true}}
\produc{exp}{Number}
\produc{exp}{Literal}
\produc{exp}{function}
\produc{exp}{tableconstructor}
\produc{prefixexp}{var \Or functioncall \Or \ter{(} exp \ter{)}}
\end{Produc}%
\IndexKW{nil}\IndexKW{false}\IndexKW{true}

An expression enclosed in parentheses always results in only one value.
Thus,
\verb|(f(x,y,z))| is always a single value,
even if \verb|f| returns several values.
(The value of \verb|(f(x,y,z))| is the first value returned by \verb|f|
or \nil{} if \verb|f| does not return any values.)

\emph{Numbers} and \emph{literal strings} are explained in \See{lexical};
variables are explained in \See{variables};
function definitions are explained in \See{func-def};
function calls are explained in \See{functioncall};
table constructors are explained in \See{tableconstructor}.

Expressions can also be built with arithmetic operators, relational operators,
and logical operadors, all of which are explained below.

\subsubsection{Arithmetic Operators}
Lua supports the usual \Index{arithmetic operators}:
the binary \verb|+| (addition),
\verb|-| (subtraction), \verb|*| (multiplication),
\verb|/| (division), and \verb|^| (exponentiation);
and unary \verb|-| (negation).
If the operands are numbers, or strings that can be converted to
numbers \see{coercion},
then all operations except exponentiation have the usual meaning,
while exponentiation calls a global function \verb|pow|; ??
otherwise, an appropriate metamethod is called \see{metatable}.
The standard mathematical library defines function \verb|pow|,
giving the expected meaning to \Index{exponentiation}
\see{mathlib}.

\subsubsection{Relational Operators}\label{rel-ops}
The \Index{relational operators} in Lua are
\begin{verbatim}
       ==    ~=    <     >     <=    >=
\end{verbatim}
These operators always result in \False{} or \True.

Equality (\verb|==|) first compares the type of its operands.
If the types are different, then the result is \False.
Otherwise, the values of the operands are compared.
Numbers and strings are compared in the usual way.
Tables, userdata, and functions are compared \emph{by reference},
that is,
two tables are considered equal only if they are the \emph{same} table.

??eq metamethod??

Every time you create a new table (or userdata, or function),
this new value is different from any previously existing value.

\NOTE
The conversion rules of \See{coercion}
\emph{do not} apply to equality comparisons.
Thus, \verb|"0"==0| evaluates to \emph{false},
and \verb|t[0]| and \verb|t["0"]| denote different
entries in a table.
\medskip

The operator \verb|~=| is exactly the negation of equality (\verb|==|).

The order operators work as follows.
If both arguments are numbers, then they are compared as such.
Otherwise, if both arguments are strings,
then their values are compared according to the current locale.
Otherwise, the ``lt'' or the ``le'' metamethod is called \see{metatable}.


\subsubsection{Logical Operators}
The \Index{logical operators} in Lua are
\index{and}\index{or}\index{not}
\begin{verbatim}
       and   or    not
\end{verbatim}
Like the control structures \see{control},
all logical operators consider both \False{} and \nil{} as false
and anything else as true.
\IndexKW{and}\IndexKW{or}\IndexKW{not}

The operator \rwd{not} always return \False{} or \True.

The conjunction operator \rwd{and} returns its first argument
if its value is \False{} or \nil;
otherwise, \rwd{and} returns its second argument.
The disjunction operator \rwd{or} returns its first argument
if it is different from \nil and \False;
otherwise, \rwd{or} returns its second argument.
Both \rwd{and} and \rwd{or} use \Index{short-cut evaluation},
that is,
the second operand is evaluated only if necessary.
For example,
\begin{verbatim}
       10 or error()       -> 10
       nil or "a"          -> "a"
       nil and 10          -> nil
       false and error()   -> false
       false and nil       -> false
       false or nil        -> nil
       10 and 20           -> 20
\end{verbatim}

\subsubsection{Concatenation} \label{concat}
The string \Index{concatenation} operator in Lua is
denoted by two dots (`\verb|..|').
If both operands are strings or numbers, then they are converted to
strings according to the rules mentioned in \See{coercion}.
Otherwise, the ``concat'' metamethod is called \see{metatable}.

\subsubsection{Precedence}
\Index{Operator precedence} in Lua follows the table below,
from lower to higher priority:
\begin{verbatim}
       or
       and
       <     >     <=    >=    ~=    ==
       ..
       +     -
       *     /
       not   - (unary)
       ^
\end{verbatim}
All binary operators are left associative,
except for \verb|^| (exponentiation),
which is right associative.
\NOTE
The pre-compiler may rearrange the order of evaluation of
associative operators,
and may exchange the operands of commutative operators,
as long as these optimizations do not change normal results.
However, these optimizations may change some results
if you define non-associative (or non-commutative)
metamethods for those operators.

\subsubsection{Table Constructors} \label{tableconstructor}
Table \Index{constructors} are expressions that create tables;
every time a constructor is evaluated, a new table is created.
Constructors can be used to create empty tables,
or to create a table and initialize some of its fields.
The general syntax for constructors is
\begin{Produc}
\produc{tableconstructor}{\ter{\{} \opt{fieldlist} \ter{\}}}
\produc{fieldlist}{field \rep{fieldsep field} \opt{fieldsep}}
\produc{field}{\ter{[} exp \ter{]} \ter{=} exp \Or
               \Nter{Name} \ter{=} exp \Or exp}
\produc{fieldsep}{\ter{,} \Or \ter{;}}
\end{Produc}%

Each field of the form \verb|[exp1] = exp2| adds to the new table an entry
with key \verb|exp1| and value \verb|exp2|.
A field of the form \verb|name = exp| is equivalent to
\verb|["name"] = exp|.
Finally, fields of the form \verb|exp| are equivalent to
\verb|[i] = exp|, where \verb|i| are consecutive numerical integers,
starting with 1.
Fields in the other formats do not affect this counting.
For example,
\begin{verbatim}
       a = {[f(1)] = g; "x", "y"; x = 1, f(x), [30] = 23; 45}
\end{verbatim}
is equivalent to
\begin{verbatim}
       do
         local temp = {}
         temp[f(1)] = g
         temp[1] = "x"         -- 1st exp
         temp[2] = "y"         -- 2nd exp
         temp.x = 1            -- temp["x"] = 1
         temp[3] = f(x)        -- 3rd exp
         temp[30] = 23
         temp[4] = 45          -- 4th exp
         a = temp
       end
\end{verbatim}

If the last expression in the list is a function call,
then all values returned by the call enter the list consecutively
\see{functioncall}.
If you want to avoid this,
enclose the function call in parentheses.

The field list may have an optional trailing separator,
as a convenience for machine-generated code.


\subsubsection{Function Calls}  \label{functioncall}
A \Index{function call} in Lua has the following syntax:
\begin{Produc}
\produc{functioncall}{prefixexp args}
\end{Produc}%
In a function call,
first \M{prefixexp} and \M{args} are evaluated.
If the value of \M{prefixexp} has type \emph{function},
then that function is called,
with the given arguments.
Otherwise, its ``call'' metamethod is called,
having as first parameter the value of \M{prefixexp},
followed by the original call arguments
\see{metatable}.

The form
\begin{Produc}
\produc{functioncall}{prefixexp \ter{:} \Nter{name} args}
\end{Produc}%
can be used to call ``methods''.
A call \verb|v:name(...)|
is syntactic sugar for \verb|v.name(v, ...)|,
except that \verb|v| is evaluated only once.

Arguments have the following syntax:
\begin{Produc}
\produc{args}{\ter{(} \opt{explist1} \ter{)}}
\produc{args}{tableconstructor}
\produc{args}{Literal}
\end{Produc}%
All argument expressions are evaluated before the call.
A call of the form \verb|f{...}| is syntactic sugar for
\verb|f({...})|, that is,
the argument list is a single new table.
A call of the form \verb|f'...'|
(or \verb|f"..."| or \verb|f[[...]]|) is syntactic sugar for
\verb|f('...')|, that is,
the argument list is a single literal string.

Because a function can return any number of results
\see{return},
the number of results must be adjusted before they are used.
If the function is called as a statement \see{funcstat},
then its return list is adjusted to~0 elements,
thus discarding all returned values.
If the function is called inside another expression,
or in the middle of a list of expressions,
then its return list is adjusted to~1 element,
thus discarding all returned values but the first one.
If the function is called as the last element of a list of expressions,
then no adjustment is made
(unless the call is enclosed in parentheses).

Here are some examples:
\begin{verbatim}
       f()                -- adjusted to 0 results
       g(f(), x)          -- f() is adjusted to 1 result
       g(x, f())          -- g gets x plus all values returned by f()
       a,b,c = f(), x     -- f() is adjusted to 1 result (and c gets nil)
       a,b,c = x, f()     -- f() is adjusted to 2
       a,b,c = f()        -- f() is adjusted to 3
       return f()         -- returns all values returned by f()
       return x,y,f()     -- returns x, y, and all values returned by f()
       {f()}              -- creates a list with all values returned by f()
       {f(), nil}         -- f() is adjusted to 1 result
\end{verbatim}

If you enclose a function call in parentheses,
then it is adjusted to return exactly one value:
\begin{verbatim}
       return x,y,(f())   -- returns x, y, and the first value from f()
       {(f())}            -- creates a table with exactly one element
\end{verbatim}

As an exception to the format-free syntax of Lua,
you cannot put a line break before the \verb|(| in a function call.
That restriction avoids some ambiguities in the language.
If you write
\begin{verbatim}
       a = f
       (g).x(a)
\end{verbatim}
Lua would read that as \verb|a = f(g).x(a)|.
So, if you want two statements, you must add a semi-colon between them.
If you actually want to call \verb|f|,
you must remove the line break before \verb|(g)|.


\subsubsection{\Index{Function Definitions}} \label{func-def}

The syntax for function definition is\IndexKW{function}
\begin{Produc}
\produc{function}{\rwd{function} funcbody}
\produc{funcbody}{\ter{(} \opt{parlist1} \ter{)} block \rwd{end}}
\end{Produc}%

The following syntactic sugar simplifies function definitions:
\begin{Produc}
\produc{stat}{\rwd{function} funcname funcbody}
\produc{stat}{\rwd{local} \rwd{function} \Nter{name} funcbody}
\produc{funcname}{\Nter{name} \rep{\ter{.} \Nter{name}} \opt{\ter{:} \Nter{name}}}
\end{Produc}%
The statement
\begin{verbatim}
       function f () ... end
\end{verbatim}
translates to
\begin{verbatim}
       f = function () ... end
\end{verbatim}
The statement
\begin{verbatim}
       function t.a.b.c.f () ... end
\end{verbatim}
translates to
\begin{verbatim}
       t.a.b.c.f = function () ... end
\end{verbatim}
The statement
\begin{verbatim}
       local function f () ... end
\end{verbatim}
translates to
\begin{verbatim}
       local f; f = function () ... end
\end{verbatim}

A function definition is an executable expression,
whose value has type \emph{function}.
When Lua pre-compiles a chunk,
all its function bodies are pre-compiled too.
Then, whenever Lua executes the function definition,
the function is \emph{instantiated} (or \emph{closed}).
This function instance (or \emph{closure})
is the final value of the expression.
Different instances of the same function
may refer to different non-local variables \see{visibility}
and may have different tables of globals \see{global-table}.

Parameters act as local variables,
initialized with the argument values:
\begin{Produc}
\produc{parlist1}{namelist \opt{\ter{,} \ter{\ldots}}}
\produc{parlist1}{\ter{\ldots}}
\end{Produc}%
\label{vararg}%
When a function is called,
the list of \Index{arguments} is adjusted to
the length of the list of parameters,
unless the function is a \Def{vararg function},
which is
indicated by three dots (`\verb|...|') at the end of its parameter list.
A vararg function does not adjust its argument list;
instead, it collects all extra arguments into an implicit parameter,
called \IndexLIB{arg}.
The value of \verb|arg| is a table,
with a field~\verb|n| whose value is the number of extra arguments,
and the extra arguments at positions 1,~2,~\ldots,~\verb|n|.

As an example, consider the following definitions:
\begin{verbatim}
       function f(a, b) end
       function g(a, b, ...) end
       function r() return 1,2,3 end
\end{verbatim}
Then, we have the following mapping from arguments to parameters:
\begin{verbatim}
       CALL            PARAMETERS

       f(3)             a=3, b=nil
       f(3, 4)          a=3, b=4
       f(3, 4, 5)       a=3, b=4
       f(r(), 10)       a=1, b=10
       f(r())           a=1, b=2

       g(3)             a=3, b=nil, arg={n=0}
       g(3, 4)          a=3, b=4,   arg={n=0}
       g(3, 4, 5, 8)    a=3, b=4,   arg={5, 8; n=2}
       g(5, r())        a=5, b=1,   arg={2, 3; n=2}
\end{verbatim}

Results are returned using the \rwd{return} statement \see{return}.
If control reaches the end of a function
without encountering a \rwd{return} statement,
then the function returns with no results.

The \emph{colon} syntax
is used for defining \IndexEmph{methods},
that is, functions that have an implicit extra parameter \IndexVerb{self}.
Thus, the statement
\begin{verbatim}
       function t.a.b.c:f (...) ... end
\end{verbatim}
is syntactic sugar for
\begin{verbatim}
       t.a.b.c.f = function (self, ...) ... end
\end{verbatim}


\subsection{Visibility Rules} \label{visibility}
\index{visibility}

Lua is a lexically scoped language.
The scope of variables begins at the first statement \emph{after}
their declaration and lasts until the end of the innermost block that
includes the declaration.
For instance:
\begin{verbatim}
  x = 10                -- global variable
  do                    -- new block
    local x = x         -- new `x', with value 10
    print(x)            --> 10
    x = x+1
    do                  -- another block
      local x = x+1     -- another `x'
      print(x)          --> 12
    end
    print(x)            --> 11
  end
  print(x)              --> 10  (the global one)
\end{verbatim}
Notice that, in a declaration like \verb|local x = x|,
the new \verb|x| being declared is not in scope yet,
so the second \verb|x| refers to the ``outside'' variable.

Because of those \Index{lexical scoping} rules,
local variables can be freely accessed by functions
defined inside their scope.
For instance:
\begin{verbatim}
  local counter = 0
  function inc (x)
    counter = counter + x
    return counter
  end
\end{verbatim}

Notice that each execution of a \rwd{local} statement
``creates'' new local variables.
Consider the following example:
\begin{verbatim}
  a = {}
  local x = 20
  for i=1,10 do
    local y = 0
    a[i] = function () y=y+1; return x+y end
  end
\end{verbatim}
In that code,
each function uses a different \verb|y| variable,
while all of them share the same \verb|x|.

\subsection{Error Handling} \label{error}

%% TODO Must be rewritten!!!

Because Lua is an extension language,
all Lua actions start from C~code in the host program
calling a function from the Lua library.
Whenever an error occurs during Lua compilation or execution,
the function \verb|_ERRORMESSAGE| is called \DefLIB{_ERRORMESSAGE}
(provided it is different from \nil),
and then the corresponding function from the library
(\verb|lua_dofile|, \verb|lua_dostring|,
\verb|lua_dobuffer|, or \verb|lua_call|)
is terminated, returning an error condition.

Memory allocation errors are an exception to the previous rule.
When memory allocation fails, Lua may not be able to execute the
\verb|_ERRORMESSAGE| function.
So, for this kind of error, Lua does not call
the \verb|_ERRORMESSAGE| function;
instead, the corresponding function from the library
returns immediately with a special error code (\verb|LUA_ERRMEM|).
This and other error codes are defined in \verb|lua.h|
\see{luado}.

The only argument to \verb|_ERRORMESSAGE| is a string
describing the error.
The default definition for
this function calls \verb|_ALERT|, \DefLIB{_ALERT}
which prints the message to \verb|stderr| \see{alert}.
The standard I/O library redefines \verb|_ERRORMESSAGE|
and uses the debug interface \see{debugI}
to print some extra information,
such as a call-stack traceback.

Lua code can explicitly generate an error by calling the
function \verb|error| \see{pdf-error}.
Lua code can ``catch'' an error using the function
\verb|call| \see{pdf-call}.


\subsection{Metatables} \label{metatable}

Every table and userdata value in Lua may have a \emph{metatable}.
This \IndexEmph{metatable} is a table that defines the behavior of
the original table and userdata under some operations.
You can query and change the metatable of an object with
functions \verb|setmetatable| and \verb|getmetatable| \see{pdf-getmetatable}.

For each of those operations Lua associates a specific key,
called an \emph{event}.
When Lua performs one of those operations over a table or a userdata,
if checks whether that object has a metatable with the corresponding event.
If so, the value associated with that key (the \IndexEmph{metamethod})
controls how Lua will perform the operation.

Metatables control the operations listed next.
Each operation is identified by its corresponding name.
The key for each operation is a string with its name prefixed by
two underscores;
for instance, the key for operation ``add'' is the
string \verb|"__add"|.
The semantics of these operations is better explained by a Lua function
describing how the interpreter executes that operation.
%Each function shows how a handler is called,
%its arguments (that is, its signature),
%its results,
%and the default behavior in the absence of a handler.
The code shown here in Lua is only illustrative;
the real behavior is hard coded in the interpreter,
and it is much more efficient than this simulation.
All functions used in these descriptions
(\verb|rawget|, \verb|tonumber|, etc.)
are described in \See{predefined}.

\begin{description}

\item[``add'':]\IndexTM{add}
the \verb|+| operation.

The function \verb|getbinhandler| below defines how Lua chooses a handler
for a binary operation.
First, Lua tries the first operand.
If its type does not define a handler for the operation,
then Lua tries the second operand.
\begin{verbatim}
       function getbinhandler (op1, op2, event)
         return metatable(op1)[event] or metatable(op2)[event]
       end
\end{verbatim}
Using that function,
the behavior of the ``add'' operation is
\begin{verbatim}
       function add_event (op1, op2)
         local o1, o2 = tonumber(op1), tonumber(op2)
         if o1 and o2 then  -- both operands are numeric
           return o1+o2  -- '+' here is the primitive 'add'
         else  -- at least one of the operands is not numeric
           local h = getbinhandler(op1, op2, "__add")
           if h then
             -- call the handler with both operands
             return h(op1, op2)
           else  -- no handler available: default behavior
             error("unexpected type at arithmetic operation")
           end
         end
       end
\end{verbatim}

\item[``sub'':]\IndexTM{sub}
the \verb|-| operation.
Behavior similar to the ``add'' operation.

\item[``mul'':]\IndexTM{mul}
the \verb|*| operation.
Behavior similar to the ``add'' operation.

\item[``div'':]\IndexTM{div}
the \verb|/| operation.
Behavior similar to the ``add'' operation.

\item[``pow'':]\IndexTM{pow}
the \verb|^| operation (exponentiation) operation.
\begin{verbatim} ??
       function pow_event (op1, op2)
         local h = getbinhandler(op1, op2, "__pow") ???
         if h then
           -- call the handler with both operands
           return h(op1, op2)
         else  -- no handler available: default behavior
           error("unexpected type at arithmetic operation")
         end
       end
\end{verbatim}

\item[``unm'':]\IndexTM{unm}
the unary \verb|-| operation.
\begin{verbatim}
       function unm_event (op)
         local o = tonumber(op)
         if o then  -- operand is numeric
           return -o  -- '-' here is the primitive 'unm'
         else  -- the operand is not numeric.
           -- Try to get a handler from the operand;
           local h = metatable(op).__unm
           if h then
             -- call the handler with the operand and nil
             return h(op, nil)
           else  -- no handler available: default behavior
             error("unexpected type at arithmetic operation")
           end
         end
       end
\end{verbatim}

\item[``lt'':]\IndexTM{lt}
the \verb|<| operation.
\begin{verbatim}
       function lt_event (op1, op2)
         if type(op1) == "number" and type(op2) == "number" then
           return op1 < op2   -- numeric comparison
         elseif type(op1) == "string" and type(op2) == "string" then
           return op1 < op2   -- lexicographic comparison
         else
           local h = getbinhandler(op1, op2, "__lt")
           if h then
             return h(op1, op2)
           else
             error("unexpected type at comparison");
           end
         end
       end
\end{verbatim}
\verb|a>b| is equivalent to \verb|b<a|.

\item[``le'':]\IndexTM{lt}
the \verb|<=| operation.
\begin{verbatim}
       function lt_event (op1, op2)
         if type(op1) == "number" and type(op2) == "number" then
           return op1 < op2   -- numeric comparison
         elseif type(op1) == "string" and type(op2) == "string" then
           return op1 < op2   -- lexicographic comparison
         else
           local h = getbinhandler(op1, op2, "__le")
           if h then
             return h(op1, op2)
           else
             h = getbinhandler(op1, op2, "__lt")
             if h then
               return not h(op2, op1)
             else
               error("unexpected type at comparison");
             end
           end
         end
       end
\end{verbatim}
\verb|a>=b| is equivalent to \verb|b<=a|.
Notice that, in the absence of a ``le'' metamethod,
Lua tries the ``lt'', assuming that \verb|a<=b| is
equivalent to \verb|not (b<a)|.


\item[``concat'':]\IndexTM{concatenation}
the \verb|..| (concatenation) operation.
\begin{verbatim}
       function concat_event (op1, op2)
         if (type(op1) == "string" or type(op1) == "number") and
            (type(op2) == "string" or type(op2) == "number") then
           return op1..op2  -- primitive string concatenation
         else
           local h = getbinhandler(op1, op2, "__concat")
           if h then
             return h(op1, op2)
           else
             error("unexpected type for concatenation")
           end
         end
       end
\end{verbatim}

\item[``index'':]\IndexTM{index}
This handler is called when Lua tries to retrieve the value of an index
not present in a table.
See the ``gettable'' operation for its semantics.

\item[``gettable'':]\IndexTM{gettable}
called whenever Lua accesses an indexed variable.
\begin{verbatim}
       function gettable_event (table, key)
         local h
         if type(table) == "table" then
           local v = rawget(table, key)
           if v ~= nil then return v end
           h = metatable(table).__index
           if h == nil then return nil end
         else
           h = metatable(table).__gettable
           if h == nil then
             error("indexed expression not a table");
           end
         end
         if type(h) == "function" then
           return h(table, key)      -- call the handler
         else return h[key]          -- or repeat operation with it
       end
\end{verbatim}

\item[``newindex'':]\IndexTM{index}
This handler is called when Lua tries to insert the value of an index
not present in a table.
See the ``settable'' operation for its semantics.

\item[``settable'':]\IndexTM{settable}
called when Lua assigns to an indexed variable.
\begin{verbatim}
       function settable_event (table, key, value)
         local h
         if type(table) == "table" then
           local v = rawget(table, key)
           if v ~= nil then rawset(table, key, value); return end
           h = metatable(table).__newindex
           if h == nil then rawset(table, key, value); return end
         else
           h = metatable(table).__settable
           if h == nil then
             error("indexed expression not a table");
           end
         end
         if type(h) == "function" then
           return h(table, key,value)    -- call the handler
         else h[key] = value             -- or repeat operation with it
       end
\end{verbatim}


\item[``call'':]\IndexTM{call}
called when Lua calls a value.
\begin{verbatim}
       function function_event (func, ...)
         if type(func) == "function" then
           return func(unpack(arg))   -- regular call
         else
           local h = metatable(func).__call
           if h then
             tinsert(arg, 1, func)
             return h(unpack(arg))
           else
             error("call expression not a function")
           end
         end
       end
\end{verbatim}

\end{description}

\subsubsection{Metatables and Garbage collection}

Metatables may also define \IndexEmph{finalizer} methods
for userdata values.
For each userdata to be collected,
Lua does the equivalent of the following function:
\begin{verbatim}
       function gc_event (obj)
         local h = metatable(obj).__gc
         if h then
           h(obj)
         end
       end
\end{verbatim}
In a garbage-collection cycle,
the finalizers for userdata are called in \emph{reverse}
order of their creation,
that is, the first finalizer to be called is the one associated
with the last userdata created in the program
(among those to be collected in the same cycle).



%------------------------------------------------------------------------------
\section{The Application Program Interface}\label{API}
\index{C API}

This section describes the API for Lua, that is,
the set of C~functions available to the host program to communicate
with Lua.
All API functions and related types and constants
are declared in the header file \verb|lua.h|.

\NOTE
Even when we use the term ``function'',
any facility in the API may be provided as a \emph{macro} instead.
All such macros use each of its arguments exactly once
(except for the first argument, which is always a Lua state),
and so do not generate hidden side-effects.


\subsection{States} \label{mangstate}

The Lua library is fully reentrant:
it has no global variables.
\index{state}
The whole state of the Lua interpreter
(global variables, stack, tag methods, etc.)\
is stored in a dynamically allocated structure of type \verb|lua_State|;
\DefAPI{lua_State}
this state must be passed as the first argument to
every function in the library (except \verb|lua_open| below).

Before calling any API function,
you must create a state by calling
\begin{verbatim}
       lua_State *lua_open (void);
\end{verbatim}
\DefAPI{lua_open}

To release a state created with \verb|lua_open|, call
\begin{verbatim}
       void lua_close (lua_State *L);
\end{verbatim}
\DefAPI{lua_close}
This function destroys all objects in the given Lua environment
(calling the corresponding garbage-collection metamethods, if any)
and frees all dynamic memory used by that state.
Usually, you do not need to call this function,
because all resources are naturally released when your program ends.
On the other hand,
long-running programs ---
like a daemon or a web server ---
might need to release states as soon as they are not needed,
to avoid growing too large.

With the exception of \verb|lua_open|,
all functions in the Lua API need a state as their first argument.


\subsection{Threads}

Lua offers a partial support for multiple threads of execution.
If you have a C~library that offers multi-threading, 
then Lua can cooperate with it to implement the equivalent facility in Lua.
Also, Lua implements its own coroutine system on top of threads.
The following function creates a new ``thread'' in Lua:
\begin{verbatim}
       lua_State *lua_newthread (lua_State *L);
\end{verbatim}
\DefAPI{lua_newthread}
The new state returned by this function shares with the original state
all global environment (such as tables, tag methods, etc.),
but has an independent run-time stack.
(The use of these multiple stacks must be ``syncronized'' with C.
How to explain that? TO BE WRITTEN.)

Each thread has an independent table for global variables.
When you create a thread, this table is the same as that of the given state,
but you can change each one independently.

You destroy threads with \DefAPI{lua_closethread}
\begin{verbatim}
       void lua_closethread (lua_State *L, lua_State *thread);
\end{verbatim}
You cannot close the sole (or last) thread of a state.
Instead, you must close the state itself.


\subsection{The Stack and Indices}

Lua uses a virtual \emph{stack} to pass values to and from C.
Each element in this stack represents a Lua value
(\nil, number, string, etc.).

Each C invocation has its own stack.
Whenever Lua calls C, the called function gets a new stack,
which is independent of previous stacks or of stacks of still
active C functions.

For convenience,
most query operations in the API do not follow a strict stack discipline.
Instead, they can refer to any element in the stack by using an \emph{index}:
A positive index represents an \emph{absolute} stack position
(starting at~1);
a negative index represents an \emph{offset} from the top of the stack.
More specifically, if the stack has \M{n} elements,
then index~1 represents the first element
(that is, the element that was pushed onto the stack first),
and
index~\M{n} represents the last element;
index~\Math{-1} also represents the last element
(that is, the element at the top),
and index \Math{-n} represents the first element.
We say that an index is \emph{valid}
if it lies between~1 and the stack top
(that is, if \verb|1 <= abs(index) <= top|).
\index{stack index} \index{valid index}

At any time, you can get the index of the top element by calling
\begin{verbatim}
       int lua_gettop (lua_State *L);
\end{verbatim}
\DefAPI{lua_gettop}
Because indices start at~1,
the result of \verb|lua_gettop| is equal to the number of elements in the stack
(and so 0~means an empty stack).

When you interact with Lua API,
\emph{you are responsible for controlling stack overflow}.
The function
\begin{verbatim}
       int lua_checkstack (lua_State *L, int extra);
\end{verbatim}
\DefAPI{lua_checkstack}
grows the stack size to \verb|top + extra| elements;
it returns false if it cannot grow the stack to that size.
This function never shrinks the stack;
if the stack is already bigger than the new size,
it is left unchanged.

Whenever Lua calls C, \DefAPI{LUA_MINSTACK}
it ensures that \verb|lua_checkstack(L, LUA_MINSTACK)| is true,
that is,
at least \verb|LUA_MINSTACK| positions are still available.
\verb|LUA_MINSTACK| is defined in \verb|lua.h| as 20,
so that usually you do not have to worry about stack space
unless your code has loops pushing elements onto the stack.

Most query functions accept as indices any value inside the
available stack space, that is, indices up to the maximum stack size
you (or Lua) have set through \verb|lua_checkstack|.
Such indices are called \emph{acceptable indices}.
More formally, we define an \IndexEmph{acceptable index}
as follows:
\begin{verbatim}
     (index < 0 && abs(index) <= top) || (index > 0 && index <= top + stackspace)
\end{verbatim}
Note that 0 is never an acceptable index.

Unless otherwise noticed,
any function that accepts valid indices can also be called with
\Index{pseudo-indices},
which represent some Lua values that are accessible to the C~code
but are not in the stack.
Pseudo-indices are used to access the table of globals \see{globals},
the registry, and the upvalues of a C function \see{c-closure}.

\subsection{Stack Manipulation}
The API offers the following functions for basic stack manipulation:
\begin{verbatim}
       void lua_settop    (lua_State *L, int index);
       void lua_pushvalue (lua_State *L, int index);
       void lua_remove    (lua_State *L, int index);
       void lua_insert    (lua_State *L, int index);
       void lua_replace   (lua_State *L, int index);
\end{verbatim}
\DefAPI{lua_settop}\DefAPI{lua_pushvalue}
\DefAPI{lua_remove}\DefAPI{lua_insert}\DefAPI{lua_replace}

\verb|lua_settop| accepts any acceptable index,
or 0,
and sets the stack top to that index.
If the new top is larger than the old one,
then the new elements are filled with \nil.
If \verb|index| is 0, then all stack elements are removed.
A useful macro defined in the \verb|lua.h| is
\begin{verbatim}
       #define lua_pop(L,n) lua_settop(L, -(n)-1)
\end{verbatim}
\DefAPI{lua_pop}
which pops \verb|n| elements from the stack.

\verb|lua_pushvalue| pushes onto the stack a copy of the element
at the given index.
\verb|lua_remove| removes the element at the given position,
shifting down the elements above that position to fill the gap.
\verb|lua_insert| moves the top element into the given position,
shifting up the elements above that position to open space.
\verb|lua_replace| moves the top element into the given position,
without shifting any element (therefore replacing the value at
the given position).
These functions accept only valid indices.
(Obviously, you cannot call \verb|lua_remove| or \verb|lua_insert| with
pseudo-indices, as they do not represent a stack position.)

As an example, if the stack starts as \verb|10 20 30 40 50*|
(from bottom to top; the \verb|*| marks the top),
then
\begin{verbatim}
       lua_pushvalue(L, 3)    --> 10 20 30 40 50 30*
       lua_pushvalue(L, -1)   --> 10 20 30 40 50 30 30*
       lua_remove(L, -3)      --> 10 20 30 40 30 30*
       lua_remove(L,  6)      --> 10 20 30 40 30*
       lua_insert(L,  1)      --> 30 10 20 30 40*
       lua_insert(L, -1)      --> 30 10 20 30 40*  (no effect)
       lua_replace(L, 2)      --> 30 40 20 30*
       lua_settop(L, -3)      --> 30 40 20*
       lua_settop(L,  6)      --> 30 40 20 nil nil nil*
\end{verbatim}



\subsection{Querying the Stack}

To check the type of a stack element,
the following functions are available:
\begin{verbatim}
       int         lua_type        (lua_State *L, int index);
       int         lua_isnil       (lua_State *L, int index);
       int         lua_isboolean   (lua_State *L, int index);
       int         lua_isnumber    (lua_State *L, int index);
       int         lua_isstring    (lua_State *L, int index);
       int         lua_istable     (lua_State *L, int index);
       int         lua_isfunction  (lua_State *L, int index);
       int         lua_iscfunction (lua_State *L, int index);
       int         lua_isuserdata  (lua_State *L, int index);
       int         lua_isdataval   (lua_State *L, int index);
\end{verbatim}
\DefAPI{lua_type}
\DefAPI{lua_isnil}\DefAPI{lua_isnumber}\DefAPI{lua_isstring}
\DefAPI{lua_istable}\DefAPI{lua_isboolean}
\DefAPI{lua_isfunction}\DefAPI{lua_iscfunction}
\DefAPI{lua_isuserdata}\DefAPI{lua_isdataval}
These functions can be called with any acceptable index.

\verb|lua_type| returns the type of a value in the stack,
or \verb|LUA_TNONE| for a non-valid index
(that is, if that stack position is ``empty'').
The types are coded by the following constants
defined in \verb|lua.h|:
\verb|LUA_TNIL|,
\verb|LUA_TNUMBER|,
\verb|LUA_TBOOLEAN|,
\verb|LUA_TSTRING|,
\verb|LUA_TTABLE|,
\verb|LUA_TFUNCTION|,
\verb|LUA_TUSERDATA|,
\verb|LUA_TLIGHTUSERDATA|.
The following function translates such constants to a type name:
\begin{verbatim}
       const char *lua_typename  (lua_State *L, int type);
\end{verbatim}
\DefAPI{lua_typename}

The \verb|lua_is*| functions return~1 if the object is compatible
with the given type, and 0 otherwise.
\verb|lua_isboolean| is an exception to this rule,
and it succeeds only for boolean values
(otherwise it would be useless,
as any value is compatible with a boolean).
They always return 0 for a non-valid index.
\verb|lua_isnumber| accepts numbers and numerical strings,
\verb|lua_isstring| accepts strings and numbers \see{coercion},
and \verb|lua_isfunction| accepts both Lua functions and C~functions.
To distinguish between Lua functions and C~functions,
you should use \verb|lua_iscfunction|.
To distinguish between numbers and numerical strings,
you can use \verb|lua_type|.

The API also has functions to compare two values in the stack:
\begin{verbatim}
       int lua_equal    (lua_State *L, int index1, int index2);
       int lua_lessthan (lua_State *L, int index1, int index2);
\end{verbatim}
\DefAPI{lua_equal} \DefAPI{lua_lessthan}
These functions are equivalent to their counterparts in Lua \see{rel-ops}.
Both functions return 0 if any of the indices are non-valid.

\subsection{Getting Values from the Stack}\label{lua-to}

To translate a value in the stack to a specific C~type,
you can use the following conversion functions:
\begin{verbatim}
       int            lua_toboolean   (lua_State *L, int index);
       lua_Number     lua_tonumber    (lua_State *L, int index);
       const char    *lua_tostring    (lua_State *L, int index);
       size_t         lua_strlen      (lua_State *L, int index);
       lua_CFunction  lua_tocfunction (lua_State *L, int index);
       void          *lua_touserdata  (lua_State *L, int index);
\end{verbatim}
\DefAPI{lua_tonumber}\DefAPI{lua_tostring}\DefAPI{lua_strlen}
\DefAPI{lua_tocfunction}\DefAPI{lua_touserdata}\DefAPI{lua_toboolean}
These functions can be called with any acceptable index.
When called with a non-valid index,
they act as if the given value had an incorrect type.

\verb|lua_toboolean| converts the Lua value at the given index
to a C ``boolean'' value (that is, 0 or 1).
Like all tests in Lua, it returns 1 for any Lua value different from
\False{} and \nil;
otherwise it returns 0.
It also returns 0 when called with a non-valid index.
(If you want to accept only real boolean values,
use \verb|lua_isboolean| to test the type of the value.)

\verb|lua_tonumber| converts the Lua value at the given index
to a number (by default, \verb|lua_Number| is \verb|double|).
\DefAPI{lua_Number}
The Lua value must be a number or a string convertible to number
\see{coercion}; otherwise, \verb|lua_tonumber| returns~0.

\verb|lua_tostring| converts the Lua value at the given index to a string
(\verb|const char*|).
The Lua value must be a string or a number;
otherwise, the function returns \verb|NULL|.
If the value is a number,
then \verb|lua_tostring| also
\emph{changes the actual value in the stack to a string}.
(This change confuses \verb|lua_next|
when \verb|lua_tostring| is applied to keys.)
\verb|lua_tostring| returns a fully aligned pointer
to a string inside the Lua environment.
This string always has a zero (\verb|'\0'|)
after its last character (as in~C),
but may contain other zeros in its body.
If you do not know whether a string may contain zeros,
you can use \verb|lua_strlen| to get its actual length.
Because Lua has garbage collection,
there is no guarantee that the pointer returned by \verb|lua_tostring|
will be valid after the corresponding value is removed from the stack.
So, if you need the string after the current function returns,
then you should duplicate it (or put it into the registry \see{registry}).

\verb|lua_tocfunction| converts a value in the stack to a C~function.
This value must be a C~function;
otherwise, \verb|lua_tocfunction| returns \verb|NULL|.
The type \verb|lua_CFunction| is explained in \See{LuacallC}.

\verb|lua_touserdata| is explained in \See{userdata}.


\subsection{Pushing Values onto the Stack}

The API has the following functions to
push C~values onto the stack:
\begin{verbatim}
       void lua_pushboolean   (lua_State *L, int b);
       void lua_pushnumber    (lua_State *L, lua_Number n);
       void lua_pushlstring   (lua_State *L, const char *s, size_t len);
       void lua_pushstring    (lua_State *L, const char *s);
       void lua_pushnil       (lua_State *L);
       void lua_pushcfunction (lua_State *L, lua_CFunction f);
       void lua_pushlightuserdata  (lua_State *L, void *p);
\end{verbatim}

\DefAPI{lua_pushnumber}\DefAPI{lua_pushlstring}\DefAPI{lua_pushstring}
\DefAPI{lua_pushcfunction}\DefAPI{lua_pushlightuserdata}\DefAPI{lua_pushboolean}
\DefAPI{lua_pushnil}\label{pushing}
These functions receive a C~value,
convert it to a corresponding Lua value,
and push the result onto the stack.
In particular, \verb|lua_pushlstring| and \verb|lua_pushstring|
make an internal copy of the given string.
\verb|lua_pushstring| can only be used to push proper C~strings
(that is, strings that end with a zero and do not contain embedded zeros);
otherwise, you should use the more general \verb|lua_pushlstring|,
which accepts an explicit size.

You can also push ``formatted'' strings:
\begin{verbatim}
       const char *lua_pushfstring  (lua_State *L, const char *fmt, ...);
       const char *lua_pushvfstring (lua_State *L, const char *fmt,
                                                   va_list argp);
\end{verbatim}
\DefAPI{lua_pushfstring}\DefAPI{lua_pushvfstring}
Both functions push onto the stack a formatted string,
and return a pointer to that string.
These functions are similar to \verb|sprintf| and \verb|vsprintf|,
but with some important differences:
\begin{itemize}
\item You do not have to allocate the space for the result;
the result is a Lua string, and Lua takes care of memory allocation
(and deallocation, later).
\item The conversion specifiers are quite restricted.
There are no flags, widths, or precisions.
The conversion specifiers can be simply
\verb|%%| (inserts a \verb|%| in the string),
\verb|%s| (inserts a zero-terminated string, with no size restrictions),
\verb|%f| (inserts a \verb|lua_Number|),
\verb|%d| (inserts an \verb|int|),
\verb|%c| (inserts an \verb|int| as a character).
\end{itemize}


\subsection{Controlling Garbage Collection}\label{GC-API}

Lua uses two numbers to control its garbage collection:
the \emph{count} and the \emph{threshold} \see{GC}.
The first counts the ammount of memory in use by Lua;
when the count reaches the threshold,
Lua runs its garbage collector.
After the collection, the count is updated,
and the threshold  is set to twice the count value.

You can access the current values of these two numbers through the
following functions:
\begin{verbatim}
       int  lua_getgccount (lua_State *L);
       int  lua_getgcthreshold (lua_State *L);
\end{verbatim}
\DefAPI{lua_getgcthreshold} \DefAPI{lua_getgccount}
Both return their respective values in Kbytes.
You can change the threshold value with
\begin{verbatim}
       void  lua_setgcthreshold (lua_State *L, int newthreshold);
\end{verbatim}
\DefAPI{lua_setgcthreshold}
Again, the \verb|newthreshold| value is given in Kbytes.
When you call this function,
Lua sets the new threshold and checks it against the byte counter.
If the new threshold is smaller than the byte counter,
then Lua immediately runs the garbage collector.
In particular
\verb|lua_setgcthreshold(L,0)| forces a garbage collectiion.
After the collection,
a new threshold is set according to the previous rule.

%% TODO do we need a new way to do that??
% If you want to change the adaptive behavior of the garbage collector,
% you can use the garbage-collection tag method for \nil{} %
% to set your own threshold
% (the tag method is called after Lua resets the threshold).


\subsection{Userdata}\label{userdata}

Userdata represents C values in Lua.
Lua supports two types of userdata:
\Def{full userdata} and \Def{light userdata}.

A full userdata represents a block of memory.
It is an object (like a table):
You must create it, it can have its own metatable,
you can detect when it is being collected.
A full userdata is only equal to itself.

A light userdata represents a pointer.
It is a value (like a number):
You do not create it, it has no metatables,
it is not collected (as it was never created).
A light userdata is equal to ``any''
light userdata with the same address.

In Lua code, there is no way to test whether a userdata is full or light;
both have type \verb|userdata|.
In C code, \verb|lua_type| returns \verb|LUA_TUSERDATA| for full userdata,
and \verb|LUA_LIGHTUSERDATA| for light userdata.

You can create new full userdata with the following function:
\begin{verbatim}
       void *lua_newuserdata (lua_State *L, size_t size);
\end{verbatim}
\DefAPI{lua_newuserdata}
It allocates a new block of memory with the given size,
pushes on the stack a new userdata with the block address,
and returns this address.

To push a light userdata into the stack you use
\verb|lua_pushlightuserdata| \see{pushing}.

\verb|lua_touserdata| \see{lua-to} retrieves the value of a userdata.
When applied on a full userdata, it returns the address of its block;
when applied on a light userdata, it returns its pointer;
when applied on a non-userdata value, it returns \verb|NULL|.

When Lua collects a full userdata,
it calls its \verb|gc| metamethod, if any,
and then it automatically frees its corresponding memory.


\subsection{Metatables}

%% TODO

\subsection{Loading Lua Chunks}
You can load a Lua chunk with
\begin{verbatim}
       typedef const char * (*lua_Chunkreader)
                                (lua_State *L, void *data, size_t *size);

       int lua_load (lua_State *L, lua_Chunkreader reader, void *data,
                                   const char *chunkname);
\end{verbatim}
\DefAPI{Chunkreader}\DefAPI{lua_load}
\verb|lua_load| uses the \emph{reader} to read the chunk.
Everytime it needs another piece of the chunk,
it calls the reader,
passing along its \verb|data| parameter.
The reader must return a pointer to a block of memory
with the part of the chunk,
and set \verb|size| to the block size.
To signal the end of the chunk, the reader must return \verb|NULL|.

In the current implementation,
the reader function cannot call any Lua function;
to ensure that, it always receives \verb|NULL| as the Lua state.

\verb|lua_load| automatically detects whether the chunk is text or binary,
and loads it accordingly (see program \IndexVerb{luac}).

The return values of \verb|lua_load| are:
\begin{itemize}
\item 0 --- no errors;
\item \IndexAPI{LUA_ERRSYNTAX} ---
syntax error during pre-compilation.
\item \IndexAPI{LUA_ERRMEM} ---
memory allocation error.
\end{itemize}
If there are no errors,
the compiled chunk is pushed as a Lua function on top of the stack.
Otherwise, an error message is pushed.

The \emph{chunkname} is used for error messages
and debug information \see{debugI}.

See the auxiliar library (\verb|lauxlib|)
for examples of how to use \verb|lua_load|,
and for some ready-to-use functions to load chunks
from files and from strings.


\subsection{Executing Lua Chunks}\label{luado}
>>>>
A host program can execute Lua chunks written in a file or in a string
by using the following functions:
\begin{verbatim}
       int lua_dofile   (lua_State *L, const char *filename);
       int lua_dostring (lua_State *L, const char *string);
       int lua_dobuffer (lua_State *L, const char *buff,
                         size_t size, const char *name);
\end{verbatim}
\DefAPI{lua_dofile}\DefAPI{lua_dostring}\DefAPI{lua_dobuffer}%
These functions return
0 in case of success, or one of the following error codes
(defined in \verb|lua.h|)
if they fail:
\begin{itemize}
\item \IndexAPI{LUA_ERRRUN} ---
error while running the chunk.
\item \IndexAPI{LUA_ERRSYNTAX} ---
syntax error during pre-compilation.
\item \IndexAPI{LUA_ERRMEM} ---
memory allocation error.
For such errors, Lua does not call \verb|_ERRORMESSAGE| \see{error}.
\item \IndexAPI{LUA_ERRERR} ---
error while running \verb|_ERRORMESSAGE|.
For such errors, Lua does not call \verb|_ERRORMESSAGE| again, to avoid loops.
\item \IndexAPI{LUA_ERRFILE} ---
error opening the file (only for \verb|lua_dofile|).
In this case,
you may want to
check \verb|errno|,
call \verb|strerror|,
or call \verb|perror| to tell the user what went wrong.
\end{itemize}


\subsection{Manipulating Tables}

Tables are created by calling
the function
\begin{verbatim}
       void lua_newtable (lua_State *L);
\end{verbatim}
\DefAPI{lua_newtable}
This function creates a new, empty table and pushes it onto the stack.

To read a value from a table that resides somewhere in the stack,
call
\begin{verbatim}
       void lua_gettable (lua_State *L, int index);
\end{verbatim}
\DefAPI{lua_gettable}
where \verb|index| points to the table.
\verb|lua_gettable| pops a key from the stack
and returns (on the stack) the contents of the table at that key.
The table is left where it was in the stack;
this is convenient for getting multiple values from a table.

As in Lua, this function may trigger a metamethod
for the ``gettable'' or ``index'' events \see{metatable}.
To get the real value of any table key,
without invoking any metamethod,
use the \emph{raw} version:
\begin{verbatim}
       void lua_rawget (lua_State *L, int index);
\end{verbatim}
\DefAPI{lua_rawget}

To store a value into a table that resides somewhere in the stack,
you push the key and the value onto the stack
(in this order),
and then call
\begin{verbatim}
       void lua_settable (lua_State *L, int index);
\end{verbatim}
\DefAPI{lua_settable}
where \verb|index| points to the table.
\verb|lua_settable| pops from the stack both the key and the value.
The table is left where it was in the stack;
this is convenient for setting multiple values in a table.

As in Lua, this operation may trigger a metamethod
for the ``settable'' or ``newindex'' events.
To set the real value of any table index,
without invoking any metamethod,
use the \emph{raw} version:
\begin{verbatim}
       void lua_rawset (lua_State *L, int index);
\end{verbatim}
\DefAPI{lua_rawset}

You can traverse a table with the function
\begin{verbatim}
       int lua_next (lua_State *L, int index);
\end{verbatim}
\DefAPI{lua_next}
where \verb|index| points to the table to be traversed.
The function pops a key from the stack,
and pushes a key-value pair from the table
(the ``next'' pair after the given key).
If there are no more elements, then \verb|lua_next| returns 0
(and pushes nothing).
Use a \nil{} key to signal the start of a traversal.

A typical traversal looks like this:
\begin{verbatim}
       /* table is in the stack at index `t' */
       lua_pushnil(L);  /* first key */
       while (lua_next(L, t) != 0) {
         /* `key' is at index -2 and `value' at index -1 */
         printf("%s - %s\n",
           lua_typename(L, lua_type(L, -2)), lua_typename(L, lua_type(L, -1)));
         lua_pop(L, 1);  /* removes `value'; keeps `key' for next iteration */
       }
\end{verbatim}

NOTE:
While traversing a table,
do not call \verb|lua_tostring| on a key,
unless you know the key is actually a string.
Recall that \verb|lua_tostring| \emph{changes} the value at the given index;
this confuses the next call to \verb|lua_next|.

\subsection{Manipulating Global Variables} \label{globals}

All global variables are kept in an ordinary Lua table.
This table is always at pseudo-index \IndexAPI{LUA_GLOBALSINDEX}.

To access and change the value of global variables,
you can use regular table operations over the global table.
For instance, to access the value of a global variable, do
\begin{verbatim}
       lua_pushstring(L, varname);
       lua_gettable(L, LUA_GLOBALSINDEX);
\end{verbatim}

You can change the global table of a Lua thread using \verb|lua_replace|.


\subsection{Using Tables as Arrays}
The API has functions that help to use Lua tables as arrays,
that is,
tables indexed by numbers only:
\begin{verbatim}
       void lua_rawgeti (lua_State *L, int index, int n);
       void lua_rawseti (lua_State *L, int index, int n);
\end{verbatim}
\DefAPI{lua_rawgeti}
\DefAPI{lua_rawseti}

\verb|lua_rawgeti| pushes the value of the \M{n}-th element of the table
at stack position \verb|index|.
\verb|lua_rawseti| sets the value of the \M{n}-th element of the table
at stack position \verb|index| to the value at the top of the stack,
removing this value from the stack.


\subsection{Calling Functions}

Functions defined in Lua
and C~functions registered in Lua
can be called from the host program.
This is done using the following protocol:
First, the function to be called is pushed onto the stack;
then, the arguments to the function are pushed
in \emph{direct order}, that is, the first argument is pushed first.
Finally, the function is called using
\begin{verbatim}
       void lua_call (lua_State *L, int nargs, int nresults);
\end{verbatim}
\DefAPI{lua_call}
\verb|nargs| is the number of arguments that you pushed onto the stack.
All arguments and the function value are popped from the stack,
and the function results are pushed.
The number of results are adjusted to \verb|nresults|,
unless \verb|nresults| is \IndexAPI{LUA_MULTRET}.
In that case, \emph{all} results from the function are pushed.
Lua takes care that the returned values fit into the stack space.
The function results are pushed onto the stack in direct order
(the first result is pushed first),
so that after the call the last result is on the top.

The following example shows how the host program may do the
equivalent to the Lua code:
\begin{verbatim}
       a = f("how", t.x, 14)
\end{verbatim}
Here it is in~C:
\begin{verbatim}
    lua_pushstring(L, "t");
    lua_gettable(L, LUA_GLOBALSINDEX);          /* global `t' (for later use) */
    lua_pushstring(L, "a");                                       /* var name */
    lua_pushstring(L, "f");                                  /* function name */
    lua_gettable(L, LUA_GLOBALSINDEX);               /* function to be called */
    lua_pushstring(L, "how");                                 /* 1st argument */
    lua_pushstring(L, "x");                            /* push the string "x" */
    lua_gettable(L, -5);                      /* push result of t.x (2nd arg) */
    lua_pushnumber(L, 14);                                    /* 3rd argument */
    lua_call(L, 3, 1);         /* call function with 3 arguments and 1 result */
    lua_settable(L, LUA_GLOBALSINDEX);             /* set global variable `a' */
    lua_pop(L, 1);                               /* remove `t' from the stack */
\end{verbatim}
Notice that the code above is ``balanced'':
at its end, the stack is back to its original configuration.
This is considered good programming practice.

(We did this example using only the raw functions provided by Lua's API,
to show all the details.
Usually programmers use several macros and auxiliar functions that
provide higher level access to Lua.)

%% TODO: pcall

\medskip

>>>>
%% TODO: mover essas 2 para algum lugar melhor.
Some special Lua functions have their own C~interfaces.
The host program can generate a Lua error calling the function
\begin{verbatim}
       void lua_error (lua_State *L, const char *message);
\end{verbatim}
\DefAPI{lua_error}
This function never returns.
If \verb|lua_error| is called from a C~function that has been called from Lua,
then the corresponding Lua execution terminates,
as if an error had occurred inside Lua code.
Otherwise, the whole host program terminates with a call to
\verb|exit(EXIT_FAILURE)|.
Before terminating execution,
the \verb|message| is passed to the error handler function,
\verb|_ERRORMESSAGE| \see{error}.
If \verb|message| is \verb|NULL|,
then \verb|_ERRORMESSAGE| is not called.

The function
\begin{verbatim}
       void lua_concat (lua_State *L, int n);
\end{verbatim}
\DefAPI{lua_concat}
concatenates the \verb|n| values at the top of the stack,
pops them, and leaves the result at the top.
If \verb|n| is 1, the result is that single string
(that is, the function does nothing);
if \verb|n| is 0, the result is the empty string.
Concatenation is done following the usual semantics of Lua
\see{concat}.


\subsection{Defining C Functions} \label{LuacallC}
Lua can be extended with functions written in~C.
These functions must be of type \verb|lua_CFunction|,
which is defined as
\begin{verbatim}
       typedef int (*lua_CFunction) (lua_State *L);
\end{verbatim}
\DefAPI{lua_CFunction}
A C~function receives a Lua environment and returns an integer,
the number of values it has returned to Lua.

In order to communicate properly with Lua,
a C~function must follow the following protocol,
which defines the way parameters and results are passed:
A C~function receives its arguments from Lua in the stack,
in direct order (the first argument is pushed first).
To return values to Lua, a C~function just pushes them onto the stack,
in direct order (the first result is pushed first),
and returns the number of results.
Like a Lua function, a C~function called by Lua can also return
many results.

As an example, the following function receives a variable number
of numerical arguments and returns their average and sum:
\begin{verbatim}
       static int foo (lua_State *L) {
         int n = lua_gettop(L);    /* number of arguments */
         lua_Number sum = 0;
         int i;
         for (i = 1; i <= n; i++) {
           if (!lua_isnumber(L, i))
             lua_error(L, "incorrect argument to function `average'");
           sum += lua_tonumber(L, i);
         }
         lua_pushnumber(L, sum/n);        /* first result */
         lua_pushnumber(L, sum);         /* second result */
         return 2;                   /* number of results */
       }
\end{verbatim}

To register a C~function to Lua,
there is the following convenience macro:
\begin{verbatim}
       #define lua_register(L,n,f) \
               (lua_pushstring(L, n), \
                lua_pushcfunction(L, f), \
                lua_settable(L, LUA_GLOBALSINDEX))
     /* const char *n;   */
     /* lua_CFunction f; */
\end{verbatim}
\DefAPI{lua_register}
which receives the name the function will have in Lua,
and a pointer to the function.
Thus,
the C~function `\verb|foo|' above may be registered in Lua as `\verb|average|'
by calling
\begin{verbatim}
       lua_register(L, "average", foo);
\end{verbatim}

\subsection{Defining C Closures} \label{c-closure}

When a C~function is created,
it is possible to associate some values to it,
thus creating a \IndexEmph{C~closure};
these values are then accessible to the function whenever it is called.
To associate values to a C~function,
first these values should be pushed onto the stack
(when there are multiple values, the first value is pushed first).
Then the function
\begin{verbatim}
       void lua_pushcclosure (lua_State *L, lua_CFunction fn, int n);
\end{verbatim}
\DefAPI{lua_pushcclosure}
is used to push the C~function onto the stack,
with the argument \verb|n| telling how many values should be
associated with the function
(\verb|lua_pushcclosure| also pops these values from the stack);
in fact, the macro \verb|lua_pushcfunction| is defined as
\verb|lua_pushcclosure| with \verb|n| set to 0.

Then, whenever the C~function is called,
those values are located at specific pseudo-indices.
Those pseudo-indices are produced by a macro \IndexAPI{lua_upvalueindex}.
The first value associated with a function is at position
\verb|lua_upvalueindex(1)|, and so on.

For examples of C~functions and closures, see files
\verb|lbaselib.c|, \verb|liolib.c|, \verb|lmathlib.c|, and \verb|lstrlib.c|
in the official Lua distribution.


\subsubsection*{Registry} \label{registry}

Lua provides a pre-defined table that can be used by any C~code to
store whatever Lua value it needs to store,
especially if the C~code needs to keep that Lua value
outside the life span of a C~function.
This table is always located at pseudo-index
\IndexAPI{LUA_REGISTRYINDEX}.
Any C~library can store data into this table,
as long as it chooses a key different from other libraries.
Typically, you can use as key a string containing the library name,
or a light userdata with the address of a C object in your code.

The integer keys in the registry are used by the reference mechanism,
implemented by the auxiliar library,
and therefore should not be used by other purposes.


%------------------------------------------------------------------------------
\section{The Debug Interface} \label{debugI}

Lua has no built-in debugging facilities.
Instead, it offers a special interface,
by means of functions and \emph{hooks},
which allows the construction of different
kinds of debuggers, profilers, and other tools
that need ``inside information'' from the interpreter.
This interface is declared in \verb|luadebug.h|.

\subsection{Stack and Function Information}

The main function to get information about the interpreter stack is
\begin{verbatim}
       int lua_getstack (lua_State *L, int level, lua_Debug *ar);
\end{verbatim}
\DefAPI{lua_getstack}
This function fills parts of a \verb|lua_Debug| structure with
an identification of the \emph{activation record}
of the function executing at a given level.
Level~0 is the current running function,
whereas level \Math{n+1} is the function that has called level \Math{n}.
Usually, \verb|lua_getstack| returns 1;
when called with a level greater than the stack depth,
it returns 0.

The structure \verb|lua_Debug| is used to carry different pieces of
information about an active function:
\begin{verbatim}
      typedef struct lua_Debug {
        const char *event;     /* "call", "return" */
        int currentline;       /* (l) */
        const char *name;      /* (n) */
        const char *namewhat;  /* (n) `global', `local', `field', `method' */
        int nups;              /* (u) number of upvalues */
        int linedefined;       /* (S) */
        const char *what;      /* (S) "Lua" function, "C" function, Lua "main" */
        const char *source;    /* (S) */
        char short_src[LUA_IDSIZE]; /* (S) */

        /* private part */
        ...
      } lua_Debug;
\end{verbatim}
\DefAPI{lua_Debug}
\verb|lua_getstack| fills only the private part
of this structure, for future use.
To fill the other fields of \verb|lua_Debug| with useful information,
call
\begin{verbatim}
       int lua_getinfo (lua_State *L, const char *what, lua_Debug *ar);
\end{verbatim}
\DefAPI{lua_getinfo}
This function returns 0 on error
(for instance, an invalid option in \verb|what|).
Each character in the string \verb|what|
selects some fields of \verb|ar| to be filled,
as indicated by the letter in parentheses in the definition of \verb|lua_Debug|
above:
`\verb|S|' fills in the fields \verb|source|, \verb|linedefined|,
and \verb|what|;
`\verb|l|' fills in the field \verb|currentline|, etc.
Moreover, `\verb|f|' pushes onto the stack the function that is
running at the given level.

To get information about a function that is not active (that is,
it is not in the stack),
you push the function onto the stack,
and start the \verb|what| string with the character `\verb|>|'.
For instance, to know in which line a function \verb|f| was defined,
you can write
\begin{verbatim}
       lua_Debug ar;
       lua_pushstring(L, "f");
       lua_gettable(L, LUA_GLOBALSINDEX);  /* get global `f' */
       lua_getinfo(L, ">S", &ar);
       printf("%d\n", ar.linedefined);
\end{verbatim}
The fields of \verb|lua_Debug| have the following meaning:
\begin{description}\leftskip=20pt

\item[source]
If the function was defined in a string,
then \verb|source| is that string;
if the function was defined in a file,
then \verb|source| starts with a \verb|@| followed by the file name.

\item[short\_src]
A ``printable'' version of \verb|source|, to be used in error messages.

\item[linedefined]
the line number where the definition of the function starts.

\item[what] the string \verb|"Lua"| if this is a Lua function,
\verb|"C"| if this is a C~function,
or \verb|"main"| if this is the main part of a chunk.

\item[currentline]
the current line where the given function is executing.
When no line information is available,
\verb|currentline| is set to \Math{-1}.

\item[name]
a reasonable name for the given function.
Because functions in Lua are first class values,
they do not have a fixed name:
Some functions may be the value of many global variables,
while others may be stored only in a table field.
The \verb|lua_getinfo| function checks whether the given
function is a tag method or the value of a global variable.
If the given function is a tag method,
then \verb|name| points to the event name.
%% TODO: mas qual o tag? Agora que temos tipos com nome, seria util saber
%% o tipo de TM. Em particular para mensagens de erro.
If the given function is the value of a global variable,
then \verb|name| points to the variable name.
If the given function is neither a tag method nor a global variable,
then \verb|name| is set to \verb|NULL|.

\item[namewhat]
Explains the previous field.
It can be \verb|"global"|, \verb|"local"|, \verb|"method"|,
\verb|"field"|, or \verb|""| (the empty string),
according to how the function was called.
(Lua uses the empty string when no other option seems to apply.)

\item[nups]
Number of upvalues of the function.

\end{description}


\subsection{Manipulating Local Variables}

For the manipulation of local variables,
\verb|luadebug.h| uses indices:
The first parameter or local variable has index~1, and so on,
until the last active local variable.

The following functions allow the manipulation of the
local variables of a given activation record:
\begin{verbatim}
       const char *lua_getlocal (lua_State *L, const lua_Debug *ar, int n);
       const char *lua_setlocal (lua_State *L, const lua_Debug *ar, int n);
\end{verbatim}
\DefAPI{lua_getlocal}\DefAPI{lua_setlocal}
The parameter \verb|ar| must be a valid activation record,
filled by a previous call to \verb|lua_getstack| or
given as argument to a hook \see{sub-hooks}.
\verb|lua_getlocal| gets the index \verb|n| of a local variable,
pushes its value onto the stack,
and returns its name.
%% TODO: why return name?
\verb|lua_setlocal| assigns the value at the top of the stack
to the variable and returns its name.
Both functions return \verb|NULL| on failure,
that is
when the index is greater than
the number of active local variables.

As an example, the following function lists the names of all
local variables for a function at a given level of the stack:
\begin{verbatim}
       int listvars (lua_State *L, int level) {
         lua_Debug ar;
         int i = 1;
         const char *name;
         if (lua_getstack(L, level, &ar) == 0)
           return 0;  /* failure: no such level in the stack */
         while ((name = lua_getlocal(L, &ar, i++)) != NULL) {
           printf("%s\n", name);
           lua_pop(L, 1);  /* remove variable value */
         }
         return 1;
       }
\end{verbatim}


\subsection{Hooks}\label{sub-hooks}

The Lua interpreter offers two hooks for debugging purposes:
a \emph{call} hook and a \emph{line} hook.
Both have type \verb|lua_Hook|, defined as follows:
\begin{verbatim}
       typedef void (*lua_Hook) (lua_State *L, lua_Debug *ar);
\end{verbatim}
\DefAPI{lua_Hook}
You can set the hooks with the following functions:
\begin{verbatim}
       lua_Hook lua_setcallhook (lua_State *L, lua_Hook func);
       lua_Hook lua_setlinehook (lua_State *L, lua_Hook func);
\end{verbatim}
\DefAPI{lua_setcallhook}\DefAPI{lua_setlinehook}
A hook is disabled when its value is \verb|NULL|,
which is the initial value of both hooks.
The functions \verb|lua_setcallhook| and \verb|lua_setlinehook|
set their corresponding hooks and return their previous values.

The call hook is called whenever the
interpreter enters or leaves a function.
The \verb|event| field of \verb|ar| has the string \verb|"call"|
or \verb|"return"|.
This \verb|ar| can then be used in calls to \verb|lua_getinfo|,
\verb|lua_getlocal|, and \verb|lua_setlocal|
to get more information about the function and to manipulate its
local variables.

The line hook is called every time the interpreter changes
the line of code it is executing.
The \verb|event| field of \verb|ar| has the string \verb|"line"|,
and the \verb|currentline| field has the new line number.
Again, you can use this \verb|ar| in other calls to the debug API.

While Lua is running a hook, it disables other calls to hooks.
Therefore, if a hook calls Lua to execute a function or a chunk,
this execution ocurrs without any calls to hooks.


%------------------------------------------------------------------------------
\section{Standard Libraries}\label{libraries}

The standard libraries provide useful functions
that are implemented directly through the standard C~API.
Some of these functions provide essential services to the language
(e.g. \verb|type| and \verb|getmetatable|);
others provide access to ``outside'' servides (e.g. I/O);
and others could be implemented in Lua itself,
but are quite useful or have critical performance to
deserve an implementation in C (e.g. \verb|sort|).

All libraries are implemented through the official C API,
and are provided as separate C~modules.
Currently, Lua has the following standard libraries:
\begin{itemize}
\item basic library;
\item string manipulation;
\item table manipulation;
\item mathematical functions (sin, log, etc.);
\item input and output;
\item operating system facilities;
\item debug facilities.
\end{itemize}
Except for the basic library,
each library provides all its functions as fields of a global table
or as methods of its objects.

To have access to these libraries,
the C~host program must call the functions
\verb|lua_baselibopen|,
\verb|lua_strlibopen|,
\verb|lua_tablibopen|,
\verb|lua_mathlibopen|,
and \verb|lua_iolibopen|, which are declared in \verb|lualib.h|.
\DefAPI{lua_baselibopen}
\DefAPI{lua_strlibopen}
\DefAPI{lua_tablibopen}
\DefAPI{lua_mathlibopen}
\DefAPI{lua_iolibopen}


\subsection{Basic Functions} \label{predefined}

The basic library provides some core functions to Lua.
If you do not include this library in your application,
you should check carefully whether you need to provide some alternative
implementation for some facilities.

The basic library also defines a global variable \IndexAPI{_VERSION}
with a string containing the current interpreter version.
The current content of this string is {\tt "Lua \Version"}.

\subsubsection*{\ff \T{assert (v [, message])}}\DefLIB{assert}
Issues an \emph{``assertion failed!''} error
when its argument \verb|v| is \nil;
otherwise, returns this argument.
This function is equivalent to the following Lua function:
\begin{verbatim}
       function assert (v, m)
         if not v then
           error(m or "assertion failed!")
         end
         return v
       end
\end{verbatim}

??\subsubsection*{\ff \T{call (func, arg [, mode [, errhandler]])}}\DefLIB{call}
\label{pdf-call}
Calls function \verb|func| with
the arguments given by the table \verb|arg|.
The call is equivalent to
\begin{verbatim}
       func(arg[1], arg[2], ..., arg[n])
\end{verbatim}
where \verb|n| is the result of \verb|getn(arg)| \see{getn}.
All results from \verb|func| are simply returned by \verb|call|.

By default,
if an error occurs during the call to \verb|func|,
the error is propagated.
If the string \verb|mode| contains \verb|"x"|,
then the call is \emph{protected}.\index{protected calls}
In this mode, function \verb|call| does not propagate an error,
regardless of what happens during the call.
Instead, it returns \nil{} to signal the error
(besides calling the appropriated error handler).

If \verb|errhandler| is provided,
the error function \verb|_ERRORMESSAGE| is temporarily set to \verb|errhandler|,
while \verb|func| runs.
In particular, if \verb|errhandler| is \nil,
no error messages will be issued during the execution of the called function.

\subsubsection*{\ff \T{collectgarbage ([limit])}}\DefLIB{collectgarbage}

Sets the garbage-collection threshold for the given limit
(in Kbytes), and checks it against the byte counter.
If the new threshold is smaller than the byte counter,
then Lua immediately runs the garbage collector \see{GC}.
If \verb|limit| is absent, it defaults to zero
(thus forcing a garbage-collection cycle).

\subsubsection*{\ff \T{dofile (filename)}}\DefLIB{dofile}
Receives a file name,
opens the named file, and executes its contents as a Lua chunk.
When called without arguments,
\verb|dofile| executes the contents of the standard input (\verb|stdin|).
Returns any value returned by the chunk.

\subsubsection*{\ff \T{error ([message])}}\DefLIB{error}\label{pdf-error}
Terminates the last protected function called,
and returns \verb|message| as the error message.
Function \verb|error| never returns.

\subsubsection*{\ff \T{getglobals (function)}}\DefLIB{getglobals}
Returns the current table of globals in use by the function.
\verb|function| can be a Lua function or a number,
meaning the function at that stack level:
Level 1 is the function calling \verb|getglobals|.
If the given function is not a Lua function,
returns the ``global'' table of globals.
The default for \verb|function| is 1.

\subsubsection*{\ff \T{getmetatable (object)}}
\DefLIB{getmetatable}\label{pdf-getmetatable}

Returns the metatable of the given object.
If the object does not have a metatable, returns \nil.

\subsubsection*{\ff \T{getmode (table)}}\DefLIB{getmode}

Returns the weak mode of a table, as a string.
Valid values for this string are \verb|""| for regular (non-weak) tables,
\verb|"k"| for weak keys, \verb|"v"| for weak values,
and \verb|"kv"| for both.

\subsubsection*{\ff \T{gcinfo ()}}\DefLIB{gcinfo}
Returns the number of Kbytes of dynamic memory Lua is using,
and (as a second result) the
current garbage collector threshold (also in Kbytes).

\subsubsection*{\ff \T{loadfile (filename)}}\DefLIB{loadfile}
Loads a file as a Lua chunk.
If there is no errors, 
returns the compiled chunk as a function;
otherwise, returns \nil{} plus an error message.

\subsubsection*{\ff \T{loadstring (string [, chunkname])}}\DefLIB{loadstring}
Loads a string as a Lua chunk.
If there is no errors, 
returns the compiled chunk as a function;
otherwise, returns \nil{} plus an error message.

The optional parameter \verb|chunkname|
is the ``name of the chunk'',
used in error messages and debug information.

To load and run a given string, use the idiom
\begin{verbatim}
      assert(loadstring(s))()
\end{verbatim}

\subsubsection*{\ff \T{next (table, [index])}}\DefLIB{next}
Allows a program to traverse all fields of a table.
Its first argument is a table and its second argument
is an index in this table.
\verb|next| returns the next index of the table and the
value associated with the index.
When called with \nil{} as its second argument,
\verb|next| returns the first index
of the table and its associated value.
When called with the last index,
or with \nil{} in an empty table,
\verb|next| returns \nil.
If the second argument is absent, then it is interpreted as \nil.

Lua has no declaration of fields;
semantically, there is no difference between a
field not present in a table or a field with value \nil.
Therefore, \verb|next| only considers fields with non-\nil{} values.
The order in which the indices are enumerated is not specified,
\emph{even for numeric indices}
(to traverse a table in numeric order,
use a numerical \rwd{for} or the function \verb|ipairs|).

The behavior of \verb|next| is \emph{undefined} if you change
the table during the traversal.

\subsubsection*{\ff \T{print (e1, e2, ...)}}\DefLIB{print}
Receives any number of arguments,
and prints their values in \verb|stdout|,
using the strings returned by \verb|tostring|.
This function is not intended for formatted output,
but only as a quick way to show a value,
typically for debugging.
For formatted output, see \verb|format| \see{format}.

\subsubsection*{\ff \T{rawget (table, index)}}\DefLIB{rawget}
Gets the real value of \verb|table[index]|,
without invoking any tag method.
\verb|table| must be a table;
\verb|index| is any value different from \nil.

\subsubsection*{\ff \T{rawset (table, index, value)}}\DefLIB{rawset}
Sets the real value of \verb|table[index]| to \verb|value|,
without invoking any tag method.
\verb|table| must be a table;
\verb|index| is any value different from \nil;
and \verb|value| is any Lua value.

\subsubsection*{\ff \T{require (packagename)}}\DefLIB{require}

Loads the given package.
The function starts by looking into the table \IndexVerb{_LOADED}
whether \verb|packagename| is already loaded.
If it is, then \verb|require| is done.
Otherwise, it searches a path looking for a file to load.

If the global variable \IndexVerb{LUA_PATH} is a string, 
this string is the path.
Otherwise, \verb|require| tries the environment variable \verb|LUA_PATH|.
In the last resort, it uses a predefined path.

The path is a sequence of \emph{templates} separated by semicolons.
For each template, \verb|require| will change an eventual interrogation
mark in the template to \verb|packagename|,
and then will try to load the resulting file name.
So, for instance, if the path is
\begin{verbatim}
  "./?.lua;./?.lc;/usr/local/?/init.lua;/lasttry"
\end{verbatim}
a \verb|require "mod"| will try to load the files
\verb|./mod.lua|,
\verb|./mod.lc|,
\verb|/usr/local/mod/init.lua|,
and \verb|/lasttry|, in that order.

The function stops the search as soon as it can load a file,
and then it runs the file.
If there is any error loading or running the file,
or if it cannot find any file in the path,
then \verb|require| signals an error. 
Otherwise, it marks in table \verb|_LOADED|
that the package is loaded, and returns.

While running a packaged file,
\verb|require| defines the global variable \IndexVerb{_REQUIREDNAME}
with the package name.

\subsubsection*{\ff \T{setglobals (function, table)}}\DefLIB{setglobals}
Sets the current table of globals to be used by the given function.
\verb|function| can be a Lua function or a number,
meaning the function at that stack level:
Level 1 is the function calling \verb|setglobals|.

\subsubsection*{\ff \T{setmetatable (table, metatable)}}\DefLIB{setmetatable}

Sets the metatable for the given table.
(You cannot change the metatable of a userdata from Lua.)
If \verb|metatable| is \nil, removes the metatable of the given table.

\subsubsection*{\ff \T{setmode (table, mode)}}\DefLIB{setmode}

Set the weak mode of a table.
The new mode is described by the \verb|mode| string.
Valid values for this string are \verb|""| for regular (non-weak) tables,
\verb|"k"| for weak keys, \verb|"v"| for weak values,
and \verb|"kv"| for both.

This function returns its first argument (\verb|table|).

\subsubsection*{\ff \T{tonumber (e [, base])}}\DefLIB{tonumber}
Tries to convert its argument to a number.
If the argument is already a number or a string convertible
to a number, then \verb|tonumber| returns that number;
otherwise, it returns \nil.

An optional argument specifies the base to interpret the numeral.
The base may be any integer between 2 and 36, inclusive.
In bases above~10, the letter `A' (in either upper or lower case)
represents~10, `B' represents~11, and so forth, with `Z' representing 35.
In base 10 (the default), the number may have a decimal part,
as well as an optional exponent part \see{coercion}.
In other bases, only unsigned integers are accepted.

\subsubsection*{\ff \T{tostring (e)}}\DefLIB{tostring}
Receives an argument of any type and
converts it to a string in a reasonable format.
For complete control of how numbers are converted,
use \verb|format| \see{format}.

\subsubsection*{\ff \T{type (v)}}\DefLIB{type}\label{pdf-type}
Returns the type of its only argument, coded as a string.
The possible results of this function are
\verb|"nil"| (a string, not the value \nil),
\verb|"number"|,
\verb|"string"|,
\verb|"table"|,
\verb|"function"|,
and \verb|"userdata"|.

\subsubsection*{\ff \T{unpack (list)}}\DefLIB{unpack}
Returns all elements from the given list.
This function is equivalent to
\begin{verbatim}
  return list[1], list[2], ..., list[n]
\end{verbatim}
except that the above code can be valid only for a fixed \M{n}.
The number \M{n} of returned values
is either the value of \verb|list.n|, if it is a number,
or one less the index of the first absent (\nil) value.

\subsection{String Manipulation}
This library provides generic functions for string manipulation,
such as finding and extracting substrings and pattern matching.
When indexing a string in Lua, the first character is at position~1
(not at~0, as in C).
Indices are allowed to be negative and are interpreted as indexing backwards,
from the end of the string.
Thus, the last character is at position \Math{-1}, and so on.

The string library provides all its functions inside the table
\DefLIB{string}.

\subsubsection*{\ff \T{string.byte (s [, i])}}\DefLIB{string.byte}
Returns the internal numerical code of the \M{i}-th character of \verb|s|.
If \verb|i| is absent, then it is assumed to be~1.
\verb|i| may be negative.

\NOTE
Numerical codes are not necessarily portable across platforms.

\subsubsection*{\ff \T{string.char (i1, i2, \ldots)}}\DefLIB{string.char}
Receives 0 or more integers.
Returns a string with length equal to the number of arguments,
in which each character has the internal numerical code equal
to its correspondent argument.

\NOTE
Numerical codes are not necessarily portable across platforms.

\subsubsection*{\ff \T{string.find (s, pattern [, init [, plain]])}}
\DefLIB{string.find}
Looks for the first \emph{match} of
\verb|pattern| in the string \verb|s|.
If it finds one, then \verb|find| returns the indices of \verb|s|
where this occurrence starts and ends;
otherwise, it returns \nil.
If the pattern specifies captures (see \verb|string.gsub| below),
the captured strings are returned as extra results.
A third, optional numerical argument \verb|init| specifies
where to start the search;
its default value is~1, and may be negative.
A value of \True{} as a fourth, optional argument \verb|plain|
turns off the pattern matching facilities,
so the function does a plain ``find substring'' operation,
with no characters in \verb|pattern| being considered ``magic''.
Note that if \verb|plain| is given, then \verb|init| must be given too.

\subsubsection*{\ff \T{string.len (s)}}\DefLIB{string.len}
Receives a string and returns its length.
The empty string \verb|""| has length 0.
Embedded zeros are counted,
and so \verb|"a\000b\000c"| has length 5.

\subsubsection*{\ff \T{string.lower (s)}}\DefLIB{string.lower}
Receives a string and returns a copy of that string with all
uppercase letters changed to lowercase.
All other characters are left unchanged.
The definition of what is an uppercase letter depends on the current locale.

\subsubsection*{\ff \T{string.rep (s, n)}}\DefLIB{string.rep}
Returns a string that is the concatenation of \verb|n| copies of
the string \verb|s|.

\subsubsection*{\ff \T{string.sub (s, i [, j])}}\DefLIB{string.sub}
Returns another string, which is a substring of \verb|s|,
starting at \verb|i|  and running until \verb|j|;
\verb|i| and \verb|j| may be negative.
If \verb|j| is absent, then it is assumed to be equal to \Math{-1}
(which is the same as the string length).
In particular,
the call \verb|string.sub(s,1,j)| returns a prefix of \verb|s|
with length \verb|j|,
and the call \verb|string.sub(s, -i)| returns a suffix of \verb|s|
with length \verb|i|.

\subsubsection*{\ff \T{string.upper (s)}}\DefLIB{string.upper}
Receives a string and returns a copy of that string with all
lowercase letters changed to uppercase.
All other characters are left unchanged.
The definition of what is a lowercase letter depends on the current locale.

\subsubsection*{\ff \T{string.format (formatstring, e1, e2, \ldots)}}
\DefLIB{string.format}\label{format}
Returns a formatted version of its variable number of arguments
following the description given in its first argument (which must be a string).
The format string follows the same rules as the \verb|printf| family of
standard C~functions.
The only differences are that the options/modifiers
\verb|*|, \verb|l|, \verb|L|, \verb|n|, \verb|p|,
and \verb|h| are not supported,
and there is an extra option, \verb|q|.
The \verb|q| option formats a string in a form suitable to be safely read
back by the Lua interpreter:
The string is written between double quotes,
and all double quotes, returns, and backslashes in the string
are correctly escaped when written.
For instance, the call
\begin{verbatim}
       string.format('%q', 'a string with "quotes" and \n new line')
\end{verbatim}
will produce the string:
\begin{verbatim}
"a string with \"quotes\" and \
 new line"
\end{verbatim}

The options \verb|c|, \verb|d|, \verb|E|, \verb|e|, \verb|f|,
\verb|g|, \verb|G|, \verb|i|, \verb|o|, \verb|u|, \verb|X|, and \verb|x| all
expect a number as argument,
whereas \verb|q| and \verb|s| expect a string.
The \verb|*| modifier can be simulated by building
the appropriate format string.
For example, \verb|"%*g"| can be simulated with
\verb|"%"..width.."g"|.

\NOTE
String values to be formatted with
\verb|%s| cannot contain embedded zeros.

\subsubsection*{\ff \T{string.gsub (s, pat, repl [, n])}}
\DefLIB{string.gsub}
Returns a copy of \verb|s|
in which all occurrences of the pattern \verb|pat| have been
replaced by a replacement string specified by \verb|repl|.
\verb|gsub| also returns, as a second value,
the total number of substitutions made.

If \verb|repl| is a string, then its value is used for replacement.
Any sequence in \verb|repl| of the form \verb|%|\M{n},
with \M{n} between 1 and 9,
stands for the value of the \M{n}-th captured substring.

If \verb|repl| is a function, then this function is called every time a
match occurs, with all captured substrings passed as arguments,
in order (see below);
if the pattern specifies no captures,
then the whole match is passed as a sole argument.
If the value returned by this function is a string,
then it is used as the replacement string;
otherwise, the replacement string is the empty string.

The last, optional parameter \verb|n| limits
the maximum number of substitutions to occur.
For instance, when \verb|n| is 1 only the first occurrence of
\verb|pat| is replaced.

Here are some examples:
\begin{verbatim}
   x = gsub("hello world", "(%w+)", "%1 %1")
   --> x="hello hello world world"

   x = gsub("hello world", "(%w+)", "%1 %1", 1)
   --> x="hello hello world"

   x = gsub("hello world from Lua", "(%w+)%s*(%w+)", "%2 %1")
   --> x="world hello Lua from"

   x = gsub("home = $HOME, user = $USER", "%$(%w+)", getenv)
   --> x="home = /home/roberto, user = roberto"  (for instance)

   x = gsub("4+5 = $return 4+5$", "%$(.-)%$", dostring)
   --> x="4+5 = 9"

   local t = {name="Lua", version="4.1"}
   x = gsub("$name - $version", "%$(%w+)", function (v) return t[v] end)
   --> x="Lua - 4.1"
\end{verbatim}


\subsubsection*{Patterns} \label{pm}

\paragraph{Character Class:}
a \Def{character class} is used to represent a set of characters.
The following combinations are allowed in describing a character class:
\begin{description}\leftskip=20pt
\item[\emph{x}] (where \emph{x} is not one of the magic characters
\verb|^$()%.[]*+-?|)
--- represents the character \emph{x} itself.
\item[\T{.}] --- (a dot) represents all characters.
\item[\T{\%a}] --- represents all letters.
\item[\T{\%c}] --- represents all control characters.
\item[\T{\%d}] --- represents all digits.
\item[\T{\%l}] --- represents all lowercase letters.
\item[\T{\%p}] --- represents all punctuation characters.
\item[\T{\%s}] --- represents all space characters.
\item[\T{\%u}] --- represents all uppercase letters.
\item[\T{\%w}] --- represents all alphanumeric characters.
\item[\T{\%x}] --- represents all hexadecimal digits.
\item[\T{\%z}] --- represents the character with representation 0.
\item[\T{\%\M{x}}] (where \M{x} is any non-alphanumeric character)  ---
represents the character \M{x}.
This is the standard way to escape the magic characters.
We recommend that any punctuation character (even the non magic)
should be preceded by a \verb|%|
when used to represent itself in a pattern.

\item[\T{[\M{set}]}] ---
represents the class which is the union of all
characters in \M{set}.
A range of characters may be specified by
separating the end characters of the range with a \verb|-|.
All classes \verb|%|\emph{x} described above may also be used as
components in \M{set}.
All other characters in \M{set} represent themselves.
For example, \verb|[%w_]| (or \verb|[_%w]|)
represents all alphanumeric characters plus the underscore,
\verb|[0-7]| represents the octal digits,
and \verb|[0-7%l%-]| represents the octal digits plus
the lowercase letters plus the \verb|-| character.

The interaction between ranges and classes is not defined.
Therefore, patterns like \verb|[%a-z]| or \verb|[a-%%]|
have no meaning.

\item[\T{[\^\null\M{set}]}] ---
represents the complement of \M{set},
where \M{set} is interpreted as above.
\end{description}
For all classes represented by single letters (\verb|%a|, \verb|%c|, \ldots),
the corresponding uppercase letter represents the complement of the class.
For instance, \verb|%S| represents all non-space characters.

The definitions of letter, space, etc.\ depend on the current locale.
In particular, the class \verb|[a-z]| may not be equivalent to \verb|%l|.
The second form should be preferred for portability.

\paragraph{Pattern Item:}
a \Def{pattern item} may be
\begin{itemize}
\item
a single character class,
which matches any single character in the class;
\item
a single character class followed by \verb|*|,
which matches 0 or more repetitions of characters in the class.
These repetition items will always match the longest possible sequence;
\item
a single character class followed by \verb|+|,
which matches 1 or more repetitions of characters in the class.
These repetition items will always match the longest possible sequence;
\item
a single character class followed by \verb|-|,
which also matches 0 or more repetitions of characters in the class.
Unlike \verb|*|,
these repetition items will always match the \emph{shortest} possible sequence;
\item
a single character class followed by \verb|?|,
which matches 0 or 1 occurrence of a character in the class;
\item
\T{\%\M{n}}, for \M{n} between 1 and 9;
such item matches a substring equal to the \M{n}-th captured string
(see below);
\item
\T{\%b\M{xy}}, where \M{x} and \M{y} are two distinct characters;
such item matches strings that start with~\M{x}, end with~\M{y},
and where the \M{x} and \M{y} are \emph{balanced}.
This means that, if one reads the string from left to right,
counting \Math{+1} for an \M{x} and \Math{-1} for a \M{y},
the ending \M{y} is the first \M{y} where the count reaches 0.
For instance, the item \verb|%b()| matches expressions with
balanced parentheses.
\end{itemize}

\paragraph{Pattern:}
a \Def{pattern} is a sequence of pattern items.
A \verb|^| at the beginning of a pattern anchors the match at the
beginning of the subject string.
A \verb|$| at the end of a pattern anchors the match at the
end of the subject string.
At other positions,
\verb|^| and \verb|$| have no special meaning and represent themselves.

\paragraph{Captures:}
A pattern may contain sub-patterns enclosed in parentheses;
they describe \Def{captures}.
When a match succeeds, the substrings of the subject string
that match captures are stored (\emph{captured}) for future use.
Captures are numbered according to their left parentheses.
For instance, in the pattern \verb|"(a*(.)%w(%s*))"|,
the part of the string matching \verb|"a*(.)%w(%s*)"| is
stored as the first capture (and therefore has number~1);
the character matching \verb|.| is captured with number~2,
and the part matching \verb|%s*| has number~3.

\NOTE
A pattern cannot contain embedded zeros.  Use \verb|%z| instead.


\subsection{Table Manipulation}
This library provides generic functions for table manipulation,
It provides all its functions inside the table \DefLIB{table}.

Most functions in the table library library assume that the table
represents an array or a list.
For those functions, an important concept is the \emph{size} of the array.
There are three ways to specify that size:
\begin{itemize}
\item the field \verb|"n"| ---
When the table has a field \verb|"n"| with a numerical value,
that value is assumed as its size.
\item \verb|setn| ---
You can call the \verb|table.setn| function to explicitly set
the size of a table.
\item implicit size ---
%% TODO
\end{itemize}
For more details, see the descriptions of the \verb|table.getn| and
\verb|table.setn| functions.

\subsubsection*{\ff \T{table.foreach (table, func)}}\DefLIB{table.foreach}
Executes the given \verb|func| over all elements of \verb|table|.
For each element, \verb|func| is called with the index and
respective value as arguments.
If \verb|func| returns a non-\nil{} value,
then the loop is broken, and this value is returned
as the final value of \verb|foreach|.

The behavior of \verb|foreach| is \emph{undefined} if you change
the table \verb|t| during the traversal.


\subsubsection*{\ff \T{table.foreachi (table, func)}}\DefLIB{table.foreachi}
Executes the given \verb|func| over the
numerical indices of \verb|table|.
For each index, \verb|func| is called with the index and
respective value as arguments.
Indices are visited in sequential order,
from~1 to \verb|n|,
where \verb|n| is the size of the table \see{getn}.
If \verb|func| returns a non-\nil{} value,
then the loop is broken, and this value is returned
as the final value of \verb|foreachi|.

\subsubsection*{\ff \T{table.getn (table)}}\DefLIB{table.getn}\label{getn}
Returns the ``size'' of a table, when seen as a list.
If the table has an \verb|n| field with a numeric value,
this value is the ``size'' of the table.
Otherwise, if there was a previous call to
\verb|table.getn| or to \verb|table.setn| over this table,
the respective value is returned.
Otherwise, the ``size'' is one less the first integer index with
a \nil{} value.

Notice that the last option happens only once for a table.
If you call \verb|table.getn| again over the same table,
it will return the same previous result,
even if the table has been modified.
The only way to change the value of \verb|table.getn| is by calling
\verb|table.setn| or assigning to field \verb|"n"| in the table.

\subsubsection*{\ff \T{table.sort (table [, comp])}}\DefLIB{table.sort}
Sorts table elements in a given order, \emph{in-place},
from \verb|table[1]| to \verb|table[n]|,
where \verb|n| is the size of the table \see{getn}.
If \verb|comp| is given,
then it must be a function that receives two table elements,
and returns true
when the first is less than the second
(so that \verb|not comp(a[i+1],a[i])| will be true after the sort).
If \verb|comp| is not given,
then the standard Lua operator \verb|<| is used instead.

The sort algorithm is \emph{not} stable
(that is, elements considered equal by the given order
may have their relative positions changed by the sort).

\subsubsection*{\ff \T{table.insert (table, [pos,] value)}}\DefLIB{table.insert}

Inserts element \verb|value| at position \verb|pos| in \verb|table|,
shifting other elements up to open space, if necessary.
The default value for \verb|pos| is \verb|n+1|,
where \verb|n| is the size of the table \see{getn},
so that a call \verb|table.insert(t,x)| inserts \verb|x| at the end
of table \verb|t|.
This function also updates the size of the table,
calling \verb|table.setn(table, n+1)|.

\subsubsection*{\ff \T{table.remove (table [, pos])}}\DefLIB{table.remove}

Removes from \verb|table| the element at position \verb|pos|,
shifting other elements down to close the space, if necessary.
Returns the value of the removed element.
The default value for \verb|pos| is \verb|n|,
where \verb|n| is the size of the table \see{getn},
so that a call \verb|tremove(t)| removes the last element
of table \verb|t|.
This function also updates the size of the table,
calling \verb|table.setn(table, n-1)|.

\subsubsection*{\ff \T{table.setn (table, n)}}\DefLIB{table.setn}

Updates the ``size'' of a table.
If the table has a field \verb|"n"| with a numerical value,
that value is changed to the given \verb|n|.
Otherwise, it updates an internal state of the \verb|table| library
so that subsequent calls to \verb|table.getn(table)| return \verb|n|.


\subsection{Mathematical Functions} \label{mathlib}

This library is an interface to most functions of the standard C~math library.
(Some have slightly different names.)
It provides all its functions inside the table \DefLIB{math}.
In addition,
it registers a ??tag method for the binary exponentiation operator \verb|^|
that returns \Math{x^y} when applied to numbers \verb|x^y|.

The library provides the following functions:
\DefLIB{math.abs}\DefLIB{math.acos}\DefLIB{math.asin}\DefLIB{math.atan}
\DefLIB{math.atan2}\DefLIB{math.ceil}\DefLIB{math.cos}
\DefLIB{math.def}\DefLIB{math.exp}
\DefLIB{math.floor}\DefLIB{math.log}\DefLIB{math.log10}
\DefLIB{math.max}\DefLIB{math.min}
\DefLIB{math.mod}\DefLIB{math.rad}\DefLIB{math.sin}
\DefLIB{math.sqrt}\DefLIB{math.tan}
\DefLIB{math.frexp}\DefLIB{math.ldexp}\DefLIB{math.random}
\DefLIB{math.randomseed}
\begin{verbatim}
       math.abs   math.acos   math.asin  math.atan math.atan2
       math.ceil  math.cos    math.deg   math.exp  math.floor
       math.log   math.log10  math.max   math.min  math.mod
       math.rad   math.sin    math.sqrt  math.tan  math.frexp
       math.ldexp math.random math.randomseed
\end{verbatim}
plus a variable \IndexLIB{math.pi}.
Most of them
are only interfaces to the homonymous functions in the C~library,
except that, for the trigonometric functions,
all angles are expressed in \emph{degrees}, not radians.
The functions \verb|math.deg| and \verb|math.rad| can be used to convert
between radians and degrees.

The function \verb|math.max| returns the maximum
value of its numeric arguments.
Similarly, \verb|math.min| computes the minimum.
Both can be used with 1, 2, or more arguments.

The functions \verb|math.random| and \verb|math.randomseed|
are interfaces to the simple random generator functions
\verb|rand| and \verb|srand|, provided by ANSI~C.
(No guarantees can be given for their statistical properties.)
When called without arguments,
\verb|math.random| returns a pseudo-random real number
in the range \Math{[0,1)}.
When called with a number \Math{n},
\verb|math.random| returns a pseudo-random integer in the range \Math{[1,n]}.
When called with two arguments, \Math{l} and \Math{u},
\verb|math.random| returns a pseudo-random integer in the range \Math{[l,u]}.


\subsection{Input and Output Facilities} \label{libio}

The I/O library provides two different styles for file manipulation.
The first one uses implicit file descriptors;
that is, there are operations to set a default input file and a
default output file,
and all input/output operations are over those default files.
The second style uses explicit file descriptors.

When using implicit file descriptors,
all operations are supplied by table \DefLIB{io}.
When using explicit file descriptors,
the operation \DefLIB{io.open} returns a file descriptor,
and then all operations are supplied as methods by the file descriptor.

Moreover, the table \verb|io| also provides
three predefined file descriptors:
\DefLIB{io.stdin}, \DefLIB{io.stdout}, and \DefLIB{io.stderr},
with their usual meaning from C.

A file handle is a userdata containing the file stream (\verb|FILE*|),
with a distinctive metatable created by the I/O library.

Unless otherwise stated,
all I/O functions return \nil{} on failure
(plus an error message as a second result)
and some value different from \nil{} on success.

\subsubsection*{\ff \T{io.close ([handle])}}\DefLIB{io.close}

Equivalent to \verb|fh:close| over the default output file.

\subsubsection*{\ff \T{io.flush ()}}\DefLIB{io.flush}

Equivalent to \verb|fh:flush| over the default output file.

\subsubsection*{\ff \T{io.input ([file])}}\DefLIB{io.input}

When called with a file name, it opens the named file (in text mode),
and sets its handle as the default input file
(and returns nothing).
When called with a file handle,
it simply sets that file handle as the default input file.
When called without parameters,
it returns the current default input file.

In case of errors this function raises the error,
instead of returning an error code.

\subsubsection*{\ff \T{io.open (filename, mode)}}\DefLIB{io.open}

This function opens a file,
in the mode specified in the string \verb|mode|.
It returns a new file handle,
or, in case of errors, \nil{} plus an error message.

The \verb|mode| string can be any of the following:
\begin{description}\leftskip=20pt
\item[``r''] read mode;
\item[``w''] write mode;
\item[``a''] append mode;
\item[``r+''] update mode, all previous data is preserved;
\item[``w+''] update mode, all previous data is erased;
\item[``a+''] append update mode, previous data is preserved,
  writing is only allowed at the end of file.
\end{description}
The \verb|mode| string may also have a \verb|b| at the end,
which is needed in some systems to open the file in binary mode.
This string is exactly what is used in the standard~C function \verb|fopen|.

\subsubsection*{\ff \T{io.output ([file])}}\DefLIB{io.output}

Similar to \verb|io.input|, but operates over the default output file.

\subsubsection*{\ff \T{io.read (format1, ...)}}\DefLIB{io.read}

Equivalent to \verb|fh:read| over the default input file.

\subsubsection*{\ff \T{io.tmpfile ()}}\DefLIB{io.tmpfile}

Returns a handle for a temporary file.
This file is open in read/write mode,
and it is automatically removed when the program ends.

\subsubsection*{\ff \T{io.write (value1, ...)}}\DefLIB{io.write}

Equivalent to \verb|fh:write| over the default output file.



\subsubsection*{\ff \T{fh:close ([handle])}}\DefLIB{fh:close}

Closes the file \verb|fh|.

\subsubsection*{\ff \T{fh:flush ()}}\DefLIB{fh:flush}

Saves any written data to the file \verb|fh|.

\subsubsection*{\ff \T{fh:read (format1, ...)}}\DefLIB{fh:read}

Reads the file \verb|fh|,
according to the given formats, which specify what to read.
For each format,
the function returns a string (or a number) with the characters read,
or \nil{} if it cannot read data with the specified format.
When called without formats,
it uses a default format that reads the entire next line
(see below).

The available formats are
\begin{description}\leftskip=20pt
\item[``*n''] reads a number;
this is the only format that returns a number instead of a string.
\item[``*a''] reads the whole file, starting at the current position.
On end of file, it returns the empty string.
\item[``*l''] reads the next line (skipping the end of line),
returning \nil{} on end of file.
This is the default format.
\item[\emph{number}] reads a string with up to that number of characters,
or \nil{} on end of file.
If number is zero,
it reads nothing and returns an empty string,
or \nil{} on end of file.
\end{description}

\subsubsection*{\ff \T{fh:seek ([whence] [, offset])}}\DefLIB{fh:seek}

Sets and gets the file position,
measured in bytes from the beginning of the file,
to the position given by \verb|offset| plus a base
specified by the string \verb|whence|, as follows:
\begin{description}\leftskip=20pt
\item[``set''] base is position 0 (beginning of the file);
\item[``cur''] base is current position;
\item[``end''] base is end of file;
\end{description}
In case of success, function \verb|seek| returns the final file position,
measured in bytes from the beginning of the file.
If this function fails, it returns \nil,
plus a string describing the error.

The default value for \verb|whence| is \verb|"cur"|,
and for \verb|offset| is 0.
Therefore, the call \verb|file:seek()| returns the current
file position, without changing it;
the call \verb|file:seek("set")| sets the position to the
beginning of the file (and returns 0);
and the call \verb|file:seek("end")| sets the position to the
end of the file, and returns its size.

\subsubsection*{\ff \T{fh:write (value1, ...)}}\DefLIB{fh:write}

Writes the value of each of its arguments to
the filehandle \verb|fh|.
The arguments must be strings or numbers.
To write other values,
use \verb|tostring| or \verb|format| before \verb|write|.
If this function fails, it returns \nil,
plus a string describing the error.


\subsection{Operating System Facilities} \label{libiosys}

This library is implemented through table \DefLIB{os}.

\subsubsection*{\ff \T{os.clock ()}}\DefLIB{os.clock}

Returns an approximation of the amount of CPU time
used by the program, in seconds.

\subsubsection*{\ff \T{os.date ([format [, time]])}}\DefLIB{os.date}

Returns a string or a table containing date and time,
formatted according to the given string \verb|format|.

If the \verb|time| argument is present,
this is the time to be formatted
(see the \verb|time| function for a description of this value).
Otherwise, \verb|date| formats the current time.

If \verb|format| starts with \verb|!|,
then the date is formatted in Coordinated Universal Time.

After that optional character,
if \verb|format| is \verb|*t|,
then \verb|date| returns a table with the following fields:
\verb|year| (four digits), \verb|month| (1--12), \verb|day| (1--31),
\verb|hour| (0--23), \verb|min| (0--59), \verb|sec| (0--61),
\verb|wday| (weekday, Sunday is 1),
\verb|yday| (day of the year),
and \verb|isdst| (daylight saving flag, a boolean).

If format is not \verb|*t|,
then \verb|date| returns the date as a string,
formatted according with the same rules as the C~function \verb|strftime|.
When called without arguments,
\verb|date| returns a reasonable date and time representation that depends on
the host system and on the current locale
(that is, \verb|os.date()| is equivalent to \verb|os.date("%c")|).

\subsubsection*{\ff \T{os.difftime (t1, t2)}}\DefLIB{os.difftime}

Returns the number of seconds from time \verb|t1| to time \verb|t2|.
In Posix, Windows, and some other systems,
this value is exactly \verb|t1|\Math{-}\verb|t2|.

\subsubsection*{\ff \T{os.execute (command)}}\DefLIB{os.execute}

This function is equivalent to the C~function \verb|system|.
It passes \verb|command| to be executed by an operating system shell.
It returns a status code, which is system-dependent.

\subsubsection*{\ff \T{os.exit ([code])}}\DefLIB{os.exit}

Calls the C~function \verb|exit|,
with an optional \verb|code|,
to terminate the host program.
The default value for \verb|code| is the success code.

\subsubsection*{\ff \T{os.getenv (varname)}}\DefLIB{os.getenv}

Returns the value of the process environment variable \verb|varname|,
or \nil{} if the variable is not defined.

\subsubsection*{\ff \T{os.remove (filename)}}\DefLIB{os.remove}

Deletes the file with the given name.
If this function fails, it returns \nil,
plus a string describing the error.

\subsubsection*{\ff \T{os.rename (name1, name2)}}\DefLIB{os.rename}

Renames file named \verb|name1| to \verb|name2|.
If this function fails, it returns \nil,
plus a string describing the error.

\subsubsection*{\ff \T{os.setlocale (locale [, category])}}\DefLIB{os.setlocale}

This function is an interface to the C~function \verb|setlocale|.
\verb|locale| is a string specifying a locale;
\verb|category| is an optional string describing which category to change:
\verb|"all"|, \verb|"collate"|, \verb|"ctype"|,
\verb|"monetary"|, \verb|"numeric"|, or \verb|"time"|;
the default category is \verb|"all"|.
The function returns the name of the new locale,
or \nil{} if the request cannot be honored.

\subsubsection*{\ff \T{os.time ([table])}}\DefLIB{os.time}

Returns the current time when called without arguments,
or a time representing the date and time specified by the given table.
This table must have fields \verb|year|, \verb|month|, and \verb|day|,
and may have fields \verb|hour|, \verb|min|, \verb|sec|, and \verb|isdst|
(for a description of these fields, see the \verb|os.date| function).

The returned value is a number, whose meaning depends on your system.
In Posix, Windows, and some other systems, this number counts the number
of seconds since some given start time (the ``epoch'').
In other systems, the meaning is not specified,
and the number returned bt \verb|time| can be used only as an argument to
\verb|date| and \verb|difftime|.

\subsubsection*{\ff \T{os.tmpname ()}}\DefLIB{os.tmpname}

Returns a string with a file name that can
be used for a temporary file.
The file must be explicitly opened before its use
and removed when no longer needed.

This function is equivalent to the \verb|tmpnam| C~function,
and many people (and even some compilers!) advise against its use,
because between the time you call the function
and the time you open the file,
it is possible for another process
to create a file with the same name.


\subsection{The Reflexive Debug Interface}

The library \verb|ldblib| provides
the functionality of the debug interface to Lua programs.
If you want to use this library,
your host application must open it,
by calling \verb|lua_dblibopen|.
\DefAPI{lua_dblibopen}

You should exert great care when using this library.
The functions provided here should be used exclusively for debugging
and similar tasks, such as profiling.
Please resist the temptation to use them as a
usual programming tool:
They can be \emph{very} slow.
Moreover, \verb|setlocal| and \verb|getlocal|
violate the privacy of local variables,
and therefore can compromise some (otherwise) secure code.


\subsubsection*{\ff \T{getinfo (function, [what])}}\DefLIB{getinfo}

This function returns a table with information about a function.
You can give the function directly,
or you can give a number as the value of \verb|function|,
which means the function running at level \verb|function| of the stack:
Level 0 is the current function (\verb|getinfo| itself);
level 1 is the function that called \verb|getinfo|;
and so on.
If \verb|function| is a number larger than the number of active functions,
then \verb|getinfo| returns \nil.

The returned table contains all the fields returned by \verb|lua_getinfo|,
with the string \verb|what| describing what to get.
The default for \verb|what| is to get all information available.
If present,
the option \verb|f|
adds a field named \verb|func| with the function itself.

For instance, the expression \verb|getinfo(1,"n").name| returns
the name of the current function, if a reasonable name can be found,
and \verb|getinfo(print)| returns a table with all available information
about the \verb|print| function.


\subsubsection*{\ff \T{getlocal (level, local)}}\DefLIB{getlocal}

This function returns the name and the value of the local variable
with index \verb|local| of the function at level \verb|level| of the stack.
(The first parameter or local variable has index~1, and so on,
until the last active local variable.)
The function returns \nil{} if there is no local
variable with the given index,
and raises an error when called with a \verb|level| out of range.
(You can call \verb|getinfo| to check whether the level is valid.)

\subsubsection*{\ff \T{setlocal (level, local, value)}}\DefLIB{setlocal}

This function assigns the value \verb|value| to the local variable
with index \verb|local| of the function at level \verb|level| of the stack.
The function returns \nil{} if there is no local
variable with the given index,
and raises an error when called with a \verb|level| out of range.
(You can call \verb|getinfo| to check whether the level is valid.)

\subsubsection*{\ff \T{setcallhook (hook)}}\DefLIB{setcallhook}

Sets the function \verb|hook| as the call hook;
this hook will be called every time the interpreter starts and
exits the execution of a function.
The only argument to the call hook is the event name (\verb|"call"| or
\verb|"return"|).
You can call \verb|getinfo| with level 2 to get more information about
the function being called or returning
(level~0 is the \verb|getinfo| function,
and level~1 is the hook function).
When called without arguments,
this function turns off call hooks.
\verb|setcallhook| returns the old call hook.

\subsubsection*{\ff \T{setlinehook (hook)}}\DefLIB{setlinehook}

Sets the function \verb|hook| as the line hook;
this hook will be called every time the interpreter changes
the line of code it is executing.
The only argument to the line hook is the line number the interpreter
is about to execute.
When called without arguments,
this function turns off line hooks.
\verb|setlinehook| returns the old line hook.


%------------------------------------------------------------------------------
\section{\Index{Lua Stand-alone}} \label{lua-sa}

Although Lua has been designed as an extension language,
to be embedded in a host C~program,
it is also frequently used as a stand-alone language.
An interpreter for Lua as a stand-alone language,
called simply \verb|lua|,
is provided with the standard distribution.
The stand-alone interpreter includes
all standard libraries plus the reflexive debug interface.
Its usage is:
\begin{verbatim}
      lua [options] [prog [args]]
\end{verbatim}
The options are:
\begin{description}\leftskip=20pt
\item[\T{-} ] executes \verb|stdin| as a file;
\item[\T{-e} \rm\emph{stat}] executes string \emph{stat};
\item[\T{-l} \rm\emph{file}] executes file \emph{file};
\item[\T{-i}] enters interactive mode after running \emph{prog};
\item[\T{-v}] prints version information;
\item[\T{--}] stop handling options.
\end{description}
After handling its options, \verb|lua| runs the given \emph{prog},
passing to it the given \emph{args}.
When called without arguments,
\verb|lua| behaves as \verb|lua -v -i| when \verb|stdin| is a terminal,
and as \verb|lua -| otherwise.

Before running any argument,
the intepreter checks for an environment variable \IndexVerb{LUA_INIT}.
If its format is \verb|@|\emph{filename},
then lua executes the file.
Otherwise, lua executes the string itself.

All options are handled in order, except \verb|-i|.
For instance, an invocation like
\begin{verbatim}
       $ lua -e'a=1' -e 'print(a)' prog.lua
\end{verbatim}
will first set \verb|a| to 1, then print \verb|a|,
and finally run the file \verb|prog.lua|.
(Here, \verb|$| is the shell prompt. Your prompt may be different.)

Before starting to run the program,
\verb|lua| collects all arguments in the command line
in a global table called \verb|arg|.
The program name is stored in index 0,
the first argument after the program goes to index 1,
and so on.
The field \verb|n| gets the number of arguments after the program name.
Any argument before the program name
(that is, the options plus the interpreter name)
goes to negative indices.
For instance, in the call
\begin{verbatim}
       $ lua -la.lua b.lua t1 t2
\end{verbatim}
the interpreter first runs the file \T{a.lua},
then creates a table
\begin{verbatim}
       arg = { [-2] = "lua", [-1] = "-la.lua", [0] = "b.lua",
               [1] = "t1", [2] = "t2"; n = 2 }
\end{verbatim}
and finally runs the file \T{b.lua}.

In interactive mode,
if you write an incomplete statement,
the interpreter waits for its completion.

If the global variable \IndexVerb{_PROMPT} is defined as a string,
then its value is used as the prompt.
Therefore, the prompt can be changed directly on the command line:
\begin{verbatim}
       $ lua -e"_PROMPT='myprompt> '" -i
\end{verbatim}
(the first pair of quotes is for the shell,
the second is for Lua),
or in any Lua programs by assigning to \verb|_PROMPT|.
Note the use of \verb|-i| to enter interactive mode; otherwise,
the program would end just after the assignment to \verb|_PROMPT|.

In Unix systems, Lua scripts can be made into executable programs
by using \verb|chmod +x| and the~\verb|#!| form,
as in \verb|#!/usr/local/bin/lua|.
(Of course,
the location of the Lua interpreter may be different in your machine.
If \verb|lua| is in your \verb|PATH|,
then a more portable solution is \verb|#!/usr/bin/env lua|.)


%------------------------------------------------------------------------------
\section*{Acknowledgments}

%% TODO rever isso?

The authors thank CENPES/PETROBRAS which,
jointly with \tecgraf, used early versions of
this system extensively and gave valuable comments.
The authors also thank Carlos Henrique Levy,
who found the name of the game.
Lua means ``moon'' in Portuguese.


\appendix

\section*{Incompatibilities with Previous Versions}
\addcontentsline{toc}{section}{Incompatibilities with Previous Versions}

We took a great care to avoid incompatibilities with
the previous public versions of Lua,
but some differences had to be introduced.
Here is a list of all these incompatibilities.


\subsection*{Incompatibilities with \Index{version 4.0}}

\subsubsection*{Changes in the Language}
\begin{itemize}

\item
Function calls written between parentheses result in exactly one value.

\item
A function call as the last expression in a list constructor
(like \verb|{a,b,f()}}|) has all its return values inserted in the list.

\item
\rwd{in} is a reserved word.

\item
When a literal string of the form \verb|[[...]]| starts with a newline,
this newline is ignored.

\item Old pre-compiled code is obsolete, and must be re-compiled.

\end{itemize}


\subsubsection*{Changes in the Libraries}
\begin{itemize}

\item
The \verb|read| option \verb|*w| is obsolete.

\item
The \verb|format| option \verb|%n$| is obsolete.

\end{itemize}


\subsubsection*{Changes in the API}
\begin{itemize}

\item
Userdata!!

\end{itemize}

%{===============================================================
\newpage
\section*{The Complete Syntax of Lua} \label{BNF}
\addcontentsline{toc}{section}{The Complete Syntax of Lua}

The notation used here is the usual extended BNF,
in which
\rep{\emph{a}}~means 0 or more \emph{a}'s, and
\opt{\emph{a}}~means an optional \emph{a}.
Non-terminals are shown in \emph{italics},
keywords are shown in {\bf bold},
and other terminal symbols are shown in {\tt typewriter} font,
enclosed in single quotes.


\renewenvironment{Produc}{\vspace{0.8ex}\par\noindent\hspace{3ex}\it\begin{tabular}{rrl}}{\end{tabular}\vspace{0.8ex}\par\noindent}

\renewcommand{\OrNL}{\\ & \Or & }
%\newcommand{\Nter}[1]{{\rm{\tt#1}}}
%\newcommand{\Nter}[1]{\ter{#1}}

\index{grammar}

\begin{Produc}

\produc{chunk}{\rep{stat \opt{\ter{;}}}}

\produc{block}{chunk}

\produc{stat}{%
	varlist1 \ter{=} explist1
\OrNL	functioncall
\OrNL	\rwd{do} block \rwd{end}
\OrNL	\rwd{while} exp \rwd{do} block \rwd{end}
\OrNL	\rwd{repeat} block \rwd{until} exp
\OrNL	\rwd{if} exp \rwd{then} block
	\rep{\rwd{elseif} exp \rwd{then} block}
	\opt{\rwd{else} block} \rwd{end}
\OrNL	\rwd{return} \opt{explist1}
\OrNL	\rwd{break}
\OrNL	\rwd{for} \Nter{Name} \ter{=} exp \ter{,} exp \opt{\ter{,} exp}
	\rwd{do} block \rwd{end}
\OrNL   \rwd{for} \Nter{Name} \rep{\ter{,} \Nter{Name}} \rwd{in} explist1
                    \rwd{do} block \rwd{end}
\OrNL	\rwd{function} funcname funcbody
\OrNL	\rwd{local} \rwd{function} \Nter{Name} funcbody
\OrNL	\rwd{local} namelist \opt{init}
}

\produc{funcname}{\Nter{Name} \rep{\ter{.} \Nter{Name}}
                              \opt{\ter{:} \Nter{Name}}}

\produc{varlist1}{var \rep{\ter{,} var}}

\produc{var}{%
	\Nter{Name}
\Or	prefixexp \ter{[} exp \ter{]}
\Or	prefixexp \ter{.} \Nter{Name}
}

\produc{namelist}{\Nter{Name} \rep{\ter{,} \Nter{Name}}}

\produc{init}{\ter{=} explist1}

\produc{explist1}{\rep{exp \ter{,}} exp}

\produc{exp}{%
	\rwd{nil}
	\rwd{false}
	\rwd{true}
\Or	\Nter{Number}
\OrNL	\Nter{Literal}
\Or	function
\Or	prefixexp
\OrNL	tableconstructor
\Or	exp binop exp
\Or	unop exp
}

\produc{prefixexp}{var \Or functioncall \Or \ter{(} exp \ter{)}}

\produc{functioncall}{%
	prefixexp args
\Or	prefixexp \ter{:} \Nter{Name} args
}

\produc{args}{%
	\ter{(} \opt{explist1} \ter{)}
\Or	tableconstructor
\Or	\Nter{Literal}
}

\produc{function}{\rwd{function} funcbody}

\produc{funcbody}{\ter{(} \opt{parlist1} \ter{)} block \rwd{end}}

\produc{parlist1}{%
	\Nter{Name} \rep{\ter{,} \Nter{Name}} \opt{\ter{,} \ter{\ldots}}
\Or	\ter{\ldots}
}

\produc{tableconstructor}{\ter{\{} \opt{fieldlist} \ter{\}}}
\produc{fieldlist}{field \rep{fieldsep field} \opt{fieldsep}}
\produc{field}{\ter{[} exp \ter{]} \ter{=} exp \Or name \ter{=} exp \Or exp}
\produc{fieldsep}{\ter{,} \Or \ter{;}}

\produc{binop}{\ter{+} \Or \ter{-} \Or \ter{*} \Or \ter{/} \Or \ter{\^{ }} \Or
  \ter{..} \Or \ter{<} \Or \ter{<=} \Or \ter{>} \Or \ter{>=}
  \Or \ter{==} \Or \ter{\~{ }=} \OrNL \rwd{and} \Or \rwd{or}}

\produc{unop}{\ter{-} \Or \rwd{not}}

\end{Produc}

%}===============================================================

% Index

\newpage
\addcontentsline{toc}{section}{Index}
% $Id: manual.tex,v 1.56 2002/06/06 12:49:28 roberto Exp roberto $

\documentclass[11pt,twoside,draft]{article}
\usepackage{fullpage}
\usepackage{bnf}
\usepackage{graphicx}

% no need for subscripts...
\catcode`\_=12

%\newcommand{\See}[1]{Section~\ref{#1}}
\newcommand{\See}[1]{\S\ref{#1}}
%\newcommand{\see}[1]{(see~\See{#1} on page \pageref{#1})}
\newcommand{\see}[1]{(see~\See{#1})}
\newcommand{\seepage}[1]{(see page~\pageref{#1})}
\newcommand{\M}[1]{{\rm\emph{#1}}}
\newcommand{\T}[1]{{\tt #1}}
\newcommand{\Math}[1]{$#1$}
\newcommand{\nil}{{\bf nil}}
\newcommand{\False}{{\bf false}}
\newcommand{\True}{{\bf true}}
%\def\tecgraf{{\sf TeC\kern-.21em\lower.7ex\hbox{Graf}}}
\def\tecgraf{{\sf Tecgraf}}

\newcommand{\Index}[1]{#1\index{#1@{\lowercase{#1}}}}
\newcommand{\IndexVerb}[1]{\T{#1}\index{#1@{\tt #1}}}
\newcommand{\IndexEmph}[1]{\emph{#1}\index{#1@{\lowercase{#1}}}}
\newcommand{\IndexTM}[1]{\index{#1 event@{``#1'' event}}\index{tag method!#1}}
\newcommand{\Def}[1]{\emph{#1}\index{#1}}
\newcommand{\IndexAPI}[1]{\T{#1}\DefAPI{#1}}
\newcommand{\IndexLIB}[1]{\T{#1}\DefLIB{#1}}
\newcommand{\DefLIB}[1]{\index{#1@{\tt #1}}}
\newcommand{\DefAPI}[1]{\index{C API!#1@{\tt #1}}}
\newcommand{\IndexKW}[1]{\index{keywords!#1@{\tt #1}}}

\newcommand{\ff}{$\bullet$\ }

\newcommand{\Version}{5.0 (alpha)}

% changes to bnf.sty by LHF
\renewcommand{\Or}{$|$ }
\renewcommand{\rep}[1]{{\rm\{}\,#1\,{\rm\}}}
\renewcommand{\opt}[1]{{\rm [}\,#1\,{\,\rm]}}
\renewcommand{\ter}[1]{{\rm`{\tt#1}'}}
\newcommand{\Nter}[1]{{\tt#1}}
\newcommand{\NOTE}{\par\medskip\noindent\emph{NOTE}: }

\makeindex

\begin{document}

%{===============================================================
\thispagestyle{empty}
\pagestyle{empty}

{
\parindent=0pt
\vglue1.5in
{\LARGE\bf
The Programming Language Lua}
\hfill
\vskip4pt \hrule height 4pt width \hsize \vskip4pt
\hfill
Reference Manual for Lua version \Version
\\
\null
\hfill
Last revised on \today
\\
\vfill
\centering
\includegraphics[width=0.7\textwidth]{nolabel.ps}
\vfill
\vskip4pt \hrule height 2pt width \hsize
}

\newpage
\begin{quotation}
\parskip=10pt
\parindent=0pt
\footnotesize
\null\vfill

\noindent
Copyright \copyright\ 2002 Tecgraf, PUC-Rio.  All rights reserved.

Permission is hereby granted, free of charge,
to any person obtaining a copy of this software
and associated documentation files (the "Software"),
to deal in the Software without restriction,
including without limitation the rights to use, copy, modify,
merge, publish, distribute, sublicense,
and/or sell copies of the Software,
and to permit persons to whom the Software is furnished to do so,
subject to the following conditions:

The above copyright notice and this permission notice shall be
included in all copies or substantial portions of the Software.

THE SOFTWARE IS PROVIDED "AS IS", WITHOUT WARRANTY OF ANY KIND,
EXPRESS OR IMPLIED,
INCLUDING BUT NOT LIMITED TO THE WARRANTIES OF MERCHANTABILITY,
FITNESS FOR A PARTICULAR PURPOSE AND NONINFRINGEMENT.
IN NO EVENT SHALL THE AUTHORS OR COPYRIGHT HOLDERS BE LIABLE
FOR ANY CLAIM, DAMAGES OR OTHER LIABILITY,
WHETHER IN AN ACTION OF CONTRACT, TORT OR OTHERWISE,
ARISING FROM, OUT OF OR IN CONNECTION WITH THE SOFTWARE
OR THE USE OR OTHER DEALINGS IN THE SOFTWARE.


Copies of this manual can be obtained at
Lua's official web site,
\verb|www.lua.org|.

\bigskip
The Lua logo was designed by A. Nakonechny.
Copyright \copyright\ 1998.  All rights reserved.
\end{quotation}
%}===============================================================
\newpage

\title{\Large\bf Reference Manual of the Programming Language Lua \Version}

\author{%
Roberto Ierusalimschy\qquad
Luiz Henrique de Figueiredo\qquad
Waldemar Celes
\vspace{1.0ex}\\
\smallskip
\small\tt lua@tecgraf.puc-rio.br
\vspace{2.0ex}\\
%MCC 08/95 ---
\tecgraf\ --- Computer Science Department --- PUC-Rio
}

%\date{{\small \tt\$Date: 2002/06/06 12:49:28 $ $}}

\maketitle

\pagestyle{plain}
\pagenumbering{roman}

\begin{abstract}
\noindent
Lua is a powerful, light-weight programming language
designed for extending applications.
Lua is also frequently used as a general-purpose, stand-alone language.
Lua combines simple procedural syntax
(similar to Pascal)
with
powerful data description constructs
based on associative arrays and extensible semantics.
Lua is
dynamically typed,
interpreted from opcodes,
and has automatic memory management with garbage collection,
making it ideal for
configuration,
scripting,
and
rapid prototyping.

This document describes version \Version\ of the Lua programming language
and the Application Program Interface (API)
that allows interaction between Lua programs and their host C~programs.
\end{abstract}

\def\abstractname{Resumo}
\begin{abstract}
\noindent
Lua \'e uma linguagem de programa\c{c}\~ao
poderosa e leve,
projetada para estender aplica\c{c}\~oes.
Lua tamb\'em \'e frequentemente usada como uma linguagem de prop\'osito geral.
Lua combina programa\c{c}\~ao procedural
(com sintaxe semelhante \`a de Pascal)
com
poderosas constru\c{c}\~oes para descri\c{c}\~ao de dados,
baseadas em tabelas associativas e sem\^antica extens\'\i vel.
Lua \'e
tipada dinamicamente,
interpretada a partir de \emph{opcodes},
e tem gerenciamento autom\'atico de mem\'oria com coleta de lixo.
Essas caracter\'{\i}sticas fazem de Lua uma linguagem ideal para
configura\c{c}\~ao,
automa\c{c}\~ao (\emph{scripting})
e prototipagem r\'apida.

Este documento descreve a vers\~ao \Version\ da linguagem de
programa\c{c}\~ao Lua e a Interface de Programa\c{c}\~ao (API) que permite
a intera\c{c}\~ao entre programas Lua e programas C~hospedeiros.
\end{abstract}

\newpage
\null
\newpage
\tableofcontents

\newpage
\setcounter{page}{1}
\pagestyle{plain}
\pagenumbering{arabic}

%------------------------------------------------------------------------------
\section{Introduction}

Lua is an extension programming language designed to support
general procedural programming with data description
facilities.
Lua is intended to be used as a powerful, light-weight
configuration language for any program that needs one.
Lua is implemented as a library, written in C.

Being an extension language, Lua has no notion of a ``main'' program:
it only works \emph{embedded} in a host client,
called the \emph{embedding program} or simply the \emph{host}.
This host program can invoke functions to execute a piece of Lua code,
can write and read Lua variables,
and can register C~functions to be called by Lua code.
Through the use of C~functions, Lua can be augmented to cope with
a wide range of different domains,
thus creating customized programming languages sharing a syntactical framework.

Lua is free software,
and is provided as usual with no guarantees,
as stated in its copyright notice.
The implementation described in this manual is available
at Lua's official web site, \verb|www.lua.org|.

Like any other reference manual,
this document is dry in places.
For a discussion of the decisions behind the design of Lua,
see the papers below,
which are available at Lua's web site.
\begin{itemize}
\item
R.~Ierusalimschy, L.~H.~de Figueiredo, and W.~Celes.
Lua---an extensible extension language.
\emph{Software: Practice \& Experience} {\bf 26} \#6 (1996) 635--652.
\item
L.~H.~de Figueiredo, R.~Ierusalimschy, and W.~Celes.
The design and implementation of a language for extending applications.
\emph{Proceedings of XXI Brazilian Seminar on Software and Hardware} (1994) 273--283.
\item
L.~H.~de Figueiredo, R.~Ierusalimschy, and W.~Celes.
Lua: an extensible embedded language.
\emph{Dr. Dobb's Journal} {\bf  21} \#12 (Dec 1996) 26--33.
\item
R.~Ierusalimschy, L.~H.~de Figueiredo, and W.~Celes.
The evolution of an extension language: a history of Lua,
\emph{Proceedings of V Brazilian Symposium on Programming Languages} (2001) B-14--B-28.
\end{itemize}

%------------------------------------------------------------------------------
\section{Lua Concepts}\label{concepts}

This section describes the main concepts of Lua as a language.
The syntax and semantics of Lua are described in \See{language}.
The discussion below is not purely conceptual;
it includes references to the C~API \see{API},
because Lua is designed to be embedded in host programs.
It also includes references to the standard libraries \see{libraries}.


\subsection{Environment and Chunks}

All statements in Lua are executed in a \Def{global environment}.
This environment is initialized with a call from the embedding program to
\verb|lua_open| and
persists until a call to \verb|lua_close|
or the end of the embedding program.
If necessary,
the host programmer can create multiple independent global
environments, and freely switch between them \see{mangstate}.

The unit of execution of Lua is called a \Def{chunk}.
A chunk is simply a sequence of statements.
Statements are described in \See{stats}.

A chunk may be stored in a file or in a string inside the host program.
When a chunk is executed, first it is pre-compiled into opcodes for
a virtual machine,
and then the compiled statements are executed
by an interpreter for the virtual machine.
All modifications a chunk effects on the global environment persist
after the chunk ends.

Chunks may also be pre-compiled into binary form and stored in files;
see program \IndexVerb{luac} for details.
Text files with chunks and their binary pre-compiled forms
are interchangeable;
Lua automatically detects the file type and acts accordingly.
\index{pre-compilation}


\subsection{\Index{Values and Types}} \label{TypesSec}

Lua is a \emph{dynamically typed language}.
That means that
variables do not have types; only values do.
There are no type definitions in the language.
All values carry their own type.

There are seven \Index{basic types} in Lua:
\Def{nil}, \Def{boolean}, \Def{number},
\Def{string}, \Def{function}, \Def{userdata}, and \Def{table}.
\emph{Nil} is the type of the value \nil,
whose main property is to be different from any other value;
usually it represents the absence of a useful value.
\emph{Boolean} is the type of the values \False{} and \True.
In Lua, both \nil{} and \False{} make a condition fails,
and any other value makes it succeeds.
\emph{Number} represents real (double-precision floating-point) numbers.
\emph{String} represents arrays of characters.
\index{eight-bit clean}
Lua is 8-bit clean,
and so strings may contain any 8-bit character,
including embedded zeros (\verb|'\0'|) \see{lexical}.

Functions are \emph{first-class values} in Lua.
That means that functions can be stored in variables,
passed as arguments to other functions, and returned as results.
Lua can call (and manipulate) functions written in Lua and
functions written in C
\see{functioncall}.

The type \emph{userdata} is provided to allow the store of
arbitrary C data in Lua variables.
This type corresponds to a block of raw memory
and has no pre-defined operations in Lua,
except assignment and identity test.
However, by using \emph{metatables},
the programmer can define operations for userdata values
\see{metatables}.
Userdata values cannot be created or modified in Lua,
only through the C~API.
This guarantees the integrity of data owned by the host program.

The type \emph{table} implements \Index{associative arrays},
that is, \Index{arrays} that can be indexed not only with numbers,
but with any value (except \nil).
Moreover,
tables can be \emph{heterogeneous},
that is, they can contain values of all types.
Tables are the sole data structuring mechanism in Lua;
they may be used not only to represent ordinary arrays,
but also symbol tables, sets, records, graphs, trees, etc.
To represent \Index{records}, Lua uses the field name as an index.
The language supports this representation by
providing \verb|a.name| as syntactic sugar for \verb|a["name"]|.
There are several convenient ways to create tables in Lua
\see{tableconstructor}.

Like indices, the value of a table field can be of any type.
In particular,
because functions are first class values,
table fields may contain functions.
So, tables may also carry \emph{methods} \see{func-def}.

Tables, functions, and userdata values are \emph{objects}:
variables do not actually \emph{contain} these values,
only \emph{references} to them.
Assignment, parameter passing, and returns from functions
always manipulate references to these values,
and do not imply any kind of copy.

The library function \verb|type| returns a string describing the type
of a given value \see{pdf-type}.


\subsubsection{Metatables}

Each table or userdata object in Lua may have a \Index{metatable}.

You can change several aspects of the behavior
of an object by setting specific fields in its metatable.
For instance, when an object is the operand of an addition,
Lua checks for a function in the field \verb|"__add"| in its metatable.
If it finds one,
Lua calls that function to perform the addition.

We call the keys in a metatable \Index{events},
and the values \Index{metamethods}.
In the previous example, \verb|"add"| is the event,
and the metamethod is the function that performs the addition.

A metatable controls how an object behaves in arithmetic operations,
order comparisons, concatenation, and indexing.
A metatable can also defines a function to be called when a userdata
is garbage collected.
\See{metatable} gives a detailed description of which events you
can control with metatables.

You can query and change the metatable of an object
through the \verb|setmetatable| and \verb|getmetatable|
functions \see{pdf-getmetatable}.



\subsection{Coercion} \label{coercion}

Lua provides automatic conversion between
string and number values at run time.
Any arithmetic operation applied to a string tries to convert
that string to a number, following the usual rules.
Conversely, whenever a number is used when a string is expected,
the number is converted to a string, in a reasonable format.
The format is chosen so that
a conversion from number to string then back to number
reproduces the original number \emph{exactly}.
For complete control of how numbers are converted to strings,
use the \verb|format| function \see{format}.


\subsection{Variables}

There are two kinds of variables in Lua:
global variables
and local variables.
Variables are assumed to be global unless explicitly declared local
\see{localvar}.
Before the first assignment, the value of a variable is \nil.

All global variables live as fields in ordinary Lua tables.
Usually, globals live in a table called \Index{table of globals}.
However, a function can individually change its global table,
so that all global variables in that function will refer to that table.
This mechanism allows the creation of \Index{namespaces} and other
modularization facilities.

\Index{Local variables} are lexically scoped.
Therefore, local variables can be freely accessed by functions
defined inside their scope \see{visibility}.


\subsection{Garbage Collection}\label{GC}

Lua does automatic memory management.
That means that
you do not have to worry about allocating memory for new objects
and freeing it when the objects are no longer needed.
Lua manages memory automatically by running
a \Index{garbage collector} from time to time
and
collecting all dead objects
(all objects that are no longer accessible from Lua).
All objects in Lua are subject to automatic management:
tables, userdata, functions, and strings.

Using the C~API,
you can set garbage-collector metamethods for userdata \see{metatable}.
When it is about to free a userdata,
Lua calls the metamethod associated with event \verb|gc| in the
userdata's metatable.
Using such facility, you can coordinate Lua's garbage collection
with external resource management
(such as closing files, network or database connections,
or freeing your own memory).

Lua uses two numbers to control its garbage-collection cycles.
One number counts how many bytes of dynamic memory Lua is using,
and the other is a threshold.
When the number of bytes crosses the threshold,
Lua runs the garbage collector,
which reclaims the memory of all dead objects.
The byte counter is corrected,
and then the threshold is reset to twice the value of the byte counter.

Through the C~API, you can query those numbers,
and change the threshold \see{GC-API}.
Setting the threshold to zero actually forces an immediate
garbage-collection cycle,
while setting it to a huge number effectively stops the garbage collector.
Using Lua code you have a more limited control over garbage-collection cycles,
through the functions \verb|gcinfo| and \verb|collectgarbage|
\see{predefined}.


\subsubsection{Weak Tables}\label{weak-table}

A \IndexEmph{weak table} is a table whose elements are
\IndexEmph{weak references}.
A weak reference is ignored by the garbage collector.
In other words,
if the only references to an object are weak references,
then the garbage collector will collect that object.

A weak table can have weak keys, weak values, or both.
A table with weak keys allows the collection of its keys,
but prevents the collection of its values.
A table with both weak keys and weak values allows the collection of
both keys and values.
In any case, if either the key or the value is collected,
the whole pair is removed from the table.
The weakness of a table is set with the \verb|setmode| function.


%------------------------------------------------------------------------------
\section{The Language}\label{language}

This section describes the lexis, the syntax, and the semantics of Lua.
In other words,
this section describes
which tokens are valid,
how they can be combined,
and what their combinations mean.

\subsection{Lexical Conventions} \label{lexical}

\IndexEmph{Identifiers} in Lua can be any string of letters,
digits, and underscores,
not beginning with a digit.
This coincides with the definition of identifiers in most languages.
(The definition of letter depends on the current locale:
any character considered alphabetic by the current locale
can be used in an identifier.)

The following \IndexEmph{keywords} are reserved,
and cannot be used as identifiers:
\index{reserved words}
\begin{verbatim}
       and       break     do        else      elseif
       end       false     for       function  global
       if        in        local     nil       not
       or        repeat    return    then      true
       until     while
\end{verbatim}

Lua is a case-sensitive language:
\T{and} is a reserved word, but \T{And} and \T{\'and}
(if the locale permits) are two different, valid identifiers.
As a convention, identifiers starting with an underscore followed by
uppercase letters (such as \verb|_VERSION|)
are reserved for internal variables.

The following strings denote other \Index{tokens}:
\begin{verbatim}
       +     -     *     /     ^     %
       ~=    <=    >=    <     >     ==    =
       (     )     {     }     [     ]
       ;     :     ,     .     ..    ...
\end{verbatim}

\IndexEmph{Literal strings}
can be delimited by matching single or double quotes,
and can contain the C-like escape sequences
`\verb|\a|' (bell),
`\verb|\b|' (backspace),
`\verb|\f|' (form feed),
`\verb|\n|' (newline),
`\verb|\r|' (carriage return),
`\verb|\t|' (horizontal tab),
`\verb|\v|' (vertical tab),
`\verb|\\|' (backslash),
`\verb|\"|' (double quote),
`\verb|\'|' (single quote),
and `\verb|\|\emph{newline}' (that is, a backslash followed by a real newline,
which  results in a newline in the string).
A character in a string may also be specified by its numerical value,
through the escape sequence `\verb|\|\emph{ddd}',
where \emph{ddd} is a sequence of up to three \emph{decimal} digits.
Strings in Lua may contain any 8-bit value, including embedded zeros,
which can be specified as `\verb|\0|'.

Literal strings can also be delimited by matching \verb|[[| $\ldots$ \verb|]]|.
Literals in this bracketed form may run for several lines,
may contain nested \verb|[[| $\ldots$ \verb|]]| pairs,
and do not interpret escape sequences.
For convenience,
when the opening \verb|[[| is immediately followed by a newline,
the newline is not included in the string.
That form is specially convenient for
writing strings that contain program pieces or
other quoted strings.
As an example, in a system using ASCII
(in which `\verb|a|' is coded as~97,
newline is coded as~10, and `\verb|1|' is coded as~49),
the four literals below denote the same string:
\begin{verbatim}
       1)   "alo\n123\""
       2)   '\97lo\10\04923"'
       3)   [[alo
            123"]]
       4)   [[
            alo
            123"]]
\end{verbatim}

\IndexEmph{Numerical constants} may be written with an optional decimal part
and an optional decimal exponent.
Examples of valid numerical constants are
\begin{verbatim}
       3     3.0     3.1416  314.16e-2   0.31416E1
\end{verbatim}

\IndexEmph{Comments} start anywhere outside a string with a
double hyphen (\verb|--|);
If the text after \verb|--| is different from \verb|[[|,
the comment is a short comment,
that runs until the end of the line.
Otherwise, it is a long comment,
that runs until the corresponding \verb|]]|.
Long comments may run for several lines,
and may contain nested \verb|[[| $\ldots$ \verb|]]| pairs.
For convenience,
the first line of a chunk is skipped if it starts with \verb|#|.
This facility allows the use of Lua as a script interpreter
in Unix systems \see{lua-sa}.


\subsection{Variables}\label{variables}

Variables are places that store values.
%In Lua, variables are given by simple identifiers or by table fields.

A single name can denote a global variable, a local variable,
or a formal parameter in a function
(formal parameters are just local variables):
\begin{Produc}
\produc{var}{\Nter{Name}}
\end{Produc}%
Square brackets are used to index a table:
\begin{Produc}
\produc{var}{prefixexp \ter{[} exp \ter{]}}
\end{Produc}%
The first expression should result in a table value,
and the second expression identifies a specific entry inside that table.

The syntax \verb|var.NAME| is just syntactic sugar for
\verb|var["NAME"]|:
\begin{Produc}
\produc{var}{prefixexp \ter{.} \Nter{Name}}
\end{Produc}%

The expression denoting the table to be indexed has a restricted syntax;
\See{expressions} for details.

The meaning of assignments and evaluations of global and
indexed variables can be changed via metatables.
An assignment to a global variable \verb|x = val|
is equivalent to the assignment
\verb|_glob.x = val|,
where \verb|_glob| is the table of globals of the running function
(\see{global-table} for a discussion about the table of globals).
An assignment to an indexed variable \verb|t[i] = val| is equivalent to
\verb|settable_event(t,i,val)|.
An access to a global variable \verb|x|
is equivalent to \verb|_glob.x|
(again, \see{global-table} for a discussion about \verb|_glob|).
An access to an indexed variable \verb|t[i]| is equivalent to
a call \verb|gettable_event(t,i)|.
See \See{metatable} for a complete description of the
\verb|settable_event| and \verb|gettable_event| functions.
(These functions are not defined in Lua.
We use them here only for explanatory purposes.)


\subsection{Statements}\label{stats}

Lua supports an almost conventional set of \Index{statements},
similar to those in Pascal or C.
The conventional commands include
assignment, control structures, and procedure calls.
Non-conventional commands include table constructors
and variable declarations.

\subsubsection{Chunks}\label{chunks}
The unit of execution of Lua is called a \Def{chunk}.
A chunk is simply a sequence of statements,
which are executed sequentially.
Each statement can be optionally followed by a semicolon:
\begin{Produc}
\produc{chunk}{\rep{stat \opt{\ter{;}}}}
\end{Produc}%

\subsubsection{Blocks}
A \Index{block} is a list of statements;
syntactically, a block is equal to a chunk:
\begin{Produc}
\produc{block}{chunk}
\end{Produc}%

A block may be explicitly delimited to produce a single statement:
\begin{Produc}
\produc{stat}{\rwd{do} block \rwd{end}}
\end{Produc}%
\IndexKW{do}
Explicit blocks are useful
to control the scope of variable declarations.
Explicit blocks are also sometimes used to
add a \rwd{return} or \rwd{break} statement in the middle
of another block \see{control}.

\subsubsection{\Index{Assignment}} \label{assignment}
Lua allows \Index{multiple assignment}.
Therefore, the syntax for assignment
defines a list of variables on the left side
and a list of expressions on the right side.
The elements in both lists are separated by commas:
\begin{Produc}
\produc{stat}{varlist1 \ter{=} explist1}
\produc{varlist1}{var \rep{\ter{,} var}}
\produc{explist1}{exp \rep{\ter{,} exp}}
\end{Produc}%
Expressions are discussed in \See{expressions}.

Before the assignment,
the list of values is \emph{adjusted} to the length of
the list of variables.\index{adjustment}
If there are more values than needed,
the excess values are thrown away.
If there are less values than needed,
the list is extended with as many  \nil's as needed.
If the list of expressions ends with a function call,
then all values returned by that function call enter in the list of values,
before the adjust
(except when the call is enclosed in parentheses; see \See{expressions}).

The assignment statement first evaluates all its expressions,
and only then makes the assignments.
So, the code
\begin{verbatim}
       i = 3
       i, a[i] = i+1, 20
\end{verbatim}
sets \verb|a[3]| to 20, without affecting \verb|a[4]|
because the \verb|i| in \verb|a[i]| is evaluated
before it is assigned 4.
Similarly, the line
\begin{verbatim}
       x, y = y, x
\end{verbatim}
exchanges the values of \verb|x| and \verb|y|.

\subsubsection{Control Structures}\label{control}
The control structures
\rwd{if}, \rwd{while}, and \rwd{repeat} have the usual meaning and
familiar syntax:
\index{while-do statement}\IndexKW{while}
\index{repeat-until statement}\IndexKW{repeat}\IndexKW{until}
\index{if-then-else statement}\IndexKW{if}\IndexKW{else}\IndexKW{elseif}
\begin{Produc}
\produc{stat}{\rwd{while} exp \rwd{do} block \rwd{end}}
\produc{stat}{\rwd{repeat} block \rwd{until} exp}
\produc{stat}{\rwd{if} exp \rwd{then} block
  \rep{\rwd{elseif} exp \rwd{then} block}
   \opt{\rwd{else} block} \rwd{end}}
\end{Produc}%
Lua also has a \rwd{for} statement, in two flavors \see{for}.

The \Index{condition expression} \M{exp} of a
control structure may return any value.
All values different from \nil{} and \False{} are considered true
(in particular, the number 0 and the empty string are also true);
both \False{} and \nil{} are considered false.

The \rwd{return} statement is used to return values
from a function or from a chunk.\IndexKW{return}
\label{return}%
\index{return statement}%
Functions and chunks may return more than one value,
and so the syntax for the \rwd{return} statement is
\begin{Produc}
\produc{stat}{\rwd{return} \opt{explist1}}
\end{Produc}%

The \rwd{break} statement can be used to terminate the execution of a
\rwd{while}, \rwd{repeat}, or \rwd{for} loop,
skipping to the next statement after the loop:\IndexKW{break}
\index{break statement}
\begin{Produc}
\produc{stat}{\rwd{break}}
\end{Produc}%
A \rwd{break} ends the innermost enclosing loop.

\NOTE
For syntactic reasons, \rwd{return} and \rwd{break}
statements can only be written as the \emph{last} statement of a block.
If it is really necessary to \rwd{return} or \rwd{break} in the
middle of a block,
then an explicit inner block can used,
as in the idioms
`\verb|do return end|' and
`\verb|do break end|',
because now \rwd{return} and \rwd{break} are the last statements in
their (inner) blocks.
In practice,
those idioms are only used during debugging.
(For instance, a line `\verb|do return end|' can be added at the
beginning of a chunk for syntax checking only.)

\subsubsection{For Statement} \label{for}\index{for statement}

The \rwd{for} statement has two forms,
one for numbers and one generic.
\IndexKW{for}\IndexKW{in}

The numerical \rwd{for} loop repeats a block of code while a
control variable runs through an arithmetic progression.
It has the following syntax:
\begin{Produc}
\produc{stat}{\rwd{for} \Nter{Name} \ter{=} exp \ter{,} exp \opt{\ter{,} exp}
                    \rwd{do} block \rwd{end}}
\end{Produc}%
The \emph{block} is repeated for \emph{name} starting at the value of
the first \emph{exp}, until it reaches the second \emph{exp} by steps of the
third \emph{exp}.
More precisely, a \rwd{for} statement like
\begin{verbatim}
       for var = e1, e2, e3 do block end
\end{verbatim}
is equivalent to the code:
\begin{verbatim}
       do
         local var, _limit, _step = tonumber(e1), tonumber(e2), tonumber(e3)
         if not (var and _limit and _step) then error() end
         while (_step>0 and var<=_limit) or (_step<=0 and var>=_limit) do
           block
           var = var+_step
         end
       end
\end{verbatim}
Note the following:
\begin{itemize}\itemsep=0pt
\item Both the limit and the step are evaluated only once,
before the loop starts.
\item \verb|_limit| and \verb|_step| are invisible variables.
The names are here for explanatory purposes only.
\item The behavior is \emph{undefined} if you assign to \verb|var| inside
the block.
\item If the third expression (the step) is absent, then a step of~1 is used.
\item You can use \rwd{break} to exit a \rwd{for} loop.
\item The loop variable \verb|var| is local to the statement;
you cannot use its value after the \rwd{for} ends or is broken.
If you need the value of the loop variable \verb|var|,
then assign it to another variable before breaking or exiting the loop.
\end{itemize}

The generic \rwd{for} statement works over functions,
called \Index{generators}.
It calls its generator to produce a new value for each iteration,
stopping when the new value is \nil.
It has the following syntax:
\begin{Produc}
\produc{stat}{\rwd{for} \Nter{Name} \rep{\ter{,} \Nter{Name}} \rwd{in} explist1
                    \rwd{do} block \rwd{end}}
\end{Produc}%
A \rwd{for} statement like
\begin{verbatim}
       for var_1, ..., var_n in explist do block end
\end{verbatim}
is equivalent to the code:
\begin{verbatim}
       do
         local _f, _s, var_1 = explist
         while 1 do
           local var_2, ..., var_n
           var_1, ..., var_n = _f(_s, var_1)
           if var_1 == nil then break end
           block
         end
       end
\end{verbatim}
Note the following:
\begin{itemize}\itemsep=0pt
\item \verb|explist| is evaluated only once.
Its results are a ``generator'' function,
a ``state'', and an initial value for the ``iterator variable''.
\item \verb|_f| and \verb|_s| are invisible variables.
The names are here for explanatory purposes only.
\item The behavior is \emph{undefined} if you assign to any
\verb|var_i| inside the block.
\item You can use \rwd{break} to exit a \rwd{for} loop.
\item The loop variables \verb|var_i| are local to the statement;
you cannot use their values after the \rwd{for} ends.
If you need these values,
then assign them to other variables before breaking or exiting the loop.
\end{itemize}


\subsubsection{Function Calls as Statements} \label{funcstat}
Because of possible side-effects,
function calls can be executed as statements:
\begin{Produc}
\produc{stat}{functioncall}
\end{Produc}%
In this case, all returned values are thrown away.
Function calls are explained in \See{functioncall}.

\subsubsection{Local Declarations} \label{localvar}
\Index{Local variables} may be declared anywhere inside a block.
The declaration may include an initial assignment:\IndexKW{local}
\begin{Produc}
\produc{stat}{\rwd{local} namelist \opt{\ter{=} explist1}}
\produc{namelist}{\Nter{Name} \rep{\ter{,} \Nter{Name}}}
\end{Produc}%
If present, an initial assignment has the same semantics
of a multiple assignment \see{assignment}.
Otherwise, all variables are initialized with \nil.

A chunk is also a block \see{chunks},
and so local variables can be declared outside any explicit block.
Such local variables die when the chunk ends.

Visibility rules for local variables are explained in \See{visibility}.


\subsection{\Index{Expressions}}\label{expressions}

%\subsubsection{\Index{Basic Expressions}}
The basic expressions in Lua are the following:
\begin{Produc}
\produc{exp}{prefixexp}
\produc{exp}{\rwd{nil} \Or \rwd{false} \Or \rwd{true}}
\produc{exp}{Number}
\produc{exp}{Literal}
\produc{exp}{function}
\produc{exp}{tableconstructor}
\produc{prefixexp}{var \Or functioncall \Or \ter{(} exp \ter{)}}
\end{Produc}%
\IndexKW{nil}\IndexKW{false}\IndexKW{true}

An expression enclosed in parentheses always results in only one value.
Thus,
\verb|(f(x,y,z))| is always a single value,
even if \verb|f| returns several values.
(The value of \verb|(f(x,y,z))| is the first value returned by \verb|f|
or \nil{} if \verb|f| does not return any values.)

\emph{Numbers} and \emph{literal strings} are explained in \See{lexical};
variables are explained in \See{variables};
function definitions are explained in \See{func-def};
function calls are explained in \See{functioncall};
table constructors are explained in \See{tableconstructor}.

Expressions can also be built with arithmetic operators, relational operators,
and logical operadors, all of which are explained below.

\subsubsection{Arithmetic Operators}
Lua supports the usual \Index{arithmetic operators}:
the binary \verb|+| (addition),
\verb|-| (subtraction), \verb|*| (multiplication),
\verb|/| (division), and \verb|^| (exponentiation);
and unary \verb|-| (negation).
If the operands are numbers, or strings that can be converted to
numbers \see{coercion},
then all operations except exponentiation have the usual meaning,
while exponentiation calls a global function \verb|pow|; ??
otherwise, an appropriate metamethod is called \see{metatable}.
The standard mathematical library defines function \verb|pow|,
giving the expected meaning to \Index{exponentiation}
\see{mathlib}.

\subsubsection{Relational Operators}\label{rel-ops}
The \Index{relational operators} in Lua are
\begin{verbatim}
       ==    ~=    <     >     <=    >=
\end{verbatim}
These operators always result in \False{} or \True.

Equality (\verb|==|) first compares the type of its operands.
If the types are different, then the result is \False.
Otherwise, the values of the operands are compared.
Numbers and strings are compared in the usual way.
Tables, userdata, and functions are compared \emph{by reference},
that is,
two tables are considered equal only if they are the \emph{same} table.

??eq metamethod??

Every time you create a new table (or userdata, or function),
this new value is different from any previously existing value.

\NOTE
The conversion rules of \See{coercion}
\emph{do not} apply to equality comparisons.
Thus, \verb|"0"==0| evaluates to \emph{false},
and \verb|t[0]| and \verb|t["0"]| denote different
entries in a table.
\medskip

The operator \verb|~=| is exactly the negation of equality (\verb|==|).

The order operators work as follows.
If both arguments are numbers, then they are compared as such.
Otherwise, if both arguments are strings,
then their values are compared according to the current locale.
Otherwise, the ``lt'' or the ``le'' metamethod is called \see{metatable}.


\subsubsection{Logical Operators}
The \Index{logical operators} in Lua are
\index{and}\index{or}\index{not}
\begin{verbatim}
       and   or    not
\end{verbatim}
Like the control structures \see{control},
all logical operators consider both \False{} and \nil{} as false
and anything else as true.
\IndexKW{and}\IndexKW{or}\IndexKW{not}

The operator \rwd{not} always return \False{} or \True.

The conjunction operator \rwd{and} returns its first argument
if its value is \False{} or \nil;
otherwise, \rwd{and} returns its second argument.
The disjunction operator \rwd{or} returns its first argument
if it is different from \nil and \False;
otherwise, \rwd{or} returns its second argument.
Both \rwd{and} and \rwd{or} use \Index{short-cut evaluation},
that is,
the second operand is evaluated only if necessary.
For example,
\begin{verbatim}
       10 or error()       -> 10
       nil or "a"          -> "a"
       nil and 10          -> nil
       false and error()   -> false
       false and nil       -> false
       false or nil        -> nil
       10 and 20           -> 20
\end{verbatim}

\subsubsection{Concatenation} \label{concat}
The string \Index{concatenation} operator in Lua is
denoted by two dots (`\verb|..|').
If both operands are strings or numbers, then they are converted to
strings according to the rules mentioned in \See{coercion}.
Otherwise, the ``concat'' metamethod is called \see{metatable}.

\subsubsection{Precedence}
\Index{Operator precedence} in Lua follows the table below,
from lower to higher priority:
\begin{verbatim}
       or
       and
       <     >     <=    >=    ~=    ==
       ..
       +     -
       *     /
       not   - (unary)
       ^
\end{verbatim}
All binary operators are left associative,
except for \verb|^| (exponentiation),
which is right associative.
\NOTE
The pre-compiler may rearrange the order of evaluation of
associative operators,
and may exchange the operands of commutative operators,
as long as these optimizations do not change normal results.
However, these optimizations may change some results
if you define non-associative (or non-commutative)
metamethods for those operators.

\subsubsection{Table Constructors} \label{tableconstructor}
Table \Index{constructors} are expressions that create tables;
every time a constructor is evaluated, a new table is created.
Constructors can be used to create empty tables,
or to create a table and initialize some of its fields.
The general syntax for constructors is
\begin{Produc}
\produc{tableconstructor}{\ter{\{} \opt{fieldlist} \ter{\}}}
\produc{fieldlist}{field \rep{fieldsep field} \opt{fieldsep}}
\produc{field}{\ter{[} exp \ter{]} \ter{=} exp \Or
               \Nter{Name} \ter{=} exp \Or exp}
\produc{fieldsep}{\ter{,} \Or \ter{;}}
\end{Produc}%

Each field of the form \verb|[exp1] = exp2| adds to the new table an entry
with key \verb|exp1| and value \verb|exp2|.
A field of the form \verb|name = exp| is equivalent to
\verb|["name"] = exp|.
Finally, fields of the form \verb|exp| are equivalent to
\verb|[i] = exp|, where \verb|i| are consecutive numerical integers,
starting with 1.
Fields in the other formats do not affect this counting.
For example,
\begin{verbatim}
       a = {[f(1)] = g; "x", "y"; x = 1, f(x), [30] = 23; 45}
\end{verbatim}
is equivalent to
\begin{verbatim}
       do
         local temp = {}
         temp[f(1)] = g
         temp[1] = "x"         -- 1st exp
         temp[2] = "y"         -- 2nd exp
         temp.x = 1            -- temp["x"] = 1
         temp[3] = f(x)        -- 3rd exp
         temp[30] = 23
         temp[4] = 45          -- 4th exp
         a = temp
       end
\end{verbatim}

If the last expression in the list is a function call,
then all values returned by the call enter the list consecutively
\see{functioncall}.
If you want to avoid this,
enclose the function call in parentheses.

The field list may have an optional trailing separator,
as a convenience for machine-generated code.


\subsubsection{Function Calls}  \label{functioncall}
A \Index{function call} in Lua has the following syntax:
\begin{Produc}
\produc{functioncall}{prefixexp args}
\end{Produc}%
In a function call,
first \M{prefixexp} and \M{args} are evaluated.
If the value of \M{prefixexp} has type \emph{function},
then that function is called,
with the given arguments.
Otherwise, its ``call'' metamethod is called,
having as first parameter the value of \M{prefixexp},
followed by the original call arguments
\see{metatable}.

The form
\begin{Produc}
\produc{functioncall}{prefixexp \ter{:} \Nter{name} args}
\end{Produc}%
can be used to call ``methods''.
A call \verb|v:name(...)|
is syntactic sugar for \verb|v.name(v, ...)|,
except that \verb|v| is evaluated only once.

Arguments have the following syntax:
\begin{Produc}
\produc{args}{\ter{(} \opt{explist1} \ter{)}}
\produc{args}{tableconstructor}
\produc{args}{Literal}
\end{Produc}%
All argument expressions are evaluated before the call.
A call of the form \verb|f{...}| is syntactic sugar for
\verb|f({...})|, that is,
the argument list is a single new table.
A call of the form \verb|f'...'|
(or \verb|f"..."| or \verb|f[[...]]|) is syntactic sugar for
\verb|f('...')|, that is,
the argument list is a single literal string.

Because a function can return any number of results
\see{return},
the number of results must be adjusted before they are used.
If the function is called as a statement \see{funcstat},
then its return list is adjusted to~0 elements,
thus discarding all returned values.
If the function is called inside another expression,
or in the middle of a list of expressions,
then its return list is adjusted to~1 element,
thus discarding all returned values but the first one.
If the function is called as the last element of a list of expressions,
then no adjustment is made
(unless the call is enclosed in parentheses).

Here are some examples:
\begin{verbatim}
       f()                -- adjusted to 0 results
       g(f(), x)          -- f() is adjusted to 1 result
       g(x, f())          -- g gets x plus all values returned by f()
       a,b,c = f(), x     -- f() is adjusted to 1 result (and c gets nil)
       a,b,c = x, f()     -- f() is adjusted to 2
       a,b,c = f()        -- f() is adjusted to 3
       return f()         -- returns all values returned by f()
       return x,y,f()     -- returns x, y, and all values returned by f()
       {f()}              -- creates a list with all values returned by f()
       {f(), nil}         -- f() is adjusted to 1 result
\end{verbatim}

If you enclose a function call in parentheses,
then it is adjusted to return exactly one value:
\begin{verbatim}
       return x,y,(f())   -- returns x, y, and the first value from f()
       {(f())}            -- creates a table with exactly one element
\end{verbatim}

As an exception to the format-free syntax of Lua,
you cannot put a line break before the \verb|(| in a function call.
That restriction avoids some ambiguities in the language.
If you write
\begin{verbatim}
       a = f
       (g).x(a)
\end{verbatim}
Lua would read that as \verb|a = f(g).x(a)|.
So, if you want two statements, you must add a semi-colon between them.
If you actually want to call \verb|f|,
you must remove the line break before \verb|(g)|.


\subsubsection{\Index{Function Definitions}} \label{func-def}

The syntax for function definition is\IndexKW{function}
\begin{Produc}
\produc{function}{\rwd{function} funcbody}
\produc{funcbody}{\ter{(} \opt{parlist1} \ter{)} block \rwd{end}}
\end{Produc}%

The following syntactic sugar simplifies function definitions:
\begin{Produc}
\produc{stat}{\rwd{function} funcname funcbody}
\produc{stat}{\rwd{local} \rwd{function} \Nter{name} funcbody}
\produc{funcname}{\Nter{name} \rep{\ter{.} \Nter{name}} \opt{\ter{:} \Nter{name}}}
\end{Produc}%
The statement
\begin{verbatim}
       function f () ... end
\end{verbatim}
translates to
\begin{verbatim}
       f = function () ... end
\end{verbatim}
The statement
\begin{verbatim}
       function t.a.b.c.f () ... end
\end{verbatim}
translates to
\begin{verbatim}
       t.a.b.c.f = function () ... end
\end{verbatim}
The statement
\begin{verbatim}
       local function f () ... end
\end{verbatim}
translates to
\begin{verbatim}
       local f; f = function () ... end
\end{verbatim}

A function definition is an executable expression,
whose value has type \emph{function}.
When Lua pre-compiles a chunk,
all its function bodies are pre-compiled too.
Then, whenever Lua executes the function definition,
the function is \emph{instantiated} (or \emph{closed}).
This function instance (or \emph{closure})
is the final value of the expression.
Different instances of the same function
may refer to different non-local variables \see{visibility}
and may have different tables of globals \see{global-table}.

Parameters act as local variables,
initialized with the argument values:
\begin{Produc}
\produc{parlist1}{namelist \opt{\ter{,} \ter{\ldots}}}
\produc{parlist1}{\ter{\ldots}}
\end{Produc}%
\label{vararg}%
When a function is called,
the list of \Index{arguments} is adjusted to
the length of the list of parameters,
unless the function is a \Def{vararg function},
which is
indicated by three dots (`\verb|...|') at the end of its parameter list.
A vararg function does not adjust its argument list;
instead, it collects all extra arguments into an implicit parameter,
called \IndexLIB{arg}.
The value of \verb|arg| is a table,
with a field~\verb|n| whose value is the number of extra arguments,
and the extra arguments at positions 1,~2,~\ldots,~\verb|n|.

As an example, consider the following definitions:
\begin{verbatim}
       function f(a, b) end
       function g(a, b, ...) end
       function r() return 1,2,3 end
\end{verbatim}
Then, we have the following mapping from arguments to parameters:
\begin{verbatim}
       CALL            PARAMETERS

       f(3)             a=3, b=nil
       f(3, 4)          a=3, b=4
       f(3, 4, 5)       a=3, b=4
       f(r(), 10)       a=1, b=10
       f(r())           a=1, b=2

       g(3)             a=3, b=nil, arg={n=0}
       g(3, 4)          a=3, b=4,   arg={n=0}
       g(3, 4, 5, 8)    a=3, b=4,   arg={5, 8; n=2}
       g(5, r())        a=5, b=1,   arg={2, 3; n=2}
\end{verbatim}

Results are returned using the \rwd{return} statement \see{return}.
If control reaches the end of a function
without encountering a \rwd{return} statement,
then the function returns with no results.

The \emph{colon} syntax
is used for defining \IndexEmph{methods},
that is, functions that have an implicit extra parameter \IndexVerb{self}.
Thus, the statement
\begin{verbatim}
       function t.a.b.c:f (...) ... end
\end{verbatim}
is syntactic sugar for
\begin{verbatim}
       t.a.b.c.f = function (self, ...) ... end
\end{verbatim}


\subsection{Visibility Rules} \label{visibility}
\index{visibility}

Lua is a lexically scoped language.
The scope of variables begins at the first statement \emph{after}
their declaration and lasts until the end of the innermost block that
includes the declaration.
For instance:
\begin{verbatim}
  x = 10                -- global variable
  do                    -- new block
    local x = x         -- new `x', with value 10
    print(x)            --> 10
    x = x+1
    do                  -- another block
      local x = x+1     -- another `x'
      print(x)          --> 12
    end
    print(x)            --> 11
  end
  print(x)              --> 10  (the global one)
\end{verbatim}
Notice that, in a declaration like \verb|local x = x|,
the new \verb|x| being declared is not in scope yet,
so the second \verb|x| refers to the ``outside'' variable.

Because of those \Index{lexical scoping} rules,
local variables can be freely accessed by functions
defined inside their scope.
For instance:
\begin{verbatim}
  local counter = 0
  function inc (x)
    counter = counter + x
    return counter
  end
\end{verbatim}

Notice that each execution of a \rwd{local} statement
``creates'' new local variables.
Consider the following example:
\begin{verbatim}
  a = {}
  local x = 20
  for i=1,10 do
    local y = 0
    a[i] = function () y=y+1; return x+y end
  end
\end{verbatim}
In that code,
each function uses a different \verb|y| variable,
while all of them share the same \verb|x|.

\subsection{Error Handling} \label{error}

%% TODO Must be rewritten!!!

Because Lua is an extension language,
all Lua actions start from C~code in the host program
calling a function from the Lua library.
Whenever an error occurs during Lua compilation or execution,
the function \verb|_ERRORMESSAGE| is called \DefLIB{_ERRORMESSAGE}
(provided it is different from \nil),
and then the corresponding function from the library
(\verb|lua_dofile|, \verb|lua_dostring|,
\verb|lua_dobuffer|, or \verb|lua_call|)
is terminated, returning an error condition.

Memory allocation errors are an exception to the previous rule.
When memory allocation fails, Lua may not be able to execute the
\verb|_ERRORMESSAGE| function.
So, for this kind of error, Lua does not call
the \verb|_ERRORMESSAGE| function;
instead, the corresponding function from the library
returns immediately with a special error code (\verb|LUA_ERRMEM|).
This and other error codes are defined in \verb|lua.h|
\see{luado}.

The only argument to \verb|_ERRORMESSAGE| is a string
describing the error.
The default definition for
this function calls \verb|_ALERT|, \DefLIB{_ALERT}
which prints the message to \verb|stderr| \see{alert}.
The standard I/O library redefines \verb|_ERRORMESSAGE|
and uses the debug interface \see{debugI}
to print some extra information,
such as a call-stack traceback.

Lua code can explicitly generate an error by calling the
function \verb|error| \see{pdf-error}.
Lua code can ``catch'' an error using the function
\verb|call| \see{pdf-call}.


\subsection{Metatables} \label{metatable}

Every table and userdata value in Lua may have a \emph{metatable}.
This \IndexEmph{metatable} is a table that defines the behavior of
the original table and userdata under some operations.
You can query and change the metatable of an object with
functions \verb|setmetatable| and \verb|getmetatable| \see{pdf-getmetatable}.

For each of those operations Lua associates a specific key,
called an \emph{event}.
When Lua performs one of those operations over a table or a userdata,
if checks whether that object has a metatable with the corresponding event.
If so, the value associated with that key (the \IndexEmph{metamethod})
controls how Lua will perform the operation.

Metatables control the operations listed next.
Each operation is identified by its corresponding name.
The key for each operation is a string with its name prefixed by
two underscores;
for instance, the key for operation ``add'' is the
string \verb|"__add"|.
The semantics of these operations is better explained by a Lua function
describing how the interpreter executes that operation.
%Each function shows how a handler is called,
%its arguments (that is, its signature),
%its results,
%and the default behavior in the absence of a handler.
The code shown here in Lua is only illustrative;
the real behavior is hard coded in the interpreter,
and it is much more efficient than this simulation.
All functions used in these descriptions
(\verb|rawget|, \verb|tonumber|, etc.)
are described in \See{predefined}.

\begin{description}

\item[``add'':]\IndexTM{add}
the \verb|+| operation.

The function \verb|getbinhandler| below defines how Lua chooses a handler
for a binary operation.
First, Lua tries the first operand.
If its type does not define a handler for the operation,
then Lua tries the second operand.
\begin{verbatim}
       function getbinhandler (op1, op2, event)
         return metatable(op1)[event] or metatable(op2)[event]
       end
\end{verbatim}
Using that function,
the behavior of the ``add'' operation is
\begin{verbatim}
       function add_event (op1, op2)
         local o1, o2 = tonumber(op1), tonumber(op2)
         if o1 and o2 then  -- both operands are numeric
           return o1+o2  -- '+' here is the primitive 'add'
         else  -- at least one of the operands is not numeric
           local h = getbinhandler(op1, op2, "__add")
           if h then
             -- call the handler with both operands
             return h(op1, op2)
           else  -- no handler available: default behavior
             error("unexpected type at arithmetic operation")
           end
         end
       end
\end{verbatim}

\item[``sub'':]\IndexTM{sub}
the \verb|-| operation.
Behavior similar to the ``add'' operation.

\item[``mul'':]\IndexTM{mul}
the \verb|*| operation.
Behavior similar to the ``add'' operation.

\item[``div'':]\IndexTM{div}
the \verb|/| operation.
Behavior similar to the ``add'' operation.

\item[``pow'':]\IndexTM{pow}
the \verb|^| operation (exponentiation) operation.
\begin{verbatim} ??
       function pow_event (op1, op2)
         local h = getbinhandler(op1, op2, "__pow") ???
         if h then
           -- call the handler with both operands
           return h(op1, op2)
         else  -- no handler available: default behavior
           error("unexpected type at arithmetic operation")
         end
       end
\end{verbatim}

\item[``unm'':]\IndexTM{unm}
the unary \verb|-| operation.
\begin{verbatim}
       function unm_event (op)
         local o = tonumber(op)
         if o then  -- operand is numeric
           return -o  -- '-' here is the primitive 'unm'
         else  -- the operand is not numeric.
           -- Try to get a handler from the operand;
           local h = metatable(op).__unm
           if h then
             -- call the handler with the operand and nil
             return h(op, nil)
           else  -- no handler available: default behavior
             error("unexpected type at arithmetic operation")
           end
         end
       end
\end{verbatim}

\item[``lt'':]\IndexTM{lt}
the \verb|<| operation.
\begin{verbatim}
       function lt_event (op1, op2)
         if type(op1) == "number" and type(op2) == "number" then
           return op1 < op2   -- numeric comparison
         elseif type(op1) == "string" and type(op2) == "string" then
           return op1 < op2   -- lexicographic comparison
         else
           local h = getbinhandler(op1, op2, "__lt")
           if h then
             return h(op1, op2)
           else
             error("unexpected type at comparison");
           end
         end
       end
\end{verbatim}
\verb|a>b| is equivalent to \verb|b<a|.

\item[``le'':]\IndexTM{lt}
the \verb|<=| operation.
\begin{verbatim}
       function lt_event (op1, op2)
         if type(op1) == "number" and type(op2) == "number" then
           return op1 < op2   -- numeric comparison
         elseif type(op1) == "string" and type(op2) == "string" then
           return op1 < op2   -- lexicographic comparison
         else
           local h = getbinhandler(op1, op2, "__le")
           if h then
             return h(op1, op2)
           else
             h = getbinhandler(op1, op2, "__lt")
             if h then
               return not h(op2, op1)
             else
               error("unexpected type at comparison");
             end
           end
         end
       end
\end{verbatim}
\verb|a>=b| is equivalent to \verb|b<=a|.
Notice that, in the absence of a ``le'' metamethod,
Lua tries the ``lt'', assuming that \verb|a<=b| is
equivalent to \verb|not (b<a)|.


\item[``concat'':]\IndexTM{concatenation}
the \verb|..| (concatenation) operation.
\begin{verbatim}
       function concat_event (op1, op2)
         if (type(op1) == "string" or type(op1) == "number") and
            (type(op2) == "string" or type(op2) == "number") then
           return op1..op2  -- primitive string concatenation
         else
           local h = getbinhandler(op1, op2, "__concat")
           if h then
             return h(op1, op2)
           else
             error("unexpected type for concatenation")
           end
         end
       end
\end{verbatim}

\item[``index'':]\IndexTM{index}
This handler is called when Lua tries to retrieve the value of an index
not present in a table.
See the ``gettable'' operation for its semantics.

\item[``gettable'':]\IndexTM{gettable}
called whenever Lua accesses an indexed variable.
\begin{verbatim}
       function gettable_event (table, key)
         local h
         if type(table) == "table" then
           local v = rawget(table, key)
           if v ~= nil then return v end
           h = metatable(table).__index
           if h == nil then return nil end
         else
           h = metatable(table).__gettable
           if h == nil then
             error("indexed expression not a table");
           end
         end
         if type(h) == "function" then
           return h(table, key)      -- call the handler
         else return h[key]          -- or repeat operation with it
       end
\end{verbatim}

\item[``newindex'':]\IndexTM{index}
This handler is called when Lua tries to insert the value of an index
not present in a table.
See the ``settable'' operation for its semantics.

\item[``settable'':]\IndexTM{settable}
called when Lua assigns to an indexed variable.
\begin{verbatim}
       function settable_event (table, key, value)
         local h
         if type(table) == "table" then
           local v = rawget(table, key)
           if v ~= nil then rawset(table, key, value); return end
           h = metatable(table).__newindex
           if h == nil then rawset(table, key, value); return end
         else
           h = metatable(table).__settable
           if h == nil then
             error("indexed expression not a table");
           end
         end
         if type(h) == "function" then
           return h(table, key,value)    -- call the handler
         else h[key] = value             -- or repeat operation with it
       end
\end{verbatim}


\item[``call'':]\IndexTM{call}
called when Lua calls a value.
\begin{verbatim}
       function function_event (func, ...)
         if type(func) == "function" then
           return func(unpack(arg))   -- regular call
         else
           local h = metatable(func).__call
           if h then
             tinsert(arg, 1, func)
             return h(unpack(arg))
           else
             error("call expression not a function")
           end
         end
       end
\end{verbatim}

\end{description}

\subsubsection{Metatables and Garbage collection}

Metatables may also define \IndexEmph{finalizer} methods
for userdata values.
For each userdata to be collected,
Lua does the equivalent of the following function:
\begin{verbatim}
       function gc_event (obj)
         local h = metatable(obj).__gc
         if h then
           h(obj)
         end
       end
\end{verbatim}
In a garbage-collection cycle,
the finalizers for userdata are called in \emph{reverse}
order of their creation,
that is, the first finalizer to be called is the one associated
with the last userdata created in the program
(among those to be collected in the same cycle).



%------------------------------------------------------------------------------
\section{The Application Program Interface}\label{API}
\index{C API}

This section describes the API for Lua, that is,
the set of C~functions available to the host program to communicate
with Lua.
All API functions and related types and constants
are declared in the header file \verb|lua.h|.

\NOTE
Even when we use the term ``function'',
any facility in the API may be provided as a \emph{macro} instead.
All such macros use each of its arguments exactly once
(except for the first argument, which is always a Lua state),
and so do not generate hidden side-effects.


\subsection{States} \label{mangstate}

The Lua library is fully reentrant:
it has no global variables.
\index{state}
The whole state of the Lua interpreter
(global variables, stack, tag methods, etc.)\
is stored in a dynamically allocated structure of type \verb|lua_State|;
\DefAPI{lua_State}
this state must be passed as the first argument to
every function in the library (except \verb|lua_open| below).

Before calling any API function,
you must create a state by calling
\begin{verbatim}
       lua_State *lua_open (void);
\end{verbatim}
\DefAPI{lua_open}

To release a state created with \verb|lua_open|, call
\begin{verbatim}
       void lua_close (lua_State *L);
\end{verbatim}
\DefAPI{lua_close}
This function destroys all objects in the given Lua environment
(calling the corresponding garbage-collection metamethods, if any)
and frees all dynamic memory used by that state.
Usually, you do not need to call this function,
because all resources are naturally released when your program ends.
On the other hand,
long-running programs ---
like a daemon or a web server ---
might need to release states as soon as they are not needed,
to avoid growing too large.

With the exception of \verb|lua_open|,
all functions in the Lua API need a state as their first argument.


\subsection{Threads}

Lua offers a partial support for multiple threads of execution.
If you have a C~library that offers multi-threading, 
then Lua can cooperate with it to implement the equivalent facility in Lua.
Also, Lua implements its own coroutine system on top of threads.
The following function creates a new ``thread'' in Lua:
\begin{verbatim}
       lua_State *lua_newthread (lua_State *L);
\end{verbatim}
\DefAPI{lua_newthread}
The new state returned by this function shares with the original state
all global environment (such as tables, tag methods, etc.),
but has an independent run-time stack.
(The use of these multiple stacks must be ``syncronized'' with C.
How to explain that? TO BE WRITTEN.)

Each thread has an independent table for global variables.
When you create a thread, this table is the same as that of the given state,
but you can change each one independently.

You destroy threads with \DefAPI{lua_closethread}
\begin{verbatim}
       void lua_closethread (lua_State *L, lua_State *thread);
\end{verbatim}
You cannot close the sole (or last) thread of a state.
Instead, you must close the state itself.


\subsection{The Stack and Indices}

Lua uses a virtual \emph{stack} to pass values to and from C.
Each element in this stack represents a Lua value
(\nil, number, string, etc.).

Each C invocation has its own stack.
Whenever Lua calls C, the called function gets a new stack,
which is independent of previous stacks or of stacks of still
active C functions.

For convenience,
most query operations in the API do not follow a strict stack discipline.
Instead, they can refer to any element in the stack by using an \emph{index}:
A positive index represents an \emph{absolute} stack position
(starting at~1);
a negative index represents an \emph{offset} from the top of the stack.
More specifically, if the stack has \M{n} elements,
then index~1 represents the first element
(that is, the element that was pushed onto the stack first),
and
index~\M{n} represents the last element;
index~\Math{-1} also represents the last element
(that is, the element at the top),
and index \Math{-n} represents the first element.
We say that an index is \emph{valid}
if it lies between~1 and the stack top
(that is, if \verb|1 <= abs(index) <= top|).
\index{stack index} \index{valid index}

At any time, you can get the index of the top element by calling
\begin{verbatim}
       int lua_gettop (lua_State *L);
\end{verbatim}
\DefAPI{lua_gettop}
Because indices start at~1,
the result of \verb|lua_gettop| is equal to the number of elements in the stack
(and so 0~means an empty stack).

When you interact with Lua API,
\emph{you are responsible for controlling stack overflow}.
The function
\begin{verbatim}
       int lua_checkstack (lua_State *L, int extra);
\end{verbatim}
\DefAPI{lua_checkstack}
grows the stack size to \verb|top + extra| elements;
it returns false if it cannot grow the stack to that size.
This function never shrinks the stack;
if the stack is already bigger than the new size,
it is left unchanged.

Whenever Lua calls C, \DefAPI{LUA_MINSTACK}
it ensures that \verb|lua_checkstack(L, LUA_MINSTACK)| is true,
that is,
at least \verb|LUA_MINSTACK| positions are still available.
\verb|LUA_MINSTACK| is defined in \verb|lua.h| as 20,
so that usually you do not have to worry about stack space
unless your code has loops pushing elements onto the stack.

Most query functions accept as indices any value inside the
available stack space, that is, indices up to the maximum stack size
you (or Lua) have set through \verb|lua_checkstack|.
Such indices are called \emph{acceptable indices}.
More formally, we define an \IndexEmph{acceptable index}
as follows:
\begin{verbatim}
     (index < 0 && abs(index) <= top) || (index > 0 && index <= top + stackspace)
\end{verbatim}
Note that 0 is never an acceptable index.

Unless otherwise noticed,
any function that accepts valid indices can also be called with
\Index{pseudo-indices},
which represent some Lua values that are accessible to the C~code
but are not in the stack.
Pseudo-indices are used to access the table of globals \see{globals},
the registry, and the upvalues of a C function \see{c-closure}.

\subsection{Stack Manipulation}
The API offers the following functions for basic stack manipulation:
\begin{verbatim}
       void lua_settop    (lua_State *L, int index);
       void lua_pushvalue (lua_State *L, int index);
       void lua_remove    (lua_State *L, int index);
       void lua_insert    (lua_State *L, int index);
       void lua_replace   (lua_State *L, int index);
\end{verbatim}
\DefAPI{lua_settop}\DefAPI{lua_pushvalue}
\DefAPI{lua_remove}\DefAPI{lua_insert}\DefAPI{lua_replace}

\verb|lua_settop| accepts any acceptable index,
or 0,
and sets the stack top to that index.
If the new top is larger than the old one,
then the new elements are filled with \nil.
If \verb|index| is 0, then all stack elements are removed.
A useful macro defined in the \verb|lua.h| is
\begin{verbatim}
       #define lua_pop(L,n) lua_settop(L, -(n)-1)
\end{verbatim}
\DefAPI{lua_pop}
which pops \verb|n| elements from the stack.

\verb|lua_pushvalue| pushes onto the stack a copy of the element
at the given index.
\verb|lua_remove| removes the element at the given position,
shifting down the elements above that position to fill the gap.
\verb|lua_insert| moves the top element into the given position,
shifting up the elements above that position to open space.
\verb|lua_replace| moves the top element into the given position,
without shifting any element (therefore replacing the value at
the given position).
These functions accept only valid indices.
(Obviously, you cannot call \verb|lua_remove| or \verb|lua_insert| with
pseudo-indices, as they do not represent a stack position.)

As an example, if the stack starts as \verb|10 20 30 40 50*|
(from bottom to top; the \verb|*| marks the top),
then
\begin{verbatim}
       lua_pushvalue(L, 3)    --> 10 20 30 40 50 30*
       lua_pushvalue(L, -1)   --> 10 20 30 40 50 30 30*
       lua_remove(L, -3)      --> 10 20 30 40 30 30*
       lua_remove(L,  6)      --> 10 20 30 40 30*
       lua_insert(L,  1)      --> 30 10 20 30 40*
       lua_insert(L, -1)      --> 30 10 20 30 40*  (no effect)
       lua_replace(L, 2)      --> 30 40 20 30*
       lua_settop(L, -3)      --> 30 40 20*
       lua_settop(L,  6)      --> 30 40 20 nil nil nil*
\end{verbatim}



\subsection{Querying the Stack}

To check the type of a stack element,
the following functions are available:
\begin{verbatim}
       int         lua_type        (lua_State *L, int index);
       int         lua_isnil       (lua_State *L, int index);
       int         lua_isboolean   (lua_State *L, int index);
       int         lua_isnumber    (lua_State *L, int index);
       int         lua_isstring    (lua_State *L, int index);
       int         lua_istable     (lua_State *L, int index);
       int         lua_isfunction  (lua_State *L, int index);
       int         lua_iscfunction (lua_State *L, int index);
       int         lua_isuserdata  (lua_State *L, int index);
       int         lua_isdataval   (lua_State *L, int index);
\end{verbatim}
\DefAPI{lua_type}
\DefAPI{lua_isnil}\DefAPI{lua_isnumber}\DefAPI{lua_isstring}
\DefAPI{lua_istable}\DefAPI{lua_isboolean}
\DefAPI{lua_isfunction}\DefAPI{lua_iscfunction}
\DefAPI{lua_isuserdata}\DefAPI{lua_isdataval}
These functions can be called with any acceptable index.

\verb|lua_type| returns the type of a value in the stack,
or \verb|LUA_TNONE| for a non-valid index
(that is, if that stack position is ``empty'').
The types are coded by the following constants
defined in \verb|lua.h|:
\verb|LUA_TNIL|,
\verb|LUA_TNUMBER|,
\verb|LUA_TBOOLEAN|,
\verb|LUA_TSTRING|,
\verb|LUA_TTABLE|,
\verb|LUA_TFUNCTION|,
\verb|LUA_TUSERDATA|,
\verb|LUA_TLIGHTUSERDATA|.
The following function translates such constants to a type name:
\begin{verbatim}
       const char *lua_typename  (lua_State *L, int type);
\end{verbatim}
\DefAPI{lua_typename}

The \verb|lua_is*| functions return~1 if the object is compatible
with the given type, and 0 otherwise.
\verb|lua_isboolean| is an exception to this rule,
and it succeeds only for boolean values
(otherwise it would be useless,
as any value is compatible with a boolean).
They always return 0 for a non-valid index.
\verb|lua_isnumber| accepts numbers and numerical strings,
\verb|lua_isstring| accepts strings and numbers \see{coercion},
and \verb|lua_isfunction| accepts both Lua functions and C~functions.
To distinguish between Lua functions and C~functions,
you should use \verb|lua_iscfunction|.
To distinguish between numbers and numerical strings,
you can use \verb|lua_type|.

The API also has functions to compare two values in the stack:
\begin{verbatim}
       int lua_equal    (lua_State *L, int index1, int index2);
       int lua_lessthan (lua_State *L, int index1, int index2);
\end{verbatim}
\DefAPI{lua_equal} \DefAPI{lua_lessthan}
These functions are equivalent to their counterparts in Lua \see{rel-ops}.
Both functions return 0 if any of the indices are non-valid.

\subsection{Getting Values from the Stack}\label{lua-to}

To translate a value in the stack to a specific C~type,
you can use the following conversion functions:
\begin{verbatim}
       int            lua_toboolean   (lua_State *L, int index);
       lua_Number     lua_tonumber    (lua_State *L, int index);
       const char    *lua_tostring    (lua_State *L, int index);
       size_t         lua_strlen      (lua_State *L, int index);
       lua_CFunction  lua_tocfunction (lua_State *L, int index);
       void          *lua_touserdata  (lua_State *L, int index);
\end{verbatim}
\DefAPI{lua_tonumber}\DefAPI{lua_tostring}\DefAPI{lua_strlen}
\DefAPI{lua_tocfunction}\DefAPI{lua_touserdata}\DefAPI{lua_toboolean}
These functions can be called with any acceptable index.
When called with a non-valid index,
they act as if the given value had an incorrect type.

\verb|lua_toboolean| converts the Lua value at the given index
to a C ``boolean'' value (that is, 0 or 1).
Like all tests in Lua, it returns 1 for any Lua value different from
\False{} and \nil;
otherwise it returns 0.
It also returns 0 when called with a non-valid index.
(If you want to accept only real boolean values,
use \verb|lua_isboolean| to test the type of the value.)

\verb|lua_tonumber| converts the Lua value at the given index
to a number (by default, \verb|lua_Number| is \verb|double|).
\DefAPI{lua_Number}
The Lua value must be a number or a string convertible to number
\see{coercion}; otherwise, \verb|lua_tonumber| returns~0.

\verb|lua_tostring| converts the Lua value at the given index to a string
(\verb|const char*|).
The Lua value must be a string or a number;
otherwise, the function returns \verb|NULL|.
If the value is a number,
then \verb|lua_tostring| also
\emph{changes the actual value in the stack to a string}.
(This change confuses \verb|lua_next|
when \verb|lua_tostring| is applied to keys.)
\verb|lua_tostring| returns a fully aligned pointer
to a string inside the Lua environment.
This string always has a zero (\verb|'\0'|)
after its last character (as in~C),
but may contain other zeros in its body.
If you do not know whether a string may contain zeros,
you can use \verb|lua_strlen| to get its actual length.
Because Lua has garbage collection,
there is no guarantee that the pointer returned by \verb|lua_tostring|
will be valid after the corresponding value is removed from the stack.
So, if you need the string after the current function returns,
then you should duplicate it (or put it into the registry \see{registry}).

\verb|lua_tocfunction| converts a value in the stack to a C~function.
This value must be a C~function;
otherwise, \verb|lua_tocfunction| returns \verb|NULL|.
The type \verb|lua_CFunction| is explained in \See{LuacallC}.

\verb|lua_touserdata| is explained in \See{userdata}.


\subsection{Pushing Values onto the Stack}

The API has the following functions to
push C~values onto the stack:
\begin{verbatim}
       void lua_pushboolean   (lua_State *L, int b);
       void lua_pushnumber    (lua_State *L, lua_Number n);
       void lua_pushlstring   (lua_State *L, const char *s, size_t len);
       void lua_pushstring    (lua_State *L, const char *s);
       void lua_pushnil       (lua_State *L);
       void lua_pushcfunction (lua_State *L, lua_CFunction f);
       void lua_pushlightuserdata  (lua_State *L, void *p);
\end{verbatim}

\DefAPI{lua_pushnumber}\DefAPI{lua_pushlstring}\DefAPI{lua_pushstring}
\DefAPI{lua_pushcfunction}\DefAPI{lua_pushlightuserdata}\DefAPI{lua_pushboolean}
\DefAPI{lua_pushnil}\label{pushing}
These functions receive a C~value,
convert it to a corresponding Lua value,
and push the result onto the stack.
In particular, \verb|lua_pushlstring| and \verb|lua_pushstring|
make an internal copy of the given string.
\verb|lua_pushstring| can only be used to push proper C~strings
(that is, strings that end with a zero and do not contain embedded zeros);
otherwise, you should use the more general \verb|lua_pushlstring|,
which accepts an explicit size.

You can also push ``formatted'' strings:
\begin{verbatim}
       const char *lua_pushfstring  (lua_State *L, const char *fmt, ...);
       const char *lua_pushvfstring (lua_State *L, const char *fmt,
                                                   va_list argp);
\end{verbatim}
\DefAPI{lua_pushfstring}\DefAPI{lua_pushvfstring}
Both functions push onto the stack a formatted string,
and return a pointer to that string.
These functions are similar to \verb|sprintf| and \verb|vsprintf|,
but with some important differences:
\begin{itemize}
\item You do not have to allocate the space for the result;
the result is a Lua string, and Lua takes care of memory allocation
(and deallocation, later).
\item The conversion specifiers are quite restricted.
There are no flags, widths, or precisions.
The conversion specifiers can be simply
\verb|%%| (inserts a \verb|%| in the string),
\verb|%s| (inserts a zero-terminated string, with no size restrictions),
\verb|%f| (inserts a \verb|lua_Number|),
\verb|%d| (inserts an \verb|int|),
\verb|%c| (inserts an \verb|int| as a character).
\end{itemize}


\subsection{Controlling Garbage Collection}\label{GC-API}

Lua uses two numbers to control its garbage collection:
the \emph{count} and the \emph{threshold} \see{GC}.
The first counts the ammount of memory in use by Lua;
when the count reaches the threshold,
Lua runs its garbage collector.
After the collection, the count is updated,
and the threshold  is set to twice the count value.

You can access the current values of these two numbers through the
following functions:
\begin{verbatim}
       int  lua_getgccount (lua_State *L);
       int  lua_getgcthreshold (lua_State *L);
\end{verbatim}
\DefAPI{lua_getgcthreshold} \DefAPI{lua_getgccount}
Both return their respective values in Kbytes.
You can change the threshold value with
\begin{verbatim}
       void  lua_setgcthreshold (lua_State *L, int newthreshold);
\end{verbatim}
\DefAPI{lua_setgcthreshold}
Again, the \verb|newthreshold| value is given in Kbytes.
When you call this function,
Lua sets the new threshold and checks it against the byte counter.
If the new threshold is smaller than the byte counter,
then Lua immediately runs the garbage collector.
In particular
\verb|lua_setgcthreshold(L,0)| forces a garbage collectiion.
After the collection,
a new threshold is set according to the previous rule.

%% TODO do we need a new way to do that??
% If you want to change the adaptive behavior of the garbage collector,
% you can use the garbage-collection tag method for \nil{} %
% to set your own threshold
% (the tag method is called after Lua resets the threshold).


\subsection{Userdata}\label{userdata}

Userdata represents C values in Lua.
Lua supports two types of userdata:
\Def{full userdata} and \Def{light userdata}.

A full userdata represents a block of memory.
It is an object (like a table):
You must create it, it can have its own metatable,
you can detect when it is being collected.
A full userdata is only equal to itself.

A light userdata represents a pointer.
It is a value (like a number):
You do not create it, it has no metatables,
it is not collected (as it was never created).
A light userdata is equal to ``any''
light userdata with the same address.

In Lua code, there is no way to test whether a userdata is full or light;
both have type \verb|userdata|.
In C code, \verb|lua_type| returns \verb|LUA_TUSERDATA| for full userdata,
and \verb|LUA_LIGHTUSERDATA| for light userdata.

You can create new full userdata with the following function:
\begin{verbatim}
       void *lua_newuserdata (lua_State *L, size_t size);
\end{verbatim}
\DefAPI{lua_newuserdata}
It allocates a new block of memory with the given size,
pushes on the stack a new userdata with the block address,
and returns this address.

To push a light userdata into the stack you use
\verb|lua_pushlightuserdata| \see{pushing}.

\verb|lua_touserdata| \see{lua-to} retrieves the value of a userdata.
When applied on a full userdata, it returns the address of its block;
when applied on a light userdata, it returns its pointer;
when applied on a non-userdata value, it returns \verb|NULL|.

When Lua collects a full userdata,
it calls its \verb|gc| metamethod, if any,
and then it automatically frees its corresponding memory.


\subsection{Metatables}

%% TODO

\subsection{Loading Lua Chunks}
You can load a Lua chunk with
\begin{verbatim}
       typedef const char * (*lua_Chunkreader)
                                (lua_State *L, void *data, size_t *size);

       int lua_load (lua_State *L, lua_Chunkreader reader, void *data,
                                   const char *chunkname);
\end{verbatim}
\DefAPI{Chunkreader}\DefAPI{lua_load}
\verb|lua_load| uses the \emph{reader} to read the chunk.
Everytime it needs another piece of the chunk,
it calls the reader,
passing along its \verb|data| parameter.
The reader must return a pointer to a block of memory
with the part of the chunk,
and set \verb|size| to the block size.
To signal the end of the chunk, the reader must return \verb|NULL|.

In the current implementation,
the reader function cannot call any Lua function;
to ensure that, it always receives \verb|NULL| as the Lua state.

\verb|lua_load| automatically detects whether the chunk is text or binary,
and loads it accordingly (see program \IndexVerb{luac}).

The return values of \verb|lua_load| are:
\begin{itemize}
\item 0 --- no errors;
\item \IndexAPI{LUA_ERRSYNTAX} ---
syntax error during pre-compilation.
\item \IndexAPI{LUA_ERRMEM} ---
memory allocation error.
\end{itemize}
If there are no errors,
the compiled chunk is pushed as a Lua function on top of the stack.
Otherwise, an error message is pushed.

The \emph{chunkname} is used for error messages
and debug information \see{debugI}.

See the auxiliar library (\verb|lauxlib|)
for examples of how to use \verb|lua_load|,
and for some ready-to-use functions to load chunks
from files and from strings.


\subsection{Executing Lua Chunks}\label{luado}
>>>>
A host program can execute Lua chunks written in a file or in a string
by using the following functions:
\begin{verbatim}
       int lua_dofile   (lua_State *L, const char *filename);
       int lua_dostring (lua_State *L, const char *string);
       int lua_dobuffer (lua_State *L, const char *buff,
                         size_t size, const char *name);
\end{verbatim}
\DefAPI{lua_dofile}\DefAPI{lua_dostring}\DefAPI{lua_dobuffer}%
These functions return
0 in case of success, or one of the following error codes
(defined in \verb|lua.h|)
if they fail:
\begin{itemize}
\item \IndexAPI{LUA_ERRRUN} ---
error while running the chunk.
\item \IndexAPI{LUA_ERRSYNTAX} ---
syntax error during pre-compilation.
\item \IndexAPI{LUA_ERRMEM} ---
memory allocation error.
For such errors, Lua does not call \verb|_ERRORMESSAGE| \see{error}.
\item \IndexAPI{LUA_ERRERR} ---
error while running \verb|_ERRORMESSAGE|.
For such errors, Lua does not call \verb|_ERRORMESSAGE| again, to avoid loops.
\item \IndexAPI{LUA_ERRFILE} ---
error opening the file (only for \verb|lua_dofile|).
In this case,
you may want to
check \verb|errno|,
call \verb|strerror|,
or call \verb|perror| to tell the user what went wrong.
\end{itemize}


\subsection{Manipulating Tables}

Tables are created by calling
the function
\begin{verbatim}
       void lua_newtable (lua_State *L);
\end{verbatim}
\DefAPI{lua_newtable}
This function creates a new, empty table and pushes it onto the stack.

To read a value from a table that resides somewhere in the stack,
call
\begin{verbatim}
       void lua_gettable (lua_State *L, int index);
\end{verbatim}
\DefAPI{lua_gettable}
where \verb|index| points to the table.
\verb|lua_gettable| pops a key from the stack
and returns (on the stack) the contents of the table at that key.
The table is left where it was in the stack;
this is convenient for getting multiple values from a table.

As in Lua, this function may trigger a metamethod
for the ``gettable'' or ``index'' events \see{metatable}.
To get the real value of any table key,
without invoking any metamethod,
use the \emph{raw} version:
\begin{verbatim}
       void lua_rawget (lua_State *L, int index);
\end{verbatim}
\DefAPI{lua_rawget}

To store a value into a table that resides somewhere in the stack,
you push the key and the value onto the stack
(in this order),
and then call
\begin{verbatim}
       void lua_settable (lua_State *L, int index);
\end{verbatim}
\DefAPI{lua_settable}
where \verb|index| points to the table.
\verb|lua_settable| pops from the stack both the key and the value.
The table is left where it was in the stack;
this is convenient for setting multiple values in a table.

As in Lua, this operation may trigger a metamethod
for the ``settable'' or ``newindex'' events.
To set the real value of any table index,
without invoking any metamethod,
use the \emph{raw} version:
\begin{verbatim}
       void lua_rawset (lua_State *L, int index);
\end{verbatim}
\DefAPI{lua_rawset}

You can traverse a table with the function
\begin{verbatim}
       int lua_next (lua_State *L, int index);
\end{verbatim}
\DefAPI{lua_next}
where \verb|index| points to the table to be traversed.
The function pops a key from the stack,
and pushes a key-value pair from the table
(the ``next'' pair after the given key).
If there are no more elements, then \verb|lua_next| returns 0
(and pushes nothing).
Use a \nil{} key to signal the start of a traversal.

A typical traversal looks like this:
\begin{verbatim}
       /* table is in the stack at index `t' */
       lua_pushnil(L);  /* first key */
       while (lua_next(L, t) != 0) {
         /* `key' is at index -2 and `value' at index -1 */
         printf("%s - %s\n",
           lua_typename(L, lua_type(L, -2)), lua_typename(L, lua_type(L, -1)));
         lua_pop(L, 1);  /* removes `value'; keeps `key' for next iteration */
       }
\end{verbatim}

NOTE:
While traversing a table,
do not call \verb|lua_tostring| on a key,
unless you know the key is actually a string.
Recall that \verb|lua_tostring| \emph{changes} the value at the given index;
this confuses the next call to \verb|lua_next|.

\subsection{Manipulating Global Variables} \label{globals}

All global variables are kept in an ordinary Lua table.
This table is always at pseudo-index \IndexAPI{LUA_GLOBALSINDEX}.

To access and change the value of global variables,
you can use regular table operations over the global table.
For instance, to access the value of a global variable, do
\begin{verbatim}
       lua_pushstring(L, varname);
       lua_gettable(L, LUA_GLOBALSINDEX);
\end{verbatim}

You can change the global table of a Lua thread using \verb|lua_replace|.


\subsection{Using Tables as Arrays}
The API has functions that help to use Lua tables as arrays,
that is,
tables indexed by numbers only:
\begin{verbatim}
       void lua_rawgeti (lua_State *L, int index, int n);
       void lua_rawseti (lua_State *L, int index, int n);
\end{verbatim}
\DefAPI{lua_rawgeti}
\DefAPI{lua_rawseti}

\verb|lua_rawgeti| pushes the value of the \M{n}-th element of the table
at stack position \verb|index|.
\verb|lua_rawseti| sets the value of the \M{n}-th element of the table
at stack position \verb|index| to the value at the top of the stack,
removing this value from the stack.


\subsection{Calling Functions}

Functions defined in Lua
and C~functions registered in Lua
can be called from the host program.
This is done using the following protocol:
First, the function to be called is pushed onto the stack;
then, the arguments to the function are pushed
in \emph{direct order}, that is, the first argument is pushed first.
Finally, the function is called using
\begin{verbatim}
       void lua_call (lua_State *L, int nargs, int nresults);
\end{verbatim}
\DefAPI{lua_call}
\verb|nargs| is the number of arguments that you pushed onto the stack.
All arguments and the function value are popped from the stack,
and the function results are pushed.
The number of results are adjusted to \verb|nresults|,
unless \verb|nresults| is \IndexAPI{LUA_MULTRET}.
In that case, \emph{all} results from the function are pushed.
Lua takes care that the returned values fit into the stack space.
The function results are pushed onto the stack in direct order
(the first result is pushed first),
so that after the call the last result is on the top.

The following example shows how the host program may do the
equivalent to the Lua code:
\begin{verbatim}
       a = f("how", t.x, 14)
\end{verbatim}
Here it is in~C:
\begin{verbatim}
    lua_pushstring(L, "t");
    lua_gettable(L, LUA_GLOBALSINDEX);          /* global `t' (for later use) */
    lua_pushstring(L, "a");                                       /* var name */
    lua_pushstring(L, "f");                                  /* function name */
    lua_gettable(L, LUA_GLOBALSINDEX);               /* function to be called */
    lua_pushstring(L, "how");                                 /* 1st argument */
    lua_pushstring(L, "x");                            /* push the string "x" */
    lua_gettable(L, -5);                      /* push result of t.x (2nd arg) */
    lua_pushnumber(L, 14);                                    /* 3rd argument */
    lua_call(L, 3, 1);         /* call function with 3 arguments and 1 result */
    lua_settable(L, LUA_GLOBALSINDEX);             /* set global variable `a' */
    lua_pop(L, 1);                               /* remove `t' from the stack */
\end{verbatim}
Notice that the code above is ``balanced'':
at its end, the stack is back to its original configuration.
This is considered good programming practice.

(We did this example using only the raw functions provided by Lua's API,
to show all the details.
Usually programmers use several macros and auxiliar functions that
provide higher level access to Lua.)

%% TODO: pcall

\medskip

>>>>
%% TODO: mover essas 2 para algum lugar melhor.
Some special Lua functions have their own C~interfaces.
The host program can generate a Lua error calling the function
\begin{verbatim}
       void lua_error (lua_State *L, const char *message);
\end{verbatim}
\DefAPI{lua_error}
This function never returns.
If \verb|lua_error| is called from a C~function that has been called from Lua,
then the corresponding Lua execution terminates,
as if an error had occurred inside Lua code.
Otherwise, the whole host program terminates with a call to
\verb|exit(EXIT_FAILURE)|.
Before terminating execution,
the \verb|message| is passed to the error handler function,
\verb|_ERRORMESSAGE| \see{error}.
If \verb|message| is \verb|NULL|,
then \verb|_ERRORMESSAGE| is not called.

The function
\begin{verbatim}
       void lua_concat (lua_State *L, int n);
\end{verbatim}
\DefAPI{lua_concat}
concatenates the \verb|n| values at the top of the stack,
pops them, and leaves the result at the top.
If \verb|n| is 1, the result is that single string
(that is, the function does nothing);
if \verb|n| is 0, the result is the empty string.
Concatenation is done following the usual semantics of Lua
\see{concat}.


\subsection{Defining C Functions} \label{LuacallC}
Lua can be extended with functions written in~C.
These functions must be of type \verb|lua_CFunction|,
which is defined as
\begin{verbatim}
       typedef int (*lua_CFunction) (lua_State *L);
\end{verbatim}
\DefAPI{lua_CFunction}
A C~function receives a Lua environment and returns an integer,
the number of values it has returned to Lua.

In order to communicate properly with Lua,
a C~function must follow the following protocol,
which defines the way parameters and results are passed:
A C~function receives its arguments from Lua in the stack,
in direct order (the first argument is pushed first).
To return values to Lua, a C~function just pushes them onto the stack,
in direct order (the first result is pushed first),
and returns the number of results.
Like a Lua function, a C~function called by Lua can also return
many results.

As an example, the following function receives a variable number
of numerical arguments and returns their average and sum:
\begin{verbatim}
       static int foo (lua_State *L) {
         int n = lua_gettop(L);    /* number of arguments */
         lua_Number sum = 0;
         int i;
         for (i = 1; i <= n; i++) {
           if (!lua_isnumber(L, i))
             lua_error(L, "incorrect argument to function `average'");
           sum += lua_tonumber(L, i);
         }
         lua_pushnumber(L, sum/n);        /* first result */
         lua_pushnumber(L, sum);         /* second result */
         return 2;                   /* number of results */
       }
\end{verbatim}

To register a C~function to Lua,
there is the following convenience macro:
\begin{verbatim}
       #define lua_register(L,n,f) \
               (lua_pushstring(L, n), \
                lua_pushcfunction(L, f), \
                lua_settable(L, LUA_GLOBALSINDEX))
     /* const char *n;   */
     /* lua_CFunction f; */
\end{verbatim}
\DefAPI{lua_register}
which receives the name the function will have in Lua,
and a pointer to the function.
Thus,
the C~function `\verb|foo|' above may be registered in Lua as `\verb|average|'
by calling
\begin{verbatim}
       lua_register(L, "average", foo);
\end{verbatim}

\subsection{Defining C Closures} \label{c-closure}

When a C~function is created,
it is possible to associate some values to it,
thus creating a \IndexEmph{C~closure};
these values are then accessible to the function whenever it is called.
To associate values to a C~function,
first these values should be pushed onto the stack
(when there are multiple values, the first value is pushed first).
Then the function
\begin{verbatim}
       void lua_pushcclosure (lua_State *L, lua_CFunction fn, int n);
\end{verbatim}
\DefAPI{lua_pushcclosure}
is used to push the C~function onto the stack,
with the argument \verb|n| telling how many values should be
associated with the function
(\verb|lua_pushcclosure| also pops these values from the stack);
in fact, the macro \verb|lua_pushcfunction| is defined as
\verb|lua_pushcclosure| with \verb|n| set to 0.

Then, whenever the C~function is called,
those values are located at specific pseudo-indices.
Those pseudo-indices are produced by a macro \IndexAPI{lua_upvalueindex}.
The first value associated with a function is at position
\verb|lua_upvalueindex(1)|, and so on.

For examples of C~functions and closures, see files
\verb|lbaselib.c|, \verb|liolib.c|, \verb|lmathlib.c|, and \verb|lstrlib.c|
in the official Lua distribution.


\subsubsection*{Registry} \label{registry}

Lua provides a pre-defined table that can be used by any C~code to
store whatever Lua value it needs to store,
especially if the C~code needs to keep that Lua value
outside the life span of a C~function.
This table is always located at pseudo-index
\IndexAPI{LUA_REGISTRYINDEX}.
Any C~library can store data into this table,
as long as it chooses a key different from other libraries.
Typically, you can use as key a string containing the library name,
or a light userdata with the address of a C object in your code.

The integer keys in the registry are used by the reference mechanism,
implemented by the auxiliar library,
and therefore should not be used by other purposes.


%------------------------------------------------------------------------------
\section{The Debug Interface} \label{debugI}

Lua has no built-in debugging facilities.
Instead, it offers a special interface,
by means of functions and \emph{hooks},
which allows the construction of different
kinds of debuggers, profilers, and other tools
that need ``inside information'' from the interpreter.
This interface is declared in \verb|luadebug.h|.

\subsection{Stack and Function Information}

The main function to get information about the interpreter stack is
\begin{verbatim}
       int lua_getstack (lua_State *L, int level, lua_Debug *ar);
\end{verbatim}
\DefAPI{lua_getstack}
This function fills parts of a \verb|lua_Debug| structure with
an identification of the \emph{activation record}
of the function executing at a given level.
Level~0 is the current running function,
whereas level \Math{n+1} is the function that has called level \Math{n}.
Usually, \verb|lua_getstack| returns 1;
when called with a level greater than the stack depth,
it returns 0.

The structure \verb|lua_Debug| is used to carry different pieces of
information about an active function:
\begin{verbatim}
      typedef struct lua_Debug {
        const char *event;     /* "call", "return" */
        int currentline;       /* (l) */
        const char *name;      /* (n) */
        const char *namewhat;  /* (n) `global', `local', `field', `method' */
        int nups;              /* (u) number of upvalues */
        int linedefined;       /* (S) */
        const char *what;      /* (S) "Lua" function, "C" function, Lua "main" */
        const char *source;    /* (S) */
        char short_src[LUA_IDSIZE]; /* (S) */

        /* private part */
        ...
      } lua_Debug;
\end{verbatim}
\DefAPI{lua_Debug}
\verb|lua_getstack| fills only the private part
of this structure, for future use.
To fill the other fields of \verb|lua_Debug| with useful information,
call
\begin{verbatim}
       int lua_getinfo (lua_State *L, const char *what, lua_Debug *ar);
\end{verbatim}
\DefAPI{lua_getinfo}
This function returns 0 on error
(for instance, an invalid option in \verb|what|).
Each character in the string \verb|what|
selects some fields of \verb|ar| to be filled,
as indicated by the letter in parentheses in the definition of \verb|lua_Debug|
above:
`\verb|S|' fills in the fields \verb|source|, \verb|linedefined|,
and \verb|what|;
`\verb|l|' fills in the field \verb|currentline|, etc.
Moreover, `\verb|f|' pushes onto the stack the function that is
running at the given level.

To get information about a function that is not active (that is,
it is not in the stack),
you push the function onto the stack,
and start the \verb|what| string with the character `\verb|>|'.
For instance, to know in which line a function \verb|f| was defined,
you can write
\begin{verbatim}
       lua_Debug ar;
       lua_pushstring(L, "f");
       lua_gettable(L, LUA_GLOBALSINDEX);  /* get global `f' */
       lua_getinfo(L, ">S", &ar);
       printf("%d\n", ar.linedefined);
\end{verbatim}
The fields of \verb|lua_Debug| have the following meaning:
\begin{description}\leftskip=20pt

\item[source]
If the function was defined in a string,
then \verb|source| is that string;
if the function was defined in a file,
then \verb|source| starts with a \verb|@| followed by the file name.

\item[short\_src]
A ``printable'' version of \verb|source|, to be used in error messages.

\item[linedefined]
the line number where the definition of the function starts.

\item[what] the string \verb|"Lua"| if this is a Lua function,
\verb|"C"| if this is a C~function,
or \verb|"main"| if this is the main part of a chunk.

\item[currentline]
the current line where the given function is executing.
When no line information is available,
\verb|currentline| is set to \Math{-1}.

\item[name]
a reasonable name for the given function.
Because functions in Lua are first class values,
they do not have a fixed name:
Some functions may be the value of many global variables,
while others may be stored only in a table field.
The \verb|lua_getinfo| function checks whether the given
function is a tag method or the value of a global variable.
If the given function is a tag method,
then \verb|name| points to the event name.
%% TODO: mas qual o tag? Agora que temos tipos com nome, seria util saber
%% o tipo de TM. Em particular para mensagens de erro.
If the given function is the value of a global variable,
then \verb|name| points to the variable name.
If the given function is neither a tag method nor a global variable,
then \verb|name| is set to \verb|NULL|.

\item[namewhat]
Explains the previous field.
It can be \verb|"global"|, \verb|"local"|, \verb|"method"|,
\verb|"field"|, or \verb|""| (the empty string),
according to how the function was called.
(Lua uses the empty string when no other option seems to apply.)

\item[nups]
Number of upvalues of the function.

\end{description}


\subsection{Manipulating Local Variables}

For the manipulation of local variables,
\verb|luadebug.h| uses indices:
The first parameter or local variable has index~1, and so on,
until the last active local variable.

The following functions allow the manipulation of the
local variables of a given activation record:
\begin{verbatim}
       const char *lua_getlocal (lua_State *L, const lua_Debug *ar, int n);
       const char *lua_setlocal (lua_State *L, const lua_Debug *ar, int n);
\end{verbatim}
\DefAPI{lua_getlocal}\DefAPI{lua_setlocal}
The parameter \verb|ar| must be a valid activation record,
filled by a previous call to \verb|lua_getstack| or
given as argument to a hook \see{sub-hooks}.
\verb|lua_getlocal| gets the index \verb|n| of a local variable,
pushes its value onto the stack,
and returns its name.
%% TODO: why return name?
\verb|lua_setlocal| assigns the value at the top of the stack
to the variable and returns its name.
Both functions return \verb|NULL| on failure,
that is
when the index is greater than
the number of active local variables.

As an example, the following function lists the names of all
local variables for a function at a given level of the stack:
\begin{verbatim}
       int listvars (lua_State *L, int level) {
         lua_Debug ar;
         int i = 1;
         const char *name;
         if (lua_getstack(L, level, &ar) == 0)
           return 0;  /* failure: no such level in the stack */
         while ((name = lua_getlocal(L, &ar, i++)) != NULL) {
           printf("%s\n", name);
           lua_pop(L, 1);  /* remove variable value */
         }
         return 1;
       }
\end{verbatim}


\subsection{Hooks}\label{sub-hooks}

The Lua interpreter offers two hooks for debugging purposes:
a \emph{call} hook and a \emph{line} hook.
Both have type \verb|lua_Hook|, defined as follows:
\begin{verbatim}
       typedef void (*lua_Hook) (lua_State *L, lua_Debug *ar);
\end{verbatim}
\DefAPI{lua_Hook}
You can set the hooks with the following functions:
\begin{verbatim}
       lua_Hook lua_setcallhook (lua_State *L, lua_Hook func);
       lua_Hook lua_setlinehook (lua_State *L, lua_Hook func);
\end{verbatim}
\DefAPI{lua_setcallhook}\DefAPI{lua_setlinehook}
A hook is disabled when its value is \verb|NULL|,
which is the initial value of both hooks.
The functions \verb|lua_setcallhook| and \verb|lua_setlinehook|
set their corresponding hooks and return their previous values.

The call hook is called whenever the
interpreter enters or leaves a function.
The \verb|event| field of \verb|ar| has the string \verb|"call"|
or \verb|"return"|.
This \verb|ar| can then be used in calls to \verb|lua_getinfo|,
\verb|lua_getlocal|, and \verb|lua_setlocal|
to get more information about the function and to manipulate its
local variables.

The line hook is called every time the interpreter changes
the line of code it is executing.
The \verb|event| field of \verb|ar| has the string \verb|"line"|,
and the \verb|currentline| field has the new line number.
Again, you can use this \verb|ar| in other calls to the debug API.

While Lua is running a hook, it disables other calls to hooks.
Therefore, if a hook calls Lua to execute a function or a chunk,
this execution ocurrs without any calls to hooks.


%------------------------------------------------------------------------------
\section{Standard Libraries}\label{libraries}

The standard libraries provide useful functions
that are implemented directly through the standard C~API.
Some of these functions provide essential services to the language
(e.g. \verb|type| and \verb|getmetatable|);
others provide access to ``outside'' servides (e.g. I/O);
and others could be implemented in Lua itself,
but are quite useful or have critical performance to
deserve an implementation in C (e.g. \verb|sort|).

All libraries are implemented through the official C API,
and are provided as separate C~modules.
Currently, Lua has the following standard libraries:
\begin{itemize}
\item basic library;
\item string manipulation;
\item table manipulation;
\item mathematical functions (sin, log, etc.);
\item input and output;
\item operating system facilities;
\item debug facilities.
\end{itemize}
Except for the basic library,
each library provides all its functions as fields of a global table
or as methods of its objects.

To have access to these libraries,
the C~host program must call the functions
\verb|lua_baselibopen|,
\verb|lua_strlibopen|,
\verb|lua_tablibopen|,
\verb|lua_mathlibopen|,
and \verb|lua_iolibopen|, which are declared in \verb|lualib.h|.
\DefAPI{lua_baselibopen}
\DefAPI{lua_strlibopen}
\DefAPI{lua_tablibopen}
\DefAPI{lua_mathlibopen}
\DefAPI{lua_iolibopen}


\subsection{Basic Functions} \label{predefined}

The basic library provides some core functions to Lua.
If you do not include this library in your application,
you should check carefully whether you need to provide some alternative
implementation for some facilities.

The basic library also defines a global variable \IndexAPI{_VERSION}
with a string containing the current interpreter version.
The current content of this string is {\tt "Lua \Version"}.

\subsubsection*{\ff \T{assert (v [, message])}}\DefLIB{assert}
Issues an \emph{``assertion failed!''} error
when its argument \verb|v| is \nil;
otherwise, returns this argument.
This function is equivalent to the following Lua function:
\begin{verbatim}
       function assert (v, m)
         if not v then
           error(m or "assertion failed!")
         end
         return v
       end
\end{verbatim}

??\subsubsection*{\ff \T{call (func, arg [, mode [, errhandler]])}}\DefLIB{call}
\label{pdf-call}
Calls function \verb|func| with
the arguments given by the table \verb|arg|.
The call is equivalent to
\begin{verbatim}
       func(arg[1], arg[2], ..., arg[n])
\end{verbatim}
where \verb|n| is the result of \verb|getn(arg)| \see{getn}.
All results from \verb|func| are simply returned by \verb|call|.

By default,
if an error occurs during the call to \verb|func|,
the error is propagated.
If the string \verb|mode| contains \verb|"x"|,
then the call is \emph{protected}.\index{protected calls}
In this mode, function \verb|call| does not propagate an error,
regardless of what happens during the call.
Instead, it returns \nil{} to signal the error
(besides calling the appropriated error handler).

If \verb|errhandler| is provided,
the error function \verb|_ERRORMESSAGE| is temporarily set to \verb|errhandler|,
while \verb|func| runs.
In particular, if \verb|errhandler| is \nil,
no error messages will be issued during the execution of the called function.

\subsubsection*{\ff \T{collectgarbage ([limit])}}\DefLIB{collectgarbage}

Sets the garbage-collection threshold for the given limit
(in Kbytes), and checks it against the byte counter.
If the new threshold is smaller than the byte counter,
then Lua immediately runs the garbage collector \see{GC}.
If \verb|limit| is absent, it defaults to zero
(thus forcing a garbage-collection cycle).

\subsubsection*{\ff \T{dofile (filename)}}\DefLIB{dofile}
Receives a file name,
opens the named file, and executes its contents as a Lua chunk.
When called without arguments,
\verb|dofile| executes the contents of the standard input (\verb|stdin|).
Returns any value returned by the chunk.

\subsubsection*{\ff \T{error ([message])}}\DefLIB{error}\label{pdf-error}
Terminates the last protected function called,
and returns \verb|message| as the error message.
Function \verb|error| never returns.

\subsubsection*{\ff \T{getglobals (function)}}\DefLIB{getglobals}
Returns the current table of globals in use by the function.
\verb|function| can be a Lua function or a number,
meaning the function at that stack level:
Level 1 is the function calling \verb|getglobals|.
If the given function is not a Lua function,
returns the ``global'' table of globals.
The default for \verb|function| is 1.

\subsubsection*{\ff \T{getmetatable (object)}}
\DefLIB{getmetatable}\label{pdf-getmetatable}

Returns the metatable of the given object.
If the object does not have a metatable, returns \nil.

\subsubsection*{\ff \T{getmode (table)}}\DefLIB{getmode}

Returns the weak mode of a table, as a string.
Valid values for this string are \verb|""| for regular (non-weak) tables,
\verb|"k"| for weak keys, \verb|"v"| for weak values,
and \verb|"kv"| for both.

\subsubsection*{\ff \T{gcinfo ()}}\DefLIB{gcinfo}
Returns the number of Kbytes of dynamic memory Lua is using,
and (as a second result) the
current garbage collector threshold (also in Kbytes).

\subsubsection*{\ff \T{loadfile (filename)}}\DefLIB{loadfile}
Loads a file as a Lua chunk.
If there is no errors, 
returns the compiled chunk as a function;
otherwise, returns \nil{} plus an error message.

\subsubsection*{\ff \T{loadstring (string [, chunkname])}}\DefLIB{loadstring}
Loads a string as a Lua chunk.
If there is no errors, 
returns the compiled chunk as a function;
otherwise, returns \nil{} plus an error message.

The optional parameter \verb|chunkname|
is the ``name of the chunk'',
used in error messages and debug information.

To load and run a given string, use the idiom
\begin{verbatim}
      assert(loadstring(s))()
\end{verbatim}

\subsubsection*{\ff \T{next (table, [index])}}\DefLIB{next}
Allows a program to traverse all fields of a table.
Its first argument is a table and its second argument
is an index in this table.
\verb|next| returns the next index of the table and the
value associated with the index.
When called with \nil{} as its second argument,
\verb|next| returns the first index
of the table and its associated value.
When called with the last index,
or with \nil{} in an empty table,
\verb|next| returns \nil.
If the second argument is absent, then it is interpreted as \nil.

Lua has no declaration of fields;
semantically, there is no difference between a
field not present in a table or a field with value \nil.
Therefore, \verb|next| only considers fields with non-\nil{} values.
The order in which the indices are enumerated is not specified,
\emph{even for numeric indices}
(to traverse a table in numeric order,
use a numerical \rwd{for} or the function \verb|ipairs|).

The behavior of \verb|next| is \emph{undefined} if you change
the table during the traversal.

\subsubsection*{\ff \T{print (e1, e2, ...)}}\DefLIB{print}
Receives any number of arguments,
and prints their values in \verb|stdout|,
using the strings returned by \verb|tostring|.
This function is not intended for formatted output,
but only as a quick way to show a value,
typically for debugging.
For formatted output, see \verb|format| \see{format}.

\subsubsection*{\ff \T{rawget (table, index)}}\DefLIB{rawget}
Gets the real value of \verb|table[index]|,
without invoking any tag method.
\verb|table| must be a table;
\verb|index| is any value different from \nil.

\subsubsection*{\ff \T{rawset (table, index, value)}}\DefLIB{rawset}
Sets the real value of \verb|table[index]| to \verb|value|,
without invoking any tag method.
\verb|table| must be a table;
\verb|index| is any value different from \nil;
and \verb|value| is any Lua value.

\subsubsection*{\ff \T{require (packagename)}}\DefLIB{require}

Loads the given package.
The function starts by looking into the table \IndexVerb{_LOADED}
whether \verb|packagename| is already loaded.
If it is, then \verb|require| is done.
Otherwise, it searches a path looking for a file to load.

If the global variable \IndexVerb{LUA_PATH} is a string, 
this string is the path.
Otherwise, \verb|require| tries the environment variable \verb|LUA_PATH|.
In the last resort, it uses a predefined path.

The path is a sequence of \emph{templates} separated by semicolons.
For each template, \verb|require| will change an eventual interrogation
mark in the template to \verb|packagename|,
and then will try to load the resulting file name.
So, for instance, if the path is
\begin{verbatim}
  "./?.lua;./?.lc;/usr/local/?/init.lua;/lasttry"
\end{verbatim}
a \verb|require "mod"| will try to load the files
\verb|./mod.lua|,
\verb|./mod.lc|,
\verb|/usr/local/mod/init.lua|,
and \verb|/lasttry|, in that order.

The function stops the search as soon as it can load a file,
and then it runs the file.
If there is any error loading or running the file,
or if it cannot find any file in the path,
then \verb|require| signals an error. 
Otherwise, it marks in table \verb|_LOADED|
that the package is loaded, and returns.

While running a packaged file,
\verb|require| defines the global variable \IndexVerb{_REQUIREDNAME}
with the package name.

\subsubsection*{\ff \T{setglobals (function, table)}}\DefLIB{setglobals}
Sets the current table of globals to be used by the given function.
\verb|function| can be a Lua function or a number,
meaning the function at that stack level:
Level 1 is the function calling \verb|setglobals|.

\subsubsection*{\ff \T{setmetatable (table, metatable)}}\DefLIB{setmetatable}

Sets the metatable for the given table.
(You cannot change the metatable of a userdata from Lua.)
If \verb|metatable| is \nil, removes the metatable of the given table.

\subsubsection*{\ff \T{setmode (table, mode)}}\DefLIB{setmode}

Set the weak mode of a table.
The new mode is described by the \verb|mode| string.
Valid values for this string are \verb|""| for regular (non-weak) tables,
\verb|"k"| for weak keys, \verb|"v"| for weak values,
and \verb|"kv"| for both.

This function returns its first argument (\verb|table|).

\subsubsection*{\ff \T{tonumber (e [, base])}}\DefLIB{tonumber}
Tries to convert its argument to a number.
If the argument is already a number or a string convertible
to a number, then \verb|tonumber| returns that number;
otherwise, it returns \nil.

An optional argument specifies the base to interpret the numeral.
The base may be any integer between 2 and 36, inclusive.
In bases above~10, the letter `A' (in either upper or lower case)
represents~10, `B' represents~11, and so forth, with `Z' representing 35.
In base 10 (the default), the number may have a decimal part,
as well as an optional exponent part \see{coercion}.
In other bases, only unsigned integers are accepted.

\subsubsection*{\ff \T{tostring (e)}}\DefLIB{tostring}
Receives an argument of any type and
converts it to a string in a reasonable format.
For complete control of how numbers are converted,
use \verb|format| \see{format}.

\subsubsection*{\ff \T{type (v)}}\DefLIB{type}\label{pdf-type}
Returns the type of its only argument, coded as a string.
The possible results of this function are
\verb|"nil"| (a string, not the value \nil),
\verb|"number"|,
\verb|"string"|,
\verb|"table"|,
\verb|"function"|,
and \verb|"userdata"|.

\subsubsection*{\ff \T{unpack (list)}}\DefLIB{unpack}
Returns all elements from the given list.
This function is equivalent to
\begin{verbatim}
  return list[1], list[2], ..., list[n]
\end{verbatim}
except that the above code can be valid only for a fixed \M{n}.
The number \M{n} of returned values
is either the value of \verb|list.n|, if it is a number,
or one less the index of the first absent (\nil) value.

\subsection{String Manipulation}
This library provides generic functions for string manipulation,
such as finding and extracting substrings and pattern matching.
When indexing a string in Lua, the first character is at position~1
(not at~0, as in C).
Indices are allowed to be negative and are interpreted as indexing backwards,
from the end of the string.
Thus, the last character is at position \Math{-1}, and so on.

The string library provides all its functions inside the table
\DefLIB{string}.

\subsubsection*{\ff \T{string.byte (s [, i])}}\DefLIB{string.byte}
Returns the internal numerical code of the \M{i}-th character of \verb|s|.
If \verb|i| is absent, then it is assumed to be~1.
\verb|i| may be negative.

\NOTE
Numerical codes are not necessarily portable across platforms.

\subsubsection*{\ff \T{string.char (i1, i2, \ldots)}}\DefLIB{string.char}
Receives 0 or more integers.
Returns a string with length equal to the number of arguments,
in which each character has the internal numerical code equal
to its correspondent argument.

\NOTE
Numerical codes are not necessarily portable across platforms.

\subsubsection*{\ff \T{string.find (s, pattern [, init [, plain]])}}
\DefLIB{string.find}
Looks for the first \emph{match} of
\verb|pattern| in the string \verb|s|.
If it finds one, then \verb|find| returns the indices of \verb|s|
where this occurrence starts and ends;
otherwise, it returns \nil.
If the pattern specifies captures (see \verb|string.gsub| below),
the captured strings are returned as extra results.
A third, optional numerical argument \verb|init| specifies
where to start the search;
its default value is~1, and may be negative.
A value of \True{} as a fourth, optional argument \verb|plain|
turns off the pattern matching facilities,
so the function does a plain ``find substring'' operation,
with no characters in \verb|pattern| being considered ``magic''.
Note that if \verb|plain| is given, then \verb|init| must be given too.

\subsubsection*{\ff \T{string.len (s)}}\DefLIB{string.len}
Receives a string and returns its length.
The empty string \verb|""| has length 0.
Embedded zeros are counted,
and so \verb|"a\000b\000c"| has length 5.

\subsubsection*{\ff \T{string.lower (s)}}\DefLIB{string.lower}
Receives a string and returns a copy of that string with all
uppercase letters changed to lowercase.
All other characters are left unchanged.
The definition of what is an uppercase letter depends on the current locale.

\subsubsection*{\ff \T{string.rep (s, n)}}\DefLIB{string.rep}
Returns a string that is the concatenation of \verb|n| copies of
the string \verb|s|.

\subsubsection*{\ff \T{string.sub (s, i [, j])}}\DefLIB{string.sub}
Returns another string, which is a substring of \verb|s|,
starting at \verb|i|  and running until \verb|j|;
\verb|i| and \verb|j| may be negative.
If \verb|j| is absent, then it is assumed to be equal to \Math{-1}
(which is the same as the string length).
In particular,
the call \verb|string.sub(s,1,j)| returns a prefix of \verb|s|
with length \verb|j|,
and the call \verb|string.sub(s, -i)| returns a suffix of \verb|s|
with length \verb|i|.

\subsubsection*{\ff \T{string.upper (s)}}\DefLIB{string.upper}
Receives a string and returns a copy of that string with all
lowercase letters changed to uppercase.
All other characters are left unchanged.
The definition of what is a lowercase letter depends on the current locale.

\subsubsection*{\ff \T{string.format (formatstring, e1, e2, \ldots)}}
\DefLIB{string.format}\label{format}
Returns a formatted version of its variable number of arguments
following the description given in its first argument (which must be a string).
The format string follows the same rules as the \verb|printf| family of
standard C~functions.
The only differences are that the options/modifiers
\verb|*|, \verb|l|, \verb|L|, \verb|n|, \verb|p|,
and \verb|h| are not supported,
and there is an extra option, \verb|q|.
The \verb|q| option formats a string in a form suitable to be safely read
back by the Lua interpreter:
The string is written between double quotes,
and all double quotes, returns, and backslashes in the string
are correctly escaped when written.
For instance, the call
\begin{verbatim}
       string.format('%q', 'a string with "quotes" and \n new line')
\end{verbatim}
will produce the string:
\begin{verbatim}
"a string with \"quotes\" and \
 new line"
\end{verbatim}

The options \verb|c|, \verb|d|, \verb|E|, \verb|e|, \verb|f|,
\verb|g|, \verb|G|, \verb|i|, \verb|o|, \verb|u|, \verb|X|, and \verb|x| all
expect a number as argument,
whereas \verb|q| and \verb|s| expect a string.
The \verb|*| modifier can be simulated by building
the appropriate format string.
For example, \verb|"%*g"| can be simulated with
\verb|"%"..width.."g"|.

\NOTE
String values to be formatted with
\verb|%s| cannot contain embedded zeros.

\subsubsection*{\ff \T{string.gsub (s, pat, repl [, n])}}
\DefLIB{string.gsub}
Returns a copy of \verb|s|
in which all occurrences of the pattern \verb|pat| have been
replaced by a replacement string specified by \verb|repl|.
\verb|gsub| also returns, as a second value,
the total number of substitutions made.

If \verb|repl| is a string, then its value is used for replacement.
Any sequence in \verb|repl| of the form \verb|%|\M{n},
with \M{n} between 1 and 9,
stands for the value of the \M{n}-th captured substring.

If \verb|repl| is a function, then this function is called every time a
match occurs, with all captured substrings passed as arguments,
in order (see below);
if the pattern specifies no captures,
then the whole match is passed as a sole argument.
If the value returned by this function is a string,
then it is used as the replacement string;
otherwise, the replacement string is the empty string.

The last, optional parameter \verb|n| limits
the maximum number of substitutions to occur.
For instance, when \verb|n| is 1 only the first occurrence of
\verb|pat| is replaced.

Here are some examples:
\begin{verbatim}
   x = gsub("hello world", "(%w+)", "%1 %1")
   --> x="hello hello world world"

   x = gsub("hello world", "(%w+)", "%1 %1", 1)
   --> x="hello hello world"

   x = gsub("hello world from Lua", "(%w+)%s*(%w+)", "%2 %1")
   --> x="world hello Lua from"

   x = gsub("home = $HOME, user = $USER", "%$(%w+)", getenv)
   --> x="home = /home/roberto, user = roberto"  (for instance)

   x = gsub("4+5 = $return 4+5$", "%$(.-)%$", dostring)
   --> x="4+5 = 9"

   local t = {name="Lua", version="4.1"}
   x = gsub("$name - $version", "%$(%w+)", function (v) return t[v] end)
   --> x="Lua - 4.1"
\end{verbatim}


\subsubsection*{Patterns} \label{pm}

\paragraph{Character Class:}
a \Def{character class} is used to represent a set of characters.
The following combinations are allowed in describing a character class:
\begin{description}\leftskip=20pt
\item[\emph{x}] (where \emph{x} is not one of the magic characters
\verb|^$()%.[]*+-?|)
--- represents the character \emph{x} itself.
\item[\T{.}] --- (a dot) represents all characters.
\item[\T{\%a}] --- represents all letters.
\item[\T{\%c}] --- represents all control characters.
\item[\T{\%d}] --- represents all digits.
\item[\T{\%l}] --- represents all lowercase letters.
\item[\T{\%p}] --- represents all punctuation characters.
\item[\T{\%s}] --- represents all space characters.
\item[\T{\%u}] --- represents all uppercase letters.
\item[\T{\%w}] --- represents all alphanumeric characters.
\item[\T{\%x}] --- represents all hexadecimal digits.
\item[\T{\%z}] --- represents the character with representation 0.
\item[\T{\%\M{x}}] (where \M{x} is any non-alphanumeric character)  ---
represents the character \M{x}.
This is the standard way to escape the magic characters.
We recommend that any punctuation character (even the non magic)
should be preceded by a \verb|%|
when used to represent itself in a pattern.

\item[\T{[\M{set}]}] ---
represents the class which is the union of all
characters in \M{set}.
A range of characters may be specified by
separating the end characters of the range with a \verb|-|.
All classes \verb|%|\emph{x} described above may also be used as
components in \M{set}.
All other characters in \M{set} represent themselves.
For example, \verb|[%w_]| (or \verb|[_%w]|)
represents all alphanumeric characters plus the underscore,
\verb|[0-7]| represents the octal digits,
and \verb|[0-7%l%-]| represents the octal digits plus
the lowercase letters plus the \verb|-| character.

The interaction between ranges and classes is not defined.
Therefore, patterns like \verb|[%a-z]| or \verb|[a-%%]|
have no meaning.

\item[\T{[\^\null\M{set}]}] ---
represents the complement of \M{set},
where \M{set} is interpreted as above.
\end{description}
For all classes represented by single letters (\verb|%a|, \verb|%c|, \ldots),
the corresponding uppercase letter represents the complement of the class.
For instance, \verb|%S| represents all non-space characters.

The definitions of letter, space, etc.\ depend on the current locale.
In particular, the class \verb|[a-z]| may not be equivalent to \verb|%l|.
The second form should be preferred for portability.

\paragraph{Pattern Item:}
a \Def{pattern item} may be
\begin{itemize}
\item
a single character class,
which matches any single character in the class;
\item
a single character class followed by \verb|*|,
which matches 0 or more repetitions of characters in the class.
These repetition items will always match the longest possible sequence;
\item
a single character class followed by \verb|+|,
which matches 1 or more repetitions of characters in the class.
These repetition items will always match the longest possible sequence;
\item
a single character class followed by \verb|-|,
which also matches 0 or more repetitions of characters in the class.
Unlike \verb|*|,
these repetition items will always match the \emph{shortest} possible sequence;
\item
a single character class followed by \verb|?|,
which matches 0 or 1 occurrence of a character in the class;
\item
\T{\%\M{n}}, for \M{n} between 1 and 9;
such item matches a substring equal to the \M{n}-th captured string
(see below);
\item
\T{\%b\M{xy}}, where \M{x} and \M{y} are two distinct characters;
such item matches strings that start with~\M{x}, end with~\M{y},
and where the \M{x} and \M{y} are \emph{balanced}.
This means that, if one reads the string from left to right,
counting \Math{+1} for an \M{x} and \Math{-1} for a \M{y},
the ending \M{y} is the first \M{y} where the count reaches 0.
For instance, the item \verb|%b()| matches expressions with
balanced parentheses.
\end{itemize}

\paragraph{Pattern:}
a \Def{pattern} is a sequence of pattern items.
A \verb|^| at the beginning of a pattern anchors the match at the
beginning of the subject string.
A \verb|$| at the end of a pattern anchors the match at the
end of the subject string.
At other positions,
\verb|^| and \verb|$| have no special meaning and represent themselves.

\paragraph{Captures:}
A pattern may contain sub-patterns enclosed in parentheses;
they describe \Def{captures}.
When a match succeeds, the substrings of the subject string
that match captures are stored (\emph{captured}) for future use.
Captures are numbered according to their left parentheses.
For instance, in the pattern \verb|"(a*(.)%w(%s*))"|,
the part of the string matching \verb|"a*(.)%w(%s*)"| is
stored as the first capture (and therefore has number~1);
the character matching \verb|.| is captured with number~2,
and the part matching \verb|%s*| has number~3.

\NOTE
A pattern cannot contain embedded zeros.  Use \verb|%z| instead.


\subsection{Table Manipulation}
This library provides generic functions for table manipulation,
It provides all its functions inside the table \DefLIB{table}.

Most functions in the table library library assume that the table
represents an array or a list.
For those functions, an important concept is the \emph{size} of the array.
There are three ways to specify that size:
\begin{itemize}
\item the field \verb|"n"| ---
When the table has a field \verb|"n"| with a numerical value,
that value is assumed as its size.
\item \verb|setn| ---
You can call the \verb|table.setn| function to explicitly set
the size of a table.
\item implicit size ---
%% TODO
\end{itemize}
For more details, see the descriptions of the \verb|table.getn| and
\verb|table.setn| functions.

\subsubsection*{\ff \T{table.foreach (table, func)}}\DefLIB{table.foreach}
Executes the given \verb|func| over all elements of \verb|table|.
For each element, \verb|func| is called with the index and
respective value as arguments.
If \verb|func| returns a non-\nil{} value,
then the loop is broken, and this value is returned
as the final value of \verb|foreach|.

The behavior of \verb|foreach| is \emph{undefined} if you change
the table \verb|t| during the traversal.


\subsubsection*{\ff \T{table.foreachi (table, func)}}\DefLIB{table.foreachi}
Executes the given \verb|func| over the
numerical indices of \verb|table|.
For each index, \verb|func| is called with the index and
respective value as arguments.
Indices are visited in sequential order,
from~1 to \verb|n|,
where \verb|n| is the size of the table \see{getn}.
If \verb|func| returns a non-\nil{} value,
then the loop is broken, and this value is returned
as the final value of \verb|foreachi|.

\subsubsection*{\ff \T{table.getn (table)}}\DefLIB{table.getn}\label{getn}
Returns the ``size'' of a table, when seen as a list.
If the table has an \verb|n| field with a numeric value,
this value is the ``size'' of the table.
Otherwise, if there was a previous call to
\verb|table.getn| or to \verb|table.setn| over this table,
the respective value is returned.
Otherwise, the ``size'' is one less the first integer index with
a \nil{} value.

Notice that the last option happens only once for a table.
If you call \verb|table.getn| again over the same table,
it will return the same previous result,
even if the table has been modified.
The only way to change the value of \verb|table.getn| is by calling
\verb|table.setn| or assigning to field \verb|"n"| in the table.

\subsubsection*{\ff \T{table.sort (table [, comp])}}\DefLIB{table.sort}
Sorts table elements in a given order, \emph{in-place},
from \verb|table[1]| to \verb|table[n]|,
where \verb|n| is the size of the table \see{getn}.
If \verb|comp| is given,
then it must be a function that receives two table elements,
and returns true
when the first is less than the second
(so that \verb|not comp(a[i+1],a[i])| will be true after the sort).
If \verb|comp| is not given,
then the standard Lua operator \verb|<| is used instead.

The sort algorithm is \emph{not} stable
(that is, elements considered equal by the given order
may have their relative positions changed by the sort).

\subsubsection*{\ff \T{table.insert (table, [pos,] value)}}\DefLIB{table.insert}

Inserts element \verb|value| at position \verb|pos| in \verb|table|,
shifting other elements up to open space, if necessary.
The default value for \verb|pos| is \verb|n+1|,
where \verb|n| is the size of the table \see{getn},
so that a call \verb|table.insert(t,x)| inserts \verb|x| at the end
of table \verb|t|.
This function also updates the size of the table,
calling \verb|table.setn(table, n+1)|.

\subsubsection*{\ff \T{table.remove (table [, pos])}}\DefLIB{table.remove}

Removes from \verb|table| the element at position \verb|pos|,
shifting other elements down to close the space, if necessary.
Returns the value of the removed element.
The default value for \verb|pos| is \verb|n|,
where \verb|n| is the size of the table \see{getn},
so that a call \verb|tremove(t)| removes the last element
of table \verb|t|.
This function also updates the size of the table,
calling \verb|table.setn(table, n-1)|.

\subsubsection*{\ff \T{table.setn (table, n)}}\DefLIB{table.setn}

Updates the ``size'' of a table.
If the table has a field \verb|"n"| with a numerical value,
that value is changed to the given \verb|n|.
Otherwise, it updates an internal state of the \verb|table| library
so that subsequent calls to \verb|table.getn(table)| return \verb|n|.


\subsection{Mathematical Functions} \label{mathlib}

This library is an interface to most functions of the standard C~math library.
(Some have slightly different names.)
It provides all its functions inside the table \DefLIB{math}.
In addition,
it registers a ??tag method for the binary exponentiation operator \verb|^|
that returns \Math{x^y} when applied to numbers \verb|x^y|.

The library provides the following functions:
\DefLIB{math.abs}\DefLIB{math.acos}\DefLIB{math.asin}\DefLIB{math.atan}
\DefLIB{math.atan2}\DefLIB{math.ceil}\DefLIB{math.cos}
\DefLIB{math.def}\DefLIB{math.exp}
\DefLIB{math.floor}\DefLIB{math.log}\DefLIB{math.log10}
\DefLIB{math.max}\DefLIB{math.min}
\DefLIB{math.mod}\DefLIB{math.rad}\DefLIB{math.sin}
\DefLIB{math.sqrt}\DefLIB{math.tan}
\DefLIB{math.frexp}\DefLIB{math.ldexp}\DefLIB{math.random}
\DefLIB{math.randomseed}
\begin{verbatim}
       math.abs   math.acos   math.asin  math.atan math.atan2
       math.ceil  math.cos    math.deg   math.exp  math.floor
       math.log   math.log10  math.max   math.min  math.mod
       math.rad   math.sin    math.sqrt  math.tan  math.frexp
       math.ldexp math.random math.randomseed
\end{verbatim}
plus a variable \IndexLIB{math.pi}.
Most of them
are only interfaces to the homonymous functions in the C~library,
except that, for the trigonometric functions,
all angles are expressed in \emph{degrees}, not radians.
The functions \verb|math.deg| and \verb|math.rad| can be used to convert
between radians and degrees.

The function \verb|math.max| returns the maximum
value of its numeric arguments.
Similarly, \verb|math.min| computes the minimum.
Both can be used with 1, 2, or more arguments.

The functions \verb|math.random| and \verb|math.randomseed|
are interfaces to the simple random generator functions
\verb|rand| and \verb|srand|, provided by ANSI~C.
(No guarantees can be given for their statistical properties.)
When called without arguments,
\verb|math.random| returns a pseudo-random real number
in the range \Math{[0,1)}.
When called with a number \Math{n},
\verb|math.random| returns a pseudo-random integer in the range \Math{[1,n]}.
When called with two arguments, \Math{l} and \Math{u},
\verb|math.random| returns a pseudo-random integer in the range \Math{[l,u]}.


\subsection{Input and Output Facilities} \label{libio}

The I/O library provides two different styles for file manipulation.
The first one uses implicit file descriptors;
that is, there are operations to set a default input file and a
default output file,
and all input/output operations are over those default files.
The second style uses explicit file descriptors.

When using implicit file descriptors,
all operations are supplied by table \DefLIB{io}.
When using explicit file descriptors,
the operation \DefLIB{io.open} returns a file descriptor,
and then all operations are supplied as methods by the file descriptor.

Moreover, the table \verb|io| also provides
three predefined file descriptors:
\DefLIB{io.stdin}, \DefLIB{io.stdout}, and \DefLIB{io.stderr},
with their usual meaning from C.

A file handle is a userdata containing the file stream (\verb|FILE*|),
with a distinctive metatable created by the I/O library.

Unless otherwise stated,
all I/O functions return \nil{} on failure
(plus an error message as a second result)
and some value different from \nil{} on success.

\subsubsection*{\ff \T{io.close ([handle])}}\DefLIB{io.close}

Equivalent to \verb|fh:close| over the default output file.

\subsubsection*{\ff \T{io.flush ()}}\DefLIB{io.flush}

Equivalent to \verb|fh:flush| over the default output file.

\subsubsection*{\ff \T{io.input ([file])}}\DefLIB{io.input}

When called with a file name, it opens the named file (in text mode),
and sets its handle as the default input file
(and returns nothing).
When called with a file handle,
it simply sets that file handle as the default input file.
When called without parameters,
it returns the current default input file.

In case of errors this function raises the error,
instead of returning an error code.

\subsubsection*{\ff \T{io.open (filename, mode)}}\DefLIB{io.open}

This function opens a file,
in the mode specified in the string \verb|mode|.
It returns a new file handle,
or, in case of errors, \nil{} plus an error message.

The \verb|mode| string can be any of the following:
\begin{description}\leftskip=20pt
\item[``r''] read mode;
\item[``w''] write mode;
\item[``a''] append mode;
\item[``r+''] update mode, all previous data is preserved;
\item[``w+''] update mode, all previous data is erased;
\item[``a+''] append update mode, previous data is preserved,
  writing is only allowed at the end of file.
\end{description}
The \verb|mode| string may also have a \verb|b| at the end,
which is needed in some systems to open the file in binary mode.
This string is exactly what is used in the standard~C function \verb|fopen|.

\subsubsection*{\ff \T{io.output ([file])}}\DefLIB{io.output}

Similar to \verb|io.input|, but operates over the default output file.

\subsubsection*{\ff \T{io.read (format1, ...)}}\DefLIB{io.read}

Equivalent to \verb|fh:read| over the default input file.

\subsubsection*{\ff \T{io.tmpfile ()}}\DefLIB{io.tmpfile}

Returns a handle for a temporary file.
This file is open in read/write mode,
and it is automatically removed when the program ends.

\subsubsection*{\ff \T{io.write (value1, ...)}}\DefLIB{io.write}

Equivalent to \verb|fh:write| over the default output file.



\subsubsection*{\ff \T{fh:close ([handle])}}\DefLIB{fh:close}

Closes the file \verb|fh|.

\subsubsection*{\ff \T{fh:flush ()}}\DefLIB{fh:flush}

Saves any written data to the file \verb|fh|.

\subsubsection*{\ff \T{fh:read (format1, ...)}}\DefLIB{fh:read}

Reads the file \verb|fh|,
according to the given formats, which specify what to read.
For each format,
the function returns a string (or a number) with the characters read,
or \nil{} if it cannot read data with the specified format.
When called without formats,
it uses a default format that reads the entire next line
(see below).

The available formats are
\begin{description}\leftskip=20pt
\item[``*n''] reads a number;
this is the only format that returns a number instead of a string.
\item[``*a''] reads the whole file, starting at the current position.
On end of file, it returns the empty string.
\item[``*l''] reads the next line (skipping the end of line),
returning \nil{} on end of file.
This is the default format.
\item[\emph{number}] reads a string with up to that number of characters,
or \nil{} on end of file.
If number is zero,
it reads nothing and returns an empty string,
or \nil{} on end of file.
\end{description}

\subsubsection*{\ff \T{fh:seek ([whence] [, offset])}}\DefLIB{fh:seek}

Sets and gets the file position,
measured in bytes from the beginning of the file,
to the position given by \verb|offset| plus a base
specified by the string \verb|whence|, as follows:
\begin{description}\leftskip=20pt
\item[``set''] base is position 0 (beginning of the file);
\item[``cur''] base is current position;
\item[``end''] base is end of file;
\end{description}
In case of success, function \verb|seek| returns the final file position,
measured in bytes from the beginning of the file.
If this function fails, it returns \nil,
plus a string describing the error.

The default value for \verb|whence| is \verb|"cur"|,
and for \verb|offset| is 0.
Therefore, the call \verb|file:seek()| returns the current
file position, without changing it;
the call \verb|file:seek("set")| sets the position to the
beginning of the file (and returns 0);
and the call \verb|file:seek("end")| sets the position to the
end of the file, and returns its size.

\subsubsection*{\ff \T{fh:write (value1, ...)}}\DefLIB{fh:write}

Writes the value of each of its arguments to
the filehandle \verb|fh|.
The arguments must be strings or numbers.
To write other values,
use \verb|tostring| or \verb|format| before \verb|write|.
If this function fails, it returns \nil,
plus a string describing the error.


\subsection{Operating System Facilities} \label{libiosys}

This library is implemented through table \DefLIB{os}.

\subsubsection*{\ff \T{os.clock ()}}\DefLIB{os.clock}

Returns an approximation of the amount of CPU time
used by the program, in seconds.

\subsubsection*{\ff \T{os.date ([format [, time]])}}\DefLIB{os.date}

Returns a string or a table containing date and time,
formatted according to the given string \verb|format|.

If the \verb|time| argument is present,
this is the time to be formatted
(see the \verb|time| function for a description of this value).
Otherwise, \verb|date| formats the current time.

If \verb|format| starts with \verb|!|,
then the date is formatted in Coordinated Universal Time.

After that optional character,
if \verb|format| is \verb|*t|,
then \verb|date| returns a table with the following fields:
\verb|year| (four digits), \verb|month| (1--12), \verb|day| (1--31),
\verb|hour| (0--23), \verb|min| (0--59), \verb|sec| (0--61),
\verb|wday| (weekday, Sunday is 1),
\verb|yday| (day of the year),
and \verb|isdst| (daylight saving flag, a boolean).

If format is not \verb|*t|,
then \verb|date| returns the date as a string,
formatted according with the same rules as the C~function \verb|strftime|.
When called without arguments,
\verb|date| returns a reasonable date and time representation that depends on
the host system and on the current locale
(that is, \verb|os.date()| is equivalent to \verb|os.date("%c")|).

\subsubsection*{\ff \T{os.difftime (t1, t2)}}\DefLIB{os.difftime}

Returns the number of seconds from time \verb|t1| to time \verb|t2|.
In Posix, Windows, and some other systems,
this value is exactly \verb|t1|\Math{-}\verb|t2|.

\subsubsection*{\ff \T{os.execute (command)}}\DefLIB{os.execute}

This function is equivalent to the C~function \verb|system|.
It passes \verb|command| to be executed by an operating system shell.
It returns a status code, which is system-dependent.

\subsubsection*{\ff \T{os.exit ([code])}}\DefLIB{os.exit}

Calls the C~function \verb|exit|,
with an optional \verb|code|,
to terminate the host program.
The default value for \verb|code| is the success code.

\subsubsection*{\ff \T{os.getenv (varname)}}\DefLIB{os.getenv}

Returns the value of the process environment variable \verb|varname|,
or \nil{} if the variable is not defined.

\subsubsection*{\ff \T{os.remove (filename)}}\DefLIB{os.remove}

Deletes the file with the given name.
If this function fails, it returns \nil,
plus a string describing the error.

\subsubsection*{\ff \T{os.rename (name1, name2)}}\DefLIB{os.rename}

Renames file named \verb|name1| to \verb|name2|.
If this function fails, it returns \nil,
plus a string describing the error.

\subsubsection*{\ff \T{os.setlocale (locale [, category])}}\DefLIB{os.setlocale}

This function is an interface to the C~function \verb|setlocale|.
\verb|locale| is a string specifying a locale;
\verb|category| is an optional string describing which category to change:
\verb|"all"|, \verb|"collate"|, \verb|"ctype"|,
\verb|"monetary"|, \verb|"numeric"|, or \verb|"time"|;
the default category is \verb|"all"|.
The function returns the name of the new locale,
or \nil{} if the request cannot be honored.

\subsubsection*{\ff \T{os.time ([table])}}\DefLIB{os.time}

Returns the current time when called without arguments,
or a time representing the date and time specified by the given table.
This table must have fields \verb|year|, \verb|month|, and \verb|day|,
and may have fields \verb|hour|, \verb|min|, \verb|sec|, and \verb|isdst|
(for a description of these fields, see the \verb|os.date| function).

The returned value is a number, whose meaning depends on your system.
In Posix, Windows, and some other systems, this number counts the number
of seconds since some given start time (the ``epoch'').
In other systems, the meaning is not specified,
and the number returned bt \verb|time| can be used only as an argument to
\verb|date| and \verb|difftime|.

\subsubsection*{\ff \T{os.tmpname ()}}\DefLIB{os.tmpname}

Returns a string with a file name that can
be used for a temporary file.
The file must be explicitly opened before its use
and removed when no longer needed.

This function is equivalent to the \verb|tmpnam| C~function,
and many people (and even some compilers!) advise against its use,
because between the time you call the function
and the time you open the file,
it is possible for another process
to create a file with the same name.


\subsection{The Reflexive Debug Interface}

The library \verb|ldblib| provides
the functionality of the debug interface to Lua programs.
If you want to use this library,
your host application must open it,
by calling \verb|lua_dblibopen|.
\DefAPI{lua_dblibopen}

You should exert great care when using this library.
The functions provided here should be used exclusively for debugging
and similar tasks, such as profiling.
Please resist the temptation to use them as a
usual programming tool:
They can be \emph{very} slow.
Moreover, \verb|setlocal| and \verb|getlocal|
violate the privacy of local variables,
and therefore can compromise some (otherwise) secure code.


\subsubsection*{\ff \T{getinfo (function, [what])}}\DefLIB{getinfo}

This function returns a table with information about a function.
You can give the function directly,
or you can give a number as the value of \verb|function|,
which means the function running at level \verb|function| of the stack:
Level 0 is the current function (\verb|getinfo| itself);
level 1 is the function that called \verb|getinfo|;
and so on.
If \verb|function| is a number larger than the number of active functions,
then \verb|getinfo| returns \nil.

The returned table contains all the fields returned by \verb|lua_getinfo|,
with the string \verb|what| describing what to get.
The default for \verb|what| is to get all information available.
If present,
the option \verb|f|
adds a field named \verb|func| with the function itself.

For instance, the expression \verb|getinfo(1,"n").name| returns
the name of the current function, if a reasonable name can be found,
and \verb|getinfo(print)| returns a table with all available information
about the \verb|print| function.


\subsubsection*{\ff \T{getlocal (level, local)}}\DefLIB{getlocal}

This function returns the name and the value of the local variable
with index \verb|local| of the function at level \verb|level| of the stack.
(The first parameter or local variable has index~1, and so on,
until the last active local variable.)
The function returns \nil{} if there is no local
variable with the given index,
and raises an error when called with a \verb|level| out of range.
(You can call \verb|getinfo| to check whether the level is valid.)

\subsubsection*{\ff \T{setlocal (level, local, value)}}\DefLIB{setlocal}

This function assigns the value \verb|value| to the local variable
with index \verb|local| of the function at level \verb|level| of the stack.
The function returns \nil{} if there is no local
variable with the given index,
and raises an error when called with a \verb|level| out of range.
(You can call \verb|getinfo| to check whether the level is valid.)

\subsubsection*{\ff \T{setcallhook (hook)}}\DefLIB{setcallhook}

Sets the function \verb|hook| as the call hook;
this hook will be called every time the interpreter starts and
exits the execution of a function.
The only argument to the call hook is the event name (\verb|"call"| or
\verb|"return"|).
You can call \verb|getinfo| with level 2 to get more information about
the function being called or returning
(level~0 is the \verb|getinfo| function,
and level~1 is the hook function).
When called without arguments,
this function turns off call hooks.
\verb|setcallhook| returns the old call hook.

\subsubsection*{\ff \T{setlinehook (hook)}}\DefLIB{setlinehook}

Sets the function \verb|hook| as the line hook;
this hook will be called every time the interpreter changes
the line of code it is executing.
The only argument to the line hook is the line number the interpreter
is about to execute.
When called without arguments,
this function turns off line hooks.
\verb|setlinehook| returns the old line hook.


%------------------------------------------------------------------------------
\section{\Index{Lua Stand-alone}} \label{lua-sa}

Although Lua has been designed as an extension language,
to be embedded in a host C~program,
it is also frequently used as a stand-alone language.
An interpreter for Lua as a stand-alone language,
called simply \verb|lua|,
is provided with the standard distribution.
The stand-alone interpreter includes
all standard libraries plus the reflexive debug interface.
Its usage is:
\begin{verbatim}
      lua [options] [prog [args]]
\end{verbatim}
The options are:
\begin{description}\leftskip=20pt
\item[\T{-} ] executes \verb|stdin| as a file;
\item[\T{-e} \rm\emph{stat}] executes string \emph{stat};
\item[\T{-l} \rm\emph{file}] executes file \emph{file};
\item[\T{-i}] enters interactive mode after running \emph{prog};
\item[\T{-v}] prints version information;
\item[\T{--}] stop handling options.
\end{description}
After handling its options, \verb|lua| runs the given \emph{prog},
passing to it the given \emph{args}.
When called without arguments,
\verb|lua| behaves as \verb|lua -v -i| when \verb|stdin| is a terminal,
and as \verb|lua -| otherwise.

Before running any argument,
the intepreter checks for an environment variable \IndexVerb{LUA_INIT}.
If its format is \verb|@|\emph{filename},
then lua executes the file.
Otherwise, lua executes the string itself.

All options are handled in order, except \verb|-i|.
For instance, an invocation like
\begin{verbatim}
       $ lua -e'a=1' -e 'print(a)' prog.lua
\end{verbatim}
will first set \verb|a| to 1, then print \verb|a|,
and finally run the file \verb|prog.lua|.
(Here, \verb|$| is the shell prompt. Your prompt may be different.)

Before starting to run the program,
\verb|lua| collects all arguments in the command line
in a global table called \verb|arg|.
The program name is stored in index 0,
the first argument after the program goes to index 1,
and so on.
The field \verb|n| gets the number of arguments after the program name.
Any argument before the program name
(that is, the options plus the interpreter name)
goes to negative indices.
For instance, in the call
\begin{verbatim}
       $ lua -la.lua b.lua t1 t2
\end{verbatim}
the interpreter first runs the file \T{a.lua},
then creates a table
\begin{verbatim}
       arg = { [-2] = "lua", [-1] = "-la.lua", [0] = "b.lua",
               [1] = "t1", [2] = "t2"; n = 2 }
\end{verbatim}
and finally runs the file \T{b.lua}.

In interactive mode,
if you write an incomplete statement,
the interpreter waits for its completion.

If the global variable \IndexVerb{_PROMPT} is defined as a string,
then its value is used as the prompt.
Therefore, the prompt can be changed directly on the command line:
\begin{verbatim}
       $ lua -e"_PROMPT='myprompt> '" -i
\end{verbatim}
(the first pair of quotes is for the shell,
the second is for Lua),
or in any Lua programs by assigning to \verb|_PROMPT|.
Note the use of \verb|-i| to enter interactive mode; otherwise,
the program would end just after the assignment to \verb|_PROMPT|.

In Unix systems, Lua scripts can be made into executable programs
by using \verb|chmod +x| and the~\verb|#!| form,
as in \verb|#!/usr/local/bin/lua|.
(Of course,
the location of the Lua interpreter may be different in your machine.
If \verb|lua| is in your \verb|PATH|,
then a more portable solution is \verb|#!/usr/bin/env lua|.)


%------------------------------------------------------------------------------
\section*{Acknowledgments}

%% TODO rever isso?

The authors thank CENPES/PETROBRAS which,
jointly with \tecgraf, used early versions of
this system extensively and gave valuable comments.
The authors also thank Carlos Henrique Levy,
who found the name of the game.
Lua means ``moon'' in Portuguese.


\appendix

\section*{Incompatibilities with Previous Versions}
\addcontentsline{toc}{section}{Incompatibilities with Previous Versions}

We took a great care to avoid incompatibilities with
the previous public versions of Lua,
but some differences had to be introduced.
Here is a list of all these incompatibilities.


\subsection*{Incompatibilities with \Index{version 4.0}}

\subsubsection*{Changes in the Language}
\begin{itemize}

\item
Function calls written between parentheses result in exactly one value.

\item
A function call as the last expression in a list constructor
(like \verb|{a,b,f()}}|) has all its return values inserted in the list.

\item
\rwd{in} is a reserved word.

\item
When a literal string of the form \verb|[[...]]| starts with a newline,
this newline is ignored.

\item Old pre-compiled code is obsolete, and must be re-compiled.

\end{itemize}


\subsubsection*{Changes in the Libraries}
\begin{itemize}

\item
The \verb|read| option \verb|*w| is obsolete.

\item
The \verb|format| option \verb|%n$| is obsolete.

\end{itemize}


\subsubsection*{Changes in the API}
\begin{itemize}

\item
Userdata!!

\end{itemize}

%{===============================================================
\newpage
\section*{The Complete Syntax of Lua} \label{BNF}
\addcontentsline{toc}{section}{The Complete Syntax of Lua}

The notation used here is the usual extended BNF,
in which
\rep{\emph{a}}~means 0 or more \emph{a}'s, and
\opt{\emph{a}}~means an optional \emph{a}.
Non-terminals are shown in \emph{italics},
keywords are shown in {\bf bold},
and other terminal symbols are shown in {\tt typewriter} font,
enclosed in single quotes.


\renewenvironment{Produc}{\vspace{0.8ex}\par\noindent\hspace{3ex}\it\begin{tabular}{rrl}}{\end{tabular}\vspace{0.8ex}\par\noindent}

\renewcommand{\OrNL}{\\ & \Or & }
%\newcommand{\Nter}[1]{{\rm{\tt#1}}}
%\newcommand{\Nter}[1]{\ter{#1}}

\index{grammar}

\begin{Produc}

\produc{chunk}{\rep{stat \opt{\ter{;}}}}

\produc{block}{chunk}

\produc{stat}{%
	varlist1 \ter{=} explist1
\OrNL	functioncall
\OrNL	\rwd{do} block \rwd{end}
\OrNL	\rwd{while} exp \rwd{do} block \rwd{end}
\OrNL	\rwd{repeat} block \rwd{until} exp
\OrNL	\rwd{if} exp \rwd{then} block
	\rep{\rwd{elseif} exp \rwd{then} block}
	\opt{\rwd{else} block} \rwd{end}
\OrNL	\rwd{return} \opt{explist1}
\OrNL	\rwd{break}
\OrNL	\rwd{for} \Nter{Name} \ter{=} exp \ter{,} exp \opt{\ter{,} exp}
	\rwd{do} block \rwd{end}
\OrNL   \rwd{for} \Nter{Name} \rep{\ter{,} \Nter{Name}} \rwd{in} explist1
                    \rwd{do} block \rwd{end}
\OrNL	\rwd{function} funcname funcbody
\OrNL	\rwd{local} \rwd{function} \Nter{Name} funcbody
\OrNL	\rwd{local} namelist \opt{init}
}

\produc{funcname}{\Nter{Name} \rep{\ter{.} \Nter{Name}}
                              \opt{\ter{:} \Nter{Name}}}

\produc{varlist1}{var \rep{\ter{,} var}}

\produc{var}{%
	\Nter{Name}
\Or	prefixexp \ter{[} exp \ter{]}
\Or	prefixexp \ter{.} \Nter{Name}
}

\produc{namelist}{\Nter{Name} \rep{\ter{,} \Nter{Name}}}

\produc{init}{\ter{=} explist1}

\produc{explist1}{\rep{exp \ter{,}} exp}

\produc{exp}{%
	\rwd{nil}
	\rwd{false}
	\rwd{true}
\Or	\Nter{Number}
\OrNL	\Nter{Literal}
\Or	function
\Or	prefixexp
\OrNL	tableconstructor
\Or	exp binop exp
\Or	unop exp
}

\produc{prefixexp}{var \Or functioncall \Or \ter{(} exp \ter{)}}

\produc{functioncall}{%
	prefixexp args
\Or	prefixexp \ter{:} \Nter{Name} args
}

\produc{args}{%
	\ter{(} \opt{explist1} \ter{)}
\Or	tableconstructor
\Or	\Nter{Literal}
}

\produc{function}{\rwd{function} funcbody}

\produc{funcbody}{\ter{(} \opt{parlist1} \ter{)} block \rwd{end}}

\produc{parlist1}{%
	\Nter{Name} \rep{\ter{,} \Nter{Name}} \opt{\ter{,} \ter{\ldots}}
\Or	\ter{\ldots}
}

\produc{tableconstructor}{\ter{\{} \opt{fieldlist} \ter{\}}}
\produc{fieldlist}{field \rep{fieldsep field} \opt{fieldsep}}
\produc{field}{\ter{[} exp \ter{]} \ter{=} exp \Or name \ter{=} exp \Or exp}
\produc{fieldsep}{\ter{,} \Or \ter{;}}

\produc{binop}{\ter{+} \Or \ter{-} \Or \ter{*} \Or \ter{/} \Or \ter{\^{ }} \Or
  \ter{..} \Or \ter{<} \Or \ter{<=} \Or \ter{>} \Or \ter{>=}
  \Or \ter{==} \Or \ter{\~{ }=} \OrNL \rwd{and} \Or \rwd{or}}

\produc{unop}{\ter{-} \Or \rwd{not}}

\end{Produc}

%}===============================================================

% Index

\newpage
\addcontentsline{toc}{section}{Index}
% $Id: manual.tex,v 1.56 2002/06/06 12:49:28 roberto Exp roberto $

\documentclass[11pt,twoside,draft]{article}
\usepackage{fullpage}
\usepackage{bnf}
\usepackage{graphicx}

% no need for subscripts...
\catcode`\_=12

%\newcommand{\See}[1]{Section~\ref{#1}}
\newcommand{\See}[1]{\S\ref{#1}}
%\newcommand{\see}[1]{(see~\See{#1} on page \pageref{#1})}
\newcommand{\see}[1]{(see~\See{#1})}
\newcommand{\seepage}[1]{(see page~\pageref{#1})}
\newcommand{\M}[1]{{\rm\emph{#1}}}
\newcommand{\T}[1]{{\tt #1}}
\newcommand{\Math}[1]{$#1$}
\newcommand{\nil}{{\bf nil}}
\newcommand{\False}{{\bf false}}
\newcommand{\True}{{\bf true}}
%\def\tecgraf{{\sf TeC\kern-.21em\lower.7ex\hbox{Graf}}}
\def\tecgraf{{\sf Tecgraf}}

\newcommand{\Index}[1]{#1\index{#1@{\lowercase{#1}}}}
\newcommand{\IndexVerb}[1]{\T{#1}\index{#1@{\tt #1}}}
\newcommand{\IndexEmph}[1]{\emph{#1}\index{#1@{\lowercase{#1}}}}
\newcommand{\IndexTM}[1]{\index{#1 event@{``#1'' event}}\index{tag method!#1}}
\newcommand{\Def}[1]{\emph{#1}\index{#1}}
\newcommand{\IndexAPI}[1]{\T{#1}\DefAPI{#1}}
\newcommand{\IndexLIB}[1]{\T{#1}\DefLIB{#1}}
\newcommand{\DefLIB}[1]{\index{#1@{\tt #1}}}
\newcommand{\DefAPI}[1]{\index{C API!#1@{\tt #1}}}
\newcommand{\IndexKW}[1]{\index{keywords!#1@{\tt #1}}}

\newcommand{\ff}{$\bullet$\ }

\newcommand{\Version}{5.0 (alpha)}

% changes to bnf.sty by LHF
\renewcommand{\Or}{$|$ }
\renewcommand{\rep}[1]{{\rm\{}\,#1\,{\rm\}}}
\renewcommand{\opt}[1]{{\rm [}\,#1\,{\,\rm]}}
\renewcommand{\ter}[1]{{\rm`{\tt#1}'}}
\newcommand{\Nter}[1]{{\tt#1}}
\newcommand{\NOTE}{\par\medskip\noindent\emph{NOTE}: }

\makeindex

\begin{document}

%{===============================================================
\thispagestyle{empty}
\pagestyle{empty}

{
\parindent=0pt
\vglue1.5in
{\LARGE\bf
The Programming Language Lua}
\hfill
\vskip4pt \hrule height 4pt width \hsize \vskip4pt
\hfill
Reference Manual for Lua version \Version
\\
\null
\hfill
Last revised on \today
\\
\vfill
\centering
\includegraphics[width=0.7\textwidth]{nolabel.ps}
\vfill
\vskip4pt \hrule height 2pt width \hsize
}

\newpage
\begin{quotation}
\parskip=10pt
\parindent=0pt
\footnotesize
\null\vfill

\noindent
Copyright \copyright\ 2002 Tecgraf, PUC-Rio.  All rights reserved.

Permission is hereby granted, free of charge,
to any person obtaining a copy of this software
and associated documentation files (the "Software"),
to deal in the Software without restriction,
including without limitation the rights to use, copy, modify,
merge, publish, distribute, sublicense,
and/or sell copies of the Software,
and to permit persons to whom the Software is furnished to do so,
subject to the following conditions:

The above copyright notice and this permission notice shall be
included in all copies or substantial portions of the Software.

THE SOFTWARE IS PROVIDED "AS IS", WITHOUT WARRANTY OF ANY KIND,
EXPRESS OR IMPLIED,
INCLUDING BUT NOT LIMITED TO THE WARRANTIES OF MERCHANTABILITY,
FITNESS FOR A PARTICULAR PURPOSE AND NONINFRINGEMENT.
IN NO EVENT SHALL THE AUTHORS OR COPYRIGHT HOLDERS BE LIABLE
FOR ANY CLAIM, DAMAGES OR OTHER LIABILITY,
WHETHER IN AN ACTION OF CONTRACT, TORT OR OTHERWISE,
ARISING FROM, OUT OF OR IN CONNECTION WITH THE SOFTWARE
OR THE USE OR OTHER DEALINGS IN THE SOFTWARE.


Copies of this manual can be obtained at
Lua's official web site,
\verb|www.lua.org|.

\bigskip
The Lua logo was designed by A. Nakonechny.
Copyright \copyright\ 1998.  All rights reserved.
\end{quotation}
%}===============================================================
\newpage

\title{\Large\bf Reference Manual of the Programming Language Lua \Version}

\author{%
Roberto Ierusalimschy\qquad
Luiz Henrique de Figueiredo\qquad
Waldemar Celes
\vspace{1.0ex}\\
\smallskip
\small\tt lua@tecgraf.puc-rio.br
\vspace{2.0ex}\\
%MCC 08/95 ---
\tecgraf\ --- Computer Science Department --- PUC-Rio
}

%\date{{\small \tt\$Date: 2002/06/06 12:49:28 $ $}}

\maketitle

\pagestyle{plain}
\pagenumbering{roman}

\begin{abstract}
\noindent
Lua is a powerful, light-weight programming language
designed for extending applications.
Lua is also frequently used as a general-purpose, stand-alone language.
Lua combines simple procedural syntax
(similar to Pascal)
with
powerful data description constructs
based on associative arrays and extensible semantics.
Lua is
dynamically typed,
interpreted from opcodes,
and has automatic memory management with garbage collection,
making it ideal for
configuration,
scripting,
and
rapid prototyping.

This document describes version \Version\ of the Lua programming language
and the Application Program Interface (API)
that allows interaction between Lua programs and their host C~programs.
\end{abstract}

\def\abstractname{Resumo}
\begin{abstract}
\noindent
Lua \'e uma linguagem de programa\c{c}\~ao
poderosa e leve,
projetada para estender aplica\c{c}\~oes.
Lua tamb\'em \'e frequentemente usada como uma linguagem de prop\'osito geral.
Lua combina programa\c{c}\~ao procedural
(com sintaxe semelhante \`a de Pascal)
com
poderosas constru\c{c}\~oes para descri\c{c}\~ao de dados,
baseadas em tabelas associativas e sem\^antica extens\'\i vel.
Lua \'e
tipada dinamicamente,
interpretada a partir de \emph{opcodes},
e tem gerenciamento autom\'atico de mem\'oria com coleta de lixo.
Essas caracter\'{\i}sticas fazem de Lua uma linguagem ideal para
configura\c{c}\~ao,
automa\c{c}\~ao (\emph{scripting})
e prototipagem r\'apida.

Este documento descreve a vers\~ao \Version\ da linguagem de
programa\c{c}\~ao Lua e a Interface de Programa\c{c}\~ao (API) que permite
a intera\c{c}\~ao entre programas Lua e programas C~hospedeiros.
\end{abstract}

\newpage
\null
\newpage
\tableofcontents

\newpage
\setcounter{page}{1}
\pagestyle{plain}
\pagenumbering{arabic}

%------------------------------------------------------------------------------
\section{Introduction}

Lua is an extension programming language designed to support
general procedural programming with data description
facilities.
Lua is intended to be used as a powerful, light-weight
configuration language for any program that needs one.
Lua is implemented as a library, written in C.

Being an extension language, Lua has no notion of a ``main'' program:
it only works \emph{embedded} in a host client,
called the \emph{embedding program} or simply the \emph{host}.
This host program can invoke functions to execute a piece of Lua code,
can write and read Lua variables,
and can register C~functions to be called by Lua code.
Through the use of C~functions, Lua can be augmented to cope with
a wide range of different domains,
thus creating customized programming languages sharing a syntactical framework.

Lua is free software,
and is provided as usual with no guarantees,
as stated in its copyright notice.
The implementation described in this manual is available
at Lua's official web site, \verb|www.lua.org|.

Like any other reference manual,
this document is dry in places.
For a discussion of the decisions behind the design of Lua,
see the papers below,
which are available at Lua's web site.
\begin{itemize}
\item
R.~Ierusalimschy, L.~H.~de Figueiredo, and W.~Celes.
Lua---an extensible extension language.
\emph{Software: Practice \& Experience} {\bf 26} \#6 (1996) 635--652.
\item
L.~H.~de Figueiredo, R.~Ierusalimschy, and W.~Celes.
The design and implementation of a language for extending applications.
\emph{Proceedings of XXI Brazilian Seminar on Software and Hardware} (1994) 273--283.
\item
L.~H.~de Figueiredo, R.~Ierusalimschy, and W.~Celes.
Lua: an extensible embedded language.
\emph{Dr. Dobb's Journal} {\bf  21} \#12 (Dec 1996) 26--33.
\item
R.~Ierusalimschy, L.~H.~de Figueiredo, and W.~Celes.
The evolution of an extension language: a history of Lua,
\emph{Proceedings of V Brazilian Symposium on Programming Languages} (2001) B-14--B-28.
\end{itemize}

%------------------------------------------------------------------------------
\section{Lua Concepts}\label{concepts}

This section describes the main concepts of Lua as a language.
The syntax and semantics of Lua are described in \See{language}.
The discussion below is not purely conceptual;
it includes references to the C~API \see{API},
because Lua is designed to be embedded in host programs.
It also includes references to the standard libraries \see{libraries}.


\subsection{Environment and Chunks}

All statements in Lua are executed in a \Def{global environment}.
This environment is initialized with a call from the embedding program to
\verb|lua_open| and
persists until a call to \verb|lua_close|
or the end of the embedding program.
If necessary,
the host programmer can create multiple independent global
environments, and freely switch between them \see{mangstate}.

The unit of execution of Lua is called a \Def{chunk}.
A chunk is simply a sequence of statements.
Statements are described in \See{stats}.

A chunk may be stored in a file or in a string inside the host program.
When a chunk is executed, first it is pre-compiled into opcodes for
a virtual machine,
and then the compiled statements are executed
by an interpreter for the virtual machine.
All modifications a chunk effects on the global environment persist
after the chunk ends.

Chunks may also be pre-compiled into binary form and stored in files;
see program \IndexVerb{luac} for details.
Text files with chunks and their binary pre-compiled forms
are interchangeable;
Lua automatically detects the file type and acts accordingly.
\index{pre-compilation}


\subsection{\Index{Values and Types}} \label{TypesSec}

Lua is a \emph{dynamically typed language}.
That means that
variables do not have types; only values do.
There are no type definitions in the language.
All values carry their own type.

There are seven \Index{basic types} in Lua:
\Def{nil}, \Def{boolean}, \Def{number},
\Def{string}, \Def{function}, \Def{userdata}, and \Def{table}.
\emph{Nil} is the type of the value \nil,
whose main property is to be different from any other value;
usually it represents the absence of a useful value.
\emph{Boolean} is the type of the values \False{} and \True.
In Lua, both \nil{} and \False{} make a condition fails,
and any other value makes it succeeds.
\emph{Number} represents real (double-precision floating-point) numbers.
\emph{String} represents arrays of characters.
\index{eight-bit clean}
Lua is 8-bit clean,
and so strings may contain any 8-bit character,
including embedded zeros (\verb|'\0'|) \see{lexical}.

Functions are \emph{first-class values} in Lua.
That means that functions can be stored in variables,
passed as arguments to other functions, and returned as results.
Lua can call (and manipulate) functions written in Lua and
functions written in C
\see{functioncall}.

The type \emph{userdata} is provided to allow the store of
arbitrary C data in Lua variables.
This type corresponds to a block of raw memory
and has no pre-defined operations in Lua,
except assignment and identity test.
However, by using \emph{metatables},
the programmer can define operations for userdata values
\see{metatables}.
Userdata values cannot be created or modified in Lua,
only through the C~API.
This guarantees the integrity of data owned by the host program.

The type \emph{table} implements \Index{associative arrays},
that is, \Index{arrays} that can be indexed not only with numbers,
but with any value (except \nil).
Moreover,
tables can be \emph{heterogeneous},
that is, they can contain values of all types.
Tables are the sole data structuring mechanism in Lua;
they may be used not only to represent ordinary arrays,
but also symbol tables, sets, records, graphs, trees, etc.
To represent \Index{records}, Lua uses the field name as an index.
The language supports this representation by
providing \verb|a.name| as syntactic sugar for \verb|a["name"]|.
There are several convenient ways to create tables in Lua
\see{tableconstructor}.

Like indices, the value of a table field can be of any type.
In particular,
because functions are first class values,
table fields may contain functions.
So, tables may also carry \emph{methods} \see{func-def}.

Tables, functions, and userdata values are \emph{objects}:
variables do not actually \emph{contain} these values,
only \emph{references} to them.
Assignment, parameter passing, and returns from functions
always manipulate references to these values,
and do not imply any kind of copy.

The library function \verb|type| returns a string describing the type
of a given value \see{pdf-type}.


\subsubsection{Metatables}

Each table or userdata object in Lua may have a \Index{metatable}.

You can change several aspects of the behavior
of an object by setting specific fields in its metatable.
For instance, when an object is the operand of an addition,
Lua checks for a function in the field \verb|"__add"| in its metatable.
If it finds one,
Lua calls that function to perform the addition.

We call the keys in a metatable \Index{events},
and the values \Index{metamethods}.
In the previous example, \verb|"add"| is the event,
and the metamethod is the function that performs the addition.

A metatable controls how an object behaves in arithmetic operations,
order comparisons, concatenation, and indexing.
A metatable can also defines a function to be called when a userdata
is garbage collected.
\See{metatable} gives a detailed description of which events you
can control with metatables.

You can query and change the metatable of an object
through the \verb|setmetatable| and \verb|getmetatable|
functions \see{pdf-getmetatable}.



\subsection{Coercion} \label{coercion}

Lua provides automatic conversion between
string and number values at run time.
Any arithmetic operation applied to a string tries to convert
that string to a number, following the usual rules.
Conversely, whenever a number is used when a string is expected,
the number is converted to a string, in a reasonable format.
The format is chosen so that
a conversion from number to string then back to number
reproduces the original number \emph{exactly}.
For complete control of how numbers are converted to strings,
use the \verb|format| function \see{format}.


\subsection{Variables}

There are two kinds of variables in Lua:
global variables
and local variables.
Variables are assumed to be global unless explicitly declared local
\see{localvar}.
Before the first assignment, the value of a variable is \nil.

All global variables live as fields in ordinary Lua tables.
Usually, globals live in a table called \Index{table of globals}.
However, a function can individually change its global table,
so that all global variables in that function will refer to that table.
This mechanism allows the creation of \Index{namespaces} and other
modularization facilities.

\Index{Local variables} are lexically scoped.
Therefore, local variables can be freely accessed by functions
defined inside their scope \see{visibility}.


\subsection{Garbage Collection}\label{GC}

Lua does automatic memory management.
That means that
you do not have to worry about allocating memory for new objects
and freeing it when the objects are no longer needed.
Lua manages memory automatically by running
a \Index{garbage collector} from time to time
and
collecting all dead objects
(all objects that are no longer accessible from Lua).
All objects in Lua are subject to automatic management:
tables, userdata, functions, and strings.

Using the C~API,
you can set garbage-collector metamethods for userdata \see{metatable}.
When it is about to free a userdata,
Lua calls the metamethod associated with event \verb|gc| in the
userdata's metatable.
Using such facility, you can coordinate Lua's garbage collection
with external resource management
(such as closing files, network or database connections,
or freeing your own memory).

Lua uses two numbers to control its garbage-collection cycles.
One number counts how many bytes of dynamic memory Lua is using,
and the other is a threshold.
When the number of bytes crosses the threshold,
Lua runs the garbage collector,
which reclaims the memory of all dead objects.
The byte counter is corrected,
and then the threshold is reset to twice the value of the byte counter.

Through the C~API, you can query those numbers,
and change the threshold \see{GC-API}.
Setting the threshold to zero actually forces an immediate
garbage-collection cycle,
while setting it to a huge number effectively stops the garbage collector.
Using Lua code you have a more limited control over garbage-collection cycles,
through the functions \verb|gcinfo| and \verb|collectgarbage|
\see{predefined}.


\subsubsection{Weak Tables}\label{weak-table}

A \IndexEmph{weak table} is a table whose elements are
\IndexEmph{weak references}.
A weak reference is ignored by the garbage collector.
In other words,
if the only references to an object are weak references,
then the garbage collector will collect that object.

A weak table can have weak keys, weak values, or both.
A table with weak keys allows the collection of its keys,
but prevents the collection of its values.
A table with both weak keys and weak values allows the collection of
both keys and values.
In any case, if either the key or the value is collected,
the whole pair is removed from the table.
The weakness of a table is set with the \verb|setmode| function.


%------------------------------------------------------------------------------
\section{The Language}\label{language}

This section describes the lexis, the syntax, and the semantics of Lua.
In other words,
this section describes
which tokens are valid,
how they can be combined,
and what their combinations mean.

\subsection{Lexical Conventions} \label{lexical}

\IndexEmph{Identifiers} in Lua can be any string of letters,
digits, and underscores,
not beginning with a digit.
This coincides with the definition of identifiers in most languages.
(The definition of letter depends on the current locale:
any character considered alphabetic by the current locale
can be used in an identifier.)

The following \IndexEmph{keywords} are reserved,
and cannot be used as identifiers:
\index{reserved words}
\begin{verbatim}
       and       break     do        else      elseif
       end       false     for       function  global
       if        in        local     nil       not
       or        repeat    return    then      true
       until     while
\end{verbatim}

Lua is a case-sensitive language:
\T{and} is a reserved word, but \T{And} and \T{\'and}
(if the locale permits) are two different, valid identifiers.
As a convention, identifiers starting with an underscore followed by
uppercase letters (such as \verb|_VERSION|)
are reserved for internal variables.

The following strings denote other \Index{tokens}:
\begin{verbatim}
       +     -     *     /     ^     %
       ~=    <=    >=    <     >     ==    =
       (     )     {     }     [     ]
       ;     :     ,     .     ..    ...
\end{verbatim}

\IndexEmph{Literal strings}
can be delimited by matching single or double quotes,
and can contain the C-like escape sequences
`\verb|\a|' (bell),
`\verb|\b|' (backspace),
`\verb|\f|' (form feed),
`\verb|\n|' (newline),
`\verb|\r|' (carriage return),
`\verb|\t|' (horizontal tab),
`\verb|\v|' (vertical tab),
`\verb|\\|' (backslash),
`\verb|\"|' (double quote),
`\verb|\'|' (single quote),
and `\verb|\|\emph{newline}' (that is, a backslash followed by a real newline,
which  results in a newline in the string).
A character in a string may also be specified by its numerical value,
through the escape sequence `\verb|\|\emph{ddd}',
where \emph{ddd} is a sequence of up to three \emph{decimal} digits.
Strings in Lua may contain any 8-bit value, including embedded zeros,
which can be specified as `\verb|\0|'.

Literal strings can also be delimited by matching \verb|[[| $\ldots$ \verb|]]|.
Literals in this bracketed form may run for several lines,
may contain nested \verb|[[| $\ldots$ \verb|]]| pairs,
and do not interpret escape sequences.
For convenience,
when the opening \verb|[[| is immediately followed by a newline,
the newline is not included in the string.
That form is specially convenient for
writing strings that contain program pieces or
other quoted strings.
As an example, in a system using ASCII
(in which `\verb|a|' is coded as~97,
newline is coded as~10, and `\verb|1|' is coded as~49),
the four literals below denote the same string:
\begin{verbatim}
       1)   "alo\n123\""
       2)   '\97lo\10\04923"'
       3)   [[alo
            123"]]
       4)   [[
            alo
            123"]]
\end{verbatim}

\IndexEmph{Numerical constants} may be written with an optional decimal part
and an optional decimal exponent.
Examples of valid numerical constants are
\begin{verbatim}
       3     3.0     3.1416  314.16e-2   0.31416E1
\end{verbatim}

\IndexEmph{Comments} start anywhere outside a string with a
double hyphen (\verb|--|);
If the text after \verb|--| is different from \verb|[[|,
the comment is a short comment,
that runs until the end of the line.
Otherwise, it is a long comment,
that runs until the corresponding \verb|]]|.
Long comments may run for several lines,
and may contain nested \verb|[[| $\ldots$ \verb|]]| pairs.
For convenience,
the first line of a chunk is skipped if it starts with \verb|#|.
This facility allows the use of Lua as a script interpreter
in Unix systems \see{lua-sa}.


\subsection{Variables}\label{variables}

Variables are places that store values.
%In Lua, variables are given by simple identifiers or by table fields.

A single name can denote a global variable, a local variable,
or a formal parameter in a function
(formal parameters are just local variables):
\begin{Produc}
\produc{var}{\Nter{Name}}
\end{Produc}%
Square brackets are used to index a table:
\begin{Produc}
\produc{var}{prefixexp \ter{[} exp \ter{]}}
\end{Produc}%
The first expression should result in a table value,
and the second expression identifies a specific entry inside that table.

The syntax \verb|var.NAME| is just syntactic sugar for
\verb|var["NAME"]|:
\begin{Produc}
\produc{var}{prefixexp \ter{.} \Nter{Name}}
\end{Produc}%

The expression denoting the table to be indexed has a restricted syntax;
\See{expressions} for details.

The meaning of assignments and evaluations of global and
indexed variables can be changed via metatables.
An assignment to a global variable \verb|x = val|
is equivalent to the assignment
\verb|_glob.x = val|,
where \verb|_glob| is the table of globals of the running function
(\see{global-table} for a discussion about the table of globals).
An assignment to an indexed variable \verb|t[i] = val| is equivalent to
\verb|settable_event(t,i,val)|.
An access to a global variable \verb|x|
is equivalent to \verb|_glob.x|
(again, \see{global-table} for a discussion about \verb|_glob|).
An access to an indexed variable \verb|t[i]| is equivalent to
a call \verb|gettable_event(t,i)|.
See \See{metatable} for a complete description of the
\verb|settable_event| and \verb|gettable_event| functions.
(These functions are not defined in Lua.
We use them here only for explanatory purposes.)


\subsection{Statements}\label{stats}

Lua supports an almost conventional set of \Index{statements},
similar to those in Pascal or C.
The conventional commands include
assignment, control structures, and procedure calls.
Non-conventional commands include table constructors
and variable declarations.

\subsubsection{Chunks}\label{chunks}
The unit of execution of Lua is called a \Def{chunk}.
A chunk is simply a sequence of statements,
which are executed sequentially.
Each statement can be optionally followed by a semicolon:
\begin{Produc}
\produc{chunk}{\rep{stat \opt{\ter{;}}}}
\end{Produc}%

\subsubsection{Blocks}
A \Index{block} is a list of statements;
syntactically, a block is equal to a chunk:
\begin{Produc}
\produc{block}{chunk}
\end{Produc}%

A block may be explicitly delimited to produce a single statement:
\begin{Produc}
\produc{stat}{\rwd{do} block \rwd{end}}
\end{Produc}%
\IndexKW{do}
Explicit blocks are useful
to control the scope of variable declarations.
Explicit blocks are also sometimes used to
add a \rwd{return} or \rwd{break} statement in the middle
of another block \see{control}.

\subsubsection{\Index{Assignment}} \label{assignment}
Lua allows \Index{multiple assignment}.
Therefore, the syntax for assignment
defines a list of variables on the left side
and a list of expressions on the right side.
The elements in both lists are separated by commas:
\begin{Produc}
\produc{stat}{varlist1 \ter{=} explist1}
\produc{varlist1}{var \rep{\ter{,} var}}
\produc{explist1}{exp \rep{\ter{,} exp}}
\end{Produc}%
Expressions are discussed in \See{expressions}.

Before the assignment,
the list of values is \emph{adjusted} to the length of
the list of variables.\index{adjustment}
If there are more values than needed,
the excess values are thrown away.
If there are less values than needed,
the list is extended with as many  \nil's as needed.
If the list of expressions ends with a function call,
then all values returned by that function call enter in the list of values,
before the adjust
(except when the call is enclosed in parentheses; see \See{expressions}).

The assignment statement first evaluates all its expressions,
and only then makes the assignments.
So, the code
\begin{verbatim}
       i = 3
       i, a[i] = i+1, 20
\end{verbatim}
sets \verb|a[3]| to 20, without affecting \verb|a[4]|
because the \verb|i| in \verb|a[i]| is evaluated
before it is assigned 4.
Similarly, the line
\begin{verbatim}
       x, y = y, x
\end{verbatim}
exchanges the values of \verb|x| and \verb|y|.

\subsubsection{Control Structures}\label{control}
The control structures
\rwd{if}, \rwd{while}, and \rwd{repeat} have the usual meaning and
familiar syntax:
\index{while-do statement}\IndexKW{while}
\index{repeat-until statement}\IndexKW{repeat}\IndexKW{until}
\index{if-then-else statement}\IndexKW{if}\IndexKW{else}\IndexKW{elseif}
\begin{Produc}
\produc{stat}{\rwd{while} exp \rwd{do} block \rwd{end}}
\produc{stat}{\rwd{repeat} block \rwd{until} exp}
\produc{stat}{\rwd{if} exp \rwd{then} block
  \rep{\rwd{elseif} exp \rwd{then} block}
   \opt{\rwd{else} block} \rwd{end}}
\end{Produc}%
Lua also has a \rwd{for} statement, in two flavors \see{for}.

The \Index{condition expression} \M{exp} of a
control structure may return any value.
All values different from \nil{} and \False{} are considered true
(in particular, the number 0 and the empty string are also true);
both \False{} and \nil{} are considered false.

The \rwd{return} statement is used to return values
from a function or from a chunk.\IndexKW{return}
\label{return}%
\index{return statement}%
Functions and chunks may return more than one value,
and so the syntax for the \rwd{return} statement is
\begin{Produc}
\produc{stat}{\rwd{return} \opt{explist1}}
\end{Produc}%

The \rwd{break} statement can be used to terminate the execution of a
\rwd{while}, \rwd{repeat}, or \rwd{for} loop,
skipping to the next statement after the loop:\IndexKW{break}
\index{break statement}
\begin{Produc}
\produc{stat}{\rwd{break}}
\end{Produc}%
A \rwd{break} ends the innermost enclosing loop.

\NOTE
For syntactic reasons, \rwd{return} and \rwd{break}
statements can only be written as the \emph{last} statement of a block.
If it is really necessary to \rwd{return} or \rwd{break} in the
middle of a block,
then an explicit inner block can used,
as in the idioms
`\verb|do return end|' and
`\verb|do break end|',
because now \rwd{return} and \rwd{break} are the last statements in
their (inner) blocks.
In practice,
those idioms are only used during debugging.
(For instance, a line `\verb|do return end|' can be added at the
beginning of a chunk for syntax checking only.)

\subsubsection{For Statement} \label{for}\index{for statement}

The \rwd{for} statement has two forms,
one for numbers and one generic.
\IndexKW{for}\IndexKW{in}

The numerical \rwd{for} loop repeats a block of code while a
control variable runs through an arithmetic progression.
It has the following syntax:
\begin{Produc}
\produc{stat}{\rwd{for} \Nter{Name} \ter{=} exp \ter{,} exp \opt{\ter{,} exp}
                    \rwd{do} block \rwd{end}}
\end{Produc}%
The \emph{block} is repeated for \emph{name} starting at the value of
the first \emph{exp}, until it reaches the second \emph{exp} by steps of the
third \emph{exp}.
More precisely, a \rwd{for} statement like
\begin{verbatim}
       for var = e1, e2, e3 do block end
\end{verbatim}
is equivalent to the code:
\begin{verbatim}
       do
         local var, _limit, _step = tonumber(e1), tonumber(e2), tonumber(e3)
         if not (var and _limit and _step) then error() end
         while (_step>0 and var<=_limit) or (_step<=0 and var>=_limit) do
           block
           var = var+_step
         end
       end
\end{verbatim}
Note the following:
\begin{itemize}\itemsep=0pt
\item Both the limit and the step are evaluated only once,
before the loop starts.
\item \verb|_limit| and \verb|_step| are invisible variables.
The names are here for explanatory purposes only.
\item The behavior is \emph{undefined} if you assign to \verb|var| inside
the block.
\item If the third expression (the step) is absent, then a step of~1 is used.
\item You can use \rwd{break} to exit a \rwd{for} loop.
\item The loop variable \verb|var| is local to the statement;
you cannot use its value after the \rwd{for} ends or is broken.
If you need the value of the loop variable \verb|var|,
then assign it to another variable before breaking or exiting the loop.
\end{itemize}

The generic \rwd{for} statement works over functions,
called \Index{generators}.
It calls its generator to produce a new value for each iteration,
stopping when the new value is \nil.
It has the following syntax:
\begin{Produc}
\produc{stat}{\rwd{for} \Nter{Name} \rep{\ter{,} \Nter{Name}} \rwd{in} explist1
                    \rwd{do} block \rwd{end}}
\end{Produc}%
A \rwd{for} statement like
\begin{verbatim}
       for var_1, ..., var_n in explist do block end
\end{verbatim}
is equivalent to the code:
\begin{verbatim}
       do
         local _f, _s, var_1 = explist
         while 1 do
           local var_2, ..., var_n
           var_1, ..., var_n = _f(_s, var_1)
           if var_1 == nil then break end
           block
         end
       end
\end{verbatim}
Note the following:
\begin{itemize}\itemsep=0pt
\item \verb|explist| is evaluated only once.
Its results are a ``generator'' function,
a ``state'', and an initial value for the ``iterator variable''.
\item \verb|_f| and \verb|_s| are invisible variables.
The names are here for explanatory purposes only.
\item The behavior is \emph{undefined} if you assign to any
\verb|var_i| inside the block.
\item You can use \rwd{break} to exit a \rwd{for} loop.
\item The loop variables \verb|var_i| are local to the statement;
you cannot use their values after the \rwd{for} ends.
If you need these values,
then assign them to other variables before breaking or exiting the loop.
\end{itemize}


\subsubsection{Function Calls as Statements} \label{funcstat}
Because of possible side-effects,
function calls can be executed as statements:
\begin{Produc}
\produc{stat}{functioncall}
\end{Produc}%
In this case, all returned values are thrown away.
Function calls are explained in \See{functioncall}.

\subsubsection{Local Declarations} \label{localvar}
\Index{Local variables} may be declared anywhere inside a block.
The declaration may include an initial assignment:\IndexKW{local}
\begin{Produc}
\produc{stat}{\rwd{local} namelist \opt{\ter{=} explist1}}
\produc{namelist}{\Nter{Name} \rep{\ter{,} \Nter{Name}}}
\end{Produc}%
If present, an initial assignment has the same semantics
of a multiple assignment \see{assignment}.
Otherwise, all variables are initialized with \nil.

A chunk is also a block \see{chunks},
and so local variables can be declared outside any explicit block.
Such local variables die when the chunk ends.

Visibility rules for local variables are explained in \See{visibility}.


\subsection{\Index{Expressions}}\label{expressions}

%\subsubsection{\Index{Basic Expressions}}
The basic expressions in Lua are the following:
\begin{Produc}
\produc{exp}{prefixexp}
\produc{exp}{\rwd{nil} \Or \rwd{false} \Or \rwd{true}}
\produc{exp}{Number}
\produc{exp}{Literal}
\produc{exp}{function}
\produc{exp}{tableconstructor}
\produc{prefixexp}{var \Or functioncall \Or \ter{(} exp \ter{)}}
\end{Produc}%
\IndexKW{nil}\IndexKW{false}\IndexKW{true}

An expression enclosed in parentheses always results in only one value.
Thus,
\verb|(f(x,y,z))| is always a single value,
even if \verb|f| returns several values.
(The value of \verb|(f(x,y,z))| is the first value returned by \verb|f|
or \nil{} if \verb|f| does not return any values.)

\emph{Numbers} and \emph{literal strings} are explained in \See{lexical};
variables are explained in \See{variables};
function definitions are explained in \See{func-def};
function calls are explained in \See{functioncall};
table constructors are explained in \See{tableconstructor}.

Expressions can also be built with arithmetic operators, relational operators,
and logical operadors, all of which are explained below.

\subsubsection{Arithmetic Operators}
Lua supports the usual \Index{arithmetic operators}:
the binary \verb|+| (addition),
\verb|-| (subtraction), \verb|*| (multiplication),
\verb|/| (division), and \verb|^| (exponentiation);
and unary \verb|-| (negation).
If the operands are numbers, or strings that can be converted to
numbers \see{coercion},
then all operations except exponentiation have the usual meaning,
while exponentiation calls a global function \verb|pow|; ??
otherwise, an appropriate metamethod is called \see{metatable}.
The standard mathematical library defines function \verb|pow|,
giving the expected meaning to \Index{exponentiation}
\see{mathlib}.

\subsubsection{Relational Operators}\label{rel-ops}
The \Index{relational operators} in Lua are
\begin{verbatim}
       ==    ~=    <     >     <=    >=
\end{verbatim}
These operators always result in \False{} or \True.

Equality (\verb|==|) first compares the type of its operands.
If the types are different, then the result is \False.
Otherwise, the values of the operands are compared.
Numbers and strings are compared in the usual way.
Tables, userdata, and functions are compared \emph{by reference},
that is,
two tables are considered equal only if they are the \emph{same} table.

??eq metamethod??

Every time you create a new table (or userdata, or function),
this new value is different from any previously existing value.

\NOTE
The conversion rules of \See{coercion}
\emph{do not} apply to equality comparisons.
Thus, \verb|"0"==0| evaluates to \emph{false},
and \verb|t[0]| and \verb|t["0"]| denote different
entries in a table.
\medskip

The operator \verb|~=| is exactly the negation of equality (\verb|==|).

The order operators work as follows.
If both arguments are numbers, then they are compared as such.
Otherwise, if both arguments are strings,
then their values are compared according to the current locale.
Otherwise, the ``lt'' or the ``le'' metamethod is called \see{metatable}.


\subsubsection{Logical Operators}
The \Index{logical operators} in Lua are
\index{and}\index{or}\index{not}
\begin{verbatim}
       and   or    not
\end{verbatim}
Like the control structures \see{control},
all logical operators consider both \False{} and \nil{} as false
and anything else as true.
\IndexKW{and}\IndexKW{or}\IndexKW{not}

The operator \rwd{not} always return \False{} or \True.

The conjunction operator \rwd{and} returns its first argument
if its value is \False{} or \nil;
otherwise, \rwd{and} returns its second argument.
The disjunction operator \rwd{or} returns its first argument
if it is different from \nil and \False;
otherwise, \rwd{or} returns its second argument.
Both \rwd{and} and \rwd{or} use \Index{short-cut evaluation},
that is,
the second operand is evaluated only if necessary.
For example,
\begin{verbatim}
       10 or error()       -> 10
       nil or "a"          -> "a"
       nil and 10          -> nil
       false and error()   -> false
       false and nil       -> false
       false or nil        -> nil
       10 and 20           -> 20
\end{verbatim}

\subsubsection{Concatenation} \label{concat}
The string \Index{concatenation} operator in Lua is
denoted by two dots (`\verb|..|').
If both operands are strings or numbers, then they are converted to
strings according to the rules mentioned in \See{coercion}.
Otherwise, the ``concat'' metamethod is called \see{metatable}.

\subsubsection{Precedence}
\Index{Operator precedence} in Lua follows the table below,
from lower to higher priority:
\begin{verbatim}
       or
       and
       <     >     <=    >=    ~=    ==
       ..
       +     -
       *     /
       not   - (unary)
       ^
\end{verbatim}
All binary operators are left associative,
except for \verb|^| (exponentiation),
which is right associative.
\NOTE
The pre-compiler may rearrange the order of evaluation of
associative operators,
and may exchange the operands of commutative operators,
as long as these optimizations do not change normal results.
However, these optimizations may change some results
if you define non-associative (or non-commutative)
metamethods for those operators.

\subsubsection{Table Constructors} \label{tableconstructor}
Table \Index{constructors} are expressions that create tables;
every time a constructor is evaluated, a new table is created.
Constructors can be used to create empty tables,
or to create a table and initialize some of its fields.
The general syntax for constructors is
\begin{Produc}
\produc{tableconstructor}{\ter{\{} \opt{fieldlist} \ter{\}}}
\produc{fieldlist}{field \rep{fieldsep field} \opt{fieldsep}}
\produc{field}{\ter{[} exp \ter{]} \ter{=} exp \Or
               \Nter{Name} \ter{=} exp \Or exp}
\produc{fieldsep}{\ter{,} \Or \ter{;}}
\end{Produc}%

Each field of the form \verb|[exp1] = exp2| adds to the new table an entry
with key \verb|exp1| and value \verb|exp2|.
A field of the form \verb|name = exp| is equivalent to
\verb|["name"] = exp|.
Finally, fields of the form \verb|exp| are equivalent to
\verb|[i] = exp|, where \verb|i| are consecutive numerical integers,
starting with 1.
Fields in the other formats do not affect this counting.
For example,
\begin{verbatim}
       a = {[f(1)] = g; "x", "y"; x = 1, f(x), [30] = 23; 45}
\end{verbatim}
is equivalent to
\begin{verbatim}
       do
         local temp = {}
         temp[f(1)] = g
         temp[1] = "x"         -- 1st exp
         temp[2] = "y"         -- 2nd exp
         temp.x = 1            -- temp["x"] = 1
         temp[3] = f(x)        -- 3rd exp
         temp[30] = 23
         temp[4] = 45          -- 4th exp
         a = temp
       end
\end{verbatim}

If the last expression in the list is a function call,
then all values returned by the call enter the list consecutively
\see{functioncall}.
If you want to avoid this,
enclose the function call in parentheses.

The field list may have an optional trailing separator,
as a convenience for machine-generated code.


\subsubsection{Function Calls}  \label{functioncall}
A \Index{function call} in Lua has the following syntax:
\begin{Produc}
\produc{functioncall}{prefixexp args}
\end{Produc}%
In a function call,
first \M{prefixexp} and \M{args} are evaluated.
If the value of \M{prefixexp} has type \emph{function},
then that function is called,
with the given arguments.
Otherwise, its ``call'' metamethod is called,
having as first parameter the value of \M{prefixexp},
followed by the original call arguments
\see{metatable}.

The form
\begin{Produc}
\produc{functioncall}{prefixexp \ter{:} \Nter{name} args}
\end{Produc}%
can be used to call ``methods''.
A call \verb|v:name(...)|
is syntactic sugar for \verb|v.name(v, ...)|,
except that \verb|v| is evaluated only once.

Arguments have the following syntax:
\begin{Produc}
\produc{args}{\ter{(} \opt{explist1} \ter{)}}
\produc{args}{tableconstructor}
\produc{args}{Literal}
\end{Produc}%
All argument expressions are evaluated before the call.
A call of the form \verb|f{...}| is syntactic sugar for
\verb|f({...})|, that is,
the argument list is a single new table.
A call of the form \verb|f'...'|
(or \verb|f"..."| or \verb|f[[...]]|) is syntactic sugar for
\verb|f('...')|, that is,
the argument list is a single literal string.

Because a function can return any number of results
\see{return},
the number of results must be adjusted before they are used.
If the function is called as a statement \see{funcstat},
then its return list is adjusted to~0 elements,
thus discarding all returned values.
If the function is called inside another expression,
or in the middle of a list of expressions,
then its return list is adjusted to~1 element,
thus discarding all returned values but the first one.
If the function is called as the last element of a list of expressions,
then no adjustment is made
(unless the call is enclosed in parentheses).

Here are some examples:
\begin{verbatim}
       f()                -- adjusted to 0 results
       g(f(), x)          -- f() is adjusted to 1 result
       g(x, f())          -- g gets x plus all values returned by f()
       a,b,c = f(), x     -- f() is adjusted to 1 result (and c gets nil)
       a,b,c = x, f()     -- f() is adjusted to 2
       a,b,c = f()        -- f() is adjusted to 3
       return f()         -- returns all values returned by f()
       return x,y,f()     -- returns x, y, and all values returned by f()
       {f()}              -- creates a list with all values returned by f()
       {f(), nil}         -- f() is adjusted to 1 result
\end{verbatim}

If you enclose a function call in parentheses,
then it is adjusted to return exactly one value:
\begin{verbatim}
       return x,y,(f())   -- returns x, y, and the first value from f()
       {(f())}            -- creates a table with exactly one element
\end{verbatim}

As an exception to the format-free syntax of Lua,
you cannot put a line break before the \verb|(| in a function call.
That restriction avoids some ambiguities in the language.
If you write
\begin{verbatim}
       a = f
       (g).x(a)
\end{verbatim}
Lua would read that as \verb|a = f(g).x(a)|.
So, if you want two statements, you must add a semi-colon between them.
If you actually want to call \verb|f|,
you must remove the line break before \verb|(g)|.


\subsubsection{\Index{Function Definitions}} \label{func-def}

The syntax for function definition is\IndexKW{function}
\begin{Produc}
\produc{function}{\rwd{function} funcbody}
\produc{funcbody}{\ter{(} \opt{parlist1} \ter{)} block \rwd{end}}
\end{Produc}%

The following syntactic sugar simplifies function definitions:
\begin{Produc}
\produc{stat}{\rwd{function} funcname funcbody}
\produc{stat}{\rwd{local} \rwd{function} \Nter{name} funcbody}
\produc{funcname}{\Nter{name} \rep{\ter{.} \Nter{name}} \opt{\ter{:} \Nter{name}}}
\end{Produc}%
The statement
\begin{verbatim}
       function f () ... end
\end{verbatim}
translates to
\begin{verbatim}
       f = function () ... end
\end{verbatim}
The statement
\begin{verbatim}
       function t.a.b.c.f () ... end
\end{verbatim}
translates to
\begin{verbatim}
       t.a.b.c.f = function () ... end
\end{verbatim}
The statement
\begin{verbatim}
       local function f () ... end
\end{verbatim}
translates to
\begin{verbatim}
       local f; f = function () ... end
\end{verbatim}

A function definition is an executable expression,
whose value has type \emph{function}.
When Lua pre-compiles a chunk,
all its function bodies are pre-compiled too.
Then, whenever Lua executes the function definition,
the function is \emph{instantiated} (or \emph{closed}).
This function instance (or \emph{closure})
is the final value of the expression.
Different instances of the same function
may refer to different non-local variables \see{visibility}
and may have different tables of globals \see{global-table}.

Parameters act as local variables,
initialized with the argument values:
\begin{Produc}
\produc{parlist1}{namelist \opt{\ter{,} \ter{\ldots}}}
\produc{parlist1}{\ter{\ldots}}
\end{Produc}%
\label{vararg}%
When a function is called,
the list of \Index{arguments} is adjusted to
the length of the list of parameters,
unless the function is a \Def{vararg function},
which is
indicated by three dots (`\verb|...|') at the end of its parameter list.
A vararg function does not adjust its argument list;
instead, it collects all extra arguments into an implicit parameter,
called \IndexLIB{arg}.
The value of \verb|arg| is a table,
with a field~\verb|n| whose value is the number of extra arguments,
and the extra arguments at positions 1,~2,~\ldots,~\verb|n|.

As an example, consider the following definitions:
\begin{verbatim}
       function f(a, b) end
       function g(a, b, ...) end
       function r() return 1,2,3 end
\end{verbatim}
Then, we have the following mapping from arguments to parameters:
\begin{verbatim}
       CALL            PARAMETERS

       f(3)             a=3, b=nil
       f(3, 4)          a=3, b=4
       f(3, 4, 5)       a=3, b=4
       f(r(), 10)       a=1, b=10
       f(r())           a=1, b=2

       g(3)             a=3, b=nil, arg={n=0}
       g(3, 4)          a=3, b=4,   arg={n=0}
       g(3, 4, 5, 8)    a=3, b=4,   arg={5, 8; n=2}
       g(5, r())        a=5, b=1,   arg={2, 3; n=2}
\end{verbatim}

Results are returned using the \rwd{return} statement \see{return}.
If control reaches the end of a function
without encountering a \rwd{return} statement,
then the function returns with no results.

The \emph{colon} syntax
is used for defining \IndexEmph{methods},
that is, functions that have an implicit extra parameter \IndexVerb{self}.
Thus, the statement
\begin{verbatim}
       function t.a.b.c:f (...) ... end
\end{verbatim}
is syntactic sugar for
\begin{verbatim}
       t.a.b.c.f = function (self, ...) ... end
\end{verbatim}


\subsection{Visibility Rules} \label{visibility}
\index{visibility}

Lua is a lexically scoped language.
The scope of variables begins at the first statement \emph{after}
their declaration and lasts until the end of the innermost block that
includes the declaration.
For instance:
\begin{verbatim}
  x = 10                -- global variable
  do                    -- new block
    local x = x         -- new `x', with value 10
    print(x)            --> 10
    x = x+1
    do                  -- another block
      local x = x+1     -- another `x'
      print(x)          --> 12
    end
    print(x)            --> 11
  end
  print(x)              --> 10  (the global one)
\end{verbatim}
Notice that, in a declaration like \verb|local x = x|,
the new \verb|x| being declared is not in scope yet,
so the second \verb|x| refers to the ``outside'' variable.

Because of those \Index{lexical scoping} rules,
local variables can be freely accessed by functions
defined inside their scope.
For instance:
\begin{verbatim}
  local counter = 0
  function inc (x)
    counter = counter + x
    return counter
  end
\end{verbatim}

Notice that each execution of a \rwd{local} statement
``creates'' new local variables.
Consider the following example:
\begin{verbatim}
  a = {}
  local x = 20
  for i=1,10 do
    local y = 0
    a[i] = function () y=y+1; return x+y end
  end
\end{verbatim}
In that code,
each function uses a different \verb|y| variable,
while all of them share the same \verb|x|.

\subsection{Error Handling} \label{error}

%% TODO Must be rewritten!!!

Because Lua is an extension language,
all Lua actions start from C~code in the host program
calling a function from the Lua library.
Whenever an error occurs during Lua compilation or execution,
the function \verb|_ERRORMESSAGE| is called \DefLIB{_ERRORMESSAGE}
(provided it is different from \nil),
and then the corresponding function from the library
(\verb|lua_dofile|, \verb|lua_dostring|,
\verb|lua_dobuffer|, or \verb|lua_call|)
is terminated, returning an error condition.

Memory allocation errors are an exception to the previous rule.
When memory allocation fails, Lua may not be able to execute the
\verb|_ERRORMESSAGE| function.
So, for this kind of error, Lua does not call
the \verb|_ERRORMESSAGE| function;
instead, the corresponding function from the library
returns immediately with a special error code (\verb|LUA_ERRMEM|).
This and other error codes are defined in \verb|lua.h|
\see{luado}.

The only argument to \verb|_ERRORMESSAGE| is a string
describing the error.
The default definition for
this function calls \verb|_ALERT|, \DefLIB{_ALERT}
which prints the message to \verb|stderr| \see{alert}.
The standard I/O library redefines \verb|_ERRORMESSAGE|
and uses the debug interface \see{debugI}
to print some extra information,
such as a call-stack traceback.

Lua code can explicitly generate an error by calling the
function \verb|error| \see{pdf-error}.
Lua code can ``catch'' an error using the function
\verb|call| \see{pdf-call}.


\subsection{Metatables} \label{metatable}

Every table and userdata value in Lua may have a \emph{metatable}.
This \IndexEmph{metatable} is a table that defines the behavior of
the original table and userdata under some operations.
You can query and change the metatable of an object with
functions \verb|setmetatable| and \verb|getmetatable| \see{pdf-getmetatable}.

For each of those operations Lua associates a specific key,
called an \emph{event}.
When Lua performs one of those operations over a table or a userdata,
if checks whether that object has a metatable with the corresponding event.
If so, the value associated with that key (the \IndexEmph{metamethod})
controls how Lua will perform the operation.

Metatables control the operations listed next.
Each operation is identified by its corresponding name.
The key for each operation is a string with its name prefixed by
two underscores;
for instance, the key for operation ``add'' is the
string \verb|"__add"|.
The semantics of these operations is better explained by a Lua function
describing how the interpreter executes that operation.
%Each function shows how a handler is called,
%its arguments (that is, its signature),
%its results,
%and the default behavior in the absence of a handler.
The code shown here in Lua is only illustrative;
the real behavior is hard coded in the interpreter,
and it is much more efficient than this simulation.
All functions used in these descriptions
(\verb|rawget|, \verb|tonumber|, etc.)
are described in \See{predefined}.

\begin{description}

\item[``add'':]\IndexTM{add}
the \verb|+| operation.

The function \verb|getbinhandler| below defines how Lua chooses a handler
for a binary operation.
First, Lua tries the first operand.
If its type does not define a handler for the operation,
then Lua tries the second operand.
\begin{verbatim}
       function getbinhandler (op1, op2, event)
         return metatable(op1)[event] or metatable(op2)[event]
       end
\end{verbatim}
Using that function,
the behavior of the ``add'' operation is
\begin{verbatim}
       function add_event (op1, op2)
         local o1, o2 = tonumber(op1), tonumber(op2)
         if o1 and o2 then  -- both operands are numeric
           return o1+o2  -- '+' here is the primitive 'add'
         else  -- at least one of the operands is not numeric
           local h = getbinhandler(op1, op2, "__add")
           if h then
             -- call the handler with both operands
             return h(op1, op2)
           else  -- no handler available: default behavior
             error("unexpected type at arithmetic operation")
           end
         end
       end
\end{verbatim}

\item[``sub'':]\IndexTM{sub}
the \verb|-| operation.
Behavior similar to the ``add'' operation.

\item[``mul'':]\IndexTM{mul}
the \verb|*| operation.
Behavior similar to the ``add'' operation.

\item[``div'':]\IndexTM{div}
the \verb|/| operation.
Behavior similar to the ``add'' operation.

\item[``pow'':]\IndexTM{pow}
the \verb|^| operation (exponentiation) operation.
\begin{verbatim} ??
       function pow_event (op1, op2)
         local h = getbinhandler(op1, op2, "__pow") ???
         if h then
           -- call the handler with both operands
           return h(op1, op2)
         else  -- no handler available: default behavior
           error("unexpected type at arithmetic operation")
         end
       end
\end{verbatim}

\item[``unm'':]\IndexTM{unm}
the unary \verb|-| operation.
\begin{verbatim}
       function unm_event (op)
         local o = tonumber(op)
         if o then  -- operand is numeric
           return -o  -- '-' here is the primitive 'unm'
         else  -- the operand is not numeric.
           -- Try to get a handler from the operand;
           local h = metatable(op).__unm
           if h then
             -- call the handler with the operand and nil
             return h(op, nil)
           else  -- no handler available: default behavior
             error("unexpected type at arithmetic operation")
           end
         end
       end
\end{verbatim}

\item[``lt'':]\IndexTM{lt}
the \verb|<| operation.
\begin{verbatim}
       function lt_event (op1, op2)
         if type(op1) == "number" and type(op2) == "number" then
           return op1 < op2   -- numeric comparison
         elseif type(op1) == "string" and type(op2) == "string" then
           return op1 < op2   -- lexicographic comparison
         else
           local h = getbinhandler(op1, op2, "__lt")
           if h then
             return h(op1, op2)
           else
             error("unexpected type at comparison");
           end
         end
       end
\end{verbatim}
\verb|a>b| is equivalent to \verb|b<a|.

\item[``le'':]\IndexTM{lt}
the \verb|<=| operation.
\begin{verbatim}
       function lt_event (op1, op2)
         if type(op1) == "number" and type(op2) == "number" then
           return op1 < op2   -- numeric comparison
         elseif type(op1) == "string" and type(op2) == "string" then
           return op1 < op2   -- lexicographic comparison
         else
           local h = getbinhandler(op1, op2, "__le")
           if h then
             return h(op1, op2)
           else
             h = getbinhandler(op1, op2, "__lt")
             if h then
               return not h(op2, op1)
             else
               error("unexpected type at comparison");
             end
           end
         end
       end
\end{verbatim}
\verb|a>=b| is equivalent to \verb|b<=a|.
Notice that, in the absence of a ``le'' metamethod,
Lua tries the ``lt'', assuming that \verb|a<=b| is
equivalent to \verb|not (b<a)|.


\item[``concat'':]\IndexTM{concatenation}
the \verb|..| (concatenation) operation.
\begin{verbatim}
       function concat_event (op1, op2)
         if (type(op1) == "string" or type(op1) == "number") and
            (type(op2) == "string" or type(op2) == "number") then
           return op1..op2  -- primitive string concatenation
         else
           local h = getbinhandler(op1, op2, "__concat")
           if h then
             return h(op1, op2)
           else
             error("unexpected type for concatenation")
           end
         end
       end
\end{verbatim}

\item[``index'':]\IndexTM{index}
This handler is called when Lua tries to retrieve the value of an index
not present in a table.
See the ``gettable'' operation for its semantics.

\item[``gettable'':]\IndexTM{gettable}
called whenever Lua accesses an indexed variable.
\begin{verbatim}
       function gettable_event (table, key)
         local h
         if type(table) == "table" then
           local v = rawget(table, key)
           if v ~= nil then return v end
           h = metatable(table).__index
           if h == nil then return nil end
         else
           h = metatable(table).__gettable
           if h == nil then
             error("indexed expression not a table");
           end
         end
         if type(h) == "function" then
           return h(table, key)      -- call the handler
         else return h[key]          -- or repeat operation with it
       end
\end{verbatim}

\item[``newindex'':]\IndexTM{index}
This handler is called when Lua tries to insert the value of an index
not present in a table.
See the ``settable'' operation for its semantics.

\item[``settable'':]\IndexTM{settable}
called when Lua assigns to an indexed variable.
\begin{verbatim}
       function settable_event (table, key, value)
         local h
         if type(table) == "table" then
           local v = rawget(table, key)
           if v ~= nil then rawset(table, key, value); return end
           h = metatable(table).__newindex
           if h == nil then rawset(table, key, value); return end
         else
           h = metatable(table).__settable
           if h == nil then
             error("indexed expression not a table");
           end
         end
         if type(h) == "function" then
           return h(table, key,value)    -- call the handler
         else h[key] = value             -- or repeat operation with it
       end
\end{verbatim}


\item[``call'':]\IndexTM{call}
called when Lua calls a value.
\begin{verbatim}
       function function_event (func, ...)
         if type(func) == "function" then
           return func(unpack(arg))   -- regular call
         else
           local h = metatable(func).__call
           if h then
             tinsert(arg, 1, func)
             return h(unpack(arg))
           else
             error("call expression not a function")
           end
         end
       end
\end{verbatim}

\end{description}

\subsubsection{Metatables and Garbage collection}

Metatables may also define \IndexEmph{finalizer} methods
for userdata values.
For each userdata to be collected,
Lua does the equivalent of the following function:
\begin{verbatim}
       function gc_event (obj)
         local h = metatable(obj).__gc
         if h then
           h(obj)
         end
       end
\end{verbatim}
In a garbage-collection cycle,
the finalizers for userdata are called in \emph{reverse}
order of their creation,
that is, the first finalizer to be called is the one associated
with the last userdata created in the program
(among those to be collected in the same cycle).



%------------------------------------------------------------------------------
\section{The Application Program Interface}\label{API}
\index{C API}

This section describes the API for Lua, that is,
the set of C~functions available to the host program to communicate
with Lua.
All API functions and related types and constants
are declared in the header file \verb|lua.h|.

\NOTE
Even when we use the term ``function'',
any facility in the API may be provided as a \emph{macro} instead.
All such macros use each of its arguments exactly once
(except for the first argument, which is always a Lua state),
and so do not generate hidden side-effects.


\subsection{States} \label{mangstate}

The Lua library is fully reentrant:
it has no global variables.
\index{state}
The whole state of the Lua interpreter
(global variables, stack, tag methods, etc.)\
is stored in a dynamically allocated structure of type \verb|lua_State|;
\DefAPI{lua_State}
this state must be passed as the first argument to
every function in the library (except \verb|lua_open| below).

Before calling any API function,
you must create a state by calling
\begin{verbatim}
       lua_State *lua_open (void);
\end{verbatim}
\DefAPI{lua_open}

To release a state created with \verb|lua_open|, call
\begin{verbatim}
       void lua_close (lua_State *L);
\end{verbatim}
\DefAPI{lua_close}
This function destroys all objects in the given Lua environment
(calling the corresponding garbage-collection metamethods, if any)
and frees all dynamic memory used by that state.
Usually, you do not need to call this function,
because all resources are naturally released when your program ends.
On the other hand,
long-running programs ---
like a daemon or a web server ---
might need to release states as soon as they are not needed,
to avoid growing too large.

With the exception of \verb|lua_open|,
all functions in the Lua API need a state as their first argument.


\subsection{Threads}

Lua offers a partial support for multiple threads of execution.
If you have a C~library that offers multi-threading, 
then Lua can cooperate with it to implement the equivalent facility in Lua.
Also, Lua implements its own coroutine system on top of threads.
The following function creates a new ``thread'' in Lua:
\begin{verbatim}
       lua_State *lua_newthread (lua_State *L);
\end{verbatim}
\DefAPI{lua_newthread}
The new state returned by this function shares with the original state
all global environment (such as tables, tag methods, etc.),
but has an independent run-time stack.
(The use of these multiple stacks must be ``syncronized'' with C.
How to explain that? TO BE WRITTEN.)

Each thread has an independent table for global variables.
When you create a thread, this table is the same as that of the given state,
but you can change each one independently.

You destroy threads with \DefAPI{lua_closethread}
\begin{verbatim}
       void lua_closethread (lua_State *L, lua_State *thread);
\end{verbatim}
You cannot close the sole (or last) thread of a state.
Instead, you must close the state itself.


\subsection{The Stack and Indices}

Lua uses a virtual \emph{stack} to pass values to and from C.
Each element in this stack represents a Lua value
(\nil, number, string, etc.).

Each C invocation has its own stack.
Whenever Lua calls C, the called function gets a new stack,
which is independent of previous stacks or of stacks of still
active C functions.

For convenience,
most query operations in the API do not follow a strict stack discipline.
Instead, they can refer to any element in the stack by using an \emph{index}:
A positive index represents an \emph{absolute} stack position
(starting at~1);
a negative index represents an \emph{offset} from the top of the stack.
More specifically, if the stack has \M{n} elements,
then index~1 represents the first element
(that is, the element that was pushed onto the stack first),
and
index~\M{n} represents the last element;
index~\Math{-1} also represents the last element
(that is, the element at the top),
and index \Math{-n} represents the first element.
We say that an index is \emph{valid}
if it lies between~1 and the stack top
(that is, if \verb|1 <= abs(index) <= top|).
\index{stack index} \index{valid index}

At any time, you can get the index of the top element by calling
\begin{verbatim}
       int lua_gettop (lua_State *L);
\end{verbatim}
\DefAPI{lua_gettop}
Because indices start at~1,
the result of \verb|lua_gettop| is equal to the number of elements in the stack
(and so 0~means an empty stack).

When you interact with Lua API,
\emph{you are responsible for controlling stack overflow}.
The function
\begin{verbatim}
       int lua_checkstack (lua_State *L, int extra);
\end{verbatim}
\DefAPI{lua_checkstack}
grows the stack size to \verb|top + extra| elements;
it returns false if it cannot grow the stack to that size.
This function never shrinks the stack;
if the stack is already bigger than the new size,
it is left unchanged.

Whenever Lua calls C, \DefAPI{LUA_MINSTACK}
it ensures that \verb|lua_checkstack(L, LUA_MINSTACK)| is true,
that is,
at least \verb|LUA_MINSTACK| positions are still available.
\verb|LUA_MINSTACK| is defined in \verb|lua.h| as 20,
so that usually you do not have to worry about stack space
unless your code has loops pushing elements onto the stack.

Most query functions accept as indices any value inside the
available stack space, that is, indices up to the maximum stack size
you (or Lua) have set through \verb|lua_checkstack|.
Such indices are called \emph{acceptable indices}.
More formally, we define an \IndexEmph{acceptable index}
as follows:
\begin{verbatim}
     (index < 0 && abs(index) <= top) || (index > 0 && index <= top + stackspace)
\end{verbatim}
Note that 0 is never an acceptable index.

Unless otherwise noticed,
any function that accepts valid indices can also be called with
\Index{pseudo-indices},
which represent some Lua values that are accessible to the C~code
but are not in the stack.
Pseudo-indices are used to access the table of globals \see{globals},
the registry, and the upvalues of a C function \see{c-closure}.

\subsection{Stack Manipulation}
The API offers the following functions for basic stack manipulation:
\begin{verbatim}
       void lua_settop    (lua_State *L, int index);
       void lua_pushvalue (lua_State *L, int index);
       void lua_remove    (lua_State *L, int index);
       void lua_insert    (lua_State *L, int index);
       void lua_replace   (lua_State *L, int index);
\end{verbatim}
\DefAPI{lua_settop}\DefAPI{lua_pushvalue}
\DefAPI{lua_remove}\DefAPI{lua_insert}\DefAPI{lua_replace}

\verb|lua_settop| accepts any acceptable index,
or 0,
and sets the stack top to that index.
If the new top is larger than the old one,
then the new elements are filled with \nil.
If \verb|index| is 0, then all stack elements are removed.
A useful macro defined in the \verb|lua.h| is
\begin{verbatim}
       #define lua_pop(L,n) lua_settop(L, -(n)-1)
\end{verbatim}
\DefAPI{lua_pop}
which pops \verb|n| elements from the stack.

\verb|lua_pushvalue| pushes onto the stack a copy of the element
at the given index.
\verb|lua_remove| removes the element at the given position,
shifting down the elements above that position to fill the gap.
\verb|lua_insert| moves the top element into the given position,
shifting up the elements above that position to open space.
\verb|lua_replace| moves the top element into the given position,
without shifting any element (therefore replacing the value at
the given position).
These functions accept only valid indices.
(Obviously, you cannot call \verb|lua_remove| or \verb|lua_insert| with
pseudo-indices, as they do not represent a stack position.)

As an example, if the stack starts as \verb|10 20 30 40 50*|
(from bottom to top; the \verb|*| marks the top),
then
\begin{verbatim}
       lua_pushvalue(L, 3)    --> 10 20 30 40 50 30*
       lua_pushvalue(L, -1)   --> 10 20 30 40 50 30 30*
       lua_remove(L, -3)      --> 10 20 30 40 30 30*
       lua_remove(L,  6)      --> 10 20 30 40 30*
       lua_insert(L,  1)      --> 30 10 20 30 40*
       lua_insert(L, -1)      --> 30 10 20 30 40*  (no effect)
       lua_replace(L, 2)      --> 30 40 20 30*
       lua_settop(L, -3)      --> 30 40 20*
       lua_settop(L,  6)      --> 30 40 20 nil nil nil*
\end{verbatim}



\subsection{Querying the Stack}

To check the type of a stack element,
the following functions are available:
\begin{verbatim}
       int         lua_type        (lua_State *L, int index);
       int         lua_isnil       (lua_State *L, int index);
       int         lua_isboolean   (lua_State *L, int index);
       int         lua_isnumber    (lua_State *L, int index);
       int         lua_isstring    (lua_State *L, int index);
       int         lua_istable     (lua_State *L, int index);
       int         lua_isfunction  (lua_State *L, int index);
       int         lua_iscfunction (lua_State *L, int index);
       int         lua_isuserdata  (lua_State *L, int index);
       int         lua_isdataval   (lua_State *L, int index);
\end{verbatim}
\DefAPI{lua_type}
\DefAPI{lua_isnil}\DefAPI{lua_isnumber}\DefAPI{lua_isstring}
\DefAPI{lua_istable}\DefAPI{lua_isboolean}
\DefAPI{lua_isfunction}\DefAPI{lua_iscfunction}
\DefAPI{lua_isuserdata}\DefAPI{lua_isdataval}
These functions can be called with any acceptable index.

\verb|lua_type| returns the type of a value in the stack,
or \verb|LUA_TNONE| for a non-valid index
(that is, if that stack position is ``empty'').
The types are coded by the following constants
defined in \verb|lua.h|:
\verb|LUA_TNIL|,
\verb|LUA_TNUMBER|,
\verb|LUA_TBOOLEAN|,
\verb|LUA_TSTRING|,
\verb|LUA_TTABLE|,
\verb|LUA_TFUNCTION|,
\verb|LUA_TUSERDATA|,
\verb|LUA_TLIGHTUSERDATA|.
The following function translates such constants to a type name:
\begin{verbatim}
       const char *lua_typename  (lua_State *L, int type);
\end{verbatim}
\DefAPI{lua_typename}

The \verb|lua_is*| functions return~1 if the object is compatible
with the given type, and 0 otherwise.
\verb|lua_isboolean| is an exception to this rule,
and it succeeds only for boolean values
(otherwise it would be useless,
as any value is compatible with a boolean).
They always return 0 for a non-valid index.
\verb|lua_isnumber| accepts numbers and numerical strings,
\verb|lua_isstring| accepts strings and numbers \see{coercion},
and \verb|lua_isfunction| accepts both Lua functions and C~functions.
To distinguish between Lua functions and C~functions,
you should use \verb|lua_iscfunction|.
To distinguish between numbers and numerical strings,
you can use \verb|lua_type|.

The API also has functions to compare two values in the stack:
\begin{verbatim}
       int lua_equal    (lua_State *L, int index1, int index2);
       int lua_lessthan (lua_State *L, int index1, int index2);
\end{verbatim}
\DefAPI{lua_equal} \DefAPI{lua_lessthan}
These functions are equivalent to their counterparts in Lua \see{rel-ops}.
Both functions return 0 if any of the indices are non-valid.

\subsection{Getting Values from the Stack}\label{lua-to}

To translate a value in the stack to a specific C~type,
you can use the following conversion functions:
\begin{verbatim}
       int            lua_toboolean   (lua_State *L, int index);
       lua_Number     lua_tonumber    (lua_State *L, int index);
       const char    *lua_tostring    (lua_State *L, int index);
       size_t         lua_strlen      (lua_State *L, int index);
       lua_CFunction  lua_tocfunction (lua_State *L, int index);
       void          *lua_touserdata  (lua_State *L, int index);
\end{verbatim}
\DefAPI{lua_tonumber}\DefAPI{lua_tostring}\DefAPI{lua_strlen}
\DefAPI{lua_tocfunction}\DefAPI{lua_touserdata}\DefAPI{lua_toboolean}
These functions can be called with any acceptable index.
When called with a non-valid index,
they act as if the given value had an incorrect type.

\verb|lua_toboolean| converts the Lua value at the given index
to a C ``boolean'' value (that is, 0 or 1).
Like all tests in Lua, it returns 1 for any Lua value different from
\False{} and \nil;
otherwise it returns 0.
It also returns 0 when called with a non-valid index.
(If you want to accept only real boolean values,
use \verb|lua_isboolean| to test the type of the value.)

\verb|lua_tonumber| converts the Lua value at the given index
to a number (by default, \verb|lua_Number| is \verb|double|).
\DefAPI{lua_Number}
The Lua value must be a number or a string convertible to number
\see{coercion}; otherwise, \verb|lua_tonumber| returns~0.

\verb|lua_tostring| converts the Lua value at the given index to a string
(\verb|const char*|).
The Lua value must be a string or a number;
otherwise, the function returns \verb|NULL|.
If the value is a number,
then \verb|lua_tostring| also
\emph{changes the actual value in the stack to a string}.
(This change confuses \verb|lua_next|
when \verb|lua_tostring| is applied to keys.)
\verb|lua_tostring| returns a fully aligned pointer
to a string inside the Lua environment.
This string always has a zero (\verb|'\0'|)
after its last character (as in~C),
but may contain other zeros in its body.
If you do not know whether a string may contain zeros,
you can use \verb|lua_strlen| to get its actual length.
Because Lua has garbage collection,
there is no guarantee that the pointer returned by \verb|lua_tostring|
will be valid after the corresponding value is removed from the stack.
So, if you need the string after the current function returns,
then you should duplicate it (or put it into the registry \see{registry}).

\verb|lua_tocfunction| converts a value in the stack to a C~function.
This value must be a C~function;
otherwise, \verb|lua_tocfunction| returns \verb|NULL|.
The type \verb|lua_CFunction| is explained in \See{LuacallC}.

\verb|lua_touserdata| is explained in \See{userdata}.


\subsection{Pushing Values onto the Stack}

The API has the following functions to
push C~values onto the stack:
\begin{verbatim}
       void lua_pushboolean   (lua_State *L, int b);
       void lua_pushnumber    (lua_State *L, lua_Number n);
       void lua_pushlstring   (lua_State *L, const char *s, size_t len);
       void lua_pushstring    (lua_State *L, const char *s);
       void lua_pushnil       (lua_State *L);
       void lua_pushcfunction (lua_State *L, lua_CFunction f);
       void lua_pushlightuserdata  (lua_State *L, void *p);
\end{verbatim}

\DefAPI{lua_pushnumber}\DefAPI{lua_pushlstring}\DefAPI{lua_pushstring}
\DefAPI{lua_pushcfunction}\DefAPI{lua_pushlightuserdata}\DefAPI{lua_pushboolean}
\DefAPI{lua_pushnil}\label{pushing}
These functions receive a C~value,
convert it to a corresponding Lua value,
and push the result onto the stack.
In particular, \verb|lua_pushlstring| and \verb|lua_pushstring|
make an internal copy of the given string.
\verb|lua_pushstring| can only be used to push proper C~strings
(that is, strings that end with a zero and do not contain embedded zeros);
otherwise, you should use the more general \verb|lua_pushlstring|,
which accepts an explicit size.

You can also push ``formatted'' strings:
\begin{verbatim}
       const char *lua_pushfstring  (lua_State *L, const char *fmt, ...);
       const char *lua_pushvfstring (lua_State *L, const char *fmt,
                                                   va_list argp);
\end{verbatim}
\DefAPI{lua_pushfstring}\DefAPI{lua_pushvfstring}
Both functions push onto the stack a formatted string,
and return a pointer to that string.
These functions are similar to \verb|sprintf| and \verb|vsprintf|,
but with some important differences:
\begin{itemize}
\item You do not have to allocate the space for the result;
the result is a Lua string, and Lua takes care of memory allocation
(and deallocation, later).
\item The conversion specifiers are quite restricted.
There are no flags, widths, or precisions.
The conversion specifiers can be simply
\verb|%%| (inserts a \verb|%| in the string),
\verb|%s| (inserts a zero-terminated string, with no size restrictions),
\verb|%f| (inserts a \verb|lua_Number|),
\verb|%d| (inserts an \verb|int|),
\verb|%c| (inserts an \verb|int| as a character).
\end{itemize}


\subsection{Controlling Garbage Collection}\label{GC-API}

Lua uses two numbers to control its garbage collection:
the \emph{count} and the \emph{threshold} \see{GC}.
The first counts the ammount of memory in use by Lua;
when the count reaches the threshold,
Lua runs its garbage collector.
After the collection, the count is updated,
and the threshold  is set to twice the count value.

You can access the current values of these two numbers through the
following functions:
\begin{verbatim}
       int  lua_getgccount (lua_State *L);
       int  lua_getgcthreshold (lua_State *L);
\end{verbatim}
\DefAPI{lua_getgcthreshold} \DefAPI{lua_getgccount}
Both return their respective values in Kbytes.
You can change the threshold value with
\begin{verbatim}
       void  lua_setgcthreshold (lua_State *L, int newthreshold);
\end{verbatim}
\DefAPI{lua_setgcthreshold}
Again, the \verb|newthreshold| value is given in Kbytes.
When you call this function,
Lua sets the new threshold and checks it against the byte counter.
If the new threshold is smaller than the byte counter,
then Lua immediately runs the garbage collector.
In particular
\verb|lua_setgcthreshold(L,0)| forces a garbage collectiion.
After the collection,
a new threshold is set according to the previous rule.

%% TODO do we need a new way to do that??
% If you want to change the adaptive behavior of the garbage collector,
% you can use the garbage-collection tag method for \nil{} %
% to set your own threshold
% (the tag method is called after Lua resets the threshold).


\subsection{Userdata}\label{userdata}

Userdata represents C values in Lua.
Lua supports two types of userdata:
\Def{full userdata} and \Def{light userdata}.

A full userdata represents a block of memory.
It is an object (like a table):
You must create it, it can have its own metatable,
you can detect when it is being collected.
A full userdata is only equal to itself.

A light userdata represents a pointer.
It is a value (like a number):
You do not create it, it has no metatables,
it is not collected (as it was never created).
A light userdata is equal to ``any''
light userdata with the same address.

In Lua code, there is no way to test whether a userdata is full or light;
both have type \verb|userdata|.
In C code, \verb|lua_type| returns \verb|LUA_TUSERDATA| for full userdata,
and \verb|LUA_LIGHTUSERDATA| for light userdata.

You can create new full userdata with the following function:
\begin{verbatim}
       void *lua_newuserdata (lua_State *L, size_t size);
\end{verbatim}
\DefAPI{lua_newuserdata}
It allocates a new block of memory with the given size,
pushes on the stack a new userdata with the block address,
and returns this address.

To push a light userdata into the stack you use
\verb|lua_pushlightuserdata| \see{pushing}.

\verb|lua_touserdata| \see{lua-to} retrieves the value of a userdata.
When applied on a full userdata, it returns the address of its block;
when applied on a light userdata, it returns its pointer;
when applied on a non-userdata value, it returns \verb|NULL|.

When Lua collects a full userdata,
it calls its \verb|gc| metamethod, if any,
and then it automatically frees its corresponding memory.


\subsection{Metatables}

%% TODO

\subsection{Loading Lua Chunks}
You can load a Lua chunk with
\begin{verbatim}
       typedef const char * (*lua_Chunkreader)
                                (lua_State *L, void *data, size_t *size);

       int lua_load (lua_State *L, lua_Chunkreader reader, void *data,
                                   const char *chunkname);
\end{verbatim}
\DefAPI{Chunkreader}\DefAPI{lua_load}
\verb|lua_load| uses the \emph{reader} to read the chunk.
Everytime it needs another piece of the chunk,
it calls the reader,
passing along its \verb|data| parameter.
The reader must return a pointer to a block of memory
with the part of the chunk,
and set \verb|size| to the block size.
To signal the end of the chunk, the reader must return \verb|NULL|.

In the current implementation,
the reader function cannot call any Lua function;
to ensure that, it always receives \verb|NULL| as the Lua state.

\verb|lua_load| automatically detects whether the chunk is text or binary,
and loads it accordingly (see program \IndexVerb{luac}).

The return values of \verb|lua_load| are:
\begin{itemize}
\item 0 --- no errors;
\item \IndexAPI{LUA_ERRSYNTAX} ---
syntax error during pre-compilation.
\item \IndexAPI{LUA_ERRMEM} ---
memory allocation error.
\end{itemize}
If there are no errors,
the compiled chunk is pushed as a Lua function on top of the stack.
Otherwise, an error message is pushed.

The \emph{chunkname} is used for error messages
and debug information \see{debugI}.

See the auxiliar library (\verb|lauxlib|)
for examples of how to use \verb|lua_load|,
and for some ready-to-use functions to load chunks
from files and from strings.


\subsection{Executing Lua Chunks}\label{luado}
>>>>
A host program can execute Lua chunks written in a file or in a string
by using the following functions:
\begin{verbatim}
       int lua_dofile   (lua_State *L, const char *filename);
       int lua_dostring (lua_State *L, const char *string);
       int lua_dobuffer (lua_State *L, const char *buff,
                         size_t size, const char *name);
\end{verbatim}
\DefAPI{lua_dofile}\DefAPI{lua_dostring}\DefAPI{lua_dobuffer}%
These functions return
0 in case of success, or one of the following error codes
(defined in \verb|lua.h|)
if they fail:
\begin{itemize}
\item \IndexAPI{LUA_ERRRUN} ---
error while running the chunk.
\item \IndexAPI{LUA_ERRSYNTAX} ---
syntax error during pre-compilation.
\item \IndexAPI{LUA_ERRMEM} ---
memory allocation error.
For such errors, Lua does not call \verb|_ERRORMESSAGE| \see{error}.
\item \IndexAPI{LUA_ERRERR} ---
error while running \verb|_ERRORMESSAGE|.
For such errors, Lua does not call \verb|_ERRORMESSAGE| again, to avoid loops.
\item \IndexAPI{LUA_ERRFILE} ---
error opening the file (only for \verb|lua_dofile|).
In this case,
you may want to
check \verb|errno|,
call \verb|strerror|,
or call \verb|perror| to tell the user what went wrong.
\end{itemize}


\subsection{Manipulating Tables}

Tables are created by calling
the function
\begin{verbatim}
       void lua_newtable (lua_State *L);
\end{verbatim}
\DefAPI{lua_newtable}
This function creates a new, empty table and pushes it onto the stack.

To read a value from a table that resides somewhere in the stack,
call
\begin{verbatim}
       void lua_gettable (lua_State *L, int index);
\end{verbatim}
\DefAPI{lua_gettable}
where \verb|index| points to the table.
\verb|lua_gettable| pops a key from the stack
and returns (on the stack) the contents of the table at that key.
The table is left where it was in the stack;
this is convenient for getting multiple values from a table.

As in Lua, this function may trigger a metamethod
for the ``gettable'' or ``index'' events \see{metatable}.
To get the real value of any table key,
without invoking any metamethod,
use the \emph{raw} version:
\begin{verbatim}
       void lua_rawget (lua_State *L, int index);
\end{verbatim}
\DefAPI{lua_rawget}

To store a value into a table that resides somewhere in the stack,
you push the key and the value onto the stack
(in this order),
and then call
\begin{verbatim}
       void lua_settable (lua_State *L, int index);
\end{verbatim}
\DefAPI{lua_settable}
where \verb|index| points to the table.
\verb|lua_settable| pops from the stack both the key and the value.
The table is left where it was in the stack;
this is convenient for setting multiple values in a table.

As in Lua, this operation may trigger a metamethod
for the ``settable'' or ``newindex'' events.
To set the real value of any table index,
without invoking any metamethod,
use the \emph{raw} version:
\begin{verbatim}
       void lua_rawset (lua_State *L, int index);
\end{verbatim}
\DefAPI{lua_rawset}

You can traverse a table with the function
\begin{verbatim}
       int lua_next (lua_State *L, int index);
\end{verbatim}
\DefAPI{lua_next}
where \verb|index| points to the table to be traversed.
The function pops a key from the stack,
and pushes a key-value pair from the table
(the ``next'' pair after the given key).
If there are no more elements, then \verb|lua_next| returns 0
(and pushes nothing).
Use a \nil{} key to signal the start of a traversal.

A typical traversal looks like this:
\begin{verbatim}
       /* table is in the stack at index `t' */
       lua_pushnil(L);  /* first key */
       while (lua_next(L, t) != 0) {
         /* `key' is at index -2 and `value' at index -1 */
         printf("%s - %s\n",
           lua_typename(L, lua_type(L, -2)), lua_typename(L, lua_type(L, -1)));
         lua_pop(L, 1);  /* removes `value'; keeps `key' for next iteration */
       }
\end{verbatim}

NOTE:
While traversing a table,
do not call \verb|lua_tostring| on a key,
unless you know the key is actually a string.
Recall that \verb|lua_tostring| \emph{changes} the value at the given index;
this confuses the next call to \verb|lua_next|.

\subsection{Manipulating Global Variables} \label{globals}

All global variables are kept in an ordinary Lua table.
This table is always at pseudo-index \IndexAPI{LUA_GLOBALSINDEX}.

To access and change the value of global variables,
you can use regular table operations over the global table.
For instance, to access the value of a global variable, do
\begin{verbatim}
       lua_pushstring(L, varname);
       lua_gettable(L, LUA_GLOBALSINDEX);
\end{verbatim}

You can change the global table of a Lua thread using \verb|lua_replace|.


\subsection{Using Tables as Arrays}
The API has functions that help to use Lua tables as arrays,
that is,
tables indexed by numbers only:
\begin{verbatim}
       void lua_rawgeti (lua_State *L, int index, int n);
       void lua_rawseti (lua_State *L, int index, int n);
\end{verbatim}
\DefAPI{lua_rawgeti}
\DefAPI{lua_rawseti}

\verb|lua_rawgeti| pushes the value of the \M{n}-th element of the table
at stack position \verb|index|.
\verb|lua_rawseti| sets the value of the \M{n}-th element of the table
at stack position \verb|index| to the value at the top of the stack,
removing this value from the stack.


\subsection{Calling Functions}

Functions defined in Lua
and C~functions registered in Lua
can be called from the host program.
This is done using the following protocol:
First, the function to be called is pushed onto the stack;
then, the arguments to the function are pushed
in \emph{direct order}, that is, the first argument is pushed first.
Finally, the function is called using
\begin{verbatim}
       void lua_call (lua_State *L, int nargs, int nresults);
\end{verbatim}
\DefAPI{lua_call}
\verb|nargs| is the number of arguments that you pushed onto the stack.
All arguments and the function value are popped from the stack,
and the function results are pushed.
The number of results are adjusted to \verb|nresults|,
unless \verb|nresults| is \IndexAPI{LUA_MULTRET}.
In that case, \emph{all} results from the function are pushed.
Lua takes care that the returned values fit into the stack space.
The function results are pushed onto the stack in direct order
(the first result is pushed first),
so that after the call the last result is on the top.

The following example shows how the host program may do the
equivalent to the Lua code:
\begin{verbatim}
       a = f("how", t.x, 14)
\end{verbatim}
Here it is in~C:
\begin{verbatim}
    lua_pushstring(L, "t");
    lua_gettable(L, LUA_GLOBALSINDEX);          /* global `t' (for later use) */
    lua_pushstring(L, "a");                                       /* var name */
    lua_pushstring(L, "f");                                  /* function name */
    lua_gettable(L, LUA_GLOBALSINDEX);               /* function to be called */
    lua_pushstring(L, "how");                                 /* 1st argument */
    lua_pushstring(L, "x");                            /* push the string "x" */
    lua_gettable(L, -5);                      /* push result of t.x (2nd arg) */
    lua_pushnumber(L, 14);                                    /* 3rd argument */
    lua_call(L, 3, 1);         /* call function with 3 arguments and 1 result */
    lua_settable(L, LUA_GLOBALSINDEX);             /* set global variable `a' */
    lua_pop(L, 1);                               /* remove `t' from the stack */
\end{verbatim}
Notice that the code above is ``balanced'':
at its end, the stack is back to its original configuration.
This is considered good programming practice.

(We did this example using only the raw functions provided by Lua's API,
to show all the details.
Usually programmers use several macros and auxiliar functions that
provide higher level access to Lua.)

%% TODO: pcall

\medskip

>>>>
%% TODO: mover essas 2 para algum lugar melhor.
Some special Lua functions have their own C~interfaces.
The host program can generate a Lua error calling the function
\begin{verbatim}
       void lua_error (lua_State *L, const char *message);
\end{verbatim}
\DefAPI{lua_error}
This function never returns.
If \verb|lua_error| is called from a C~function that has been called from Lua,
then the corresponding Lua execution terminates,
as if an error had occurred inside Lua code.
Otherwise, the whole host program terminates with a call to
\verb|exit(EXIT_FAILURE)|.
Before terminating execution,
the \verb|message| is passed to the error handler function,
\verb|_ERRORMESSAGE| \see{error}.
If \verb|message| is \verb|NULL|,
then \verb|_ERRORMESSAGE| is not called.

The function
\begin{verbatim}
       void lua_concat (lua_State *L, int n);
\end{verbatim}
\DefAPI{lua_concat}
concatenates the \verb|n| values at the top of the stack,
pops them, and leaves the result at the top.
If \verb|n| is 1, the result is that single string
(that is, the function does nothing);
if \verb|n| is 0, the result is the empty string.
Concatenation is done following the usual semantics of Lua
\see{concat}.


\subsection{Defining C Functions} \label{LuacallC}
Lua can be extended with functions written in~C.
These functions must be of type \verb|lua_CFunction|,
which is defined as
\begin{verbatim}
       typedef int (*lua_CFunction) (lua_State *L);
\end{verbatim}
\DefAPI{lua_CFunction}
A C~function receives a Lua environment and returns an integer,
the number of values it has returned to Lua.

In order to communicate properly with Lua,
a C~function must follow the following protocol,
which defines the way parameters and results are passed:
A C~function receives its arguments from Lua in the stack,
in direct order (the first argument is pushed first).
To return values to Lua, a C~function just pushes them onto the stack,
in direct order (the first result is pushed first),
and returns the number of results.
Like a Lua function, a C~function called by Lua can also return
many results.

As an example, the following function receives a variable number
of numerical arguments and returns their average and sum:
\begin{verbatim}
       static int foo (lua_State *L) {
         int n = lua_gettop(L);    /* number of arguments */
         lua_Number sum = 0;
         int i;
         for (i = 1; i <= n; i++) {
           if (!lua_isnumber(L, i))
             lua_error(L, "incorrect argument to function `average'");
           sum += lua_tonumber(L, i);
         }
         lua_pushnumber(L, sum/n);        /* first result */
         lua_pushnumber(L, sum);         /* second result */
         return 2;                   /* number of results */
       }
\end{verbatim}

To register a C~function to Lua,
there is the following convenience macro:
\begin{verbatim}
       #define lua_register(L,n,f) \
               (lua_pushstring(L, n), \
                lua_pushcfunction(L, f), \
                lua_settable(L, LUA_GLOBALSINDEX))
     /* const char *n;   */
     /* lua_CFunction f; */
\end{verbatim}
\DefAPI{lua_register}
which receives the name the function will have in Lua,
and a pointer to the function.
Thus,
the C~function `\verb|foo|' above may be registered in Lua as `\verb|average|'
by calling
\begin{verbatim}
       lua_register(L, "average", foo);
\end{verbatim}

\subsection{Defining C Closures} \label{c-closure}

When a C~function is created,
it is possible to associate some values to it,
thus creating a \IndexEmph{C~closure};
these values are then accessible to the function whenever it is called.
To associate values to a C~function,
first these values should be pushed onto the stack
(when there are multiple values, the first value is pushed first).
Then the function
\begin{verbatim}
       void lua_pushcclosure (lua_State *L, lua_CFunction fn, int n);
\end{verbatim}
\DefAPI{lua_pushcclosure}
is used to push the C~function onto the stack,
with the argument \verb|n| telling how many values should be
associated with the function
(\verb|lua_pushcclosure| also pops these values from the stack);
in fact, the macro \verb|lua_pushcfunction| is defined as
\verb|lua_pushcclosure| with \verb|n| set to 0.

Then, whenever the C~function is called,
those values are located at specific pseudo-indices.
Those pseudo-indices are produced by a macro \IndexAPI{lua_upvalueindex}.
The first value associated with a function is at position
\verb|lua_upvalueindex(1)|, and so on.

For examples of C~functions and closures, see files
\verb|lbaselib.c|, \verb|liolib.c|, \verb|lmathlib.c|, and \verb|lstrlib.c|
in the official Lua distribution.


\subsubsection*{Registry} \label{registry}

Lua provides a pre-defined table that can be used by any C~code to
store whatever Lua value it needs to store,
especially if the C~code needs to keep that Lua value
outside the life span of a C~function.
This table is always located at pseudo-index
\IndexAPI{LUA_REGISTRYINDEX}.
Any C~library can store data into this table,
as long as it chooses a key different from other libraries.
Typically, you can use as key a string containing the library name,
or a light userdata with the address of a C object in your code.

The integer keys in the registry are used by the reference mechanism,
implemented by the auxiliar library,
and therefore should not be used by other purposes.


%------------------------------------------------------------------------------
\section{The Debug Interface} \label{debugI}

Lua has no built-in debugging facilities.
Instead, it offers a special interface,
by means of functions and \emph{hooks},
which allows the construction of different
kinds of debuggers, profilers, and other tools
that need ``inside information'' from the interpreter.
This interface is declared in \verb|luadebug.h|.

\subsection{Stack and Function Information}

The main function to get information about the interpreter stack is
\begin{verbatim}
       int lua_getstack (lua_State *L, int level, lua_Debug *ar);
\end{verbatim}
\DefAPI{lua_getstack}
This function fills parts of a \verb|lua_Debug| structure with
an identification of the \emph{activation record}
of the function executing at a given level.
Level~0 is the current running function,
whereas level \Math{n+1} is the function that has called level \Math{n}.
Usually, \verb|lua_getstack| returns 1;
when called with a level greater than the stack depth,
it returns 0.

The structure \verb|lua_Debug| is used to carry different pieces of
information about an active function:
\begin{verbatim}
      typedef struct lua_Debug {
        const char *event;     /* "call", "return" */
        int currentline;       /* (l) */
        const char *name;      /* (n) */
        const char *namewhat;  /* (n) `global', `local', `field', `method' */
        int nups;              /* (u) number of upvalues */
        int linedefined;       /* (S) */
        const char *what;      /* (S) "Lua" function, "C" function, Lua "main" */
        const char *source;    /* (S) */
        char short_src[LUA_IDSIZE]; /* (S) */

        /* private part */
        ...
      } lua_Debug;
\end{verbatim}
\DefAPI{lua_Debug}
\verb|lua_getstack| fills only the private part
of this structure, for future use.
To fill the other fields of \verb|lua_Debug| with useful information,
call
\begin{verbatim}
       int lua_getinfo (lua_State *L, const char *what, lua_Debug *ar);
\end{verbatim}
\DefAPI{lua_getinfo}
This function returns 0 on error
(for instance, an invalid option in \verb|what|).
Each character in the string \verb|what|
selects some fields of \verb|ar| to be filled,
as indicated by the letter in parentheses in the definition of \verb|lua_Debug|
above:
`\verb|S|' fills in the fields \verb|source|, \verb|linedefined|,
and \verb|what|;
`\verb|l|' fills in the field \verb|currentline|, etc.
Moreover, `\verb|f|' pushes onto the stack the function that is
running at the given level.

To get information about a function that is not active (that is,
it is not in the stack),
you push the function onto the stack,
and start the \verb|what| string with the character `\verb|>|'.
For instance, to know in which line a function \verb|f| was defined,
you can write
\begin{verbatim}
       lua_Debug ar;
       lua_pushstring(L, "f");
       lua_gettable(L, LUA_GLOBALSINDEX);  /* get global `f' */
       lua_getinfo(L, ">S", &ar);
       printf("%d\n", ar.linedefined);
\end{verbatim}
The fields of \verb|lua_Debug| have the following meaning:
\begin{description}\leftskip=20pt

\item[source]
If the function was defined in a string,
then \verb|source| is that string;
if the function was defined in a file,
then \verb|source| starts with a \verb|@| followed by the file name.

\item[short\_src]
A ``printable'' version of \verb|source|, to be used in error messages.

\item[linedefined]
the line number where the definition of the function starts.

\item[what] the string \verb|"Lua"| if this is a Lua function,
\verb|"C"| if this is a C~function,
or \verb|"main"| if this is the main part of a chunk.

\item[currentline]
the current line where the given function is executing.
When no line information is available,
\verb|currentline| is set to \Math{-1}.

\item[name]
a reasonable name for the given function.
Because functions in Lua are first class values,
they do not have a fixed name:
Some functions may be the value of many global variables,
while others may be stored only in a table field.
The \verb|lua_getinfo| function checks whether the given
function is a tag method or the value of a global variable.
If the given function is a tag method,
then \verb|name| points to the event name.
%% TODO: mas qual o tag? Agora que temos tipos com nome, seria util saber
%% o tipo de TM. Em particular para mensagens de erro.
If the given function is the value of a global variable,
then \verb|name| points to the variable name.
If the given function is neither a tag method nor a global variable,
then \verb|name| is set to \verb|NULL|.

\item[namewhat]
Explains the previous field.
It can be \verb|"global"|, \verb|"local"|, \verb|"method"|,
\verb|"field"|, or \verb|""| (the empty string),
according to how the function was called.
(Lua uses the empty string when no other option seems to apply.)

\item[nups]
Number of upvalues of the function.

\end{description}


\subsection{Manipulating Local Variables}

For the manipulation of local variables,
\verb|luadebug.h| uses indices:
The first parameter or local variable has index~1, and so on,
until the last active local variable.

The following functions allow the manipulation of the
local variables of a given activation record:
\begin{verbatim}
       const char *lua_getlocal (lua_State *L, const lua_Debug *ar, int n);
       const char *lua_setlocal (lua_State *L, const lua_Debug *ar, int n);
\end{verbatim}
\DefAPI{lua_getlocal}\DefAPI{lua_setlocal}
The parameter \verb|ar| must be a valid activation record,
filled by a previous call to \verb|lua_getstack| or
given as argument to a hook \see{sub-hooks}.
\verb|lua_getlocal| gets the index \verb|n| of a local variable,
pushes its value onto the stack,
and returns its name.
%% TODO: why return name?
\verb|lua_setlocal| assigns the value at the top of the stack
to the variable and returns its name.
Both functions return \verb|NULL| on failure,
that is
when the index is greater than
the number of active local variables.

As an example, the following function lists the names of all
local variables for a function at a given level of the stack:
\begin{verbatim}
       int listvars (lua_State *L, int level) {
         lua_Debug ar;
         int i = 1;
         const char *name;
         if (lua_getstack(L, level, &ar) == 0)
           return 0;  /* failure: no such level in the stack */
         while ((name = lua_getlocal(L, &ar, i++)) != NULL) {
           printf("%s\n", name);
           lua_pop(L, 1);  /* remove variable value */
         }
         return 1;
       }
\end{verbatim}


\subsection{Hooks}\label{sub-hooks}

The Lua interpreter offers two hooks for debugging purposes:
a \emph{call} hook and a \emph{line} hook.
Both have type \verb|lua_Hook|, defined as follows:
\begin{verbatim}
       typedef void (*lua_Hook) (lua_State *L, lua_Debug *ar);
\end{verbatim}
\DefAPI{lua_Hook}
You can set the hooks with the following functions:
\begin{verbatim}
       lua_Hook lua_setcallhook (lua_State *L, lua_Hook func);
       lua_Hook lua_setlinehook (lua_State *L, lua_Hook func);
\end{verbatim}
\DefAPI{lua_setcallhook}\DefAPI{lua_setlinehook}
A hook is disabled when its value is \verb|NULL|,
which is the initial value of both hooks.
The functions \verb|lua_setcallhook| and \verb|lua_setlinehook|
set their corresponding hooks and return their previous values.

The call hook is called whenever the
interpreter enters or leaves a function.
The \verb|event| field of \verb|ar| has the string \verb|"call"|
or \verb|"return"|.
This \verb|ar| can then be used in calls to \verb|lua_getinfo|,
\verb|lua_getlocal|, and \verb|lua_setlocal|
to get more information about the function and to manipulate its
local variables.

The line hook is called every time the interpreter changes
the line of code it is executing.
The \verb|event| field of \verb|ar| has the string \verb|"line"|,
and the \verb|currentline| field has the new line number.
Again, you can use this \verb|ar| in other calls to the debug API.

While Lua is running a hook, it disables other calls to hooks.
Therefore, if a hook calls Lua to execute a function or a chunk,
this execution ocurrs without any calls to hooks.


%------------------------------------------------------------------------------
\section{Standard Libraries}\label{libraries}

The standard libraries provide useful functions
that are implemented directly through the standard C~API.
Some of these functions provide essential services to the language
(e.g. \verb|type| and \verb|getmetatable|);
others provide access to ``outside'' servides (e.g. I/O);
and others could be implemented in Lua itself,
but are quite useful or have critical performance to
deserve an implementation in C (e.g. \verb|sort|).

All libraries are implemented through the official C API,
and are provided as separate C~modules.
Currently, Lua has the following standard libraries:
\begin{itemize}
\item basic library;
\item string manipulation;
\item table manipulation;
\item mathematical functions (sin, log, etc.);
\item input and output;
\item operating system facilities;
\item debug facilities.
\end{itemize}
Except for the basic library,
each library provides all its functions as fields of a global table
or as methods of its objects.

To have access to these libraries,
the C~host program must call the functions
\verb|lua_baselibopen|,
\verb|lua_strlibopen|,
\verb|lua_tablibopen|,
\verb|lua_mathlibopen|,
and \verb|lua_iolibopen|, which are declared in \verb|lualib.h|.
\DefAPI{lua_baselibopen}
\DefAPI{lua_strlibopen}
\DefAPI{lua_tablibopen}
\DefAPI{lua_mathlibopen}
\DefAPI{lua_iolibopen}


\subsection{Basic Functions} \label{predefined}

The basic library provides some core functions to Lua.
If you do not include this library in your application,
you should check carefully whether you need to provide some alternative
implementation for some facilities.

The basic library also defines a global variable \IndexAPI{_VERSION}
with a string containing the current interpreter version.
The current content of this string is {\tt "Lua \Version"}.

\subsubsection*{\ff \T{assert (v [, message])}}\DefLIB{assert}
Issues an \emph{``assertion failed!''} error
when its argument \verb|v| is \nil;
otherwise, returns this argument.
This function is equivalent to the following Lua function:
\begin{verbatim}
       function assert (v, m)
         if not v then
           error(m or "assertion failed!")
         end
         return v
       end
\end{verbatim}

??\subsubsection*{\ff \T{call (func, arg [, mode [, errhandler]])}}\DefLIB{call}
\label{pdf-call}
Calls function \verb|func| with
the arguments given by the table \verb|arg|.
The call is equivalent to
\begin{verbatim}
       func(arg[1], arg[2], ..., arg[n])
\end{verbatim}
where \verb|n| is the result of \verb|getn(arg)| \see{getn}.
All results from \verb|func| are simply returned by \verb|call|.

By default,
if an error occurs during the call to \verb|func|,
the error is propagated.
If the string \verb|mode| contains \verb|"x"|,
then the call is \emph{protected}.\index{protected calls}
In this mode, function \verb|call| does not propagate an error,
regardless of what happens during the call.
Instead, it returns \nil{} to signal the error
(besides calling the appropriated error handler).

If \verb|errhandler| is provided,
the error function \verb|_ERRORMESSAGE| is temporarily set to \verb|errhandler|,
while \verb|func| runs.
In particular, if \verb|errhandler| is \nil,
no error messages will be issued during the execution of the called function.

\subsubsection*{\ff \T{collectgarbage ([limit])}}\DefLIB{collectgarbage}

Sets the garbage-collection threshold for the given limit
(in Kbytes), and checks it against the byte counter.
If the new threshold is smaller than the byte counter,
then Lua immediately runs the garbage collector \see{GC}.
If \verb|limit| is absent, it defaults to zero
(thus forcing a garbage-collection cycle).

\subsubsection*{\ff \T{dofile (filename)}}\DefLIB{dofile}
Receives a file name,
opens the named file, and executes its contents as a Lua chunk.
When called without arguments,
\verb|dofile| executes the contents of the standard input (\verb|stdin|).
Returns any value returned by the chunk.

\subsubsection*{\ff \T{error ([message])}}\DefLIB{error}\label{pdf-error}
Terminates the last protected function called,
and returns \verb|message| as the error message.
Function \verb|error| never returns.

\subsubsection*{\ff \T{getglobals (function)}}\DefLIB{getglobals}
Returns the current table of globals in use by the function.
\verb|function| can be a Lua function or a number,
meaning the function at that stack level:
Level 1 is the function calling \verb|getglobals|.
If the given function is not a Lua function,
returns the ``global'' table of globals.
The default for \verb|function| is 1.

\subsubsection*{\ff \T{getmetatable (object)}}
\DefLIB{getmetatable}\label{pdf-getmetatable}

Returns the metatable of the given object.
If the object does not have a metatable, returns \nil.

\subsubsection*{\ff \T{getmode (table)}}\DefLIB{getmode}

Returns the weak mode of a table, as a string.
Valid values for this string are \verb|""| for regular (non-weak) tables,
\verb|"k"| for weak keys, \verb|"v"| for weak values,
and \verb|"kv"| for both.

\subsubsection*{\ff \T{gcinfo ()}}\DefLIB{gcinfo}
Returns the number of Kbytes of dynamic memory Lua is using,
and (as a second result) the
current garbage collector threshold (also in Kbytes).

\subsubsection*{\ff \T{loadfile (filename)}}\DefLIB{loadfile}
Loads a file as a Lua chunk.
If there is no errors, 
returns the compiled chunk as a function;
otherwise, returns \nil{} plus an error message.

\subsubsection*{\ff \T{loadstring (string [, chunkname])}}\DefLIB{loadstring}
Loads a string as a Lua chunk.
If there is no errors, 
returns the compiled chunk as a function;
otherwise, returns \nil{} plus an error message.

The optional parameter \verb|chunkname|
is the ``name of the chunk'',
used in error messages and debug information.

To load and run a given string, use the idiom
\begin{verbatim}
      assert(loadstring(s))()
\end{verbatim}

\subsubsection*{\ff \T{next (table, [index])}}\DefLIB{next}
Allows a program to traverse all fields of a table.
Its first argument is a table and its second argument
is an index in this table.
\verb|next| returns the next index of the table and the
value associated with the index.
When called with \nil{} as its second argument,
\verb|next| returns the first index
of the table and its associated value.
When called with the last index,
or with \nil{} in an empty table,
\verb|next| returns \nil.
If the second argument is absent, then it is interpreted as \nil.

Lua has no declaration of fields;
semantically, there is no difference between a
field not present in a table or a field with value \nil.
Therefore, \verb|next| only considers fields with non-\nil{} values.
The order in which the indices are enumerated is not specified,
\emph{even for numeric indices}
(to traverse a table in numeric order,
use a numerical \rwd{for} or the function \verb|ipairs|).

The behavior of \verb|next| is \emph{undefined} if you change
the table during the traversal.

\subsubsection*{\ff \T{print (e1, e2, ...)}}\DefLIB{print}
Receives any number of arguments,
and prints their values in \verb|stdout|,
using the strings returned by \verb|tostring|.
This function is not intended for formatted output,
but only as a quick way to show a value,
typically for debugging.
For formatted output, see \verb|format| \see{format}.

\subsubsection*{\ff \T{rawget (table, index)}}\DefLIB{rawget}
Gets the real value of \verb|table[index]|,
without invoking any tag method.
\verb|table| must be a table;
\verb|index| is any value different from \nil.

\subsubsection*{\ff \T{rawset (table, index, value)}}\DefLIB{rawset}
Sets the real value of \verb|table[index]| to \verb|value|,
without invoking any tag method.
\verb|table| must be a table;
\verb|index| is any value different from \nil;
and \verb|value| is any Lua value.

\subsubsection*{\ff \T{require (packagename)}}\DefLIB{require}

Loads the given package.
The function starts by looking into the table \IndexVerb{_LOADED}
whether \verb|packagename| is already loaded.
If it is, then \verb|require| is done.
Otherwise, it searches a path looking for a file to load.

If the global variable \IndexVerb{LUA_PATH} is a string, 
this string is the path.
Otherwise, \verb|require| tries the environment variable \verb|LUA_PATH|.
In the last resort, it uses a predefined path.

The path is a sequence of \emph{templates} separated by semicolons.
For each template, \verb|require| will change an eventual interrogation
mark in the template to \verb|packagename|,
and then will try to load the resulting file name.
So, for instance, if the path is
\begin{verbatim}
  "./?.lua;./?.lc;/usr/local/?/init.lua;/lasttry"
\end{verbatim}
a \verb|require "mod"| will try to load the files
\verb|./mod.lua|,
\verb|./mod.lc|,
\verb|/usr/local/mod/init.lua|,
and \verb|/lasttry|, in that order.

The function stops the search as soon as it can load a file,
and then it runs the file.
If there is any error loading or running the file,
or if it cannot find any file in the path,
then \verb|require| signals an error. 
Otherwise, it marks in table \verb|_LOADED|
that the package is loaded, and returns.

While running a packaged file,
\verb|require| defines the global variable \IndexVerb{_REQUIREDNAME}
with the package name.

\subsubsection*{\ff \T{setglobals (function, table)}}\DefLIB{setglobals}
Sets the current table of globals to be used by the given function.
\verb|function| can be a Lua function or a number,
meaning the function at that stack level:
Level 1 is the function calling \verb|setglobals|.

\subsubsection*{\ff \T{setmetatable (table, metatable)}}\DefLIB{setmetatable}

Sets the metatable for the given table.
(You cannot change the metatable of a userdata from Lua.)
If \verb|metatable| is \nil, removes the metatable of the given table.

\subsubsection*{\ff \T{setmode (table, mode)}}\DefLIB{setmode}

Set the weak mode of a table.
The new mode is described by the \verb|mode| string.
Valid values for this string are \verb|""| for regular (non-weak) tables,
\verb|"k"| for weak keys, \verb|"v"| for weak values,
and \verb|"kv"| for both.

This function returns its first argument (\verb|table|).

\subsubsection*{\ff \T{tonumber (e [, base])}}\DefLIB{tonumber}
Tries to convert its argument to a number.
If the argument is already a number or a string convertible
to a number, then \verb|tonumber| returns that number;
otherwise, it returns \nil.

An optional argument specifies the base to interpret the numeral.
The base may be any integer between 2 and 36, inclusive.
In bases above~10, the letter `A' (in either upper or lower case)
represents~10, `B' represents~11, and so forth, with `Z' representing 35.
In base 10 (the default), the number may have a decimal part,
as well as an optional exponent part \see{coercion}.
In other bases, only unsigned integers are accepted.

\subsubsection*{\ff \T{tostring (e)}}\DefLIB{tostring}
Receives an argument of any type and
converts it to a string in a reasonable format.
For complete control of how numbers are converted,
use \verb|format| \see{format}.

\subsubsection*{\ff \T{type (v)}}\DefLIB{type}\label{pdf-type}
Returns the type of its only argument, coded as a string.
The possible results of this function are
\verb|"nil"| (a string, not the value \nil),
\verb|"number"|,
\verb|"string"|,
\verb|"table"|,
\verb|"function"|,
and \verb|"userdata"|.

\subsubsection*{\ff \T{unpack (list)}}\DefLIB{unpack}
Returns all elements from the given list.
This function is equivalent to
\begin{verbatim}
  return list[1], list[2], ..., list[n]
\end{verbatim}
except that the above code can be valid only for a fixed \M{n}.
The number \M{n} of returned values
is either the value of \verb|list.n|, if it is a number,
or one less the index of the first absent (\nil) value.

\subsection{String Manipulation}
This library provides generic functions for string manipulation,
such as finding and extracting substrings and pattern matching.
When indexing a string in Lua, the first character is at position~1
(not at~0, as in C).
Indices are allowed to be negative and are interpreted as indexing backwards,
from the end of the string.
Thus, the last character is at position \Math{-1}, and so on.

The string library provides all its functions inside the table
\DefLIB{string}.

\subsubsection*{\ff \T{string.byte (s [, i])}}\DefLIB{string.byte}
Returns the internal numerical code of the \M{i}-th character of \verb|s|.
If \verb|i| is absent, then it is assumed to be~1.
\verb|i| may be negative.

\NOTE
Numerical codes are not necessarily portable across platforms.

\subsubsection*{\ff \T{string.char (i1, i2, \ldots)}}\DefLIB{string.char}
Receives 0 or more integers.
Returns a string with length equal to the number of arguments,
in which each character has the internal numerical code equal
to its correspondent argument.

\NOTE
Numerical codes are not necessarily portable across platforms.

\subsubsection*{\ff \T{string.find (s, pattern [, init [, plain]])}}
\DefLIB{string.find}
Looks for the first \emph{match} of
\verb|pattern| in the string \verb|s|.
If it finds one, then \verb|find| returns the indices of \verb|s|
where this occurrence starts and ends;
otherwise, it returns \nil.
If the pattern specifies captures (see \verb|string.gsub| below),
the captured strings are returned as extra results.
A third, optional numerical argument \verb|init| specifies
where to start the search;
its default value is~1, and may be negative.
A value of \True{} as a fourth, optional argument \verb|plain|
turns off the pattern matching facilities,
so the function does a plain ``find substring'' operation,
with no characters in \verb|pattern| being considered ``magic''.
Note that if \verb|plain| is given, then \verb|init| must be given too.

\subsubsection*{\ff \T{string.len (s)}}\DefLIB{string.len}
Receives a string and returns its length.
The empty string \verb|""| has length 0.
Embedded zeros are counted,
and so \verb|"a\000b\000c"| has length 5.

\subsubsection*{\ff \T{string.lower (s)}}\DefLIB{string.lower}
Receives a string and returns a copy of that string with all
uppercase letters changed to lowercase.
All other characters are left unchanged.
The definition of what is an uppercase letter depends on the current locale.

\subsubsection*{\ff \T{string.rep (s, n)}}\DefLIB{string.rep}
Returns a string that is the concatenation of \verb|n| copies of
the string \verb|s|.

\subsubsection*{\ff \T{string.sub (s, i [, j])}}\DefLIB{string.sub}
Returns another string, which is a substring of \verb|s|,
starting at \verb|i|  and running until \verb|j|;
\verb|i| and \verb|j| may be negative.
If \verb|j| is absent, then it is assumed to be equal to \Math{-1}
(which is the same as the string length).
In particular,
the call \verb|string.sub(s,1,j)| returns a prefix of \verb|s|
with length \verb|j|,
and the call \verb|string.sub(s, -i)| returns a suffix of \verb|s|
with length \verb|i|.

\subsubsection*{\ff \T{string.upper (s)}}\DefLIB{string.upper}
Receives a string and returns a copy of that string with all
lowercase letters changed to uppercase.
All other characters are left unchanged.
The definition of what is a lowercase letter depends on the current locale.

\subsubsection*{\ff \T{string.format (formatstring, e1, e2, \ldots)}}
\DefLIB{string.format}\label{format}
Returns a formatted version of its variable number of arguments
following the description given in its first argument (which must be a string).
The format string follows the same rules as the \verb|printf| family of
standard C~functions.
The only differences are that the options/modifiers
\verb|*|, \verb|l|, \verb|L|, \verb|n|, \verb|p|,
and \verb|h| are not supported,
and there is an extra option, \verb|q|.
The \verb|q| option formats a string in a form suitable to be safely read
back by the Lua interpreter:
The string is written between double quotes,
and all double quotes, returns, and backslashes in the string
are correctly escaped when written.
For instance, the call
\begin{verbatim}
       string.format('%q', 'a string with "quotes" and \n new line')
\end{verbatim}
will produce the string:
\begin{verbatim}
"a string with \"quotes\" and \
 new line"
\end{verbatim}

The options \verb|c|, \verb|d|, \verb|E|, \verb|e|, \verb|f|,
\verb|g|, \verb|G|, \verb|i|, \verb|o|, \verb|u|, \verb|X|, and \verb|x| all
expect a number as argument,
whereas \verb|q| and \verb|s| expect a string.
The \verb|*| modifier can be simulated by building
the appropriate format string.
For example, \verb|"%*g"| can be simulated with
\verb|"%"..width.."g"|.

\NOTE
String values to be formatted with
\verb|%s| cannot contain embedded zeros.

\subsubsection*{\ff \T{string.gsub (s, pat, repl [, n])}}
\DefLIB{string.gsub}
Returns a copy of \verb|s|
in which all occurrences of the pattern \verb|pat| have been
replaced by a replacement string specified by \verb|repl|.
\verb|gsub| also returns, as a second value,
the total number of substitutions made.

If \verb|repl| is a string, then its value is used for replacement.
Any sequence in \verb|repl| of the form \verb|%|\M{n},
with \M{n} between 1 and 9,
stands for the value of the \M{n}-th captured substring.

If \verb|repl| is a function, then this function is called every time a
match occurs, with all captured substrings passed as arguments,
in order (see below);
if the pattern specifies no captures,
then the whole match is passed as a sole argument.
If the value returned by this function is a string,
then it is used as the replacement string;
otherwise, the replacement string is the empty string.

The last, optional parameter \verb|n| limits
the maximum number of substitutions to occur.
For instance, when \verb|n| is 1 only the first occurrence of
\verb|pat| is replaced.

Here are some examples:
\begin{verbatim}
   x = gsub("hello world", "(%w+)", "%1 %1")
   --> x="hello hello world world"

   x = gsub("hello world", "(%w+)", "%1 %1", 1)
   --> x="hello hello world"

   x = gsub("hello world from Lua", "(%w+)%s*(%w+)", "%2 %1")
   --> x="world hello Lua from"

   x = gsub("home = $HOME, user = $USER", "%$(%w+)", getenv)
   --> x="home = /home/roberto, user = roberto"  (for instance)

   x = gsub("4+5 = $return 4+5$", "%$(.-)%$", dostring)
   --> x="4+5 = 9"

   local t = {name="Lua", version="4.1"}
   x = gsub("$name - $version", "%$(%w+)", function (v) return t[v] end)
   --> x="Lua - 4.1"
\end{verbatim}


\subsubsection*{Patterns} \label{pm}

\paragraph{Character Class:}
a \Def{character class} is used to represent a set of characters.
The following combinations are allowed in describing a character class:
\begin{description}\leftskip=20pt
\item[\emph{x}] (where \emph{x} is not one of the magic characters
\verb|^$()%.[]*+-?|)
--- represents the character \emph{x} itself.
\item[\T{.}] --- (a dot) represents all characters.
\item[\T{\%a}] --- represents all letters.
\item[\T{\%c}] --- represents all control characters.
\item[\T{\%d}] --- represents all digits.
\item[\T{\%l}] --- represents all lowercase letters.
\item[\T{\%p}] --- represents all punctuation characters.
\item[\T{\%s}] --- represents all space characters.
\item[\T{\%u}] --- represents all uppercase letters.
\item[\T{\%w}] --- represents all alphanumeric characters.
\item[\T{\%x}] --- represents all hexadecimal digits.
\item[\T{\%z}] --- represents the character with representation 0.
\item[\T{\%\M{x}}] (where \M{x} is any non-alphanumeric character)  ---
represents the character \M{x}.
This is the standard way to escape the magic characters.
We recommend that any punctuation character (even the non magic)
should be preceded by a \verb|%|
when used to represent itself in a pattern.

\item[\T{[\M{set}]}] ---
represents the class which is the union of all
characters in \M{set}.
A range of characters may be specified by
separating the end characters of the range with a \verb|-|.
All classes \verb|%|\emph{x} described above may also be used as
components in \M{set}.
All other characters in \M{set} represent themselves.
For example, \verb|[%w_]| (or \verb|[_%w]|)
represents all alphanumeric characters plus the underscore,
\verb|[0-7]| represents the octal digits,
and \verb|[0-7%l%-]| represents the octal digits plus
the lowercase letters plus the \verb|-| character.

The interaction between ranges and classes is not defined.
Therefore, patterns like \verb|[%a-z]| or \verb|[a-%%]|
have no meaning.

\item[\T{[\^\null\M{set}]}] ---
represents the complement of \M{set},
where \M{set} is interpreted as above.
\end{description}
For all classes represented by single letters (\verb|%a|, \verb|%c|, \ldots),
the corresponding uppercase letter represents the complement of the class.
For instance, \verb|%S| represents all non-space characters.

The definitions of letter, space, etc.\ depend on the current locale.
In particular, the class \verb|[a-z]| may not be equivalent to \verb|%l|.
The second form should be preferred for portability.

\paragraph{Pattern Item:}
a \Def{pattern item} may be
\begin{itemize}
\item
a single character class,
which matches any single character in the class;
\item
a single character class followed by \verb|*|,
which matches 0 or more repetitions of characters in the class.
These repetition items will always match the longest possible sequence;
\item
a single character class followed by \verb|+|,
which matches 1 or more repetitions of characters in the class.
These repetition items will always match the longest possible sequence;
\item
a single character class followed by \verb|-|,
which also matches 0 or more repetitions of characters in the class.
Unlike \verb|*|,
these repetition items will always match the \emph{shortest} possible sequence;
\item
a single character class followed by \verb|?|,
which matches 0 or 1 occurrence of a character in the class;
\item
\T{\%\M{n}}, for \M{n} between 1 and 9;
such item matches a substring equal to the \M{n}-th captured string
(see below);
\item
\T{\%b\M{xy}}, where \M{x} and \M{y} are two distinct characters;
such item matches strings that start with~\M{x}, end with~\M{y},
and where the \M{x} and \M{y} are \emph{balanced}.
This means that, if one reads the string from left to right,
counting \Math{+1} for an \M{x} and \Math{-1} for a \M{y},
the ending \M{y} is the first \M{y} where the count reaches 0.
For instance, the item \verb|%b()| matches expressions with
balanced parentheses.
\end{itemize}

\paragraph{Pattern:}
a \Def{pattern} is a sequence of pattern items.
A \verb|^| at the beginning of a pattern anchors the match at the
beginning of the subject string.
A \verb|$| at the end of a pattern anchors the match at the
end of the subject string.
At other positions,
\verb|^| and \verb|$| have no special meaning and represent themselves.

\paragraph{Captures:}
A pattern may contain sub-patterns enclosed in parentheses;
they describe \Def{captures}.
When a match succeeds, the substrings of the subject string
that match captures are stored (\emph{captured}) for future use.
Captures are numbered according to their left parentheses.
For instance, in the pattern \verb|"(a*(.)%w(%s*))"|,
the part of the string matching \verb|"a*(.)%w(%s*)"| is
stored as the first capture (and therefore has number~1);
the character matching \verb|.| is captured with number~2,
and the part matching \verb|%s*| has number~3.

\NOTE
A pattern cannot contain embedded zeros.  Use \verb|%z| instead.


\subsection{Table Manipulation}
This library provides generic functions for table manipulation,
It provides all its functions inside the table \DefLIB{table}.

Most functions in the table library library assume that the table
represents an array or a list.
For those functions, an important concept is the \emph{size} of the array.
There are three ways to specify that size:
\begin{itemize}
\item the field \verb|"n"| ---
When the table has a field \verb|"n"| with a numerical value,
that value is assumed as its size.
\item \verb|setn| ---
You can call the \verb|table.setn| function to explicitly set
the size of a table.
\item implicit size ---
%% TODO
\end{itemize}
For more details, see the descriptions of the \verb|table.getn| and
\verb|table.setn| functions.

\subsubsection*{\ff \T{table.foreach (table, func)}}\DefLIB{table.foreach}
Executes the given \verb|func| over all elements of \verb|table|.
For each element, \verb|func| is called with the index and
respective value as arguments.
If \verb|func| returns a non-\nil{} value,
then the loop is broken, and this value is returned
as the final value of \verb|foreach|.

The behavior of \verb|foreach| is \emph{undefined} if you change
the table \verb|t| during the traversal.


\subsubsection*{\ff \T{table.foreachi (table, func)}}\DefLIB{table.foreachi}
Executes the given \verb|func| over the
numerical indices of \verb|table|.
For each index, \verb|func| is called with the index and
respective value as arguments.
Indices are visited in sequential order,
from~1 to \verb|n|,
where \verb|n| is the size of the table \see{getn}.
If \verb|func| returns a non-\nil{} value,
then the loop is broken, and this value is returned
as the final value of \verb|foreachi|.

\subsubsection*{\ff \T{table.getn (table)}}\DefLIB{table.getn}\label{getn}
Returns the ``size'' of a table, when seen as a list.
If the table has an \verb|n| field with a numeric value,
this value is the ``size'' of the table.
Otherwise, if there was a previous call to
\verb|table.getn| or to \verb|table.setn| over this table,
the respective value is returned.
Otherwise, the ``size'' is one less the first integer index with
a \nil{} value.

Notice that the last option happens only once for a table.
If you call \verb|table.getn| again over the same table,
it will return the same previous result,
even if the table has been modified.
The only way to change the value of \verb|table.getn| is by calling
\verb|table.setn| or assigning to field \verb|"n"| in the table.

\subsubsection*{\ff \T{table.sort (table [, comp])}}\DefLIB{table.sort}
Sorts table elements in a given order, \emph{in-place},
from \verb|table[1]| to \verb|table[n]|,
where \verb|n| is the size of the table \see{getn}.
If \verb|comp| is given,
then it must be a function that receives two table elements,
and returns true
when the first is less than the second
(so that \verb|not comp(a[i+1],a[i])| will be true after the sort).
If \verb|comp| is not given,
then the standard Lua operator \verb|<| is used instead.

The sort algorithm is \emph{not} stable
(that is, elements considered equal by the given order
may have their relative positions changed by the sort).

\subsubsection*{\ff \T{table.insert (table, [pos,] value)}}\DefLIB{table.insert}

Inserts element \verb|value| at position \verb|pos| in \verb|table|,
shifting other elements up to open space, if necessary.
The default value for \verb|pos| is \verb|n+1|,
where \verb|n| is the size of the table \see{getn},
so that a call \verb|table.insert(t,x)| inserts \verb|x| at the end
of table \verb|t|.
This function also updates the size of the table,
calling \verb|table.setn(table, n+1)|.

\subsubsection*{\ff \T{table.remove (table [, pos])}}\DefLIB{table.remove}

Removes from \verb|table| the element at position \verb|pos|,
shifting other elements down to close the space, if necessary.
Returns the value of the removed element.
The default value for \verb|pos| is \verb|n|,
where \verb|n| is the size of the table \see{getn},
so that a call \verb|tremove(t)| removes the last element
of table \verb|t|.
This function also updates the size of the table,
calling \verb|table.setn(table, n-1)|.

\subsubsection*{\ff \T{table.setn (table, n)}}\DefLIB{table.setn}

Updates the ``size'' of a table.
If the table has a field \verb|"n"| with a numerical value,
that value is changed to the given \verb|n|.
Otherwise, it updates an internal state of the \verb|table| library
so that subsequent calls to \verb|table.getn(table)| return \verb|n|.


\subsection{Mathematical Functions} \label{mathlib}

This library is an interface to most functions of the standard C~math library.
(Some have slightly different names.)
It provides all its functions inside the table \DefLIB{math}.
In addition,
it registers a ??tag method for the binary exponentiation operator \verb|^|
that returns \Math{x^y} when applied to numbers \verb|x^y|.

The library provides the following functions:
\DefLIB{math.abs}\DefLIB{math.acos}\DefLIB{math.asin}\DefLIB{math.atan}
\DefLIB{math.atan2}\DefLIB{math.ceil}\DefLIB{math.cos}
\DefLIB{math.def}\DefLIB{math.exp}
\DefLIB{math.floor}\DefLIB{math.log}\DefLIB{math.log10}
\DefLIB{math.max}\DefLIB{math.min}
\DefLIB{math.mod}\DefLIB{math.rad}\DefLIB{math.sin}
\DefLIB{math.sqrt}\DefLIB{math.tan}
\DefLIB{math.frexp}\DefLIB{math.ldexp}\DefLIB{math.random}
\DefLIB{math.randomseed}
\begin{verbatim}
       math.abs   math.acos   math.asin  math.atan math.atan2
       math.ceil  math.cos    math.deg   math.exp  math.floor
       math.log   math.log10  math.max   math.min  math.mod
       math.rad   math.sin    math.sqrt  math.tan  math.frexp
       math.ldexp math.random math.randomseed
\end{verbatim}
plus a variable \IndexLIB{math.pi}.
Most of them
are only interfaces to the homonymous functions in the C~library,
except that, for the trigonometric functions,
all angles are expressed in \emph{degrees}, not radians.
The functions \verb|math.deg| and \verb|math.rad| can be used to convert
between radians and degrees.

The function \verb|math.max| returns the maximum
value of its numeric arguments.
Similarly, \verb|math.min| computes the minimum.
Both can be used with 1, 2, or more arguments.

The functions \verb|math.random| and \verb|math.randomseed|
are interfaces to the simple random generator functions
\verb|rand| and \verb|srand|, provided by ANSI~C.
(No guarantees can be given for their statistical properties.)
When called without arguments,
\verb|math.random| returns a pseudo-random real number
in the range \Math{[0,1)}.
When called with a number \Math{n},
\verb|math.random| returns a pseudo-random integer in the range \Math{[1,n]}.
When called with two arguments, \Math{l} and \Math{u},
\verb|math.random| returns a pseudo-random integer in the range \Math{[l,u]}.


\subsection{Input and Output Facilities} \label{libio}

The I/O library provides two different styles for file manipulation.
The first one uses implicit file descriptors;
that is, there are operations to set a default input file and a
default output file,
and all input/output operations are over those default files.
The second style uses explicit file descriptors.

When using implicit file descriptors,
all operations are supplied by table \DefLIB{io}.
When using explicit file descriptors,
the operation \DefLIB{io.open} returns a file descriptor,
and then all operations are supplied as methods by the file descriptor.

Moreover, the table \verb|io| also provides
three predefined file descriptors:
\DefLIB{io.stdin}, \DefLIB{io.stdout}, and \DefLIB{io.stderr},
with their usual meaning from C.

A file handle is a userdata containing the file stream (\verb|FILE*|),
with a distinctive metatable created by the I/O library.

Unless otherwise stated,
all I/O functions return \nil{} on failure
(plus an error message as a second result)
and some value different from \nil{} on success.

\subsubsection*{\ff \T{io.close ([handle])}}\DefLIB{io.close}

Equivalent to \verb|fh:close| over the default output file.

\subsubsection*{\ff \T{io.flush ()}}\DefLIB{io.flush}

Equivalent to \verb|fh:flush| over the default output file.

\subsubsection*{\ff \T{io.input ([file])}}\DefLIB{io.input}

When called with a file name, it opens the named file (in text mode),
and sets its handle as the default input file
(and returns nothing).
When called with a file handle,
it simply sets that file handle as the default input file.
When called without parameters,
it returns the current default input file.

In case of errors this function raises the error,
instead of returning an error code.

\subsubsection*{\ff \T{io.open (filename, mode)}}\DefLIB{io.open}

This function opens a file,
in the mode specified in the string \verb|mode|.
It returns a new file handle,
or, in case of errors, \nil{} plus an error message.

The \verb|mode| string can be any of the following:
\begin{description}\leftskip=20pt
\item[``r''] read mode;
\item[``w''] write mode;
\item[``a''] append mode;
\item[``r+''] update mode, all previous data is preserved;
\item[``w+''] update mode, all previous data is erased;
\item[``a+''] append update mode, previous data is preserved,
  writing is only allowed at the end of file.
\end{description}
The \verb|mode| string may also have a \verb|b| at the end,
which is needed in some systems to open the file in binary mode.
This string is exactly what is used in the standard~C function \verb|fopen|.

\subsubsection*{\ff \T{io.output ([file])}}\DefLIB{io.output}

Similar to \verb|io.input|, but operates over the default output file.

\subsubsection*{\ff \T{io.read (format1, ...)}}\DefLIB{io.read}

Equivalent to \verb|fh:read| over the default input file.

\subsubsection*{\ff \T{io.tmpfile ()}}\DefLIB{io.tmpfile}

Returns a handle for a temporary file.
This file is open in read/write mode,
and it is automatically removed when the program ends.

\subsubsection*{\ff \T{io.write (value1, ...)}}\DefLIB{io.write}

Equivalent to \verb|fh:write| over the default output file.



\subsubsection*{\ff \T{fh:close ([handle])}}\DefLIB{fh:close}

Closes the file \verb|fh|.

\subsubsection*{\ff \T{fh:flush ()}}\DefLIB{fh:flush}

Saves any written data to the file \verb|fh|.

\subsubsection*{\ff \T{fh:read (format1, ...)}}\DefLIB{fh:read}

Reads the file \verb|fh|,
according to the given formats, which specify what to read.
For each format,
the function returns a string (or a number) with the characters read,
or \nil{} if it cannot read data with the specified format.
When called without formats,
it uses a default format that reads the entire next line
(see below).

The available formats are
\begin{description}\leftskip=20pt
\item[``*n''] reads a number;
this is the only format that returns a number instead of a string.
\item[``*a''] reads the whole file, starting at the current position.
On end of file, it returns the empty string.
\item[``*l''] reads the next line (skipping the end of line),
returning \nil{} on end of file.
This is the default format.
\item[\emph{number}] reads a string with up to that number of characters,
or \nil{} on end of file.
If number is zero,
it reads nothing and returns an empty string,
or \nil{} on end of file.
\end{description}

\subsubsection*{\ff \T{fh:seek ([whence] [, offset])}}\DefLIB{fh:seek}

Sets and gets the file position,
measured in bytes from the beginning of the file,
to the position given by \verb|offset| plus a base
specified by the string \verb|whence|, as follows:
\begin{description}\leftskip=20pt
\item[``set''] base is position 0 (beginning of the file);
\item[``cur''] base is current position;
\item[``end''] base is end of file;
\end{description}
In case of success, function \verb|seek| returns the final file position,
measured in bytes from the beginning of the file.
If this function fails, it returns \nil,
plus a string describing the error.

The default value for \verb|whence| is \verb|"cur"|,
and for \verb|offset| is 0.
Therefore, the call \verb|file:seek()| returns the current
file position, without changing it;
the call \verb|file:seek("set")| sets the position to the
beginning of the file (and returns 0);
and the call \verb|file:seek("end")| sets the position to the
end of the file, and returns its size.

\subsubsection*{\ff \T{fh:write (value1, ...)}}\DefLIB{fh:write}

Writes the value of each of its arguments to
the filehandle \verb|fh|.
The arguments must be strings or numbers.
To write other values,
use \verb|tostring| or \verb|format| before \verb|write|.
If this function fails, it returns \nil,
plus a string describing the error.


\subsection{Operating System Facilities} \label{libiosys}

This library is implemented through table \DefLIB{os}.

\subsubsection*{\ff \T{os.clock ()}}\DefLIB{os.clock}

Returns an approximation of the amount of CPU time
used by the program, in seconds.

\subsubsection*{\ff \T{os.date ([format [, time]])}}\DefLIB{os.date}

Returns a string or a table containing date and time,
formatted according to the given string \verb|format|.

If the \verb|time| argument is present,
this is the time to be formatted
(see the \verb|time| function for a description of this value).
Otherwise, \verb|date| formats the current time.

If \verb|format| starts with \verb|!|,
then the date is formatted in Coordinated Universal Time.

After that optional character,
if \verb|format| is \verb|*t|,
then \verb|date| returns a table with the following fields:
\verb|year| (four digits), \verb|month| (1--12), \verb|day| (1--31),
\verb|hour| (0--23), \verb|min| (0--59), \verb|sec| (0--61),
\verb|wday| (weekday, Sunday is 1),
\verb|yday| (day of the year),
and \verb|isdst| (daylight saving flag, a boolean).

If format is not \verb|*t|,
then \verb|date| returns the date as a string,
formatted according with the same rules as the C~function \verb|strftime|.
When called without arguments,
\verb|date| returns a reasonable date and time representation that depends on
the host system and on the current locale
(that is, \verb|os.date()| is equivalent to \verb|os.date("%c")|).

\subsubsection*{\ff \T{os.difftime (t1, t2)}}\DefLIB{os.difftime}

Returns the number of seconds from time \verb|t1| to time \verb|t2|.
In Posix, Windows, and some other systems,
this value is exactly \verb|t1|\Math{-}\verb|t2|.

\subsubsection*{\ff \T{os.execute (command)}}\DefLIB{os.execute}

This function is equivalent to the C~function \verb|system|.
It passes \verb|command| to be executed by an operating system shell.
It returns a status code, which is system-dependent.

\subsubsection*{\ff \T{os.exit ([code])}}\DefLIB{os.exit}

Calls the C~function \verb|exit|,
with an optional \verb|code|,
to terminate the host program.
The default value for \verb|code| is the success code.

\subsubsection*{\ff \T{os.getenv (varname)}}\DefLIB{os.getenv}

Returns the value of the process environment variable \verb|varname|,
or \nil{} if the variable is not defined.

\subsubsection*{\ff \T{os.remove (filename)}}\DefLIB{os.remove}

Deletes the file with the given name.
If this function fails, it returns \nil,
plus a string describing the error.

\subsubsection*{\ff \T{os.rename (name1, name2)}}\DefLIB{os.rename}

Renames file named \verb|name1| to \verb|name2|.
If this function fails, it returns \nil,
plus a string describing the error.

\subsubsection*{\ff \T{os.setlocale (locale [, category])}}\DefLIB{os.setlocale}

This function is an interface to the C~function \verb|setlocale|.
\verb|locale| is a string specifying a locale;
\verb|category| is an optional string describing which category to change:
\verb|"all"|, \verb|"collate"|, \verb|"ctype"|,
\verb|"monetary"|, \verb|"numeric"|, or \verb|"time"|;
the default category is \verb|"all"|.
The function returns the name of the new locale,
or \nil{} if the request cannot be honored.

\subsubsection*{\ff \T{os.time ([table])}}\DefLIB{os.time}

Returns the current time when called without arguments,
or a time representing the date and time specified by the given table.
This table must have fields \verb|year|, \verb|month|, and \verb|day|,
and may have fields \verb|hour|, \verb|min|, \verb|sec|, and \verb|isdst|
(for a description of these fields, see the \verb|os.date| function).

The returned value is a number, whose meaning depends on your system.
In Posix, Windows, and some other systems, this number counts the number
of seconds since some given start time (the ``epoch'').
In other systems, the meaning is not specified,
and the number returned bt \verb|time| can be used only as an argument to
\verb|date| and \verb|difftime|.

\subsubsection*{\ff \T{os.tmpname ()}}\DefLIB{os.tmpname}

Returns a string with a file name that can
be used for a temporary file.
The file must be explicitly opened before its use
and removed when no longer needed.

This function is equivalent to the \verb|tmpnam| C~function,
and many people (and even some compilers!) advise against its use,
because between the time you call the function
and the time you open the file,
it is possible for another process
to create a file with the same name.


\subsection{The Reflexive Debug Interface}

The library \verb|ldblib| provides
the functionality of the debug interface to Lua programs.
If you want to use this library,
your host application must open it,
by calling \verb|lua_dblibopen|.
\DefAPI{lua_dblibopen}

You should exert great care when using this library.
The functions provided here should be used exclusively for debugging
and similar tasks, such as profiling.
Please resist the temptation to use them as a
usual programming tool:
They can be \emph{very} slow.
Moreover, \verb|setlocal| and \verb|getlocal|
violate the privacy of local variables,
and therefore can compromise some (otherwise) secure code.


\subsubsection*{\ff \T{getinfo (function, [what])}}\DefLIB{getinfo}

This function returns a table with information about a function.
You can give the function directly,
or you can give a number as the value of \verb|function|,
which means the function running at level \verb|function| of the stack:
Level 0 is the current function (\verb|getinfo| itself);
level 1 is the function that called \verb|getinfo|;
and so on.
If \verb|function| is a number larger than the number of active functions,
then \verb|getinfo| returns \nil.

The returned table contains all the fields returned by \verb|lua_getinfo|,
with the string \verb|what| describing what to get.
The default for \verb|what| is to get all information available.
If present,
the option \verb|f|
adds a field named \verb|func| with the function itself.

For instance, the expression \verb|getinfo(1,"n").name| returns
the name of the current function, if a reasonable name can be found,
and \verb|getinfo(print)| returns a table with all available information
about the \verb|print| function.


\subsubsection*{\ff \T{getlocal (level, local)}}\DefLIB{getlocal}

This function returns the name and the value of the local variable
with index \verb|local| of the function at level \verb|level| of the stack.
(The first parameter or local variable has index~1, and so on,
until the last active local variable.)
The function returns \nil{} if there is no local
variable with the given index,
and raises an error when called with a \verb|level| out of range.
(You can call \verb|getinfo| to check whether the level is valid.)

\subsubsection*{\ff \T{setlocal (level, local, value)}}\DefLIB{setlocal}

This function assigns the value \verb|value| to the local variable
with index \verb|local| of the function at level \verb|level| of the stack.
The function returns \nil{} if there is no local
variable with the given index,
and raises an error when called with a \verb|level| out of range.
(You can call \verb|getinfo| to check whether the level is valid.)

\subsubsection*{\ff \T{setcallhook (hook)}}\DefLIB{setcallhook}

Sets the function \verb|hook| as the call hook;
this hook will be called every time the interpreter starts and
exits the execution of a function.
The only argument to the call hook is the event name (\verb|"call"| or
\verb|"return"|).
You can call \verb|getinfo| with level 2 to get more information about
the function being called or returning
(level~0 is the \verb|getinfo| function,
and level~1 is the hook function).
When called without arguments,
this function turns off call hooks.
\verb|setcallhook| returns the old call hook.

\subsubsection*{\ff \T{setlinehook (hook)}}\DefLIB{setlinehook}

Sets the function \verb|hook| as the line hook;
this hook will be called every time the interpreter changes
the line of code it is executing.
The only argument to the line hook is the line number the interpreter
is about to execute.
When called without arguments,
this function turns off line hooks.
\verb|setlinehook| returns the old line hook.


%------------------------------------------------------------------------------
\section{\Index{Lua Stand-alone}} \label{lua-sa}

Although Lua has been designed as an extension language,
to be embedded in a host C~program,
it is also frequently used as a stand-alone language.
An interpreter for Lua as a stand-alone language,
called simply \verb|lua|,
is provided with the standard distribution.
The stand-alone interpreter includes
all standard libraries plus the reflexive debug interface.
Its usage is:
\begin{verbatim}
      lua [options] [prog [args]]
\end{verbatim}
The options are:
\begin{description}\leftskip=20pt
\item[\T{-} ] executes \verb|stdin| as a file;
\item[\T{-e} \rm\emph{stat}] executes string \emph{stat};
\item[\T{-l} \rm\emph{file}] executes file \emph{file};
\item[\T{-i}] enters interactive mode after running \emph{prog};
\item[\T{-v}] prints version information;
\item[\T{--}] stop handling options.
\end{description}
After handling its options, \verb|lua| runs the given \emph{prog},
passing to it the given \emph{args}.
When called without arguments,
\verb|lua| behaves as \verb|lua -v -i| when \verb|stdin| is a terminal,
and as \verb|lua -| otherwise.

Before running any argument,
the intepreter checks for an environment variable \IndexVerb{LUA_INIT}.
If its format is \verb|@|\emph{filename},
then lua executes the file.
Otherwise, lua executes the string itself.

All options are handled in order, except \verb|-i|.
For instance, an invocation like
\begin{verbatim}
       $ lua -e'a=1' -e 'print(a)' prog.lua
\end{verbatim}
will first set \verb|a| to 1, then print \verb|a|,
and finally run the file \verb|prog.lua|.
(Here, \verb|$| is the shell prompt. Your prompt may be different.)

Before starting to run the program,
\verb|lua| collects all arguments in the command line
in a global table called \verb|arg|.
The program name is stored in index 0,
the first argument after the program goes to index 1,
and so on.
The field \verb|n| gets the number of arguments after the program name.
Any argument before the program name
(that is, the options plus the interpreter name)
goes to negative indices.
For instance, in the call
\begin{verbatim}
       $ lua -la.lua b.lua t1 t2
\end{verbatim}
the interpreter first runs the file \T{a.lua},
then creates a table
\begin{verbatim}
       arg = { [-2] = "lua", [-1] = "-la.lua", [0] = "b.lua",
               [1] = "t1", [2] = "t2"; n = 2 }
\end{verbatim}
and finally runs the file \T{b.lua}.

In interactive mode,
if you write an incomplete statement,
the interpreter waits for its completion.

If the global variable \IndexVerb{_PROMPT} is defined as a string,
then its value is used as the prompt.
Therefore, the prompt can be changed directly on the command line:
\begin{verbatim}
       $ lua -e"_PROMPT='myprompt> '" -i
\end{verbatim}
(the first pair of quotes is for the shell,
the second is for Lua),
or in any Lua programs by assigning to \verb|_PROMPT|.
Note the use of \verb|-i| to enter interactive mode; otherwise,
the program would end just after the assignment to \verb|_PROMPT|.

In Unix systems, Lua scripts can be made into executable programs
by using \verb|chmod +x| and the~\verb|#!| form,
as in \verb|#!/usr/local/bin/lua|.
(Of course,
the location of the Lua interpreter may be different in your machine.
If \verb|lua| is in your \verb|PATH|,
then a more portable solution is \verb|#!/usr/bin/env lua|.)


%------------------------------------------------------------------------------
\section*{Acknowledgments}

%% TODO rever isso?

The authors thank CENPES/PETROBRAS which,
jointly with \tecgraf, used early versions of
this system extensively and gave valuable comments.
The authors also thank Carlos Henrique Levy,
who found the name of the game.
Lua means ``moon'' in Portuguese.


\appendix

\section*{Incompatibilities with Previous Versions}
\addcontentsline{toc}{section}{Incompatibilities with Previous Versions}

We took a great care to avoid incompatibilities with
the previous public versions of Lua,
but some differences had to be introduced.
Here is a list of all these incompatibilities.


\subsection*{Incompatibilities with \Index{version 4.0}}

\subsubsection*{Changes in the Language}
\begin{itemize}

\item
Function calls written between parentheses result in exactly one value.

\item
A function call as the last expression in a list constructor
(like \verb|{a,b,f()}}|) has all its return values inserted in the list.

\item
\rwd{in} is a reserved word.

\item
When a literal string of the form \verb|[[...]]| starts with a newline,
this newline is ignored.

\item Old pre-compiled code is obsolete, and must be re-compiled.

\end{itemize}


\subsubsection*{Changes in the Libraries}
\begin{itemize}

\item
The \verb|read| option \verb|*w| is obsolete.

\item
The \verb|format| option \verb|%n$| is obsolete.

\end{itemize}


\subsubsection*{Changes in the API}
\begin{itemize}

\item
Userdata!!

\end{itemize}

%{===============================================================
\newpage
\section*{The Complete Syntax of Lua} \label{BNF}
\addcontentsline{toc}{section}{The Complete Syntax of Lua}

The notation used here is the usual extended BNF,
in which
\rep{\emph{a}}~means 0 or more \emph{a}'s, and
\opt{\emph{a}}~means an optional \emph{a}.
Non-terminals are shown in \emph{italics},
keywords are shown in {\bf bold},
and other terminal symbols are shown in {\tt typewriter} font,
enclosed in single quotes.


\renewenvironment{Produc}{\vspace{0.8ex}\par\noindent\hspace{3ex}\it\begin{tabular}{rrl}}{\end{tabular}\vspace{0.8ex}\par\noindent}

\renewcommand{\OrNL}{\\ & \Or & }
%\newcommand{\Nter}[1]{{\rm{\tt#1}}}
%\newcommand{\Nter}[1]{\ter{#1}}

\index{grammar}

\begin{Produc}

\produc{chunk}{\rep{stat \opt{\ter{;}}}}

\produc{block}{chunk}

\produc{stat}{%
	varlist1 \ter{=} explist1
\OrNL	functioncall
\OrNL	\rwd{do} block \rwd{end}
\OrNL	\rwd{while} exp \rwd{do} block \rwd{end}
\OrNL	\rwd{repeat} block \rwd{until} exp
\OrNL	\rwd{if} exp \rwd{then} block
	\rep{\rwd{elseif} exp \rwd{then} block}
	\opt{\rwd{else} block} \rwd{end}
\OrNL	\rwd{return} \opt{explist1}
\OrNL	\rwd{break}
\OrNL	\rwd{for} \Nter{Name} \ter{=} exp \ter{,} exp \opt{\ter{,} exp}
	\rwd{do} block \rwd{end}
\OrNL   \rwd{for} \Nter{Name} \rep{\ter{,} \Nter{Name}} \rwd{in} explist1
                    \rwd{do} block \rwd{end}
\OrNL	\rwd{function} funcname funcbody
\OrNL	\rwd{local} \rwd{function} \Nter{Name} funcbody
\OrNL	\rwd{local} namelist \opt{init}
}

\produc{funcname}{\Nter{Name} \rep{\ter{.} \Nter{Name}}
                              \opt{\ter{:} \Nter{Name}}}

\produc{varlist1}{var \rep{\ter{,} var}}

\produc{var}{%
	\Nter{Name}
\Or	prefixexp \ter{[} exp \ter{]}
\Or	prefixexp \ter{.} \Nter{Name}
}

\produc{namelist}{\Nter{Name} \rep{\ter{,} \Nter{Name}}}

\produc{init}{\ter{=} explist1}

\produc{explist1}{\rep{exp \ter{,}} exp}

\produc{exp}{%
	\rwd{nil}
	\rwd{false}
	\rwd{true}
\Or	\Nter{Number}
\OrNL	\Nter{Literal}
\Or	function
\Or	prefixexp
\OrNL	tableconstructor
\Or	exp binop exp
\Or	unop exp
}

\produc{prefixexp}{var \Or functioncall \Or \ter{(} exp \ter{)}}

\produc{functioncall}{%
	prefixexp args
\Or	prefixexp \ter{:} \Nter{Name} args
}

\produc{args}{%
	\ter{(} \opt{explist1} \ter{)}
\Or	tableconstructor
\Or	\Nter{Literal}
}

\produc{function}{\rwd{function} funcbody}

\produc{funcbody}{\ter{(} \opt{parlist1} \ter{)} block \rwd{end}}

\produc{parlist1}{%
	\Nter{Name} \rep{\ter{,} \Nter{Name}} \opt{\ter{,} \ter{\ldots}}
\Or	\ter{\ldots}
}

\produc{tableconstructor}{\ter{\{} \opt{fieldlist} \ter{\}}}
\produc{fieldlist}{field \rep{fieldsep field} \opt{fieldsep}}
\produc{field}{\ter{[} exp \ter{]} \ter{=} exp \Or name \ter{=} exp \Or exp}
\produc{fieldsep}{\ter{,} \Or \ter{;}}

\produc{binop}{\ter{+} \Or \ter{-} \Or \ter{*} \Or \ter{/} \Or \ter{\^{ }} \Or
  \ter{..} \Or \ter{<} \Or \ter{<=} \Or \ter{>} \Or \ter{>=}
  \Or \ter{==} \Or \ter{\~{ }=} \OrNL \rwd{and} \Or \rwd{or}}

\produc{unop}{\ter{-} \Or \rwd{not}}

\end{Produc}

%}===============================================================

% Index

\newpage
\addcontentsline{toc}{section}{Index}
\input{manual.id}

\end{document}


\end{document}


\end{document}

\end{theindex}


\end{document}
